% AER-Article.tex for AEA last revised 22 June 2011

% \documentclass[AER]{AEA} modified for the full path to my AEA.cls file
\documentclass[AER]{/Users/zyl357/Documents/GitHub/research-dissertation-case-for-alt-ed/papers/alt-ed-survey/aea-latex-templates/AEA}

\usepackage{tikz}
\usetikzlibrary{calc,matrix}

% The mathtime package uses a Times font instead of Computer Modern.
% Uncomment the line below if you wish to use the mathtime package:
%\usepackage[cmbold]{mathtime}
% Note that miktex, by default, configures the mathtime package to use commercial fonts
% which you may not have. If you would like to use mathtime but you are seeing error
% messages about missing fonts (mtex.pfb, mtsy.pfb, or rmtmi.pfb) then please see
% the technical support document at http://www.aeaweb.org/templates/technical_support.pdf
% for instructions on fixing this problem.

% Note: you may use either harvard or natbib (but not both) to provide a wider
% variety of citation commands than latex supports natively. See below.

% Uncomment the next line to use the natbib package with bibtex 
%\usepackage{natbib}

% Uncomment the next line to use the harvard package with bibtex
\usepackage[abbr]{harvard}

% This command determines the leading (vertical space between lines) in draft mode
% with 1.5 corresponding to "double" spacing.
\draftSpacing{1.5}

\begin{document}

\title{Attitudinal Trends in Alternative Postsecondary Learning}
\shortTitle{Trends in Alternative Learning}
\author{John Vandivier\thanks{Vandivier: George Mason University, 4400 University Dr, Fairfax, VA 22030, jvandivi@masonlive.gmu.edu. The author acknowledges valuable input from Bryan Caplan at George Mason University.}}
\date{\today}
\pubMonth{Month}
\pubYear{Year}
\pubVolume{Vol}
\pubIssue{Issue}
\JEL{D12, I21, I22, I24, I25, I26}
\Keywords{Education economics, alternative education, debt crisis, signaling}

\begin{abstract}
Traditional postsecondary learning in the United States is associated
with increased earnings and employment, but these benefits come at
substantial cost. Nontraditional education provides a partial workaround
with a variety of caveats including nonconformity stigma. This paper
explores a novel data set (n = 1190) to understand trends in public
disposition on alternative credentials. Results indicate that
favorability is high, declining in the short term, and may reverse trend
over time. Leveraging alternative pathways to reduce cost and accelerate
completion of accredited education is a recommended strategy. Some
policy considerations are discussed.
\end{abstract}

\maketitle

\section{Background and Motivation}

The concept of a student debt crisis has found durable academic and media
coverage. In 1999, Roots called the student loan debt crisis a lesson in
unintended consequences \cite{roots1999student}. He identified the issue as attributable in large
part to the Guaranteed Student Loan Program of 1965.

It was neither a new lesson at that time, nor a lesson finally learned at
that time. Hansendn and Rhodes discuss the student debt crisis in 1988 \cite{hansendn1988student}.
Van Dusen published a genuinely prescient paper, The Coming Crisis in
Student Aid, in February 1979 \cite{van1979coming}. Forbes noted in 2019\cite{friedman2018student} that “Student loan
debt in 2019 is the highest ever…There are more than 44 million borrowers
who collectively owe \$1.5 trillion in student loan debt in the U.S.
alone.”

Recent work has called into question both the social return and the
individual return to spending on education\footnote{Eg, \cite{caplan2018case}}. Alternatives to the status quo
in education present the opportunity for significant economic benefit.
From 1989 to 2012, the average cost of a year of undergraduate education
in the US rose 79 percent\footnote{This represents a price increase from \$11,862 to \$21,222 in constant 2016
dollars. This price includes tuition and fees and room and board rates charged for full-time students in
degree-granting postsecondary institutions. Data derived from \cite{nces2017averageundergraduatetuition}}.
Over the same period, per pupil public expenditure for
K-12 students increased 27 percent\footnote{This represents an increase from \$8,654 to \$11,011 in constant
2014 dollars. Data derived from \cite{nces2015expendituresperpupil}}. This indicates that
postsecondary education presents a particularly valuable area of exploration.

From 1989 to 2012, K-12 student expenditure increased significantly and
the cost of a year of undergraduate education grew nearly three times more
quickly, but the adjusted average starting salary of a college graduate
decreased. In real terms, the average starting salary of a college
graduate decreased about 9 percent\footnote{From 1989 to 2012, a decrease of \$4,385 from \$49,487 to
\$45,102 in constant 2016 dollars is observed. (4385/49487) = 0.089. From 1960 to 2012 an increase from
\$47,442 to \$50,219 is observed. Data derived from \cite{koncz2016}}.
Additional temporal sampling from 1960 to
2015 indicates that the longer trend for education is modestly positive,
with a real increase of about 6 percent over that period. It’s worth noting that
the highest paying years for the degree were observed around 1970 in real
terms, and salaries after the Great Recession have remained lower than the
early 2000s.

Because the price of college is rising several times faster than the rate
at which the salary of new graduates is increasing, the traditional degree
is becoming a dynamically worse financial investment, and current research
shows it is already a relatively poor choice compared to investing in a
standard fund from the social point of view, although it is clearly
lucrative from the individual perspective. Caplan estimates the private
average annual return to attempting a year of college is about 4.9 percent in
Chapter 5 of The Case Against Education. In Chapter 6 he calculates the
social return to college at less than 2 percent.

Annualized 20 year returns on an S\&P index, in contrast, typically
return between 4 and 9 percent\cite{isbitts_2018}. It’s worth noting that Caplan uses 2011-2012
in-state tuition and fees at public four-year universities when
calculating return. From the 2011-2012 school year to the 2018-2019 school
year, the relevant cost figure increased from 8,244 in 2011 dollars to
10,230 in 2018 dollars\cite{collegeboard2018fiveyearchange}. This represents a real cost increase of about 11 percent
from 9,203 to 10,230 in 2018 dollars, after accounting for a cumulative
rate of inflation of about 11.6 percent from 2011 to 2018\footnote{Cumulative inflation calculated using https://www.usinflationcalculator.com/}.

It’s also worth noting that this return on investment assumes no
expenditure for residence. Ostensibly the student will be living at home.
In practice, this greatly reduces the number of actual universities
available for a learner to choose from, and this may force a student to
choose a more expensive school. Many schools have an on-campus residency
requirement, and many other schools will be too far to commute for most
learners. Learning exclusively online from a provider which has no
on-campus residency requirement remains an option, but this can safely be
considered a non-traditional practice. In 2015, less than 15 percent of students
were exclusively distance learners\cite{nces2016percentexclusivelydistance}, although the number is trending up
over time.

Mattern and Wyatt note that college students live an average distance of
268 miles from home and a median of 94 miles. This indicates that most
students don’t live at home, and as a result housing costs should be
included in both the average and median analysis of college price. Baum
and Ma give the \$8,244 tuition and fees figure used by Caplan, and they
also provide a room and board figure of \$8,887 for the 2011-2012 academic
year. Abodo reports that median rent in 2018 for a 1/1 was about \$1,010
per month\cite{abodo_2019}, and this works out to about \$803 in 2011
dollars\footnote{Calculated from rent-specific inflation, not general inflation
figures. Derived from \cite{alioth2019}}. At 9 months
per year, it would have been a more affordable \$7,227 for the typical
student to live in a 1/1 apartment than to consume university room and
board.

% TODO: the Minnesota State quiz is an interesting resource but maybe not for this paper
Consider that the modal degree is a Business degree, and the average
salary for a business degree graduate is about \$54,000\cite{adams2013college}. A common,
in-demand occupation for someone holding a business degree is a business
analyst role. Business analysts earn about \$67,000 per year, and there
many reputable bootcamp-style alternative learning programs for this
occupation\cite{white_2018}. Many of these bootcamps are online and take less than 6 months
to complete and cost less than half of the \$7,227 price of housing alone.
Online learning is not equally suited to all individuals. Minnesota State
has a short quiz which helps individuals identify if online learning is a
good fit\cite{minnstate2019areonlinecoursesright}.

While traditional education is a decent investment, once we ignore typical
housing costs, alternative education is a strictly better investment. This
is because an individual can utilize alternative education to obtain a
traditional credential for a lower price at an accelerated rate, and
employers pay based on credentialing. The meaning of alternative education
for the purposes of this paper is detailed in section 2, but one example
of an alternative pathway to a traditional degree would be to leverage
credit by examination. Almost any degree program can be partially
fulfilled through credit by examination. In section 2.4 the price of
credit by examination is calculated as about 15 percent of the traditional way to
obtain credit in the average case.

% TODO: per youtube dude, placement rate has changed importantly over time
% this may be related to shifting employer preference. underlying quality change
In addition to generalized cost reductions from alternative pathways,
alternative credentialing programs sometimes allow learners to obtain
better salary or employment for even less than 15 percent of the price of a
traditional degree, and often over a much shorter time. In other cases,
alternative education may be more expensive but desirable even so. General
Assembly is an example of an Information Technology industry bootcamp.
They teach business, marketing, and design courses in addition to the
typical programming courses offered at similar programs. General Assembly
costs \$14,950 for its priciest full-time, immersive course, but students
can finance in creative ways like \$0 upfront with an income sharing
agreement, where a student doesn’t need to pay until employed full-time\cite{ga2019}.
The immersive lasts about 3 months. 88 percent of GA students found full-time
work within 90 days of graduation, and 99 percent found full-time work within 180
days\cite{kirkham_2017}. This is in notable contrast to the traditional degree, where 54 percent of
the class of 2015 had found a standard, full-time job 6 months after
graduation\cite{wexler_2016}.

Bootcamps can sometimes be used as a college substitute, but they can also
be used after college graduation to differentiate a job candidate from
competitors, or to switch careers or brush up on recent changes
mid-career. Finally, many traditional universities now offer through prior
learning assessments or credit by portfolio, so that bootcamps can result
in college credit even without officially partnering with a university\cite{aceposttraditionallearners}.

Traditional education might still be an optimal consumption choice if
students demand higher education as leisure, but survey data indicates
that this is not the case. Among a mix of prospective and first year
college students from ages 16-40\cite{fishman_2015}, Rachel Fishman finds that the top three
reasons to go to college are improved employment, making more money, and
getting a good job. Over 90 percent of respondents affirmed at least one of these
reasons.

In A New U, Craig documents several faster and cheaper alternatives to
college\cite{craig_2018}. Craig establishes that many of these alternative education
solutions are quickly growing in both supply and demand, but it is not
obvious whether the programs Craig discusses are representative of the
broader ecosystem of alternative learning. Prior to Craig’s writing, Bryan
Caplan argues for the signaling model of postsecondary credential value\cite{caplan2018case}.
On Caplan’s view, the consumer of alternative credentials faces a signal
composition problem which threatens the value of the credential.
Traditional credentials may do a better job of signaling things like work
ethic and conformity.

Alternative education, however, may endow real skills at a better rate
than traditional education. Caplan estimates, for example, that the value
of vocational education benefits is 40 percent signaling, in contrast with 80 percent
signaling for the usual college education\footnote{Chapter 8, \cite{caplan2018case}}. If employers can obtain
better-skilled workers for lower cost, they would be expected to have some
willingness to give on conformity. In addition, as alternative credentials
become more widely accepted, any stigma or nonconformity costs from
pursuing alternative education is expected to diminish. Additionally,
prior research has yet to establish magnitudes and dynamic trends on those
magnitudes for many of these important effects.

This paper explores a novel attitudinal data set on alternative
credentials\footnote{ The data used in this survey are publicly accessible
at https://github.com/Vandivier/research-dissertation-case-for-alt-ed/tree/master/papers/alt-ed-survey/190201-feb-survey-monkey/data}.
This paper tests the thesis that employers will favor
alternative credentials as a mechanism to identify suitable entry level
employment. As a secondary interest, changes over time to the relation of
interest are investigated. The structure of included survey data allows
for exploration of several other interesting tertiary relations.

The first section describes the organization of the paper, the motivation,
and the main thesis. The second section gives theoretical and historical
context. The third section presents findings. The fourth section describes
applications or use cases for findings.

\section{Introduction to the Theory of Alternative Education}

This second section is broken into six subsections. The first subsection
gives an overview of the subsequent five sections.

Because alternative education is characterized as the negation of
traditional education, the second subsection begins by conceptualizing
tradition. Section 2.3 moves from theoretical conceptualization of
tradition into a brief inventory of actual American history, including
legislation relevant to the topic of interest. The description of
traditional credentials toward the end of the third section flows
naturally into a discussion on alternative credentials in the fourth
section. The three subtypes of alternative education are detailed
throughout the fourth section. The subtypes include alternative
credentials, alternative pathways, and alternative pedagogies.

Section 2.5 synthesizes these subtypes into a discussion of alternative
education and alternativeness in a general way. The fifth subsection also
discusses minor results from outside of the main data set which help
concretize the concept of alternativeness intended throughout the paper.
To further clarify the intuition of the thesis, Subsection 2.6 describes a
game-theoretic model of dynamic norms.

\section{Traditions Conceptualized}

Throughout the paper, the variable of interest will be referred to in a
few ways. The variable of interest is most concisely described as
entry-level suitability. More completely, the variable of interest is
favorability on the use of alternative credentials to qualify an
entry-level candidate who is applying for a career position. Technically,
and most completely, the variable of interest is a response between 1 and
10 to the question, “For many professions, alternative credentials can
qualify a person for an entry-level position.” A response of 10 represents
a strong agreement.

Alternative credentials fit into the broader research area on alternative
education. Alternative education is defined as all education other than
traditional education. Alternative education decomposes into three
subtopics including alternative credentials, alternative pathways, and
alternative pedagogy. Each of these alternative entities is defined by the
negation of their traditional counterpart.

In service of an effective description of alternative credentials,
traditions are described in this subsection. Traditions can be concisely
described as intertemporal social norms. As such, traditions exist in
socio-temporal space. The maximum socio-temporal space that such a
tradition could occupy would be from the dawn of humankind until today,
and among all humans. Tradition can be viewed on a spectrum, where some
processes are relatively traditional, and the modal process among a group
of comparable processes is the strictly defined traditional process,
implied when the singular traditional process is referred to.

\section{Actual Traditions: A History of Accreditation and Student
Lending in the United States}

From the vantage point just established, postsecondary accreditation is a
peculiar and infant approach to learning and education. Private
accrediting agencies began forming in the 1880s in the United States, and
private accreditation had become a well-established element of the higher
education landscape by the 1930s\cite{ace_2019}. The G.I. Bill was signed into law in
1944\cite{sullings_2019}, and provisions of the bill boosted consumption of higher education
through subsidy to military service members. The number of degrees awarded
by US colleges and universities more than doubled between 1940 and 1950.
The increased demand for education stimulated the formation of many new
colleges, and some of these were perceived to be of dubious quality\cite{wellman_1998}. The
G.I. Bill was reauthorized in 1952, but this time the educational benefits
it included were restricted in availability to those students enrolling at
an accredited institution, and the U.S. Commissioner of Education began
publishing a list of federally recognized accredited institutions.

Over time, federal recognition criteria became more elaborate. The 1992
Amendments to the Higher Education Act is a notable act in this regard.
Federal lending began in a military-oriented fashion with the National
Defense Education Act of 1958, but lending was expanded to the general
population with the 1965 Higher Education Act and subsequent legislation\cite{lumina_2019}.
As earlier noted, Roots and other scholars identify this legislative trend
as essentially causal to our present student debt crisis.

This brief history indicates that federal postsecondary accreditation is
not only new as a human institution, but also new within the much more
limited context of United States history. It’s true that market-driven
accreditation has existed since the 1880s, and therefore may be considered
a traditional process in United States education, but it is not true that
the federal accreditation process which exists today has been around
through most of American history. In this sense, federal accreditation is
both causal in our present debt crisis and decidedly nontraditional.

The point of this exercise is both to familiarize the reader with a bit of
relevant history, and to initiate conceptualization of traditional
education as a special case of alternative education, rather than
something altogether different. The traditional education of today was
itself an alternative form of education at some point in time, and it
remains a minority approach to education within a variety of nontrivial
timeframes.

Caplan rightly argues that part of the signal of a traditional degree is
to signal conformity, but throughout most of history it would be the
nonconformist who possesses the thing we now call a traditional degree.
Given this prior information, it becomes more plausible that society might
one day return to such a situation. Obtaining a federally accredited
undergraduate degree is a rather new practice, although we call it a
tradition, and it has always been dubiously socially beneficial.

Traditional education is loosely synonymous with accredited education in
the United States, but there are important technical differences.
Accredited credentials in the context of the US include the high school
diploma, the accredited undergraduate degree, and accredited graduate
education. While graduate education is accredited, it is also excluded
from the concept of traditional education. While tradition indicates a
normal activity, graduate education is unusual. About 9 percent of U.S. adults
had a graduate degree in 2000, and about 13 percent had such a degree as of 2018\cite{americacountsstaff_2019}.

In fact, it’s not technically traditional to get a college degree.
Technically speaking, the American tradition is to enroll in college
degree and never graduate. The history of factors leading to higher
enrollment in higher education in the United States, was previously
discussed, but it’s an important historical watermark to notice that 51%
of Americans immediately enrolled in college after high school completion
beginning in 1975\cite{aud2013condition}. Between 1975 and 2011, the immediate college enrollment
rate increased from 51 percent to 68 percent. Immediate transition to
college plateaued after the turn of the century. The immediate college
enrollment rates for 4-year and for 2-year colleges in 2016 were not
measurably different from 2000\cite{nces2019condition}.

Enrolling in college has been a tradition since 1975, but obtaining a
degree never was a tradition in the same way. The fact that the trend on
immediate enrollment has slowed from a positive trend into a plateau,
which has now remained stably flat for more than a decade, casts doubt on
the plausibility of a move back to a positive trend. For the foreseeable
future, the expected trend on immediate enrollment is between flat and the
possibility of a small decline.

In 2016, the percentage of students enrolling in college in the fall
immediately following high school completion was 69.8 percent\cite{nces_2019}, but in 2016 the
percentage of the adult population with a bachelor’s degree or higher was
33.4 percent\cite{censusbureau_2017} for “the first time in decades of data.” To reiterate the point, the
strictly modal pattern of educational attainment would be for an American
student to obtain a diploma, enroll in an accredited undergraduate degree
program, and never obtain an undergraduate degree.

\section{Three Subtypes of Alternative Education}

While the final paragraph of the above section describes the strictly
modal pattern of educational attainment in the modern United States, the
pattern of enrollment plus noncompletion does not describe the state of
being desired by those who enroll. The desired situation would be college
graduation. As all ideas are antecedent to action, the desire to complete
college is identified as more traditional than the actualization of
college completion. The nuanced difference between a desire and its
actualization is lost when speaking loosely, and as a result the actual
four-year degree is loosely identified as a traditional credential.

Alternative education is a general term which includes alternative
credentials, alternative pathways, and alternative pedagogies. Each of
these alternative entities is characterized by the negation of a
traditional counterpart.

Traditional pedagogy is the lecture format. K-12 education and higher
education have both typically utilized this teaching method during and
outside of the post-1975 period of interest, despite wide knowledge on the
ineffectiveness of this approach. In 2014, for example, a meta-analysis of
225 studies found that undergraduate students in classes with traditional
lectures are 1.5 times more likely to fail than students in classes that
use active learning methods\cite{freeman2014active}.

A pathway is a series of actions culminating in the attainment of
education or a credential. The traditional pathway always culminates in
the traditional credential, but alternative pathways may also culminate in
the traditional credential. For example, a student may obtain significant
college credit or even complete an entire degree program through credit by
examination. This competency-based pathway is importantly different
compared to the traditional pathway based on the credit hour.

While it’s common for students to self-study in preparation for credit by
examination, it’s also common for students to attend preparatory classes
or even obtain knowledge for the purposes of testing out of one course by
sitting in another traditional course. That situation could occur if a
student took a course at one university, changed universities, and the
credit would not transfer for the original course. Although their credit
did not automatically transfer, the student might be able to test out at
the second school. In cases like these, the student may have obtained a
traditional credential through education using a traditional lecture-based
pedagogy, and yet the pathway was not traditional.

The distinction of pathways might seem like splitting hairs in theory, but
in practice alternative pathways like credit by examination, prior
learning assessment, credit by portfolio, and similar processes hold
immense potential as a time and cost savings mechanism for the student,
while holding constant any concerns over lack of an accredited degree. To
briefly illustrate, the price of a CLEP test is \$89 in 2019 dollars\cite{collegeboard_2019}, while
the average cost per credit hour at an accredited college is \$594 in 2018
dollars\cite{kirkham2018study}. A CLEP test may substitute for a 4-credit course\footnote{Credit
may vary and is generally decided by the awarding institution rather than
the exam provider. Some CLEP tests are offered at other price points. CLEP
tests are a specific product provided by The College Board and other
providers may charge different prices. Other exams widely recognized for
credit by examination include AP, Cambridge International, DSST, Excelsior
College, and TECEP exams.}. This means
credit by examination is approximately 15 percent of the price of credit by
credit hour.

\section{Alternative Education Broadly Conceptualized}

Alternative education broadly encompasses all forms of formal and informal
learning, but such a process space exceeds feasible study in a single
paper, and frankly eludes sufficient study after combining many papers
across several fields. Instead of studying alternative education
holistically, researchers typically identify and studying a special case
or particular implementation of alternative education.

One benefit of this approach is that the researcher may identify specific
instances of alternative education which are faster, cheaper, or otherwise
preferred in some way relative to traditional education, but a weakness of
such an approach is that findings appear small, rare, disbursed, and ad
hoc. To collect such effects into a strong case against the existing norm,
a systematic approach is required which establishes alternativeness as an
independent factor which can then be tested for effect.

Alternativeness can be conceptualized ordinally or cardinally. Remember
that the three subtypes of alternative education are alternative
credentials, alternative pathways, and alternative pedagogies. Within each
of these three subtypes, solutions within a given subtype can be
identified and ranked according to popularity. After ranking from most
popular to n-popularity, an increase in rank number synonymously
represents decreasing traditional status and increasing alternativeness.

By directly utilizing the underlying measure of popularity, a cardinal
operationalization is achieved. Examples of popularity measures include
number of applications, number of enrollments, expenditure toward a
program, or survey-based measures of familiarity and favorability with
respect to a specific program.

As a brief concretization, a secondary data set is investigated. The
College Board, which administers the SAT, recognizes four types of high
school in 2014\cite{collegeboard_2014} and other years. These types include public, religiously
affiliated, independent, and a group combining other and unknown types of
schools. Table 1 shows reported measures of SAT performance by type of
high school, augmented with third party data for homeschoolers\cite{mullins_2016}. Table 2
shows basic model results for cardinal and ordinal operationalizations,
with linear and marginal effects in their expected directions. Low
significance is attributable to small sample size. Notice the non-trivial
R-square values identified despite the small sample size.

Other research indicates that charter schools\cite{di2011evidence} perform modestly better than
public schools when standardizing by SAT score, although nationally
representative charter school data could not be found, and gains vary
importantly by state and other factors.

\begin{table}
\caption{Rank Alternativeness and SAT Score by School Type}
\begin{tabular}{llll}
School Type & Test Taker Count & Rank Alternativeness & Total Score \\
Home School & 13549 & 5 & 1623 \\
Public & 1306039 & 1 & 1471 \\
Religious & 142783 & 2 & 1597 \\
Independent & 102358 & 4 & 1657 \\
Other and Unknown & 121215 & 3 & 1521 %
\end{tabular}
\begin{tablenotes}[Source]
Home School data from https://hslda.org/content/docs/news/2016/201606240.asp,
and other data from \cite{collegeboard_2014} % https://secure-media.collegeboard.org/digitalServices/pdf/sat/TotalGroup-2014.pdf
\end{tablenotes}
\end{table}

% TODO: remove from concise paper
\begin{table}
\caption{SAT Score by Section and School Type}
\begin{tabular}{lllll}
School Type & Test Taker Count & Math & Reading & Writing \\
Home School & 13549 & 521 & 567 & 535 \\
Public & 1306039 & 501 & 492 & 478 \\
Religious & 142783 & 537 & 533 & 527 \\
Independent & 102358 & 580 & 535 & 542 \\
Other and Unknown & 121215 & 558 & 479 & 484 %
\end{tabular}
\begin{tablenotes}[Source]
Home School data from https://hslda.org/content/docs/news/2016/201606240.asp,
and other data from \cite{collegeboard_2014} % https://secure-media.collegeboard.org/digitalServices/pdf/sat/TotalGroup-2014.pdf
\end{tablenotes}
\end{table}

A simple OLS regression of total score on rank alternativeness with a
constant yields an effect coefficient of 36.40 with a p-value of 0.14. The
constant takes a value of 1464.6. R-squared is 0.57 and the adjusted value
is 0.43. Diminishing returns are expected, but the lack of degrees of
freedom preclude deeper validation in this quick exploration.

\section{A Game-Theoretic Model of Dynamic Norms}

It’s expected and intuitive that rank alternativeness would have a
positive linear and negative marginal effect. Traditions are a kind of
durable norm or institution. It’s a foundational lesson of New
Institutional Economics that norms, institutions, and other classes of
informal rules are self-sustaining and socially valued\cite{dequech2006institutions}. If alternativeness
continued to move in a positive direction ad infinitum, this would seem to
indicate that traditions are perfectly opposed to maximal social value.
That would be a jarring result which would fly in the face of much of
mainline economics. Instead, the observed positive linear and negative
marginal effects collectively indicate something much more compatible with
orthodox economics. The indication is that some of tradition’s near
neighbors represent an improvement, but continuing into the deeply
alternative eventually detracts from value.

Education fits a non-special case of norms models. As a completely
standard example, Conley and Neilson use a prisoner’s dilemma to
demonstrate equilibrium adoption of social norms\cite{conley2009endogenous}. Suppose we modify this
approach to account for dynamic technical improvement. In the present
approach, consider an infinitely repeated prisoner’s dilemma where each
round adds an additional option to choose cooperateN. CooperateN pays off
(1 + cooperateN-1), and in the first round the participants are known to
choose cooperate, because they have already equilibrated on the cooperate
choice as a social norm. For the sake of modelling, also suppose there is
a defectN added in each round as well, and it pays off (defectN-1 – 1),
although it’s immediately obvious that coordination on such defects never
obtains.

Alternative credentials include both technological improvements, and
ostensibly technological degradation, relative to present-day norms, but
according to the above game-theoretic representation, it’s not expected
for society to equilibrate on any of the technologically degraded choices.
Instead, it’s observed in the model that society will tend to adopt those
preferred alternatives as they become available over time.

In a more complex model, suppose that instead of both players gaining
certain knowledge of the new cooperate option, each player had some
probability of knowledge of the new cooperate option and some level of
risk aversion. It’s now seen that there is some delay in adoption of the
new cooperate choice, and in some rounds one or both players may prefer to
remain on the prior cooperate space, but eventually all players tend
toward the highest value cooperate choice.

In a third model of highest complexity, suppose that two new cooperate
choices are added each round instead of a single choice each round, but
players only probabilistically know about the new choices. Figure 1
presents a diagram of this game. The first new cooperate choice is
revealed to both players with 99 percent probability, and the second cooperate
choice has a payoff which is larger by one unit, but it is revealed to
each player with a probability of 10 percent. Given standard models of nonlinear
risk aversion, players will coordinate on the choice which is revealed
with near certainty.

% ref: https://tex.stackexchange.com/questions/148607/problem-typesetting-a-prisoners-dilemma-table\begin{tikzpicture}[element/.style={minimum width=1.75cm,minimum height=0.85cm}]
\begin{figure}
    \centering
    \caption{Modified Iterated Prisoner's Dilemma}
    \label{Modified Iterated PD Label}
    \begin{tikzpicture}[element/.style={minimum width=1.75cm, minimum height=0.85cm}]

    \matrix (m1) [matrix of nodes,
    	nodes={element},
	column sep=-\pgflinewidth,
	row sep=-.05cm,]
    {
        & P1C2 & P1C1 \\
        P2C2 & |[draw]|1,4,4 & |[draw]|1,0,2 \\
        P2C1 & |[draw]|1,2,0 & |[draw]|1,0,0 \\
    };
    
    % m1 label node
    \node[above=0.25cm] at ($(m1-1-2)!0.5!(m1-1-3)$) {\textbf{Stage 1}};

    \matrix (m2) [matrix of nodes,
    	nodes={element},
	column sep=-\pgflinewidth,
	row sep=-.05cm,
	anchor=north,]
		 at ($(m1-1-2)!0.5!(m1-1-3) + (-.75, -3)$)
    {
        & P1C4 & P1C3 & P1C2 & P1C1 \\
        P2C4 & |[draw]|.01,8,8 & |[draw]|.1,0,2 & |[draw]|.1,0,2 & |[draw]|.1,0,2 \\
        P2C3 & |[draw]|.1,2,0 & |[draw]|.98,6,6 & |[draw]|.99,0,2 & |[draw]|.99,0,2 \\
        P2C2 & |[draw]|.1,2,0 & |[draw]|.99,2,0 & |[draw]|1,4,4 & |[draw]|1,0,2 \\
        P2C1 & |[draw]|.1,2,0 & |[draw]|.99,2,0 & |[draw]|1,2,0 & |[draw]|1,0,0 \\
    };
    
    % m2 label node
    \node[above=0.25cm] at ($(m1-1-2)!0.5!(m1-1-3) + (0, -3.5)$) {\textbf{Stage 2}};

    \end{tikzpicture}
    \begin{figurenotes}
        Probability of cell existence rounded to the nearest hundredth.
    \end{figurenotes}
\end{figure}

Other research echoes this story of dynamic curvilinear adoption of new
technology under risk and uncertainty, with or without the game-theoretic
explanation in similar or other forms. Marra et al covers this literature
well in a paper on adoption of agricultural innovation\cite{marra2003economics}. Marra emphasizes
that agriculture is just one instance of a general learning concern, and
the present paper considers itself similarly.

This model clarifies risk aversion as a powerful mechanism for the
incremental transition of norms from tradition toward alternatives. I
propose that additional unmodeled mechanisms exist in the real world, but
that the real-world trend nevertheless moves in the modeled direction. One
example of an unmodeled consideration is that when an individual consumes
alternative education in the real world, the payoff might be smaller than
the traditional payoff.

\section{Actual Alternatives: A History of Alternative Education in
the United States}

Several other papers do a great job of assessing the history of
alternative education prior to 2000. One important point in much of this
literature is that differing concepts of alternative education are used.
Aron\cite{aron2006overview} states that the term alternative education, “in its broadest sense
covers all educational activities that fall outside the traditional K-12
school system (including home schooling, GED preparation K-12 school
system (including home schooling, GED preparation programs, special
programs for gifted children, charter schools, etc.), although the term is
often used to describe programs serving vulnerable youth who are no longer
in traditional schools.” In the same literature, Lange and Sletten are not
abnormal in nearly using the term as a synonym for K-12 special education.
They collect earlier research on the subject beginning in the 1960s\cite{lange2002alternative}.

The present paper takes an even broader look at alternative education by
considering alternative post-secondary education including professional
certifications and a crop of new, non-accredited, digital credentials like
the Udacity Nanodegree. By looking at post-secondary education, the
present paper unifies the literature on K-12 alternative education with
the literature on nontraditional students. One might think nontraditional
students exist in the K-12 space, and they do, but the nontraditional
student literature focuses on college students in particular. Dill and
Henly\cite{dill1998stressors}, operationalize a nontraditional student as “as having multiple
roles (e.g., parent, employee, student) and at least 1 year between high
school and college.” Taniguchi and Kaufman\cite{taniguchi2005degree} define a nontradional student
as, “those who enter four-year colleges or universities as adults, or at
age 21 or older.”

A variety of definitions are used for alternative education and
nontraditional learning, but many of the solutions are robust across these
definitions. The present paper emphasizes postsecondary alternative
learning, but many of the technical solutions which are helpful in this
space are also helpful elsewhere. Partially or fully online learning,
personalized learning, and active learning are examples of alternative
pedagogies which improve results across many of these categories.

Harasim gives a good history of online education from the founding of the
world wide web in 1992 through the year 2000\cite{harasim2000shift}. Allen and Seaman gave an
early look at online education beginning in 2003 and proceeding annually,
then they released an informative 10-year review in 2013\cite{allen2013changing}. From these
papers we can already see some clear problems and advantages with online
learning. Advantages include lower cost and improved learning outcomes
from web-assisted or hybrid courses. The effectiveness of online learning
grew importantly over the early 2000s. In 2003, 57.2 percent of academic
leaders rated learning outcomes from online education as the same or
better than face-to-face, while the number rose to 77 percent in 2012.

Massive Open Online Courses, or MOOCs, are a major topic in the modern
digital learning literature. In the Allen and Seaman note that, “Academic
leaders remain unconvinced that MOOCs represent a sustainable method for
offering online courses, but do believe they provide an important means
for institutions to learn about online pedagogy.” The literature on MOOCs
is broadly pessimistic about effectiveness. A well-designed study in 2015
found that a majority of MOOCs scored highly on organization of material
and low on instructional design\cite{margaryan2015instructional}.

Universities are shifting to become more like their alternative
competitors, but an interesting finding is that their competitors are also
shifting to become more university-like, so that traditional educators and
disruptive educators both appear to be equilibrating around a hybrid
model, and even partnering directly with each other so that online
education providers are beginning to offer college credit.

The present paper is focused on suitability of alternative credential. A
prototypical example of the type of modern, digital, alternative
credential the research is intended to relate to would be the Udacity
Nanodegree. This credential is specifically mentioned during survey
administration. The survey includes a brief statement on alternative
credentials as follows:

% TODO: block quote the below or reformat to AER convention
Alternative credentials include certificates, documents, and other proof
of receiving education, other than traditional credentials. Traditional
credentials include a high school diploma or an undergraduate degree from
an accredited university. An example of an alternative credential is a
Nanodegree from Udacity.

While there is some ongoing discussion in the current literature to the
effect that online education is still contentious, the present section
shows that online and hybrid learning are in fact a new normal which has
been developing for some time. The disruptive education of Clayton
Christiansen and Michael Horn has already been incorporated into the
typical university\cite{horn2008disrupting}.

While the present paper strongly argues for alternative education, it
agrees with scholars like Jeffrey Selingo. Jeff agrees that the bundled
service model of the traditional 4-year undergraduate degree is
collapsing, but rather than foreseeing a market takeover by nimbler
competitors, Jeff is optimistic that universities will be able to adapt.
This theme of universities successfully adapting to become like their
competitors, rather than being ousted by competitors, is a major theme of
the book.

In Robinson\cite{robinson2016creative} and Selingo\cite{selingo2013college} we see traditional providers including four-year
universities adapting and innovating by adopting best-of-breed
technologies, pedagogies, and program structures piloted by alternative
providers. Craig shows that this flow is bidirectional, rather than
unidirectional. Craig notes that Udacity's latest innovation is the
in-person course\cite{craig_2018}. Besomebody Paths are fully offline. The typical course
seems to be achieving equilibrium among a range of subtly different, but
substantively similar, hybrid modes. Far from being an artifact of
overfitting this literature, the elucidated pattern is consistent with
multiple stories that none of these authors mention.

An elementary conversation on alternative education consists in
contrasting disruptive private education and online learning providers
with traditional education. Clayton Christiansen lead development of
thought on disruptive innovation\cite{bower1995disruptive} and worked with Michael Horn to bring
this analysis to the field of education in 2008 with a book entitled
Disrupting Class: How Disruptive Innovation Will Change the Way the World
Learns. Notice the change in tense between 2008 and 2011, when Clayton
publishes a second book on the subject called "The Innovative University:
Changing the DNA of Higher Education from the Inside Out\cite{christensen2011innovative}".

A correct understanding of those in the movement of the disruption of
education is not that they are anticipating declining university
enrollment any time soon. Southern New Hampshire University and Western
Governors University are mentioned as leaders in competency-based
education. In one part of Selingo's College (Un)Bound, he mentions Paul
LeBlanc. Paul LeBlanc identifies himself as a disrupter in the vein of
Christiansen. He considers the online learning revolution old news and
competency-based education is the more recent innovation. LeBlanc believes
that competency-based education will come from existing institutions, and
will exist alongside four-year programs.

In 2011, the same year that Christiansen noticed the present-tense
changing of higher education’s DNA, Sebastian Thrun manifested the
Christiansen of 2008 and ignored the Christiansen of 2011 by relinquishing
his tenure at Stanford to found Udacity. Udacity's first courses began
taking students in early 2012\cite{desantis2012stanford}. Only in 2013 did Udacity begin contributing
to changing the university by offering some courses for college credit. In
2014, Udacity entered into its first full-fledged partnership with a
university. The same year, Udacity released its first signature
alternative credential, the Nanodegree. Excitement filled the air. About
that time, as Craig informs us, the evidence on weak outcomes for
Udacity's courses and other MOOCs began to cause significant doubt. In
2017, Udacity Connect was launched. This product is a hybrid learning
solution which occurs partly in a classroom setting and partly online. Now
that the results have started coming in, as earlier mentioned, this
approach has shown about a 500 percent increase in graduation rate.

% TODO: remove 1870 proposals for weak citation. 1904 figure is good
In 2016, Khan Academy applied for the \$100 million dollar grant by
100\&Change in order to create a globally recognized secondary
education diploma. 1904 organizations applied for the grant\cite{daver_2016}. About 1870
proposals are documented in a solution explorer made public by
100\&Change\cite{100change_2017}. 375 proposals are in the education category. When
decisions were rendered in 2017, Khan Academy's proposal earned an
honorable mention as one of the top ten in the education category, but it
did not place among the 8 semi-finalists across categories and did not
earn a financial award\cite{playworks_2017}.

Like Udacity, Khan Academy is an online learning provider which went
through a period of immense excitement followed by failure, and also like
Udacity, Khan Academy achieved a remarkable success on a different project
during the same calendar year as their disenchanting loss. In 2017, Khan
Academy released the results of a study they conducted with the College
Board. It showed that studying for the SAT using Khan Academy is
associated with 115-point average score increase\cite{khanacademy_2017studying20hours}. Khan Academy also became
the official practice partner for AP exams in 2017\cite{khanacademy_2017}.

Not only do Udacity and Khan Academy share a Jungian hero typology, they
have both evolved from traditional learning competitors to traditional
learning allies. Like Coursera, edX, and others, the best-of-breed
alternative learning providers of today are not substituting for
traditional education providers, they are integrating with them. Likewise,
the best-of-breed traditional providers are not rejecting new learning
approaches, they are partnering with them, awarding credit to students for
alternative learning, and even supplying online education providers with
content.

While literature can be found further back that applies in ways
increasingly less direct, I consider Christiansen's 2008 piece to be a
watermark in the literature. With Udacity's course offerings in early
2012, I observe a bright line in the actualization of modern alternative
education provision. With Udacity, Coursera, Khan Academy, and other major
online learning providers having been through significant revision in just
a few years, I consider 2018 to be a new age of alternative education.
Significant changes occurred during 2017, so 2018 will have been the first
year in which these changes were available throughout the period.

The general trend is toward integration of traditional and alternative
providers, but alternative learning systems are heterogenous and these
heterogenous solutions are not equal in optimality for consumption by any
particular individual, nor at the social level. Portfolios, and digital
portfolios in particular, are in demand by employers. Digital portfolios
have recently become trendy among universities, while they have been in
fashion with alternative providers for some time. Standardized portfolio
artifact generation and evaluation is becoming more standardized over
time, assisted by certain dedicated evaluation providers.

Apprenticeship programs are making a comeback as a matter of fact,
although it is controversial to claim that this is an obvious social good. (https://www.insidehighered.com/quicktakes/2019/06/26/trump-administration-proposes-new-apprenticeship-structure)
These programs never truly left Germany, but they have revived from
slumber in the UK. Under Trump, the US saw a major step forward for a
particular implementation of apprenticeship, but it has been a
controversial policy implementation, as discussed in the atemporal
findings section. Like apprenticeship, many scholars find themselves
supporting evidence-based learning, even while criticizing particular
implementations of learning assessment. It is particularly fashionable to
criticize standardized examination.

Standardized exams typically have certain question formats, including
multiple choice or essay response questions which may be graded according
to a rubric. Examples of standardized exams include the SAT, the ACT, the
international PISA, and many state-level exams including the Texas STAAR.
Standardized exams are generally cost-effective means of generating
meaningful signals, but these signals are often systematically imperfect,
and those imperfections are the source of much discussion. While
portfolios may be costlier to generate and evaluate, they seem to be
substantially less controversial in both the literature and when surveying
professionals. Portfolios may include a broad range of expression, and the
complexity of normalizing these expressions may be one reason for a
silence in the academic literature. The best explanation for high
professional opinion, on the other hand, may be plain efficacy of skill
demonstration by the candidate to a knowledgeable portfolio reviewer.

\section{Atemporal Findings in the Academic Literature}

A history of alternative education was just given, evidencing an argument
about the directional trend of education. Earlier a similar thing was done
with traditional education. This section notes some important points on
education which seem to hold in a non-trending way.

Intellectually, there is a clean distinction between a disruption camp and
a non-disruption camp. In the real world, the leaders in either camp exist
much more in the middle of the two than toward the extremes. The leading
thinkers and practitioners in favor of the university system acknowledge
that universities should continue to innovate, and adopting
non-traditional features is practically, and in some sense tautologically,
the means to that end. The leading thinkers and practitioners opposed to
the legacy system acknowledge that the legacy system is firmly entrenched
and will continue to be so for many years, and so the optimal course of
action for individuals and for society is for new providers to integrate,
cooperate, and coordinate with those legacy systems, instead of attempting
to steal consumers in a winner-take-all fashion.

The result is that leading thinkers and practitioners on both sides
largely support both alternative education and traditional education.
Moreover, leaders on both sides generally agree on which pedagogies,
technologies, and so on, provide optimal results in most cases. A cursory
glance at the non-book literature demonstrates the difficulty involved in
locating a formal paper with citations in excess of the single digits
which promotes the traditional lecture mode of class instruction. I could
only find one from by Kalogeras in 1976. While the magazines occasionally
headline an article which pretends to defend lecture, seldom make and
evidence-based case and often essentially concede the point. Consider
<em>In Defense of the Lecture</em>, a 2014 article from the Chronicle of
Higher Education. This article points out specific virtues of lecture,
even while conceding "...lecturing as a means of transferring basic
factual information is a poor way to teach. I agree..."

The differences in opinion seem rather limited, and they almost
universally apply to macroeconomic issues. The closest thing to
disagreement on microeconomic choices seems to be threefold:

\begin{itemize}
    \item Some scholars are aware of interesting programs or technologies which
    other scholars are not aware of.
    \item Scholars heterogeneously trust the payoff claims of particular programs of
    alternative learning.
    \item Some scholars oppose particular alternative education techniques.
\end{itemize}

Expanding on the third point above, Eric Hanushek recently stated “We
should not delude ourselves into thinking that Trump’s apprenticeship
expansion will substitute for our failing K-12 schooling
system...Vocationally-trained workers with relatively narrow skills face a
harsher labor market with time as the nature of production changes.” Along
the same lines, a Brooking study to which Hanushek contributed found that
excess utilization of an apprenticeship model at the social level could
generate a skill gap. Despite Hanushek's particular position on the
Trump plan, and perhaps his general opposition to excess utilization of
apprenticeship, it is clear that he does not oppose alternative education
writ large, and that he does not deny the utility of vocational training
for particular individuals.

It is worth mentioning that both of these main microeconomic concerns are
being addressed over time by projects like Credential Engine, which seeks
to comprehensively catalog non-accredited credentials and standardize
their outcome measurement and reporting. An alternative strategy is
provided by firms like Degreed, which markets a generalized service to
measure any skill. This approach simplifies the measurement trust problem
from a need to trust heterogenous providers to the need to trust a single
provider of learning measurement, Degreed.

Many less comprehensive skill measurement providers exist. Pluralsight,
for example, is a relatively well known and reputed firm in the IT market.
Pluralsight provides a standard measurement service for a specific range
of skills. This measurement process obtains independent from where the
learning or skill development occurred. In this sense, employers can
choose to trust a skill measurement provider instead of directly trusting
a learning provider. This simplifies the analysis problem for employers,
and it also creates an additional incentive for learning providers to
provide good content. If many students pass through an alternative
learning process, then perform poorly during standard evaluation, the poor
performance becomes attributable in part or whole to that alternative
learning process.

Three main sources of debate seem to be on macroeconomic concerns. These
concerns are completely out of my interest, but they are worth noting:

Firstly, what is the best course of action for the mean or median student?

Secondly, what should be done with public funds for education?

Thirdly, should the existing educational requirements for certain professional
licenses be reevaluated, or should certain licenses be created or
destroyed?

It seems to me that if there is to be any meaning to the labelling of a
scholar as in favor of traditional education, it merely means that, with
respect to the macroeconomic points mentioned above, the scholar believes
at least one of, and possibly all of, the following:

Firstly, the mean or median student should obtain a four-year degree.

Secondly, the growth in public education spending should remain constant, or perhaps
grow.

Thirdly, at least some professions benefit from licensing, and at least some
licenses benefit from requiring accredited education.

Caplan would represent, then, a non-traditional position. Hanushek would
represent a moderate position between these extremes, arguing that
policymakers should not grow spending, or perhaps cut it non-drastically,
but mainly focus on spending in a more intelligent way.

Under this macroeconomic categorization, even the proponent of traditional
education need not say a particular student should necessarily obtain a
four-year degree, although it would seem to be a null hypothesis. It is
this null hypothesizing mechanism which finally allows us to obtain some
meaningful distinction at the microeconomic level, whereas breaking
scholars into camps according to their attitude on disruption seems like a
red herring.

\section{Methodology and Organization of Findings}

Comparable survey questions obtained a maximum of 1190 responses during
four administration windows including portions of February 2018, October
2018, February 2019, and May 2019. Appendix 1 details the wording of
questions and the answers available. Appendix 2 identifies which factors
were included in each administration, notes significant factors by
administration, and gives shorthand factor names. In general, significant
factors and variables of interest were persisted across administration.
Weak factors or tangential curiosities were included only in a subset of
administrations.

The survey was created in SurveyMonkey and responses were gathered using
SurveyMonkey Paid Audiences, Amazon Mechanical Turk, social media posts,
and word of mouth. Responses were grouped according to their origin using
a construct in SurveyMonkey which is called a collector. Collector effects
were insignificant. This is interesting for two reasons. First, the source
populations are known to be systematically different. Perhaps the most
notable known systematic difference is that Amazon Mechanical Turk
respondents were guaranteed to be U.S. High School graduates. A second
reason the insignificance of collector effects is important is that
response prices were significantly different. Amazon Mechanical Turk
responses were more than 20 percent cheaper than SurveyMonkey Paid Audience
responses on average.

Systematic analysis of the novel data set includes 106 right-hand
variables and two left-hand variables. Ad hoc analysis checked another 8
selectively created interaction or similar variables.

Variable-level sample sizes range from 240 to 1190. Appendix 3 lists
technical variable names in alphabetical order along with summary
statistics. Appendix 4 lists variable names in alphabetical order, and
summarizes factor strength across models. Factors are generally
operationalized into multiple variables. Appendix 4 makes this
factor-to-variable mapping clear by identifying the factor short name
related to each variable. For example, 9 gender variables were explored.
These variables are sometimes complimentary, and in other cases they are
directly redundant with another representation of the same construct.

Data exploration began and by investigating arbitrary relations of
interest. These ad hoc findings of interest are discussed in section 3.5.
The primary variable of interest goes by the variable name voi and is
referred to in shorthand as entry-level suitability. It is structured as a
favorability question on a scale from 1 to 10. See the description of
question number 2 in appendix 1 for the wording of the question.

The secondary variable of interest goes by the variable name ioi and is
referred to in shorthand as the index of interest. This is a 3-factor
index which establishes a more general favorability measure of alternative
education, whereas entry-level suitability is narrowly focused on the
favorability of using alternative credentials in entry-level job
application. The index of interest includes the variable of interest, and
the two factors are strongly correlated, so general comments on
favorability of alternative education should be considered a reference to
both. Results for the index of interest were mainly uninteresting, as they
were very much in line with an attenuated form of the results for the
variable of interest.

Systematic analysis leveraged ordinary least squares regression analysis
and identification of four key models for each administration year. The
first model is a long model which involves multiple regression of every
available right-hand variable. The second model of interest is the weak
model which involves regressing all variables with a p-value less than .5.
The third model is the adjusted r-squared maximizing model, and the fourth
model, also called the strong model, involves regressing variables which
have a p-value less than .1.

Systematic exploration began with the long model and variables were
eliminated one at a time by significance until the next model of interest
was discovered. The long model is interesting because it shows the maximum
explanatory power of the available data set. The weak model is interesting
because each variable which survives to this model is more likely than not
to have an effect on the left-hand variable. The adjusted r-squared model
is interesting because it balances between model complexity and
explanatory power in a standard way. The strong model is interesting
because it includes factors which have had an effect identified rigorously
to a high degree of precision. With the probable exception of the long
model, any of these models might be useful in varying applied business
scenarios. While the long models seem to have high complexity relative to
added value, it is interesting to note that the raw explanatory power of
.5635 is greater than .5. This is important because it means the long
model explains more of the variation than it fails to explain among the
observations it is fit against.

Because the October 2018 administration variables are a superset of the
February 2018 variables, a single systematic exploration was conducted
concerning the 2018 administration year. Similarly, May 2019 variables are
a superset of February 2019, and a single systematic exploration was
conducted for 2019 variables. This analysis did not restrict the sample,
however, and it turned out exegetically that the 2019 strong model holds
for 2018 as well. That is, the most significant factors identified in the
2019 samples were also measured in the 2018 administrations. This is
likely a case of statistical endogeneity of significance, however, as
these variables may be significantly identified precisely because they
were oversampled.

% TODO: remove from concise paper
\section{Complexity and The Preferred Model}

4 of the 8 systematically derived models are reported in Table 3. Long and
weak models are not recorded in the table for brevity, but factor strength
across all models is reported in Appendix 4, and discussion is given to
these models and their weak and super-weak factors when relevant. Adjusted
r-squared maximizing models are also called medium models, and the
variables in these models is considered to have medium importance.

Overall, the 2019 medium model is preferred. It obtains high raw and
adjusted explanatory power while maintaining relatively low complexity. It
is not, however, the highest of all adjusted-r squared among the four
models. The medium model including 2018-only variables had a higher
adjusted r-squared. There are a couple potential reasons for this. First,
variables which were included in 2018 and not included in 2019 are likely
to contain important effects which would add to adjusted explanatory
power. Secondly, there might be additional variation in the newer samples
which would cause weaker fit even in the presence of 2018 variables.

Initial investigation at the time of the 2018 survey administrations
indicated that stem identification and religiosity were insignificant, so
they were removed from later investigation. More recent replication of
those results uncovered that those effects were moderately important, and
inclusion in future research is recommended for reanalysis.

During 2018 investigation, there was suspect improvement to simplicity
when filtering by factor significance from the weak into the medium model.
7 variables were eliminating, leading to nontrivial gain in adjusted
r-squared, however the practical complexity of the matter did not
significantly reduce because only one survey question could be eliminated
in order to implement the medium model instead of the weak model. This is
a different perspective on complexity which is an economically important
distinction. As a result, a metric called q-complexity was checked for all
models. This metric simply measures the number of questions which would be
asked during a survey in order to implement a model.

Q-complexity is more directly connected to expenditure when constructing a
survey or other data collection system, and it also more accurately
assesses the interpretive complexity compared to adjusted r-squared in
some cases. A model with 9 regional effects instead of 10 is unimportantly
simpler compared to the reduction in complexity achieved when a survey
with 9 yes or no questions is administered in comparison to a survey with
10 such questions. Q-complexity detects this nuance while adjusted
r-squared is blind to it.

This background is an important reason for selecting the 2019 medium
model. The 2018 medium model decreased q-complexity by 1 relative to the
2018 weak model, so practically the weak model is not more expensive to
implement, and it may gain substantively in raw explanatory power. In the
2019 case, however, the medium model reduces q-complexity from the 2019
weak model by 4, or 25 percent, so it achieves an economically important
reduction in implementation use and it is also substantively simpler to
reason about.

% TODO: Table: Medium and Strong Models

\section{Systematic Exploration of the 2018 Data Set}

There were 168 observations in the 2018 long model. Model hardening from
the long model to the weak model for 2018 resulted in a reduction from 69
to 39 variables. The 69 variables were associated with 15 questions, and
the surviving 39 variables in the weak model were associated with 14
questions. Surprisingly, the only question which was fully filtered out
was the question on employment status.

The original hypothesis was that employers would be willing to support
alternative credentials. The systematically derived 2018 models indicate
that the attitudes of those individuals who make hiring and firing
decisions are not significantly different than the general population, but
this simply begs a question about whether the general population supports
alternative credentials. Appendix 3 answers this question by presenting
summary data on all variables. The mean of the variable of interest is
about 6.614, which is significantly more favorable than not across the
population. Table 4 adds a bit of detail with respect to the specific
sample included in the 2018 long and weak models. Individuals within this
sample had an average favorability of about 6.351, although this is
insignificantly different from average.

% TODO: Table: Cross Tab VOI by is2018LongModelResponse

While cross tabulation within this sample seems to hint at weak positive
temporal trend, direct interrogation of time variables yields a mixed
confirmation. Simple regression of linear time on the variable of interest
has a super-weak (p > .5) negative effect. A regression of linear and
squared time on the variable of interest yields stronger effects on each
factor, but the effects are still weak (.5 > p > .1). Interestingly,
the weak temporal effects are opposite expectation. Linear positive
effects with a negative marginal effect would be intuitive, but the
observed weak temporal effects are linearly negative and marginally
positive. Positive marginal effects are generally considered
unsustainable, but this finding may indicate that entry-level suitability
resides on the early portion of an s-curve for adoption.

When cubic time is introduced to the right hand, linear time becomes
omitted due to collinearity. The p-values of marginal and cubic effects
are slightly better than the p-values of the linear and marginal effects
in the simpler model, but the marginal effect has a negative value in this
model. Theoretically, as time increases arbitrarily the cubic effect would
dominate, so that this model also suggests unbounded increasing returns to
scale. As earlier mentioned, this is generally considered theoretically
unsustainable, and so a more plausible interpretation of this temporally
complex model is simply that it is replicating the suggestion that
entry-level suitability exists at the early phase of an adoption growth
curve, prior to inflection, sometimes called the lag phase.

Based on exploration up to this point, the working answer to the
hypothesis is that employers are favorable toward using alternative
credentials, but so is everyone else. In addition, there is weak evidence
that exponential favorability is down the road, but favorability may
decrease in the immediate future and for some time. The date construct
used is the number of days since 1960. Time effect coefficients indicate
that the total time effect will net positive with central estimates of
80-120 years. The variety of events which could occur over such a time
jeopardize reasonable confidence about the magnitude of these effects, and
statistical significance in these effects is low enough that it would be
unsurprising for inflection to obtain within 5 years. High variability in
temporal estimation underscores the potential value of additional temporal
sampling.

Reducing the weak 2018 model into the 2018 adjusted r-squared maximizing
model eliminates Christian identification as an important variable. This
variable competed with generic religious identification, and linear
religiosity survives to this model with a positive effect. Religiosity is
typically associated with political conservatism, and conservatism is
thought to move with status quo bias. The present article gives mixed
confirmation of a positive relation between religiosity and
anti-innovation bias, but education appears to be an important exception.

Innovation proxy variables include favorability to artificial
intelligence, cryptocurrency, and online education. These variables are
cross-correlated with one another with a p-value of less than .001.
Religiosity is negatively related only to artificial intelligence, but
artificial intelligence is the only variable in this set of three which
survives to the strong model. The negative linear correlation between
religiosity and artificial intelligence is also more significant and
larger in magnitude compared to the relation of religiosity to other
innovation proxies.

Conservatism is characterized by high religiosity and high favorability to
market-based solutions. Regulatory favorability is positively associated
with all proxies of innovation. This amounts to confirmation on the
association of market favorability with status quo bias, but it also
presents two paradoxes. First, the market is considered an effective tool
of innovation, so individuals seeking to maintain the status quo ought to
disfavor it rather than favor it. Second, traditional education is
regulated education, and alternative credentials are deregulated, so
individuals committed to high levels of regulation ought to disfavor
alternative credentials. One hypothesized explanation to this apparent
paradox is oriented around individual personality. If those on the
political left are high in openness, then they might also favor
alternative credentials.

Industrial effects are common in weaker models, but fail to survive into
the strong model for 2018. The most significant industrial effect was for
those who chose other as their industry. The second most significant
effect was for information technology. Two regions have significant
effects in the strong model. The mid-atlantic region, including much of
Washington DC, is associated with a positive effect. The west south
central region is associated with a large negative effect. This region
consists of Arkansas, Louisiana, Oklahoma, and Texas. Gender, age, and
income were significant. Anti-foreign bias was tested and identified, but
it’s explanation is not intuitive. Anti-foreign bias is positively
correlated with favorability on alternative education.

\section{Systematic Exploration of the 2019 Data Set}

The 2019 strong model identifies gender, innovation bias, expected
conventionalism, online education favorability, and regulatory
favorability as the strongest factors. Effects move in their expected
directions, except for regulatory favorability which is linearly positive
with respect to the entry level suitability. It’s interesting that the
question about whether alternative credentials will be conventional soon
survives into the strong model, because this reinforces two key
theoretical stories in the literature.

First, it highlights the importance of education as a norm, which is key
to Caplan’s criticism of alternative credentials. Second, the surviving
positive quadratic and negative cubic effects reinforces the story that
alternative credential adoption is progressing through an s-curve. Figure
3 shows the that the effect of expected conventionality on entry-level
suitability follows an s-curve. This only reinforces our temporal story if
time moves with expected conventionality, and indeed with nonlinear
conventionality, but it turns out that this is exactly the case. While our
earlier simple analysis of time on the variable of interest involved
p-values on time variables in the neighborhood of .4, a regression of
linear time on nonlinear expected conventionality reveals a positive
coefficient with a p-value of .024.

% TODO: Figure 3 – Effect of Expected Conventionality on Entry-Level Suitability in the 2019 Strong Model

The vertical line in Figure 3 occurs at a value of about 6.1, which is the
mean value of expected conventionality in the survey. Notice that this
story about s-curve adoption is slightly different than our earlier story.
The conventionality-based adoption analysis indicates that alternative
credentials are past the lagged phase of adoption and recently past the
point of inflection. Extrapolating far into the future seems to indicate
an eventual demise to suitability, but this extrapolation is inappropriate
for a few reasons. First, the model turns negative around an expected
conventionality value of 11, but the maximum value this construct is
capable of taking on is 10. At 10, we seem to see a marginal value near
zero, which is consistent with the second extrapolation issue. In theory,
we have good reason to expect decreasing marginal effects, but we have not
identified any reason to expected negative marginal effects. Section 2.6
discusses some of this theory on growth curves, learning curves, and so
on.

Finally, the shape of the curve is the result of analytical design.
Quadratic and cubic factors were constructed rather than directly
measured. These constructs are useful because they offer simple detection
of non-linear effects, but not because they are optimal for all analytical
purposes. S-curves are prototypically modeled as a sigmoid function, and
log-log modelling is also common for learning or experience curves. While
individuals are not treated and measured for learning in this paper, the
idea is that society as a whole is learning about alternative credentials
over time. Log-log regression for time on conventionalism was checked, and
indeed it has an even better p-value of .004. Under that model there would
be no extrapolative decline in entry-level suitability as a function of
arbitrarily, indeed impossibly, large expected conventionality. For
practical purposes there is little substantive difference in these
approaches with respect to the variables in question. Expected
conventionality is not binary, but transformation of this variable is
possible to allow logistic regression to model a sigmoid.

Logarithmic analysis obtains higher confidence, but the relation is
indirect to the variable of interest. Unfortunately, log-linear and
log-log analysis of time directly to the variable of interest is
super-weak, so the indirect relation seems to be both our most accurate
story and also a relatively complex story to predict on and reason about.
The log-linear regression of expected conventionality on the variable of
interest is exceedingly significance with a p-value under .001, but it is
less explanatory than a multiple regression of linear, quadratic, and
cubic expected conventionality on the variable of interest, and in the
longer regression all right-hand variables are significant with p-values
under .08.

Because the direct relation between the variable of interest and time is
insignificant, but each step of an indirect relationship is significant,
an indirect model is tested by generating the predicted log of expected
conventionality from log time, and the variable of interest is regressed
on predicted log expected conventionality. Yet, this relation is also
insignificant with a p-value of .811 for the predicted coefficient in the
regression.

Still, following the intuition of this indirect relation, nonlinear
regressions are explored for significance. Eventually, three interesting
models are identified. One strong temporal model was identified and two
nonlinear regressions of expected conventionality on the variable of
interest. A dynamic model was identified with the form:

This temporal model obtained an r-squared of .8691 and b2 had a p-value
less than .001. The estimate of b2 was less than 1, indicating exponential
decay rather than exponential growth. This is the best fit temporal trend
for the observed data, indicating a decreasing nonlinear trend in
alternative credential suitability over the sample. Combining insights
from 2018 and 2019 trend analysis, the evidence toward a short run
reduction in alternative credential favorability is strong, and there is
some comparatively weak evidence for a longer run reversal.

Regarding nonlinear regression of conventionality, a two-factor
exponential expansion obtains an r-squared of .9029. Let X represent
expected conventionality, then two-factor model takes the form:

A three-factor exponential expansion obtains an r-squared of .2621. The
three-factor model takes the form:

The three-factor expansion is interesting because the exponentiated
parameter is identified with a t statistic of 1077.23. This immense
t-statistic seems to indicate the parameter is identified with high
precision. However, the constant in this model takes the implausible value
of about -171, and the exponentiated parameter takes the implausible value
of about 175. Remember that the variable of interest is observed between 1
and 10. The two-factor model estimates the exponentiated parameter at
about 4. The two-factor model also estimates the exponentiated parameter
with a t-statistic of about 33.8, and an associated p-value of about 0.
While this lower t-statistic is technically a less strong identification,
it is practically unimportant, and the estimated value is plausible.

The positive association between conventionality and entry-level
suitability is already firmly established, as is the short run negative
association between time and the variable of interest. One interesting
note to add is that when expected conventionality is interacted with time,
a multiple regression of time, conventionality, and the interacted
variable reveals a positive relation between the interacted variable and
entry-level suitability. This may point to long run normalization of
alternative credentials as a mechanism toward eventual recovery in
entry-level suitability.

Employer effects refer to the effect associated with an individual’s
statement that they influence hiring and firing decisions in their place
of work. Employer effects were weak in the 2018 sample, but additional
sampling across 2018 and 2019 allowed employer effects to survive into the
preferred 2019 model. The preferred 2019 model is the model which
maximizes adjusted r-squared. Employer effects are negative in that model,
with a coefficient of about -.47. Employer effects are slightly negative
to a lesser extent in a simple regression against the variable of
interest, with a coefficient of about -.1.

A simple interpretation is that employers are more pessimistic than others
on alternative credentials. Another interesting possibility views this
effect from a process perspective. From a process perspective, employers
are a driver of changes to the labor market, so that population
favorability lags employer favorability. It’s clear that entry-level
suitability will decline in the short term, and this is consistent with
employers having a more negative view than average. The interesting
finding here is to note that when we interact time with employer status,
the employer effects are already reversed from the general population. A
regression of four parameters on the variable of interest is depicted in
Figure 4, which illustrates a hypothetical reversal in entry-level
suitability. The figure is conceptual and not to scale. The population
trend is illustrated at A, and employer views are represented at B. At A,
time effects are linearly negative and marginally positive. Linear
employer-time effects are positive, but marginal employer-time effects are
negative. The plausibility of a reversal story is enhanced when noting
that interacted manager-time effects are more positive than and
significant compared to ordinary time effects.

% TODO: Figure 4 – Employer-Driven Favorability

Other interesting findings from the 2019 data analysis includes the fact
that age group had a more robust effect compared to exact age, which may
indicate something like a cohort effect. Prior analysis indicated that
regional effects were moderately important. Ethnicity was introduced into
the survey in part to distinguish between underlying policy or culture
partials of regional effects. Regional effects were significant after the
introduction of ethnicity, but ethnic effects were also moderately
significant. Future analysis could identify state of residence to partial
out policy effects to a greater extent.

Educational attainment obtained an important effect which was more
significant than either age or income effects. In addition to level of
education, a dummy variable for whether education was at or greater than
obtaining a college degree was found to be significant, and it had a
positive effect on favorability to alternative education. It seems that
individuals who have obtained a traditional degree are more appreciative
of alternative education.

% TODO: remove from concise paper
\section{Other Interesting Results}

Previous research found student indifference toward debt on the part of
undergraduate students. The present paper replicates and extends such
findings by identifying youth antagonism to alternative credentials. Prior
research often measured debt attitudes among college students, but such
evidence is susceptible to selection bias because debt-tolerant
individuals might have a propensity to consume higher education. In
contrast, the present paper identifies generalized youth antagonism to
alternative credentials.

A simple regression of exact age on the variable of interest yields a
slight negative effect. Age group was more important than exact age, and
including age group and exact age simultaneously replicates linear
negative association across both variables. These regression results
obfuscate a narrative which is readily apparent in Tables 5 and 6, a
crosstab of age group on entry level suitability, and a summary of mean
response to the variable of interest by age group. Notice that the most
positive group is not the youngest group, but the age group actively
attending or having just graduated college.

30 percent of minors gave the lowest possible entry-level suitability response,
and only 10 percent gave the highest response. Minors are the only age group
which is unfavorable toward alternative credentials on average, with an
average response of 4.6. One age group up, less than 3 percent of college-aged
individuals gave a response of 1, while more than 20 percent gave a response of
10. Entry-level suitability attenuates downward for age groups 3 and 4,
but it is still positive on average. The oldest age group also has the
highest proportion of individuals in maximal favor of alternative
credentials, with about 1 in 4 giving a response of 10. The oldest age
group has a strongly bimodal response, and they are on average less
favorable than other groups except minors, but the pessimistic peak among
the elderly is still favorable, at a value of 6.

The youngest group is a group of small sample size, and therefore not
weighed heavily into the lines of best fit, and they are also the most
pessimistic about alternative education. Contrary to the stereotype of the
innovative youth against the in-their-ways elderly, the present paper
indicates that the youth are less innovative than any other group. In the
preferred model, which maximizes adjusted r-squared across all 2018 and
2019 data, we see that educational effects are important, including a
dummy variable for having received a college-level or better education.
While neither age nor educational attainment survive into the strong
model, educational effects are more significant in the preferred model. It
seems that having been through the education system is the more
explanatory factor, and age is a side-effect, rather than the other way
around. The uneducated, including the youth, appear to be less innovative
than the elderly.

This information provides for a better marketing strategy for alternative
credentials. Instead of marketing to those about to enter college, market
to their parents. Marketing to active college students is also a plausible
path, although these students are already invested. Some approaches to
alternative education, however, work in concert with traditional
education. For example, credit by examination is an affordable, fast-paced
alternative pathway toward a traditional degree.

Plausible explanations for elderly favorability include memory of a time
before the 1980s when a degree wasn’t as essential. These individuals have
also often obtained a degree and worked for a substantial amount of time,
and they may have noticed only a small attainment of job-related skills
from the degree. Younger individuals may have a lack of skin in the game
and a longer time horizon for repayment.

Another interesting, if tangential, result is that to the author’s
knowledge, the present paper is the first to look at the effect of
nonbinary gender identification on not only the variable of interest, but
other items contained in the survey. Nonbinary gender identification
obtained for 16 respondents, and it was motivated for inclusion by a
desire to reduce noise in known gender effects, but it turns out to have a
significant relation to the variable of interest by itself.

A simple regression of nonbinary gender identification on the variable of
interest reveals a coefficient of about -1.3 with a p-value less than .05.
Gender nonbinary individuals are also pessimistic about online education,
with a coefficient of about -1.1 and a p-value of less than .08 in that
simple regression. Gender effects survive into the strong model, but not
in the form of the nonbinary identification variable. Substituting
nonbinary identification in for other gender variables in the strong model
maintains the negative direction of effect, but the magnitude of effect is
attenuated to -.48, this time with relatively low significance and a
p-value of .374.

A final interesting note is that every factor tested had at least one
variable representation that was moderately significant in at least one of
the administrations, with two exceptions. Collector effects were utterly
insignificant, and Christian identification was weak at best. Ethnicity
effects varied by ethnicity. Two were moderately important and none were
strong. Hispanics, represented by isethnicity4, and also other
ethnicities, represented by isethnicity6, were the two ethnicity variables
present in the preferred model. Both ethnic effects were positive with
nontrivial magnitudes of .87 and 1.68. These effects were nearly strong
with a maximum p-value of .128 in the preferred model.

% TODO: Table: Crosstab of Age Group on Entry Level Suitability
% TODO: Table: Mean VOI by Age Group

\section{Applications}

There are several important microeconomic applications of the present
research. Key applications include accelerating and reducing cost for
traditional education, improving employment and earnings through
alternative credentials, individual application during the interview
process, individual application in the context of corporate politics, firm
application in competitive analysis, and individual application while
facing the education consumption decision.

Accelerating and reducing cost for traditional education can be
accomplished in several ways. First, identify the average public in-state
tuition for four-year public universities in the learner’s state of
residence. Then, filter possible learning providers to ensure this is the
maximum amount paid. Second, utilize online learning providers if the
professional education desired support than, and if the learner is
comfortable doing so. If the professional education desired involves
hand-on experience, like science lab or medical, pure online solutions may
not be ideal. Business, information technology, and liberal arts degrees
are largely consumable online. Identify the learner’s desired career path
and work backwards from that, taking note of relevant certifications, and
directly pursuing certification as part of or in lieu of traditional
education if possible. Leverage credit by examination and prior learning
assessments when possible. Make it a goal to work while going to school
and obtain employer reimbursement for college expenses to the maximum
available amount.

During the application process, an individual who has received alternative
education should bear in mind the preferred model of alternative education
favorability. The employment candidate will have opportunities to observe
interviewers who will interview on behalf of the employer and contribute
to an employment decision. The candidate can strategically communicate
their educational history by observing interviewers and roughly
calculating their favorability to alternative education.

In the context of corporate politics, an individual may already be
employed and may be seeking to garner consensus within the organization
for a policy change. An example of a desired policy change might be to
eliminate the requirement for a traditional degree from certain job
requisitions, or to allow specific alternative credentials to substitute
for that requirement in some cases. Many corporations offer thousands of
dollars per employee in tuition assistance. A second example of a desired
policy change might be to modify tuition assistance to target CLEP
testing, so that recipients would be able to more quickly and cheaply
obtain college credit, and potentially reduce assistance outlays from the
employer. Bearing in mind the preferred model might assist a change
advocate in identifying those individuals best predisposed to agreement
with the change, facilitating consensus building and execution of that
change.

For the two above scenarios, a key rhetorical strategy is to ask a person
about whether they are familiar with alternative credentials. If they are
not, talk a bit about them. After ensuring the concept is familiar,
proceed to ask whether the person thinks these will soon become
conventional. This is a key non-observable factor which is extremely
explanatory in the model, but when asked in conversation it comes across
in a non-technical, comfortable way. Handled properly, this question can
be a good ice breaker and help the person asking the question to
understand their audience without giving away the views of the person
asking the question. The findings in the present paper indicate that
people are receptive to alternative credentials even if they aren’t
familiar with the topic, and that they become more favorable as they learn
more. Outside of formal processes, these positive effects may indicate
that conversation around alternative credentials is generally positive,
and it might be applicable as ordinary leisure conversation material,
which might eventually contribute to wider social acceptance by word of
mouth.

Regarding competitive analysis from the firm perspective, particularly in
the case of labor competition, firms already know that alternative
education is important. People often learn about alternative learning
providers through their employer. This is reflected in the findings from
the present research in that unemployed status has a highly significant
association with lack of knowledge about alternative learning providers.
While employers are already driving alternative learning adoption, this
kind of learning is typically used as a layer of professional learning,
upskilling, or continuous education on top of a prior traditional degree.

The competitive trend is the tendency to allow that learning to substitute
for the degree. This improvement to the prior human resource process
allows access to a larger pool of qualified candidates who tend to accept
offers at lower salary. Google was in early on this trend. In 2013, Laszlo
Bock, Senior Vice President at Google, was interviewed by Adam Bryant of
The New York Times. He stated that Google’s data at that time indicated
that on the job performance was insignificantly related to GPA or test
scores after 2-3 years, and the proportion of people without any college
education at Google was increased over time. Years later, in 2018, a
well-known salary aggregator called Glassdoor reported on 15 major
companies, including Google, which no longer required a degree. Glassdoor
stated, “Increasingly, there are many companies offering well-paying jobs
to those with non-traditional education or a high-school diploma.”

Alternative learning providers are also a key approach to improving
workforce diversity. In order to align with other industry-leading firms,
drive down labor cost, and improve workforce diversity, the present
findings suggest a best practice policy is to marginally reduce
traditional educational requirements in as many professional positions as
feasible for a given firm.

Facing the education consumption decision includes at least two
sub-scenarios. In one scenario the consumer is the student, and in another
scenario the consumer is financing a third-party student. Typically, a
financier would be a parent paying for their child to receive additional
education, but there are many non-parental cases of third-party financing.
Employers are a key example of non-parental education financing.

The important takeaway from the findings for individuals facing education
consumption choices is that most people are favorable to the idea of
alternative education, although we may soon enter a period where that
favorability decreases substantially. Even facing lower favorability,
though, it may still be worth exploring alternative credentials due to
their affordability and rapid ability to attain. Finally, alternative
education is broader than alternative credentials, and it’s possible to
leverage alternative education as a way to accelerate or cheapen the
completion of a traditional education.

Learning while employed makes a learner a nontraditional student, but it
greatly enhances the return to college. This for at least three reasons.
First, because foregone earnings are a major college expense. Second,
gaining experience allows the learner to obtain even higher earnings at
graduation time. Third, many employers reimburse a significant amount of
employee expenses toward college.

This research also informs several potential macroeconomic policy
enhancements. Federal lending programs, the G.I. Bill, and similar
programs could be redirected, growth-limited, frozen, amended with a
sunset provision, or terminated. Licensing regulation entailing formal
education could be written to target evidence-based competency in lieu of
accredited education. Internship requirements could be relaxed, or the
minimum wage could be reduced. Finally, tax write-offs and tax-privileged
investment vehicles targeted at accredited education could be liberalized
to allow various forms of alternative education. While the present paper
has focused on post-secondary credentials, it finds itself in harmony with
a broad literature identifying favorable outcomes for school choice and
self-directed learning at all ages.

\section{References}

% ref: https://www.youtube.com/watch?v=KS9GvK7cvmo
% https://tex.stackexchange.com/a/51501/197312
\bibliographystyle{/Users/zyl357/Documents/GitHub/research-dissertation-case-for-alt-ed/papers/alt-ed-survey/aea-latex-templates/aea}
\bibliography{BibFile}

% The appendix command is issued once, prior to all appendices, if any.
\appendix

\section{Question Reference}

\end{document}
