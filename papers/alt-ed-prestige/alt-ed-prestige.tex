% cleaning note: run:
% `deno run --allow-read --allow-write ../../../../deno-utility-tex-to-text/tex-to-text.ts --infile /c/Users/vandi/workspace/research-dissertation-case-for-alt-ed/papers/alt-ed-prestige/alt-ed-prestige.tex'

% using Elseveir template per https://www.elsevier.com/authors/author-schemas/latex-instructions
\documentclass[review]{elsarticle}
\usepackage{lineno,hyperref}
\modulolinenumbers[5]
\bibliographystyle{elsarticle-num}

\usepackage{booktabs}
\usepackage{graphicx}
\graphicspath{{../alt-ed-survey/figures-and-tables}}
\usepackage{hyperref}
\usepackage{threeparttable}  
\usepackage{tikz}
\usetikzlibrary{calc,matrix}

\begin{document}
\begin{frontmatter}

    \title{
        Hirability and Educational Prestige
    }

    \author[mymainaddress]{John Vandivier}
    \address[mymainaddress]{4400 University Dr, Fairfax, VA 22030}
    \ead{jvandivi@masonlive.gmu.edu}

    \begin{abstract}
        % a4.1
        Alternative credentials
        offer a partial solution to the skill gap and student debt crises,
        supernormal returns for some students,
        and a tool to support diversity hiring for firms.
        This paper tests the hypothesis that educational prestige explains hirability
        better than accreditation.
        Results from an original questionnaire ($n = 454$)
        confirm that prestige explains comparatively more hirability variance.
        Accredited credentials have higher average prestige,
        but alternative credentials have a larger variance in prestige,
        so a significant number of job opportunities favor the nontraditional student.
        When prestige is low, hirability for alternative credentials remains nontrivial.
        Analysis using ordinary least squares and linear mixed models demonstrate that
        industry, state, individual, and other effects
        favor the nontraditional student in specific cases.
        % below line is true per a4.2 but it's a very weak increase so let's not mention
        % Exposure to quality ratings for learning providers
        % is associated with improved relative prestige on alternative credentials.
        % The conclusion includes a discussion on policy,
        % and actionable recommendations for employers, students, and quality rating aggregators.
        % Do I really need to offer policy solutions?
        % At the same time, situations are revealed in which alternative credentials are preferred.
        % Hirability is high for alternative credentials,
        % and these credentials can be obtained at a lower cost and difficulty compared
        % to certain degrees.
        % As a result, alternative credentials are sometimes preferred even when comparative prestige is low.
        % Descriptive statistics show that prestige for alternative credentials is
        % more dispersed than prestige for the accredited degree,
        % so some proprtion of respondents prefer alternative credentials.
    \end{abstract}

    \begin{keyword}
        debt crisis, skill gap, prestige, social economics, education economics, alternative education    %%% not grammatical
        \MSC[2010] I20, I24, J24, B55                                                                     %%% not grammatical
    \end{keyword}

\end{frontmatter}

\pagebreak
\linenumbers

\section{Introduction}

The accredited degree is an established means toward desirable labor outcomes,
but proliferation of the degree is associated with a variety of well-understood issues including
the student debt crisis, skill gaps, grade inflation, low social return,
and contribution to lack of diversity in the labor market.
Alternative credentials, or non-accredited credentials,
are a broad category of offerings that exhibit greater variation intensity, price, and outcomes\cite{urdan_2020}.
This paper hypothesizes that variation in the properties of alternative credentials
contemporaneously inhibit normal usage and support occasional superior results.

Strategic usage of alternative credentials requires a qualitative description of the occasions in which they provide superior results.
Decomposition of alternative credentials can be accomplished through a variety of lenses,
and this paper takes the lens of prestige.
This paper tests the hypothesis that prestige is a better explanation of willingness to hire
than accreditation.
Results are made practical through the description of low-effort methods to identify of high prestige alternative credentials.

The motivation for the lens of prestige extends from the literatures on education economics and the economics of social norms.
Education economics provides two mainstream accounts of the value of a degree.
One account is the human capital model and the other is the signaling model.
The human capital model explains that improved labor outcomes result from skills gained by a student in the course of education.

Alternative credentials are regarded as preferred to the traditional degree for the attainment of specific technical skills\cite{craig2018new}.
For this reason, many college graduates supplement using alternative credentials.
Some alternative learning providers specifically target this market with a special kind of alternative education called last-mile training.
This presents an explanatory problem for the human capital model.
If better labor outcomes arise from skill enhancement,
then alternatively educated individuals should enjoy better wages, employment rates, and so on,
compared to college graduates.

The signaling model holds that credentials signal a basket of applicant qualities that are valued by employers.
Proponents of the signaling model commonly argue that the college degree signals intelligence,
work ethic, and conformity\cite{caplan_2012}.
This presents a testable contrast to the signal of an alternative credential.
Alternative credentials also signal intelligence,
but they may not signal work ethic,
and they are generally expected to signal non-conformity rather than conformity.

This paper hypothesizes that prestige is valued by employers as a signal,
and indeed it is in part a signal of conformity.
Google is a prestigious employer and also an alternative learning provider.
From the point of view of Google, their own credential is a preferred conformity signal as well as a signal of skill.
The case of employer-provided credentials is interesting,
but it is not the main argument in this paper.
While conformity and prestige intersect at times,
this paper does not suppose they are identical nor generally correlated.
Instead, this paper argues that these are two social characteristics that are valued by employers
and a lack in one may be compensated for by the presence of the other.
% This paper hypothesizes that alternative credentials will have low average prestige,
% but that some particular credentials from prestigious providers like Google will prove valuable.

In a broad review of economics and norm types, hiring decisions exist within what Elster would identify as work norms\cite{elster1989social}.
Elster supports a rational model of work norms, with the caveat that social interactions may involve unobserved emotional effects.
Similarly, the neoclassical model utilized in this paper comes with the caveat that
applicability of results is constrained in cases where a hiring decision is made subject to abnormal emotional effects.
This paper will also make use of the distinction between social and legal norms provided by Elster.

Within the economics of work norms, Rivera is one scholar to have recently operationalized social norms as prestige\cite{rivera2016pedigree}.
Rivera finds that prestige is important in her analysis, but the scope of her analysis is focused
within an analysis of traditional education and a few specific industries including health and law.
The current paper extends the analysis of prestige and hiring norms
across many industries and to include alternative credentials.
% arguably I emphasize coding bootcamps / other bootcamp industries / information technology industry and perhaps sales

% Vocational schools are only sometimes accredited, and non-accredited means of learning some vocations are common.
% For this reason, it is widely accepted that there are particular jobs for which alternative credentials are preferred,
% but the explanation for the lack of general non-accreditated higher learning is commonly made on the basis of social norms.
% As two sides of the same coin, some argue that vocational schooling and alternative credentials bear some stigma.
% Others argue that there is a positive social expectation, and indeed systematized facilitation,
% of educational progress from a high school to an accredited university.

As a preview, statistical evidence confirms that prestige independently explains hirability better than accreditation alone,
but accreditation fails to be explained away.
Instead, models that use both factors produce superior estimates of willingness to hire.
The independent importance of accreditation indicates that asymptotic improvement to alternative credentials
are unlikely to fully compete away the traditional education system.
The failure of arbitrary technical and social gains in alternative credentials to fully crowd out traditional education
points to a need to investigate legal norms for further remedy.
The conclusion describes policy options that solve for the remainder of concerns in higher education that survive competition from alternative credentials.
% At the same time, there is a large level of economic value/arbitrage to be had/exploited from the socialization of alternative credentials
% before the binding constraint of legal stability is encountered.
% TODO: maybe estimate the level of economic growth (it would be proportional utility growth which is merely analgous tho)

\section{Description of Data and Methodology}

This paper investigates an original set of online questionnaire responses ($n = 454$).
Responses are cross-sectional data obtained in March of 2021.
Respondents are United States citizens at or over the age of eighteen.
Qualified respondents participated in the survey through the Amazon Mechanical Turk platform.

Appendix A contains the wording and response options for each question.
Appendix A also contains the wording for a priming message presented at the start of the survey.
The priming message lays out the definition of alternative credentials for the purposes of the study.
The message also provides several concrete examples of alternative credentials,
including ``a Certified Project Manager certification,
a portfolio of work, a Khan Academy profile, or a Nanodegree from Udacity.''

The dependent variable of interest is called hirability.
This variable measures individual response on a 10-point scale to the question,
``For many professions, alternative credentials can qualify a person for an entry-level position.''
The questionnaire is composed of three sections.
The first section collects respondent characteristics and baseline hirability.
The second section collects hirability and prestige responses with respect to nine specific learning providers.
The third section collects hirability and prestige responses with respect to eight vignette learning providers.

Data from the first section is used to optimize an ordinary least squares model.
Vignette data is analyzed as panel data in a mixed model with individual random effects.
The vignette model allows comparison between prestige and accreditation coefficients,
but it encounters a practical problem in that the schools are only vignettes rather than actual learning providers.
To address the practical concern,
descriptive statistics are compared between vignette and actual schools using information from the second section.

Additionally, half of respondents were randomly selected for exposure to an informational message about actual schools.
The message is included in Appendix A.
The message provides rating data from two leading credential aggregator websites.
University ratings are US News ranking information for the 2021 school year.
Coding bootcamp ratings are Course Report ratings from December 2020.
% TODO: maybe cite the above two sources

% TODO: write the below content into the paper when we get to results on concrete providers
% why do we not use a mixed model of concrete providers?
% three reasons:
%   first, the whole point of using an aggregator is to obtain individual-independent data
%   second, ratings are heterogenous so there would be some noise introduced by indexing them together
%   third, i use stipulated prestige categories as a pseudo-index
%       the pseudo-index can be used via simpler descriptive stats,
%       and the pseudo-index has evidence supporting it; that is, high / low stipulated prestige is sig correlated to response prestige.
% if all those objections fail, then fine we can do it but i suspect the answer will be:
%   'findings are insignificant bc of low variation and injected noise not bc no such level exists'

Respondent characteristics are measured as categorical variables.
Hirability and prestige are measured as 10-point likert-type responses.
Prestige also takes a secondary representation as a stipulated boolean.
Stipulating schools as high or low prestige allows the paper to verify
that prestige response is correlated to stipulated prestige.
For example, a vignette school is identified to the respondent as well-known for being prestigious.
This corresponds to a stipulated boolean with a value of true.
When the respondent reads that a school is known for being high in prestige,
they are then asked for their own prestige rating on a 10-point scale.

As a preview of results, stipulated high prestige is strongly correlated with high prestige response.
At the same time, there are cases where a respondent gives a low response rating to,
for example, the University of Chicago,
which is a school that happens to be stipulated as high prestige on the basis of aggregator website ratings.

Two-way representation of prestige enables better general application of findings into the real world.
In the real world, an individual can easily access aggregator website ratings.
In the real world, an individual cannot readily access questionnaire results for many credentials.
Results from this paper include the identification of rules of thumb
that a person can use to identify actual learning providers as high prestige.
To ensure clarity of results,
stipulated prestige always refers to the boolean and prestige response refers to the 10-point measure.

Stipulated prestige is used in the vignette section and the section on actual schools.
All other variables are either 10-point likert-type responses or categorical variables\footnote{
    It is an accepted practice to treat Likert-type responses as either categorical or continuous for regression analysis.
    Jaccard and Wan provide support for continuous analysis of Likert-type data.
    They note that severe departures from the assumptions on cardinality ``do not seem to affect Type I and Type II errors dramatically,''
    particularly when the Likert scale is five or more points\cite{jaccard1996lisrel}.
    This paper treats responses on a 10-point scale as continuous.
}.
Categorical variables are exclusively respondent characteristics.
There are four other respondent measures that are likert-type responses.
Vignette responses include responses for hirability and prestige,
while actual schools only receive responses for hirability.
% Prestige is measured two-ways in the vignette section, but it is only stipulated in the section on actual schools.

Respondent characteristics include eight standard controls and four questions unique to this study.
The eight controls include
age, gender, ethnicity, income,
level of education, employment status, the industry of occupation, and state of residence.
A unique question on work norms records whether the respondent tends ``to work more closely with coworkers at your company or customers and external business partners.''
The motivation for this question is to test whether prestige disproportionately impacts roles that are outward or client-facing.
Respondents are also directly asked whether they
``prefer to hire or work with a person that has a college degree rather a person that holds a reputable certification or non-college credential.''

Another unique control is support for online education.
This is useful to distinguish preference for alternative education which is due to unobserved preference for online education.
The fourth control is called expected conventionality.
This variable measures whether the respondent believes
that it will soon be common for an individual to obtain an alternative credential instead of going to college.
This is a useful correction variable for two reasons.
First, it seperates willingness to hire on the basis of the preferences of others from
willingness to hire on the basis of own preferences.

Second, surveys sometimes overreport demand effects because of the lack of cost constraint on respondent expression.
This bias is sometimes called budget constraint bias or omitted budget constraint bias\cite{ahlheim1998contingent, pachali2020omitted}.
% This is also in part corrected for by collecting income...so there isn't an 'unobserved budget' really...see sources cited
Without a cost constraint, there is a risk that the respondent may exagerate their true willingness to hire.
For individuals that reveal such an exageration effect,
it is plausible that their expected conventionality is similarly affected,
so using this variable as a control attenuates this concern.

Vignette questions are formatted following Atzm{\"u}ller and Steiner\cite{atzmuller2010experimental}.
Each vignette stipulates whether a school is accredited,
whether the respondent should imagine the school as impressive,
and whether the respondent should imagine that other people consider the school impressive.
Each stipulated factor can take a value of true or false,
resulting in eight vignette questions.

This study uses multiple regression and descriptive statistics to generate results.
% \footnote{
%     While the data for this analysis is not public, the analytical code is open-source.
%     See \url{https://github.com/Vandivier/research-dissertation-case-for-alt-ed/tree/master/papers/alt-ed-prestige}
% }.
Multiple regression is conducted using ordinary least squares (OLS)
for baseline hirability analysis
and linear mixed models (LMM)
are used for for vignette analysis.
OLS specification of vignette data is inappropriate because repeated measures of hirability
from a single participant introduce an individual-level bias into resulting coefficients.
LMM yields linear coefficients that can be interpreted as similar to OLS coefficients.
One difference of note is that adjusted r-squared is not available for an LMM model.
Following Magezi\cite{magezi2015linear},
linear mixed models in this paper use a within-participant random factor,
or individual random effects,
to correct for individual-level repeated measures bias.
% For this reasons, an OLS model is optimized for baseline hirability,
% and then that specification is trivially modified into an LMM model for further analysis.
% formula for LMM at https://www.statsmodels.org/stable/mixed_linear.html
% individual random effects https://www.statsmodels.org/stable/examples/notebooks/generated/mixed_lm_example.html#Growth-curves-of-pigs

% 2. how was nonresponse bias addressed? - maybe not at all
% - main way to address nonresponse bias is to explicitly capture and correct for all of the individual characteristics that matter: ethnicity, age, income...
% - it would not be enough to show nonresponse bias exists;
% - it would need to be shown that it exists in the direction of some effect that moves the relation of interest in a predictable and meaningful way;
% - else the criticism is an argument from ignorance which due dilligence has been undertaken to preclude.
% - https://forum.effectivealtruism.org/posts/a6LMQcER6Awhawtqq/using-amazon-s-mechanical-turk-for-animal-advocacy-studies
% - above indicates overstatement of effects...i would want more info...there is a paper internally cited
% - above also deflates income nonresponse bias consern (these don't pay much so systematic bias from rich ppl) also i explicitly capture income anyway
% - "AMT was found to be a reliable source of data and to diminish the potential for non-response error in online research"
% - https://www.ncbi.nlm.nih.gov/pmc/articles/PMC4397064/
% - https://duckofminerva.com/2013/07/mechanical-turk-and-experiments-in-the-social-sciences.html
% - https://www.tandfonline.com/doi/abs/10.1080/10967494.2016.1276493
% 3. How were ratings subjects selected? min 2*2*2 (isQuality)*(isBootcamp)*(isKnown) social and individual ratings [10 point likert-type unit]
% 4. a few correction variables based on literature review and computed norm factors how

\section{Results}

% top line results; alt creds are lower prestige on average but still important
Results ($n = 454$) indicate that accredited degrees are generally higher in prestige compared to alternative credentials.
At the same, alternative credentials are associated with significant hirability,
and alternative credentials are preferred to accredited degrees in a certain common situations.

Three specific situations are identified in which an alternative credential is preferred to a degree with respect to hirability.
First, specific alternative credentials are of particularly high prestige.
In this study, the prestige response for the average accredited degree is about equal to the prestige of a credential from Google.

Second, some individuals award prestige preferentially to alternative learning providers.
% analysis_5_transfer table a5.1
When comparing actual learning providers,
71 percent of respondents prefer at least one alternative credential to at least one university degree.
This proportion increases to about 75 percent when respondents are given rating data provided from online aggregator and review sites.
These sites include US News and Course Report, and they aggregate learning providers,
report standard information about those providers,
and allow users to leave reviews.

Third, in some cases there are indirect compensating factors, such as industry or state effects,
that enhance support for alternative credentials to the extent that they become competitive with an accredited degree.
For example, the state effect for California is positive on hirability
and it retains a magnitude that compensates almost exactly for the hirability penalty from non-accreditation.

% summary statistics


Mean baseline hirability is 7.58 on a 10-point scale, and the median response is 8.
Table \ref{tab:desc_stats} gives average hirability and prestige for interesting segments of respondents.
Four basic results in the table are worth noting.
First, stipulated prestige always moves with prestige response as expected.
Second, accredited schools are generally higher than non-accredited schools as expected.

Third, the difference in average hirability between high and low prestige providers
is more than twice the difference in hirability between accredited and unaccredited providers.
This supports the possibility that at some level of prestige,
alternative education becomes competitive with traditional education.
% The fourth result is that the average actual school with stipulated high prestige
% is too low in prestige to be competitive with the average actual school with an accredited status.
The fourth result is an initial attempt at a prestige rule of thumb.
For both vignette and actual schools,
if a school can obtain a prestige score of 7 or more,
it will be at least as prestigious as the average accredited school.

\begin{table}
    \caption{Average Hirability and Prestige}
    \resizebox{\columnwidth}{!}{
        {
\def\sym#1{\ifmmode^{#1}\else\(^{#1}\)\fi}
\begin{tabular}{llcc}
    \toprule
                              &                                   & \textbf{Average Hirability} & \textbf{Average Prestige} \\
    \midrule
    \textbf{Actual Schools}   &                                   &                             & 6.50                      \\
                              & \textbf{Accredited}               &                             & 7.05                      \\
                              & \textbf{Unaccredited}             &                             & 6.07                      \\
                              & \textbf{Difference}               &                             & 0.98                      \\
    \cmidrule{2-4}
                              & \textbf{Stipulated High Prestige} &                             & 6.72                      \\
                              & \textbf{Stipulated Low Prestige}  &                             & 6.23                      \\
                              & \textbf{Difference}               &                             & 0.49                      \\
    \cmidrule{2-4}
    \textbf{Vignette Schools} &                                   & 6.49                        & 6.21                      \\
                              & \textbf{Accredited}               & 6.97                        & 6.49                      \\
                              & \textbf{Unaccredited}             & 6.02                        & 5.93                      \\
                              & \textbf{Difference}               & 0.95                        & 0.56                      \\
    \cmidrule{2-4}
                              & \textbf{Stipulated High Prestige} & 7.59                        & 7.69                      \\
                              & \textbf{Stipulated Low Prestige}  & 5.63                        & 4.94                      \\
                              & \textbf{Difference}               & 1.96                        & 2.75                      \\
    %   \midrule
    % \textbf{Pooled}           &                                   &                             & 6.37                      \\
    %                           & \textbf{Accredited}               &                             & 6.77                      \\
    %                           & \textbf{Unaccredited}             &                             & 6.01                      \\
    %                           & \textbf{Difference}               &                             & 0.76                      \\
    %                           & \textbf{Stipulated High Prestige} &                             & 7.00                      \\
    %                           & \textbf{Stipulated Low Prestige}  &                             & 5.80                      \\
    %                           & \textbf{Difference}               &                             & 1.20                      \\
    \bottomrule
    \bottomrule
\end{tabular}
}

    }
    \label{tab:desc_stats}
\end{table}

Google is the only unaccredited learning provider to acheive a strong competitive status.
The mean prestige response for Google was 7.10 and the median response was 7.
Two lower bars for competitive status are interesting.
First, an alternative provider can be described as moderately competitive if it fails to beat the average university,
but it succeeds in beating at least one university on average.
Second, an alternative provider can be described as weakly competitive if it fails to beat any university on average,
but it succeeds in beating at least one university in a significant percentage of individual responses.
No alternative credentials investigated in this study meet the criteria for moderate competitiveness.
% App Academy, General Assembly, and Google are the three alternative learning providers with stipulated high prestige.
All stipulated high prestige learning providers are at least weakly competitive.

When asked directly, 41.6 percent of respondents indicated that they would not prefer to
``work with a person that has a college degree rather a person that holds a reputable certification or non-college credential.''
When comparing prestige responses instead of asking directly,
over 70 percent of respondents preferred at least one actual alternative credential to at least one university credential.
Over half of respondents preferred at least one actual alternative credential that was stipulated as high prestige
to at least one university credential that was stipulated as high prestige.
When Google is excluded, over one-quarter of respondents preferred at least one actual alternative credential that was stipulated as high prestige
to at least one university credential that was stipulated as high prestige.

% TODO: Does zety belong in the conclusion?
Zety is in part a job search support platform.
Zety finds that one in six job applicants are given an interview,
and the average conversion rate from interview to offer was 19.78 in 2016\cite{turczynski_2021}.
Assuming rejections are independent enables naive estimation that most job searches consist of at least four interviews\footnote{
    Four independent games that each include an eighty percent chance of rejection yields $0.8^4 = 0.4096$.
    The associated probability of having at least one offer result from four interviews would be about $1 - 0.41 = 0.59$,
    or 59 percent, which is more likely than not.
} and dozens of applications.
Given the rates at which respondents prefer alternative credentials to accredited degrees,
a job search of typical length is likely to include several applications and at least one interview
with one or more employers that would prefer an alternative credential stipulated as high prestige to an accredited degree.

More than half of respondents prefer a high prestige alternative credential to at least one high prestige accredited degree.
After excluding the highest prestige alternative credential from Google,
more than one-quarter of respondents still prefer one of the remaining high prestige alternative credentials to at least one high prestige accredited degree.
% a2.2
When asked directly, about 42 percent of respondents state that they do not prefer
to work with a person that has a college degree rather a person that holds a reputable non-college credential.

%% TODO: BASELINE REGRESSION YIELDS PANELIZED REGRESSION

% m3, m4, m6

% summary statistics
% a2.1
Average hirability among the sample of 

1. summary Results (percent prefer degree, prestige high-low stipulated vs prestige yes-no accreditation)
2. vignette Results (prestige vs accreditation on hirability)
3. concrete results (translating prestige into aggregation site metrics; no fav reg)
    b. context effect

arguments roughly
1. summary data: stipulated high quality alt creds vs others and proportion that would weakly affirm
2. non-panel models of favorability
3. panelized model of vignette (primary regression analysis)
4. analysis of concrete data using descriptive statistics and insights from the vignette study

conclusion - implications for application from the consumer perspective and supply-side (getting on an aggregation site nbd, naming important, public branding)

theoretically (low-prestige + not well known) > (low prestige + well known). did that seem true?

\section{Conclusions}

Two rules of thumb:
1. Get a Google. (general competitive rule: Amazon and Microsoft are other examples)
2. Aggregator cream of the crop (secondary rule; may result in a longer job search but still viable)

In any case, also seek to leverage compensating factors (third rule)

The synthesis of results from this paper and the data from Zety on job search length forms a practical rule of thumb.
The selection criteria for stipulated high prestige alternative credentials other than Google was to use a reputable bootcamp aggregator,
in this case Course Report,
and to choose a learning provider with more than four hundred reviews and an average rating of more than 4.25 on a 5-point scale.
Google was selected on the basis of industry leadership.
Credential producers that are also Fortune 50 companies fit this description.

policy recommendations are constrained in recognition of the political economy of education as a turbulent, endogenous, entangled thing,
in contrast to a simplistic analysis that might treat politics as a thing distinct from and able to exogenously move the economy
(wagner: % https://mason.gmu.edu/~rwagner/Wirth%20Keynote.pdf)
Wagner argues contra Becker and other neoclassicals that so-called political competition doesn't get you to the efficient economic outcome.
``Unlike market entities, political entities cannot generate their own revenue.
To support their activities, political entities must attach themselves parasitically to
market entities and activities''
so, rents still exist in some form or another, usually vis a vis an indirect transaction:
'an indirect form of transaction, not different in form from the various indirect transactions that arise in the presence of price controls'
from american democracy comes a forced triad
traces the problem all the way to the american constitutional architecture.
at some level, a full resolution to the problem would require competition at the level of the constitution; insert plug for seasteading
it's not actually even clear that the success of seasteading or another form of constutional competition is less feasible than
change in the below 8 laws;

also cite leeson/boettke robust political economy to emphasize the perhaps unexpectedly high difficulty of substantial change
perhaps refer to the friedman rule about how, even if we got the legal change we asked for,
administrative costs and so forth could result in cost/quality differentials that are in excess of what we want

perhaps a short history of educational change policy; media and social reaction (eg faciliation under trump was decried)
to emphasize the general pessimism extends into education and it perhaps accentuated rather than attenuated in this particular policy field

so, constitutional competition is the great height at which the last remainder of educational policy consequence can be addressed at the macro level
express optimism that much can be done at the mesoeconomic and micro levels; individuals, and firms and even meso entities like states and industries
can and are already doing much...

The conclusion describes policy options that solve for the remainder of concerns in higher education that survive competition from alternative credentials.

1. remove accreditation
2. lower the bar for accreditation
3. remove professional requirements that include requirement for accreditation
4. remove legally enshrined preferential treatment of accredited credentials (esp government pay scales and in government contracts)
5. how to fill the gap? define skill-based requirements instead of accreditation (better outcomes for govt and all)
6. (institution based vs skill or course based accreditation)
    1. ...see 8 for a non-policy workaround (but working around is likely more costly than direct fix)
6. aggregators need to do a better job
7. individuals don't need to wait for social movement to reap individual-level benefits
8. hybrid approach; universities can (and increasingly already do) partner with external providers for a win-win
    1. better university placement rates
    2. better university solutioning of the skill gap
    3. accreditation given fully or partially (ACE; professional credit) to unaccredited partners (institution based vs skill or course based accreditation)

1. overall is there evidence that bootcamp can replace college due to prestige effects? what caveats?
- we need to consider price in the real world
- we had a threshold of name recognition to ensure statistical confidence; what happens if we drop below that? say a bootcamp w less than 30 reviews?
2. online education isn't a good fit for all learners
3. parental preference matters; if parents will finance university but not bootcamp (or vice-versa) then effective cost to student is modified
- this also presents a channel for pro-ACNG movement
4. policy differences: government or employer may provide financing or employment conditinoal on a certain type of educational participation

overall advice: as an individual, work backwards from a particular desired job description
as a society: push for policies which make comparing across types easier, generate lower cost and higher skill outputs,
and clearer signals of productivity, reducing demand for prestige which is really 1) a second-best / fallback signal of productivity,
and 2) a form of slack and a biased means toward inefficient selection (anti-diversity as a byproduct; again see Pedigree)

\bibliography{./BibFile}

\end{document}
