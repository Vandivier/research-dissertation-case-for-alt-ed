% cleaning note: run:
% `deno run --allow-read --allow-write ../../../../deno-utility-tex-to-text/tex-to-text.ts --infile /c/Users/vandi/workspace/research-dissertation-case-for-alt-ed/papers/alt-ed-prestige/alt-ed-prestige.tex'

% using Elseveir template per https://www.elsevier.com/authors/author-schemas/latex-instructions
\documentclass[review]{elsarticle}
\usepackage{lineno,hyperref}
\modulolinenumbers[5]
\bibliographystyle{elsarticle-num}

\usepackage{booktabs}
\usepackage{graphicx}
\graphicspath{{../alt-ed-survey/figures-and-tables}}
\usepackage{hyperref}
\usepackage{threeparttable}  
\usepackage{tikz}
\usetikzlibrary{calc,matrix}

\begin{document}
\begin{frontmatter}

    \title{
        Hirability and Educational Prestige
    }

    \author[mymainaddress]{John Vandivier}
    \address[mymainaddress]{4400 University Dr, Fairfax, VA 22030}
    \ead{jvandivi@masonlive.gmu.edu}

    \begin{abstract}
        % a4.1
        Alternative credentials
        offer a partial solution to the skill gap and student debt crises,
        supernormal returns for some students,
        and a tool to support diversity hiring for firms.
        This paper tests the hypothesis that educational prestige explains hirability
        better than accreditation.
        Results from an original questionnaire ($n = 454$)
        confirm that prestige explains comparatively more hirability variance.
        Accredited credentials have higher average prestige,
        but alternative credentials have a larger variance in prestige,
        so a significant number of job opportunities favor the nontraditional student.
        When prestige is low, hirability for alternative credentials remains nontrivial.
        Analysis using ordinary least squares and linear mixed models demonstrate that
        industry, state, individual, and other effects
        favor the nontraditional student in specific cases.
        % below line is true per a4.2 but it's a very weak increase so let's not mention
        % Exposure to quality ratings for learning providers
        % is associated with improved relative prestige on alternative credentials.
        % The conclusion includes a discussion on policy,
        % and actionable recommendations for employers, students, and quality rating aggregators.
        % Do I really need to offer policy solutions?
        % At the same time, situations are revealed in which alternative credentials are preferred.
        % Hirability is high for alternative credentials,
        % and these credentials can be obtained at a lower cost and difficulty compared
        % to certain degrees.
        % As a result, alternative credentials are sometimes preferred even when comparative prestige is low.
        % Descriptive statistics show that prestige for alternative credentials is
        % more dispersed than prestige for the accredited degree,
        % so some proprtion of respondents prefer alternative credentials.
    \end{abstract}

    \begin{keyword}
        debt crisis, skill gap, prestige, social economics, education economics, alternative education    %%% not grammatical
        \MSC[2010] I20, I24, J24, B55                                                                     %%% not grammatical
    \end{keyword}

\end{frontmatter}

\pagebreak
\linenumbers

\section{Introduction}

The accredited degree is an established means toward desirable labor outcomes,
but proliferation of the degree is associated with a variety of well-understood issues including
the student debt crisis, skill gaps, grade inflation, low social return,
and contribution to lack of diversity in the labor market.
Alternative credentials, or non-accredited credentials,
are a broad category of offerings that exhibit greater variation intensity, price, and outcomes.
This paper hypothesizes that variation in the properties of alternative credentials
contemporaneously inhibit normal usage and support occasional superior results.

Strategic usage of alternative credentials requires a qualitative description of the occasions in which they provide superior results.
Decomposition of alternative credentials can be accomplished through a variety of lenses,
and this paper takes the lens of prestige.
This paper tests the hypothesis that prestige is a better explanation of willingness to hire
than accreditation.
Results are made practical through the description of low-effort methods to identify of high prestige alternative credentials.

The motivation for the lens of prestige extends from the literatures on education economics and the economics of social norms.
In a broad review of economics and norm types, hiring decisions exist within what Elster would identify as work norms\cite{elster1989social}.
Elster finds support for a rational explanation of work norms, concluding with the caveat that actual social interactions may involve unobserved emotional effects.
Similarly, the neoclassical model utilized in this paper comes with the caveat that
applicability of results is constrained in cases where a hiring decision is made subject to abnormal emotional effects.
Elster importantly distinguishes between social and legal norms, in a move that becomes surprisingly useful to the current analysis.

Within the economics of work norms, Rivera is one scholar to have recently operationalized social norms as prestige\cite{rivera2016pedigree}.
Rivera finds that prestige is important in her analysis, but the scope of her analysis is focused
within an analysis of traditional education and a few specific industries including health and law.
The current paper extends the analysis of prestige and hiring norms
across all industries and to include alternative credentials.
% arguably I emphasize coding bootcamps / other bootcamp industries / information technology industry and perhaps sales

TODO: continue below 4/4/21 5:30 PM

Vocational schools are only sometimes accredited, and non-accredited means of learning some vocations are common.
For this reason, it is widely accepted that there are particular jobs for which alternative credentials are preferred,
but the explanation for the lack of general non-accreditated higher learning is commonly made on the basis of social norms.
As two sides of the same coin, some argue that vocational schooling and alternative credentials bear some stigma.
Others argue that there is a positive social expectation, and indeed systematized facilitation,
of educational progress from a high school to an accredited university.


TODO: DRAW ON ELSTER TO CONSTRUCT A NORMS-WEIGHTED UTILITY FUNCTION
we are specifically concerned with work norms

% Elster, smithian connection

As a preview, statistical evidence confirms that prestige independently explains hirability better than accreditation alone,
but accreditation fails to be explained away.
Instead, models that use both factors produce superior estimates of willingness to hire.
The independent importance of accreditation indicates that asymptotic improvement to alternative credentials
are unlikely to fully compete away the traditional education system.
A change in legal norms 
The conclusion describes policy options that solve for the remainder of concerns in higher education that survive competition from alternative credentials.

% extends university prestige literature. book Pedigree by Lauren Rivera: https://www.youtube.com/watch?v=_70KfdHV7Nk
% sprinkle in some norms economics:
% https://www.behavioraleconomics.com/resources/mini-encyclopedia-of-be/social-norm/
% https://www.aeaweb.org/articles?id=10.1257/jep.3.4.99

theory: prestige substantially explains hirability
rationally conspicuous consumption: part of the degree value is prestige
suppose prestige is a combination of quality and familiarity; a simple combination would be the product of those factors.

prestige vs well-known vs reputable; I'm using "economics of prestige" like "economics of reputation"
which is well-known + highly regarded
as opposed to well-known + lowly regarded
(a note on question 13 and 14 wording:
in common parlance: prestigious implies "elite" while reputable indicates generally good but not necessarily elite)

\section{Data and Methodology}

1. Online questionnaire
2. how was nonresponse bias addressed? - maybe not at all
- main way to address nonresponse bias is to explicitly capture and correct for all of the individual characteristics that matter: ethnicity, age, income...
- it would not be enough to show nonresponse bias exists;
- it would need to be shown that it exists in the direction of some effect that moves the relation of interest in a predictable and meaningful way;
- else the criticism is an argument from ignorance which due dilligence has been undertaken to preclude.
- https://forum.effectivealtruism.org/posts/a6LMQcER6Awhawtqq/using-amazon-s-mechanical-turk-for-animal-advocacy-studies
- above indicates overstatement of effects...i would want more info...there is a paper internally cited
- above also deflates income nonresponse bias consern (these don't pay much so systematic bias from rich ppl) also i explicitly capture income anyway
- "AMT was found to be a reliable source of data and to diminish the potential for non-response error in online research"
- https://www.ncbi.nlm.nih.gov/pmc/articles/PMC4397064/
- https://duckofminerva.com/2013/07/mechanical-turk-and-experiments-in-the-social-sciences.html
- https://www.tandfonline.com/doi/abs/10.1080/10967494.2016.1276493
3. How were ratings subjects selected? min 2*2*2 (isQuality)*(isBootcamp)*(isKnown) social and individual ratings [10 point likert-type unit]
4. a few correction variables based on literature review and computed norm factors how

Test making them aware of the normative data and price data for rating quality and hirability (not for rating name recognition)

\section{Results}

% top line results; alt creds are lower prestige on average but still important

Results ($n = 454$) indicate that accredited degrees are generally higher in prestige compared to alternative credentials.
At the same, alternative credentials are associated with significant hirability,
and alternative credentials are preferred to accredited degrees in a certain common situations.

Three specific situations are identified in which an alternative credential is preferred to a degree from a hirability perspective.
First, specific alternative credentials are of particularly high prestige.
This study found that a credential from Google was regarded as better than some accredited degrees.

Second, some individuals place high value on alternative education.
% analysis_5_transfer table a5.1
This study found that 71 percent of respondents preferred at least one specific alternative credential to at least one specific accredited degree.
This proportion increases to about 75 percent when respondents are given rating data provided from online aggregator and review sites.
These sites include US News and Course Report, and they aggregate learning providers, report standard information about those providers, and allow users to leave reviews.

More than half of respondents prefer a high prestige alternative credential to at least one high prestige accredited degree.
After excluding the highest prestige alternative credential from Google,
more than one-quarter of respondents still prefer one of the remaining high prestige alternative credentials to at least one high prestige accredited degree.
% a2.2
When asked directly, about 42 percent of respondents state that they do not prefer
to work with a person that has a college degree rather a person that holds a reputable non-college credential.

Third, in some cases there are indirect compensating factors, such as industry or state effects,
that enhance support for alternative credentials to the extent that they become competitive with an accredited degree.
In one regression model discussed later on, a state effect for California is positive on hirability
to the extent that it compensates almost exactly for the hirability penalty from non-accreditation.

Zety is in part a job search support platform.
Zety finds that one in six job applicants are given an interview,
and the average conversion rate from interview to offer was 19.78 in 2016\cite{turczynski_2021}.
Assuming rejections are independent allows us to naively estimate that most job searches consist of at least four interviews\footnote{
    Four independent games that each include an eighty percent chance of rejection yields $0.8^4 = 0.4096$.
    The associated probability of having at least one offer result from four interviews would be about $1 - 0.41 = 0.59$,
    or 59 percent, which is more likely than not.
} and dozens of applications.
Given the rates at which respondents prefer alternative credentials to accredited degrees,
a job search of typical length is likely to include several applications and at least one interview
with one or more employers that would prefer an alternative credential to an accredited degree.

% summary statistics



% summary statistics
% a2.1
Average hirability among the sample of 

1. summary Results (percent prefer degree, prestige high-low stipulated vs prestige yes-no accreditation)
2. vignette Results (prestige vs accreditation on hirability)
3. concrete results (translating prestige into aggregation site metrics; no fav reg)
    b. context effect
4. [optional] did stipulated prestige match response prestige? did response prestige correlate to aggregation site metrics?

1. did perceived prestige explain willingness to hire?
2. did normal prestige explain willingness to hire?
3. is a prestigious alt cred better than a non-prestigious university? where are these prestige cutoffs? are they 1:1?
4. how substantially did individual prestige vary from normal prestige (more variation indicates that it's easier for low-prestige to find a match somewhere)




arguments roughly
1. summary data: stipulated high quality alt creds vs others and proportion that would weakly affirm
2. non-panel models of favorability
3. panelized model of vignette (primary regression analysis)
4. analysis of concrete data using descriptive statistics and insights from the vignette study

conclusion - implications for application from the consumer perspective and supply-side (getting on an aggregation site nbd, naming important, public branding)

theoretically (low-prestige + not well known) > (low prestige + well known). did that seem true?

\section{Conclusions}

policy recommendations are constrained in recognition of the political economy of education as a turbulent, endogenous, entangled thing,
in contrast to a simplistic analysis that might treat politics as a thing distinct from and able to exogenously move the economy
(wagner: % https://mason.gmu.edu/~rwagner/Wirth%20Keynote.pdf)
Wagner argues contra Becker and other neoclassicals that so-called political competition doesn't get you to the efficient economic outcome.
"Unlike market entities, political entities cannot generate their own revenue.
To support their activities, political entities must attach themselves parasitically to
market entities and activities"
so, rents still exist in some form or another, usually vis a vis an indirect transaction:
'an indirect form of transaction, not different in form from the various indirect transactions that arise in the presence of price controls'
from american democracy comes a forced triad
traces the problem all the way to the american constitutional architecture.
at some level, a full resolution to the problem would require competition at the level of the constitution; insert plug for seasteading
it's not actually even clear that the success of seasteading or another form of constutional competition is less feasible than
change in the below 8 laws;

also cite leeson/boettke robust political economy to emphasize the perhaps unexpectedly high difficulty of substantial change
perhaps refer to the friedman rule about how, even if we got the legal change we asked for,
administrative costs and so forth could result in cost/quality differentials that are in excess of what we want

perhaps a short history of educational change policy; media and social reaction (eg faciliation under trump was decried)
to emphasize the general pessimism extends into education and it perhaps accentuated rather than attenuated in this particular policy field

so, constitutional competition is the great height at which the last remainder of educational policy consequence can be addressed at the macro level
express optimism that much can be done at the mesoeconomic and micro levels; individuals, and firms and even meso entities like states and industries
can and are already doing much...

The conclusion describes policy options that solve for the remainder of concerns in higher education that survive competition from alternative credentials.

1. remove accreditation
2. lower the bar for accreditation
3. remove professional requirements that include requirement for accreditation
4. remove legally enshrined preferential treatment of accredited credentials (esp government pay scales and in government contracts)
5. how to fill the gap? define skill-based requirements instead of accreditation (better outcomes for govt and all)
6. (institution based vs skill or course based accreditation)
    1. ...see 8 for a non-policy workaround (but working around is likely more costly than direct fix)
6. aggregators need to do a better job
7. individuals don't need to wait for social movement to reap individual-level benefits
8. hybrid approach; universities can (and increasingly already do) partner with external providers for a win-win
    1. better university placement rates
    2. better university solutioning of the skill gap
    3. accreditation given fully or partially (ACE; professional credit) to unaccredited partners (institution based vs skill or course based accreditation)

1. overall is there evidence that bootcamp can replace college due to prestige effects? what caveats?
- we need to consider price in the real world
- we had a threshold of name recognition to ensure statistical confidence; what happens if we drop below that? say a bootcamp w less than 30 reviews?
2. online education isn't a good fit for all learners
3. parental preference matters; if parents will finance university but not bootcamp (or vice-versa) then effective cost to student is modified
- this also presents a channel for pro-ACNG movement
4. policy differences: government or employer may provide financing or employment conditinoal on a certain type of educational participation

overall advice: as an individual, work backwards from a particular desired job description
as a society: push for policies which make comparing across types easier, generate lower cost and higher skill outputs,
and clearer signals of productivity, reducing demand for prestige which is really 1) a second-best / fallback signal of productivity,
and 2) a form of slack and a biased means toward inefficient selection (anti-diversity as a byproduct; again see Pedigree)

\bibliography{./BibFile}

\end{document}
