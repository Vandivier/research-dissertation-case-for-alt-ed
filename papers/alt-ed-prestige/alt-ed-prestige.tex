% using Elseveir template per https://www.elsevier.com/authors/author-schemas/latex-instructions
\documentclass[review]{elsarticle}
\usepackage{lineno,hyperref}
\modulolinenumbers[5]
\bibliographystyle{elsarticle-num}

\usepackage{booktabs}
\usepackage{graphicx}
\graphicspath{{../alt-ed-survey/figures-and-tables}}
\usepackage{hyperref}
\usepackage{threeparttable}  
\usepackage{tikz}
\usetikzlibrary{calc,matrix}

\begin{document}
\begin{frontmatter}

    \title{
        Hireability and Educational Prestige
    }

    \author[mymainaddress]{John Vandivier}
    \address[mymainaddress]{4400 University Dr, Fairfax, VA 22030}
    \ead{jvandivi@masonlive.gmu.edu}

    \begin{abstract}
        This paper investigates the effect of educational prestige on willingness to hire.
        This paper makes an important contribution to the existing literature on educational prestige
        in that it extends analysis beyond university education by considering coding bootcamps.
        This paper further decomposes prestige into seperate factors for familiarity and quality.
        Prestige is assessed from a normative perspective by comparing the explanatory power of aggregate and individual familiarity and quality ratings.

        % This analysis generalizes on both the university prestige literature
        % and the literature on hirability of alternatively credentialed non-college graduates (ACNG).

        % What do we find? Sample size? Caveats and implications?
    \end{abstract}

    \begin{keyword}
        TODO
        \MSC[2010] TODO123 %%% not grammatical
    \end{keyword}

\end{frontmatter}

\pagebreak
\linenumbers

\section{Introduction}

extends university prestige literature. book Pedigree by Lauren Rivera: https://www.youtube.com/watch?v=_70KfdHV7Nk
merge literature on alt ed hirability: https://www.tandfonline.com/doi/abs/10.1080/13504851.2020.1767279
sprinkle in some norms economics:
https://www.behavioraleconomics.com/resources/mini-encyclopedia-of-be/social-norm/
https://www.aeaweb.org/articles?id=10.1257/jep.3.4.99
factors and caveats?

theory: prestige substantially explains hirability
rationally conspicuous consumption: part of the degree value is prestige
suppose prestige is a combination of quality and familiarity; a simple combination would be the product of those factors.

\section{Data and Methodology}

1. Online questionnaire
2. how was nonresponse bias addressed? - maybe not at all
- main way to address nonresponse bias is to explicitly capture and correct for all of the individual characteristics that matter: ethnicity, age, income...
- it would not be enough to show nonresponse bias exists;
- it would need to be shown that it exists in the direction of some effect that moves the relation of interest in a predictable and meaningful way;
- else the criticism is an argument from ignorance which due dilligence has been undertaken to preclude.
- https://forum.effectivealtruism.org/posts/a6LMQcER6Awhawtqq/using-amazon-s-mechanical-turk-for-animal-advocacy-studies
- above indicates overstatement of effects...i would want more info...there is a paper internally cited
- above also deflates income nonresponse bias consern (these don't pay much so systematic bias from rich ppl) also i explicitly capture income anyway
- "AMT was found to be a reliable source of data and to diminish the potential for non-response error in online research"
- https://www.ncbi.nlm.nih.gov/pmc/articles/PMC4397064/
- https://duckofminerva.com/2013/07/mechanical-turk-and-experiments-in-the-social-sciences.html
- https://www.tandfonline.com/doi/abs/10.1080/10967494.2016.1276493
3. How were ratings subjects selected? min 2*2*2 (isQuality)*(isBootcamp)*(isKnown) social and individual ratings [10 point likert-type unit]
4. a few correction variables based on literature review and computed norm factors how

Test making them aware of the normative data and price data for rating quality and hirability (not for rating name recognition)

\section{Results}

1. did perceived prestige explain willingness to hire?
2. did normal prestige explain willingness to hire?
3. is a prestigious alt cred better than a non-prestigious university? where are these prestige cutoffs? are they 1:1?
4. how substantially did individual prestige vary from normal prestige (more variation indicates that it's easier for low-prestige to find a match somewhere)

theoretically (low-prestige + not well known) > (low prestige + well known). did that seem true?

\section{Conclusions}

1. overall is there evidence that bootcamp can replace college due to prestige effects? what caveats?
- we need to consider price in the real world
- we had a threshold of name recognition to ensure statistical confidence; what happens if we drop below that? say a bootcamp w less than 30 reviews?
2. online education isn't a good fit for all learners
3. parental preference matters; if parents will finance university but not bootcamp (or vice-versa) then effective cost to student is modified
- this also presents a channel for pro-ACNG movement
4. policy differences: government or employer may provide financing or employment conditinoal on a certain type of educational participation

overall advice: as an individual, work backwards from a particular desired job description
as a society: push for policies which make comparing across types easier, generate lower cost and higher skill outputs,
and clearer signals of productivity, reducing demand for prestige which is really 1) a second-best / fallback signal of productivity,
and 2) a form of slack and a biased means toward inefficient selection (anti-diversity as a byproduct; again see Pedigree)

\bibliography{./BibFile}

\end{document}
