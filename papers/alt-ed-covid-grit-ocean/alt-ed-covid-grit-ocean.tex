% using Elseveir template per https://www.elsevier.com/authors/author-schemas/latex-instructions
\documentclass[review]{elsarticle}
\usepackage{lineno,hyperref}
\modulolinenumbers[5]
\journal{Journal of \LaTeX\ Templates}
\bibliographystyle{elsarticle-num}

\usepackage{booktabs}
\usepackage{graphicx}
\graphicspath{{../alt-ed-survey/figures-and-tables}}
\usepackage{hyperref}
\usepackage{threeparttable}  
\usepackage{tikz}
\usetikzlibrary{calc,matrix}

\begin{document}
\begin{frontmatter}

    \title{
        % COVID-19 Effects on Interest in Alternative Postsecondary Learning
        % calls for papers: Policy Analysis and Management
        % https://www.journals.elsevier.com/journal-of-public-economics/call-for-papers/call-for-papers-the-public-economics-of-covid-19
        % https://www.journals.elsevier.com/social-sciences-and-humanities-open/call-for-papers/coronavirus-society-call-for-papers
        Personality and Ideological Factors of Alternative Learning Favorability
        % \tnoteref{titlenotes}
    }
    % \tnotetext[titlenotes]{
    %     Go to \url{https://github.com/Vandivier/research-dissertation-case-for-alt-ed/tree/master/papers/alt-ed-survey}
    %     for additional materials including the online appendix,
    %     survey data, and data analysis source code.
    % }

    %% Group authors per affiliation:
    \author[mymainaddress]{John Vandivier} % \fnref{authorlinefootnote}}
    \address[mymainaddress]{4400 University Dr, Fairfax, VA 22030}
    \ead{jvandivi@masonlive.gmu.edu}
    % \fntext[authorlinefootnote]{
    %     Vandivier: George Mason University,
    %     4400 University Dr, Fairfax, VA 22030,
    %     jvandivi@masonlive.gmu.edu.
    %     The author acknowledges valuable input from Bryan Caplan at George Mason University.
    % }

    \begin{abstract}
        This paper investigates an original data set (n = 217) to understand public
        and employer disposition toward alternative postsecondary learning.
        This study builds on the literatures of alternative learning and personality
        to solve an apparent contradiction where conservatives reject alternative learning.
        This paper specifically tests whether personality is a solving mechanism.
        % eg ideologically they support, but when reacting quickly in a survey,
        % anti-innocation conservative bias dominates
    \end{abstract}

    \begin{keyword}
        education economics, alternative education, debt crisis, big 5
        \MSC[2010] I21, I22, J20 % Unused at AEL: D12, J23, I24, I25, I26
    \end{keyword}

\end{frontmatter}

\pagebreak
\linenumbers

\section{Introduction and Description of Data}

% TODO: cite old paper and make the problem to be solved clear
% excerpt from prior paper commented below:
% ***
% Respondent favorability to government regulation moves positively with innovation
% while religiosity is associated with reduced innovation favorability.
% (ie religiosity as conservative proxy)...
% but really it's social conservatism which is a distinct sub-branch of conservatism and only weakly tied to free market principles.

% Status quo bias among conservatives is a common theme in the literature\cite{eidelman2012bias},
% but it is paradoxical in this situation.
% Markets facilitate innovation\cite{baumol2002free},
% so individuals seeking to maintain the status quo ought to disfavor it.
% Traditional education is heavily regulated,
% so individuals committed to high levels of regulation ought to disfavor
% alternatives.

% so the paradox arises from our concept of conservatism which is mixing two very different subtypes
% religious conservatives may oppose innovation, but among conservatives in general there isn't opposition to alt education
% low AI would be a status quo bias independent of religiosity

% A Kahneman-like explanation is one resolution.
% Survey respondents may be thinking fast\cite{kahneman2011thinking}.
% The preference of some conservatives for the status quo in education becomes explained by
% risk aversion, lack of openness, and related factors.
% It may be the case that many of these same individuals would favor alternative
% credentials when a logical mode of thought is activated.
% *** end excerpt

\section{Methodology and Model}

We assume indepdence of personality and ideology, along with stability of both over time.
These assumptions are necessary because multiple regression of both sets of effects cannot be accomplished with the present data set.

We have ideological effects, which I am distinguishing from cultural effects.
Cultural effects include regional and ethnic effects.

Non-cultural ideological effects include religiosity,
christianity,
favorability to regulation,
favorability to AI (conservatism and anti-innovation bias proxy),
STEM employment measure (scientism proxy),
and whether American education is important (nationalist / anti-foreign prox)

then there are personality effects (OCEAN, grit)

then there are standard sociological controls (age, gender, education, income)

\section{Results}

% start w replication section, then new OLS results, then implied OLS results (regional effects), then special checks (interactions)

how does this relate to hiring and firing or industry growth trends?
answer: personality answer: managers tend to have certain personality traits
do they also tend to have a certain ideology...? idk
industry growth isn't really effected bc personality is taken as socially stable,
this info could be relevant for product marketing; some personalities being more friendly to alt education
however, preliminary analysis indicates these effects are small

%

ai favorability is lower among low regulation supporters
highly significant and positive relation found, but total r2 is low
this is counterintuitive because intellectual conservatives should embrace technological advancement and the free market
potential solution: as a matter of personality, or Kahneman's System 1 response, conservatives may exhibit anti-innovation bias
raising more regulation and ai (eg techno-liberal or scientistic progressive) is associated with a reduction in alt ed cred support
reducing ai and reducing regulation (eg anti-innovation conservative) is associated with more support for alt ed cred (indicates ideological dominance over personality at survey time)
but, both of these effects are weak.
ideological dominance over personality is consistent with results in this paper: weak personality effects relative to ideology (2:1, without multiple regression).

% in preferred model a reason not to think grit attenuates concientiousness: eliminating the latter does not reverse the sign of the former

\section{Conclusions}

% 1. Personality matters more than a bunch of other things (time, ethnicity, industrial effects, familiarity/provider effects)
% 2. Personality matters less than ideology (regulation favorability)
% 3. For the purposes of this paper, grit > OCEAN and is not a synonym of concientiousness. read conclusion food comment in basic exploration
% personality not strong in the sense that p < 0.1, but additional sampling would likely resolve this
% if we believe we can improve alternative education favorability through exposure, we need to attenuate those expectations due to personality interaction
% 4. support for regulation retains positive assocation after personality correction
% 5. is personality even related to regulation positivity? (not significantly)
% 6. is personality interacted with familiarity? specifically, grit, concientiousness, or openness (neuroticism irrational pessimism)
%   grit-provider interaction is the only one that is significant, and it's negative
%   stronger than ideological (even AI) or other personality effects
%   the reasoning behind this isn't obvious but it may be that gritty folks are losers in the alternative education system

% what did prior survey say about regional effects? they're missing now

\bibliography{./BibFile}

\end{document}
