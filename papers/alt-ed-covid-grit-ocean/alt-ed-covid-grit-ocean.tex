% using Elseveir template per https://www.elsevier.com/authors/author-schemas/latex-instructions
\documentclass[review]{elsarticle}
\usepackage{lineno,hyperref}
\modulolinenumbers[5]
\journal{Journal of \LaTeX\ Templates}
\bibliographystyle{elsarticle-num}

\usepackage{booktabs}
\usepackage{graphicx}
\graphicspath{{../alt-ed-survey/figures-and-tables}}
\usepackage{hyperref}
\usepackage{threeparttable}  
\usepackage{tikz}
\usetikzlibrary{calc,matrix}

\begin{document}
\begin{frontmatter}

    \title{
        % COVID-19 Effects on Interest in Alternative Postsecondary Learning
        % calls for papers: Policy Analysis and Management
        % https://www.journals.elsevier.com/journal-of-public-economics/call-for-papers/call-for-papers-the-public-economics-of-covid-19
        % https://www.journals.elsevier.com/social-sciences-and-humanities-open/call-for-papers/coronavirus-society-call-for-papers
        Personality and Ideological Factors of Alternative Learning Favorability
        % \tnoteref{titlenotes}
    }
    % \tnotetext[titlenotes]{
    %     Go to \url{https://github.com/Vandivier/research-dissertation-case-for-alt-ed/tree/master/papers/alt-ed-survey}
    %     for additional materials including the online appendix,
    %     survey data, and data analysis source code.
    % }

    %% Group authors per affiliation:
    \author[mymainaddress]{John Vandivier} % \fnref{authorlinefootnote}}
    \address[mymainaddress]{4400 University Dr, Fairfax, VA 22030}
    \ead{jvandivi@masonlive.gmu.edu}
    % \fntext[authorlinefootnote]{
    %     Vandivier: George Mason University,
    %     4400 University Dr, Fairfax, VA 22030,
    %     jvandivi@masonlive.gmu.edu.
    %     The author acknowledges valuable input from Bryan Caplan at George Mason University.
    % }

    \begin{abstract}
        This paper investigates an original data set (n = 217) to understand public
        and employer disposition toward alternative postsecondary learning.
        This study builds on the literatures of alternative learning and personality
        to solve an apparent contradiction where conservatives reject alternative learning.
        This paper specifically tests whether personality is a solving mechanism.
        % eg ideologically they support, but when reacting quickly in a survey,
        % anti-innocation conservative bias dominates
    \end{abstract}

    \begin{keyword}
        education economics, alternative education, debt crisis, big 5
        \MSC[2010] I21, I22, J20 % Unused at AEL: D12, J23, I24, I25, I26
    \end{keyword}

\end{frontmatter}

\pagebreak
\linenumbers

\section{Introduction and Description of Data}

% TODO: cite old paper and make the problem to be solved clear
% excerpt from prior paper commented below:
% ***
% Innovation proxies include favorability to artificial
% intelligence, cryptocurrency, and online education. These factors are
% cross-correlated with a p-value of less than .001.
% Respondent favorability to government regulation moves positively with
% innovation, while religiosity is associated with reduced
% innovation favorability.

% Status quo bias among conservatives is a common theme in the literature\cite{eidelman2012bias},
% but it is paradoxical in this situation.
% Markets facilitate innovation\cite{baumol2002free},
% so individuals seeking to maintain the status quo ought to disfavor it.
% Traditional education is heavily regulated,
% so individuals committed to high levels of regulation ought to disfavor
% alternatives.

% A Kahneman-like explanation is one resolution.
% Survey respondents may be thinking fast\cite{kahneman2011thinking}.
% The preference of some conservatives for the status quo in education becomes explained by
% risk aversion, lack of openness, and related factors.
% It may be the case that many of these same individuals would favor alternative
% credentials when a logical mode of thought is activated.
% *** end excerpt

\section{Methodology and Model}

We assume indepdence of personality and ideology, along with stability of both over time.
These assumptions are necessary because multiple regression of both sets of effects cannot be accomplished with the present data set.

We have ideological effects, which I am distinguishing from cultural effects.
Cultural effects include regional and ethnic effects.

Non-cultural ideological effects include religiosity,
christianity,
favorability to regulation,
favorability to AI (conservatism and anti-innovation bias proxy),
STEM employment measure (scientism proxy),
and whether American education is important (nationalist / anti-foreign prox)

then there are personality effects (OCEAN, grit)

then there are standard sociological controls (age, gender, education, income)

\section{Results}

how does this relate to hiring and firing or industry growth trends?
answer: personality answer: managers tend to have certain personality traits
do they also tend to have a certain ideology...? idk
industry growth isn't really effected bc personality is taken as socially stable,
this info could be relevant for product marketing; some personalities being more friendly to alt education
however, preliminary analysis indicates these effects are small

\section{Conclusions}

% TODO

\bibliography{./BibFile}

\end{document}
