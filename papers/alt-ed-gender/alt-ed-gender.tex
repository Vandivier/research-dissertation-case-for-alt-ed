% using Elseveir template per https://www.elsevier.com/authors/author-schemas/latex-instructions
\documentclass[review]{elsarticle}
\usepackage{lineno,hyperref}
\modulolinenumbers[5]
\bibliographystyle{elsarticle-num}

\usepackage{booktabs}
\usepackage{graphicx}
\graphicspath{{../alt-ed-survey/figures-and-tables}}
\usepackage{hyperref}
\usepackage{threeparttable}
\usepackage{tikz}
\usetikzlibrary{calc,matrix}

\usepackage[capposition=top]{floatrow}

% ref: https://tex.stackexchange.com/questions/50747/options-for-appearance-of-links-in-hyperref
\hypersetup{
    breaklinks = true,   %%% not grammatical
    hidelinks = true,    %%% not grammatical
}

\begin{document}
\begin{frontmatter}

    \title{
        % Alternative Credentials and Gender Equity
        Coding Bootcamps and Gender Equity
    }

    \author[mymainaddress]{John Vandivier}
    \ead{john@afterecon.com}

    \begin{abstract}
        TODO
    \end{abstract}

    \begin{keyword}
        TODO %%% not grammatical
        \MSC[2010] TODO %%% not grammatical
    \end{keyword}

\end{frontmatter}

\pagebreak
\linenumbers


\section{Introduction}

This paper provides a review of current knowledge and original empirical evidence to solve for the present gendered debt crisis.
Given that sixty percent of college students are female[1],
it is no surprise that about sixty percent of student debt is held by women in the United States[2].
The problem of debt payoff is a harder task for the average woman.
In 2020, the average female worker in the United States made eighty-four cents on the dollar compared to a male.
The female labor force participation rate is historically lower than the rate for men,
and further,
the recent pandemic has disproportionately impacted women, pushing labor force participation to a 33-year low in 2021[4].

Two naive solutions to the gendered debt crisis include
increasing earnings for women
and reducing the college debt obtained by women.
This paper will argue that these naive solutions are exactly appropriate.
I will argue that digital bootcamps are an ideal tool for equitable wage normalization,
and that reductions in educational debt subsidy solve the problem of excess debt retention.
It turns out that non-females benefit from the same activities.

% https://twitter.com/JohnVandivier/status/1447343762024419334

Glassdoor is a leading job search platform.
Glassdoor reports that in the United States and abroad, occupational sorting alone explains the majority of the wage gap[9].
The present paper provides original research on the degree to which this sorting is voluntary.
If sorting is involuntary or switching costs are high, we have identified the root cause of the issue.
If sorting is voluntary and switching costs are low, disparities in wages are not a problem to be solved.

Burning Glass is another job market analytics platform, similar to Glassdoor.
Burning Glass reports that in 2013,
the average entry-level STEM job paid 66,123 dollars,
with 52,299 dollars for the average non-STEM job.
This difference represents a twenty-six percent premium for a STEM job[5].
Some institutions including NASA have begun to support initiatives related to STEAM, or STEM plus art[6].
Artistic disciplines are lower-earning than ordinary STEM roles.
Institutional support of STEAM over STEM is a move in the wrong direction.
It is a move that enables perpetuation or  of existing wage gaps.

As the acronym grows, the salary advantage is diluted, and the reverse is also true.
Top-paying degree fields include computer-related, engineering, and mathematical fields[3].
Computer-related jobs are at the top in pay but at the bottom in female program enrollment and industry employment.
Pew notes that women make up half of those employed in STEM fields,
but only twenty-five percent of computer jobs
and fifteen percent of engineering jobs[7].

Software engineering and data engineering are at the intersection of computer-related jobs and engineering.
A degree in computer science is the most common degree among software engineers[8].
Figure [FIG 1] provides historic data by gender on computer science degree enrollment,
software engineering as a profession,
and female graduation from coding bootcamps.

Software engineers are also referred to as software developers and a number of other job titles on the market.
% The division of Occupational Employment and Wage Statistics (OEWS)
% within the Bureau of Labor Statistics (BLS) currently captures these roles using OES 15-1256,
% 15-1257, and 15-1251.
In 2013, the average salary for software developers was 92,820 dollars[10].
This is a forty percent premium compared to the average STEM job using the earlier Burning Glass data.
In short, increasing female labor in software development is a potent strategy to increase wages.
Coding bootcamps are a known path to such increased participation, and the utilization of coding bootcamps
involves a double-benefit because the cost of attendence is generally lower.

If coding bootcamps are so effective, why are they not already widely used?
The present metastudy answers this question in three ways.
First, I document that coding bootcamps are already widely used and usage is increasing.
Second, I show that usage is associated with hirability.
This provides evidence for specific predictive models that are reused in this paper for an updated future outlook.
Third, I review ways in which the positive rate of adoption is currently constrained and can be strategically improved.

% note: the idea here is that present hirability explains both present and future usage
%    I then technically predict hirability and implicitly predict on usage


[FIG 1] - SOURCE
purpose 1: show share of female enrollment in CS programs vs coding bootcamps over time
1. from 2017 to 2021, stack overflow consistently shows that about 75 percent of professional programmers have a bachelor's degree or higher
  - https://insights.stackoverflow.com/survey/2021
  - this measure wasn't available in prior years (and neither was gender breakdown by pro dev status)
2. females among SO pro devs: 7.2 percent to 4.8 percent from 2017 to 2021
  - males: 89.5 percent to 92.8 percent from 2017 to 2021
  - other: 2.1 to 2.0
  - non-male within USA was 9.1 percent in 2021, although apples-to-apples trend data isn't present.
      * this is the highest percentage of all countries, but still quite asmmetrical
3. cs program female attendance over time
  - note this is only a loose correlation bc many programmers don't have a CS degree
  - https://codeorg.medium.com/women-computer-science-graduates-finally-surpass-record-set-17-years-ago-20a79a76275
4. course report bootcamp gender share over time
5. are social media, edtech, and digital marketing a larger share of female? what about people management in tech? (experience blocker)

- in tech, traditional credentials have particularly low value because employers in this industry are particularly quick learners
   - `https://econfaculty.gmu.edu/bcaplan/StudyGuide.pdf'
   - "Full catch up takes over 10 years"
   - But, Google says that degree-oriented models are insignificant predictors after 3 years
   - Since then, Google and many other tech firms have dropped the degree requirement. Coincidence? hardly.
   - So it's true that women are overrepresented in college and underrepresented in CS programs, but that is a less material fact in Third Age of Online Education

[FIG 2] - SOURCE (could be a table)
purpose: show bootcamps have been increasing relative to cs prog enrollment, and there is a coding bootcamp growth + favorability association
1. cs program enrollment over time
2. coding bootcamp enrollment over time
3. coding bootcamp favorability over time


\section{Methodology}

% note: any time 'novel data' is referred to in this paper, it's referring to the new data I gathered in this paper
The method of this paper follows a five-step process.
First, I systematically review claims in the existing literature on hirability and alternative credentials.
Second, I collect novel data for the purpose of replicating and testing claims in the existing literature.
Third, I test some new hypotheses on the novel data set.
Fourth, I perform a meta-analysis across current and prior findings.

% note: dataset_vandivier includes "four survey administrations from February 2018 to May 2019"
The papers in this systematic meta-analysis use a technical definition of hirability that first appears in [11].
The seminal paper uses a public dataset that is hosted by Mendeley Data[12].
The first group of papers considered for inclusion in this meta-analysis are
papers that cite the seminal work or the associated dataset.
In addition, all papers from any author in the first group are also considered.
% Finally, the present paper is included as a special case.
Papers that do not provide new data nor make any new predictions or claims related to hirability are excluded.

Searches were conducted on Google Scholar, Google Dataset Search, Mendeley Data, and the Social Science Research Network (SSRN).
Ten papers, including the present paper, were considered for inclusion.
Five candidate papers are included in the analysis and five were excluded\footnote{
    Three papers are excluded because they are unrelated to a study of hirability.
    Two other papers are excluded because they lack new data, predictions, and claims.
}.

Each of the five papers in the meta-analysis follow a questionnaire design.
Appendex TODO contains a copy of the questionnaire used to collect the novel data analyzed in this paper.
This questionnaire was administered in October of 2021.
Respondents are United States citizens at or over the age of eighteen.
Qualified respondents participated in the survey through the Amazon Mechanical Turk platform.

Responses are investigated using an exploratory data analysis (EDA),
descriptive statistics,
robust feature selection,
and regression analysis.
Descriptive statistics focus on identifying differences in questionnaire response by gender.
Feature selection and regression analysis focus on identifying whether gender or gender-interacted variables
explain differences in evaluation of alternative credentials or demand for a career in programming.
Novel results are subsequently compared with other findings from the systematic review.

EDA uncovered an interesting four-way interaction between
the industry of occupation, gender, risk preference, and naive preference for a programming career.
The subsequent section provides a detailed description of the whole dataset,
but this singular finding resulted in a situation where the number of independent variables exceeded the number of samples.
This fact of the dataset informs the methods of feature selection that are used.

Robust feature selection is accomplished by conducting multiple feature selection algorithms.
These algorithms include the elastic net, univariate selection, and sequential forward-selection (SFS).
Leave-one-out cross-validation is used to tune the elastic net and terminate the SFS process\footnote{
    Specifically, I use the sklearn.linear_model.ElasticNetCV module of the Scikit-learn library in Python[21].
    The analytical code and novel data used in this study are available at TODO ANONYMIZE.
}.
The elastic net is an ideal feature selection tool because it combines the benefits of ridge regression and the lasso regression
while mitigating their respective drawbacks.
These benefits include stable feature selection and accounting for collinearity.
Elastic net can be used with linear and logistic regression, with categorical variables or otherwise,
and it is a recommended variable selection tool
when the number of predictors ($k$) is larger than the number of observations ($n$)[13].

Hirability is used as the main dependent variable for regression analysis,
but two other dependent variables were also studied to ensure completeness and robustness of findings.
Univariate selection is a trivial process that correlates features one-by-one to the dependent variable,
or hirability in the case of this study.
Univariate selection is not preferred to elastic net selection in large part because it does not account for colinearity.
Still, features that are selected by univariate selection are relatively strong, so employing the method adds an interesting nuance to results.

Several prior studies in the meta-analysis used recursive feature elimination (RFE).
Recursive feature elimination involves including a large set of variables concurrently in a model and removing them one at a time
until some condition is reached.
Because $k > n$ in the current analysis, it is not possible to run a long regression of this type.
RFE is relatively stable and generally preferred to SFS, but SFS is otherwise similar in nature to RFE.
As a result, it is a useful tool for the purpose of replication and also for checking the robustness of findings produced by the elastic net.
SFS is conducted using linear regression and also using scaled vector regression, which is somewhat orthogonal to the way the elastic net is computed.
This means that SFS could potentially uncover interesting nonlinear relations that the elastic net may not consider.

Finally, regression analysis is used to examine the magnitudes of the relatively small
number of interesting independent variables identified through robust feature selection.
Ordinary least squares (OLS) regression is conducted with two different dependent variables.
Logistic regression is conducted with a third dependent variable.


\section{Description of Data}

This paper leverages a novel set of online questionnaire responses ($n = 114$)\footnote{
    The data is available here TODO ANONYMIZE
}.
The questionnaire includes ninety-four questions.
The questionnaire is organized into ten sections.
The last section is administrative in nature.
The other nine sections correspond to groups of related factors.
Personality information includes continuous measures of the Big Five personality traits and for grit.
Other than those factors, all other variables are categorical or Likert-type responses.
The majority of Likert-type questions use a 10-point scale, and such variables are often analyzed as continuous data.
% \footnote{
%     It is an accepted practice to treat Likert-type responses as either categorical or continuous for regression analysis.
%     Jaccard and Wan provide support for continuous analysis of Likert-type data.
%     They note that severe departures from the assumptions on cardinality ``do not seem to affect Type I and Type II errors dramatically,''
%     particularly when the Likert scale is five or more points\cite{jaccard1996lisrel}.
% }.

Appendix TODO contains summary statistics for each variable.
% Variables in this study are a superset of variables from the other four studies in the meta-analysis, plus three additional questions.
% The three new questions are risk preference, naive preference for a programming career, and a categorical variable for worldview.
% The three additional questions were each specifically motivated by one or more claims or predictions in one of the other four studies.
% cite: weak correlation between self-reported and actual risk aversion https://wol.iza.org/articles/measuring-individual-risk-preferences/long
% author above says `Survey measures lack a clear connection to theory' -> signaling model solves that (perceived risk > actual risk in hiring decision)
% 'enjoy taking risks' phrasing a. taken from above and b. consistent with programming career orientation (an alternative to people-thing orientation measure)
The main dependent variable of interest is hirability,
a non-price indicator of willingness to hire on the basis of an alternative credential.
Two other variables that are investigated as both dependent and independent variables are
employment in the information technology industry
and naive preference for a programming career.
% Investigating these three factors allows for both supply-side and demand-side analysis of the importance of gender with respect to alternative credential supply and demand.


% my empirical argument that sometimes, including in my use case, ar2>aic for model selection
% 1. ar2 selects vars w p-value < .4 (not sure always true, but true for samples in 30-3000 range)
% 2. i know from empirical evidence that industry is an important variable; including industry favored by ar2 and not by aic
% 3. i should prefer aicc, dic, or waic to aic based on my knowledge anyway bc K>n for me
% 4. I'm interested in factor identification and I consider a 'true model' to be one with no spurious or random variables; all vars must p<0.5


% test/train split LASSO and Elstatic Net were unable to find a decent model
% i've been told it's due to sample size https://stats.stackexchange.com/questions/548958/why-are-my-elastic-net-and-lasso-r-squared-measures-negative
% but small-n regularization/penalization is OK, standard rule ~30+ for t-tests is still advised
% https://stats.stackexchange.com/questions/200242/minimum-number-of-observations-needed-for-penalized-regression

% %%% a discussion on selection criteria
% %%% AIC vs BIC vs AICc vs DIC vs WAIC vs MDL vs PCA vs stepwise
% %%% maybe should be a completeley different paper
% Economic Efficiency over Information-Theoretic Efficiency
% AIC has been called an 'efficient selection criterion' but this has an information-theoretic meaning rather than an economic-theoretic meaning
% it can be easily shown that a model with AIC informatic inefficiency is efficient under economic analysis.
% Further, AIC can be shown to overfit even on information-theoretic ground
% BIC is supposed to select a 'true model' and AIC makes no such claim but optimizes on error minimization instead
% There seems to be a conse
% # "Is the true model finite-dimensional or infinite-dimensional? There seems to be a consensus that for the
% #   former case, BIC should be preferred and AIC should be chosen for the latter."
% # 1) i think they use 'true model' differently than me, 2) I think my model is infinite-dimensional (and I would expect that for most models)
% # http://users.stat.umn.edu/~yangx374/papers/Pre-Print_2003-10_Biometrika.pdf
% for me, I consider a 'true model' to be one with neither omitted variable bias nor included variables bias, so p should be under 0.5
%   - and it's risky to exclude variables with p < 0.2 (ordinal logic plus signifiance:
%   we can naively dismiss due to possible model or measurement error
%   but it's hard to dismiss when we are still significant after an ordinal correction)
% % https://stats.stackexchange.com/questions/408267/justification-for-and-optimality-of-r2-adj-as-a-model-selection-criterion/548190


\section{Results}

% Figure \ref{fig:one}
The novel questionnaire administered for the present gender analysis resulted in one hundred and fourteen responses.
The mean hirability response is $8.1$ out of ten, and the median is $8$.
Mean hirability among males is not significantly different from mean hirability among females ($p < 0.51$).
Eight responses were dropped for three reasons and the remaining responses include some partial responses.
In the cleaned dataset ($n = 106$), mean hirability is $8.2$ out of ten,
the median remains at $8$, and hirability remains invariant by gender ($p < 0.68$).
Invariance of hirability by gender is consistent with the insignificance of gender in three of four other papers in the meta-analysis.

The purpose of the current study is to detect gender effects, but only two observations of "Other" gender were obtained.
These responses were dropped because they are insufficient to yield meaningful statistical analysis.
Three more responses were dropped for lack of variation in skill question responses.
Three last responses were dropped as negative outliers in hirability response.
These three negative outliers responded with a hirability of less than five.
The remaining responses for hirability are approximately normal in distribution.

Table TODO shows dates of observation, sample sizes, and mean hirability for the five papers in the meta-analysis.
The collective data indicates a positive trend over time.
Two explanations of a positive trend to hirability exist prior in the literature.
The earliest theory is the concept of employer-led favorability[14].
The second explanation is a positive exposure effect that results from coronavirus-induced remote activities[15].
These theories are compatible with each other and they indiciate that hirability is expected to remain high
to the extent that social activity remains remote, particularly with respect to industry norms.

% note: does this block provide evidence that female openness dominates risk aversion?
Risk preference did not differ significantly by gender ($n = 99, p < 0.57$)
and naive preference for a programming career did not differ by gender ($n = 99, p < 0.74$).
Preference for working with a peer that is a college graduate did differ significantly by gender ($n = 105, p < 0.05$).
% TODO: is there a latex for start/end quotes?
When asked to agree or disagree that, "I prefer to hire or work with a person that has a college degree
rather [sic] a person that holds a reputable certification or non-college credential,"
seventy-seven percent of men agreed while only fifty-eight percent of women agreed.
High female enrollment into coding bootcamps
relative to accredited computer science programs is not simply explained
by differential interest in programming as a career,
risk preference,
or perceived value of alternative credentials.

% TODO: cite men are things-oriented
Professional psychology suggests that men tend to be things-oriented and therefore will prefer a career in software development.
The current study presents conflicting evidence in that women and men do not differ in preference for a career in programming.
The conflict can be reconciled by realizing that modern software development is largely collaborative and social.
The social nature of modern programming is consistent with high value for soft skills as emphasized by one paper in the meta-study[16].

A survey of software developers from Stack Overflow adds nuance to the relationship between programming and things-oriented work[17].
Programmers may specialize in a number of areas.
Graphic designers, web developers, and mobile application developers are more frequently female.
Systems and database administrators are more often male.
Notice that those specialties that tend female are associated with user interfaces and graphics.
This indicates that an explanation from orientation towards things or people works as a partial explanation of specialty selection,
but the explanation fails at the aggregate level of the modern software development job family.
Contrasting holistic and typical software development practices with the curriculum of a typical computer science program
indicates an emphasis on things that may discourage female interest in the field of study,
even while interest in the programming occupation persists.

% # each page of survey has a factor group. they are:
% # 1. thoughts on alt creds
% # 2. thoughts on rulebreakers
% # 3. occupational information
% # 4. demographic information
% # 5. personality information
% # 6. ideological
% # 7. covid impact
% # 8. learning provider questions
% # 9. perceived skill questions
% 10th page is a thank you page that allows for paying participants
The absence of differences in means is noteworthy,
but a better analysis accounts for interactions and colinearity between gender and other variables.
As previously noted, risk aversion did not differ significantly by gender.
This is surprising, because substantial external literature demonstrates a link between gender and risk aversion[18, 19].
Further research invariant to gender shows a link between risk aversion and occupational preference[20].
These prior findings motivated exploration of a three-way interaction that turned out to significantly explain hirability.
Further exploration showed that naive preference for a programming career varies importantly with respect to the three-way interaction.

When categorical response options are treated as distinct dummy variables,
the ninety-four question survey produces one hundred and sixty-five variables.
When the four-way interaction plus related lower-level interactions are counted,
there are two hundred and fourty-six variables.
% note: intuitively, the fact that risk preference interacts, for example, captures heterogeneous within-industry risk preference effects.

GENDER IS IMPORTANT BC SURVIVES NET AND MORE IMPORTANT WHEN INTERACTED
***TODO: REDO PREFERRED MODEL
PREFERED NX MODEL l1_ratio=0.015, INCLUDES GENDER, MALE SURVIVES TO 0.03 l1_ratio
# note "0.03 l1_ratio -> top 57/143 -> top 40 percent of factors"
***
% This is a count of actual variables rather than possible variables because many categorical response options were never selected.
These variable pools were seperately subjected to elastic net regularization with leave-one-out cross-validation.
Male identification is the only gender variable in the pool of one hundred and sixty-five variables.
Male identification is retained in a second-best regularized model ($ \lambda_1 / \lambda_2 = 0.019, k = 57, R^2=0.40 $).
This places gender in the top thirty-five percent of independent variables.
A distinct model derived from the same variable pool has trivially better out-of-sample explanatory power,
but does so with substantially fewer variables ($ \lambda_1 / \lambda_2 = 0.05, k = 34, R^2=0.41 $).

When the variable pool that includes interactions is reduced, many gender-interacted variables
are retained in the preferred model ($ \lambda_1 / \lambda_2 = 0.0001, k = 79, R^2=0.14 $).
From a pool of two hundred and fourty-six variables, seventy-nine variables are retained in this model.
This represents about thirty-two percent of the variable pool.
Eighteen of the retained variables are gender-related.
This confirms the idea that interacted gender effects are more stable and important predictors of hirability
compared to a simple dummy variable.

% This preferred model doesn't perform as well as the model reduced from pool 1, so an ideal model needs to superset factors and re-reduce
A second, heavily penalized, non-preferred model that is identified from this pool is also interesting for identifying the most important variables.
This heavily penalized model contains five factors,
including one gender-interacted variable ($ \lambda_1 / \lambda_2 = 0.005, k = 5, R^2=0.08 $).
Five factors represent the most important two percent of factors according to this factor evaluation strategy.
Interestingly, the retained gender variable is gender interacted with risk preference, preference for a programming career,
and occupation in the information technology industry.

% TODO: do I guard overfit by removing once-identified vars? or do I try all of them?
Two kinds of univariate selection and two kinds of SFS are also used to robustly identify interacted gender effects.
now put all the variables together and reduce using AIC+AR2 (elastic net model 1 plus model 2 plus other factors)
now compare remaining gender vars to multiply-attested gender vars; test 'big ones' again (2+ or 3+?)


% TODO: replicate overqualification effects are bad and extend into comparative overqualification effects

% do i need to explain how i tuned the elastic net?
% The coefficient estimates from the elastic net are defined by Equation TODO:
% % TODO: latex equation: https://en.wikipedia.org/wiki/Elastic_net_regularization
% The elastic net is tuned by testing many values of \lambda_1 while holding constant $\lambda_2 = 1$.
% Table TODO shows the count of retained features, the proportion of retained gender features,
% and the explanatory power of various models obtained
% Elastic net multiple regression with leave-one-out cross validation accounts for colinearity and protects against overfit.




Table TODO shows explanatory power by factor group on hirability.
Independent factors for ethnicity, industry of employment, and state of residence
have been consistently identified as important in other papers,
so these factor groups provide a meaningful standard for identifying other factors as important.
Table TODO (the same table) also includes explanatory power of the preferred linear model,
the model without gender-derivative features,
and a model of the gender-derivative features in isolation.


% # sample size for grit: 99
% # grit mean diff test by gender: 0.33946185189778477


% Did prestige differ by gender
% Does time value or value of money differ by gender?

2. reveal explanatory power by factor group
3. theory-driven discovery of four way interaction validated by elastic net and other feature selection
4. preferred model and direction of coefficients - appendix w full model and maybe another appendix w summary stats including n and table w selected variables of interest
5. ways to improve the analysis (constraints/shortcomings)
6. conclusion!

% pca and mca explored for dimensionality reduction of high-dimension categorical industry and state variables, but not feasible due to partial responses
% alternatives include: shrinkage (ridge, lasso, etc) or variable selection (best-subset, BFE), manual feature elimination (if a whole set is unimportant)

Specifically, likelihood to work in information technology and grit
% https://www.sciencedirect.com/science/article/pii/S0732118X1930234X

% TODO: mention cross-validation and also two samples w/ how much of raw is captured in deskewed


1. in non-pooled sample (Oct 2021) factors outnumber samples. so i tested models by factor group


1. findings: replicated and disputed compared to prior results
    * OVERQUALIFICATION IS A FICTION - get Degreed guys to talk about this too; cross-validated in a new temporal context, see factor groups
2. novel results
    * no relative overqualification; simpler mental model without loss of (statistically significant or economically important) generality

note: we are theoretically safe by identifying a marginal effect as important invariant to it's level
    so no I don't agree that if you include a higher order interaction you must include the lower order interaction (theoretical reason)
    although I do agree it's worth checking - looks like stack exchange illuminati say check it and if it's not significant nbd
    % https://stats.stackexchange.com/questions/236113/are-lower-order-interactions-a-prequisite-for-three-way-interactions-in-regressi

%%% REVIEW PRIOR RESULTS, THEORIZE TRENDS
1. one would be the covid effect, so test covid_impact*gender
2.  Conformity  and  perceived  skillgaps  explain  about  one-third  of  the  hirability  variance -> alt ed matching effects 2.pdf
    * soft skills matter more than industry effects
    * conformity reduces hirability on average
    * Respondents tend to perceive ACNGcandidates as an even mix of high and low performers.  Evidence favors employerrisk aversion toward labor productivity as a preferred explanation of low ACNGdemand.

m18_a vs m18 for anti-AIC

%%% OCTOBER 2021 NOVEL RESULTS

1. no diff in means for hirability, favor_programming_career, favor_seeking_risk by gender
2. grit is closer to significance then those above, but still failed to detect (n=0.34)
    - may indicate our sample is underpowered compared to other literature, will include this anyway.
3. I do detect a strong mean diff with is_prefer_college_peer (men scored significantly higher)
4. is_tech mean diff by gender expected but not found
    - about 34 percent of my sample claims to be in information tech industry
    - but overall economy would expect 7 percent: https://www.prnewswire.com/news-releases/us-tech-employment-surpasses-12-million-workers-accounts-for-10-of-nations-economy-301044415.html
5. let's try to introduce gender. should it be interacted, added, both, or neither?
    - we know that gender influences industry choice, so it might already be 'embedded' adding little to overall model power
    % ref: m1-m4
    - finding: gender insig in naive simple reg
    - adding or interacting gender to industry reduces ar2 and aic





%%% POOLED / PANELED SAMPLES BELOW

Begin by testing industry alone. Prior research shows that gender plays a role in selecting industry.
Now test interacting gender. Notice this reduces AIC and AR2. In other words,

Now,

Begin by creating an interaction for gender and risk seeking
Then an interaction for gender and favor_programming_career
Finally, an interaction for gender and industry


Actionable Things:
1.  assertiveness training:
 - https://www.shrm.org/resourcesandtools/hr-topics/behavioral-competencies/global-and-cultural-effectiveness/pages/study-gender-pay-gap-narrows-but-still-exists.aspx

note: elastic net that tested with interaction effects found no gain from comparative skill gaps!
(without interaction effects, comp skill gaps are important...so did the interactions partial out the comparative skill gaps?)


note: only two gender interactions were duplicatively identified between SFS and elastic net:
+ gender_male:favor_programming_career:favor_seeking_risk:industry[T.Information Technology]
+ gender_male:favor_programming_career:favor_seeking_risk:industry[T.Retail]
// my guess would be neg coefficient on retail just bc who needs alt ed or higher ed for retail...?

\section{Conclusions}

%# two begged questions in my conclusion: expected_conventionality, favor_programming_career -> scope for another study

% Do females drop out disproportionately? that might tell if cs aversion is developed on exposure (instititional culture prob) or pre-existing
% An agnostic estimate would say "some of both effects" however there is a bit of fiction to be dealt with on the pre-enrollment story
%   - the hollywood theory post-dates that men worked in industry so it can't be causal it can only be a source of lock-in or persitance
%   other theory: war effects (men and women, men go to war, men come back and hire men) -> but this could be a 'military man' effect rather than a pure male effect
% theory 1: institutional culture problem; college exposure and some priming through social network effects; curriculum and profs in part to blame
% theory 2: extra-collegiate culture problem; girls before signing up to college are pre-averse to CS; check freshman enrollment rates.
% theory 3: EQ repulsion or differential low eq tolerance...again we would prob expect disproportionate drop out (unless eq repulsion is expected prior)
% theory 4: male tendencey to perfectionism and gatekeeping

my analysis would be improved by:
  1. use multivariate selection instead of GenericUnivariateSelect; currently no widely-supported library seems to exist so this would need to be from scratch
  2. merge data sets
  3. more sophisticated analysis over time
  4. backwards selection instead of forward selection or k-fold cross validation instead of leave-one-out; basically, requires far larger sample


\bibliography{./BibFile}

[1]https://www.wsj.com/articles/college-university-fall-higher-education-men-women-enrollment-admissions-back-to-school-11630948233
[2]https://educationdata.org/student-loan-debt-by-gender
[3]https://fortune.com/2021/09/01/highest-paying-college-majors-starting-salaries/
[4]https://nwlc.org/resources/january-jobs-day-2021/
[5]https://www.burning-glass.com/research-project/stem/
[6]https://www.nasa.gov/press-release/nasa-announces-virtual-webb-steam-day-event-for-students-educators
[7]https://www.pewresearch.org/science/2021/04/01/stem-jobs-see-uneven-progress-in-increasing-gender-racial-and-ethnic-diversity/
[8]https://www.careerexplorer.com/careers/software-engineer/education/
[9]https://www.glassdoor.com/research/gender-pay-gap-2019/
[10]https://codesubmit.io/blog/the-evolution-of-developer-salaries/#from-2013-to-2019
[11]https://www.tandfonline.com/doi/abs/10.1080/13504851.2020.1767279?journalCode=rael20
[12]https://data.mendeley.com/datasets/75t88s6sdt/1
[13]https://rss.onlinelibrary.wiley.com/doi/full/10.1111/j.1467-9868.2005.00503.x
[14]Preliminary Attitudinal Trends in Alternative Postsecondary Learning
[15]COVID-19 and Alternative Postsecondary Learning
[16]Conformity and Soft Skills as Determinants of Alternatively Credentialed Non-College Graduate Hirability
[17]https://insights.stackoverflow.com/survey/2017
[18]https://academic.oup.com/jeea/article-abstract/7/2-3/649/2296043
[19]https://psycnet.apa.org/buy/1999-13573-004
[20]https://www.sciencedirect.com/science/article/pii/S105353571000154X
[21]https://scikit-learn.org/stable/about.html#citing-scikit-learn

\end{document}

% POST-PAPER TODO ie FUTURE STUDY

% TODO:
% "girls drop out of STEM subjects in their teens, discouraged by a lack of role models and a plethora of gender stereotypes."

% TODO: some other study: correct for agreement bias
% people saying yes govt is good and yes ai is good and yes free trade is good...maybe better 'true feelings' if we normalize for
% 1. individual effects and
% 2. cross-individual agreement effects

% % # the unhelpful STEM distinction
% % 1. all fields are STEM. Specifically, technology cuts across every industry. Science is moderately useful and math isn't a professional field. Engineering is the only useful component of STEM.
% % 2. The move towards STEAM, inclusive of art, is even worse in regards to the amibguity problem and meaningful salary differences.
% % 3. my regressions show STEM is insignificant in a multiple regression and many people don't know whether their role or industry is STEM.
% % 4. focus on what matters: industry is more predictive than STEM and more meaningful, useful, and specific. It also makes action clearer with respect to the inequality issues that we all care about.
% Women earned 53 percent of STEM college degrees in 2018,
% Women earned 53% of STEM college degrees in 2018, smaller than their 58% share of all college degrees. The gender dynamics in STEM degree attainment mirror many of those seen across STEM job clusters. For instance, women earned 85% of the bachelor’s degrees in health-related fields, but just 22% in engineering and 19% in computer science as of 2018.



ASSORTED NOTES TO SELF

single big question: what gender differences, if any, exist in favorability to alternative credentials?

bigger questions:
1. is the college degree going away?
2. should people prefer alternative credentials to the degree?
3. if people should, why are they not doing so?
4. can we create interesting predictions or interventions that generate value?
5. all of the above, with an eye toward gender.
[6. as a result, yes people-things distinction matters]

intro 1. review scope (lit review + limitted replication + contribute test for robustness over time)
pt 2 - do the lit review (review papers, factors, theories previously tested / established)
pt 3 - preview results

% in our use case if we ultimately care about landing programmers a job, then RIASEC is not only overkill but it is also a noisey and indirect indicator
% future work which seeks to look at alternative credentials outside of programming should incorporate a full RIASEC,
% but i introduce a direct programming interest question instead for now.
% Not only does RIASEC + PCA introduce measurement error into thing-orientation,
% but reapplication of the construct onto programming creates an a secondary measurement error because programming is not a purely thing-oriented activity.
% my wording (enjoyable) is based on the open version https://openpsychometrics.org/tests/RIASEC
% further, my wording is meant to cause the respondent to think of the whole career, not the narrow activity
% further, my wording is meant to address the possibility that a respondent may already have a job programming, but I don't want them to anchor on their present job.

+ add a "people person" question -> person-orientation theory (scale of 1 to 10: beware Dunning-Kruger / overconfidence here) (extroversion)
caveat: speak to dunning-kruger; an actual eq test would be more accurate,
    correlation expected though, roughly linear,
    weakly cubic relation: https://www.researchgate.net/publication/12688660_Unskilled_and_Unaware_of_It_How_Difficulties_in_Recognizing_One's_Own_Incompetence_Lead_to_Inflated_Self-Assessments )

people-oriented
https://www.linkedin.com/feed/update/urn:li:activity:6858771848033898497/
```https://onlinelibrary.wiley.com/doi/full/10.1111/j.1751-9004.2010.00320.x
Results show that gender differences in Big Five personality traits are ‘small’ to ‘moderate,’
with the largest differences occurring for agreeableness and neuroticism
(respective ds = 0.40 and 0.34; women higher than men). In contrast,
gender differences on the people–things dimension of interests are ‘very large’
(d = 1.18), with women more people-oriented and less thing-oriented than men.
...
Research on gender differences in personality needs to focus more attention on Big Five facets and on traits that may not have good representations in the Big Five model
'''

specific questions modeled after https://pubmed.ncbi.nlm.nih.gov/18712468/
```
These results suggest that biological factors may contribute to sex differences in personality and that culture plays a negligible to small role in moderating sex differences in personality.
'''
https://pubmed.ncbi.nlm.nih.gov/9569655/
http://psych.fullerton.edu/rlippa/
% https://www.hawaiipublicschools.org/DOE%20Forms/CTE/RIASEC.pdf
how do i complete people-things from RIASEC?
% Gender-Related Individual Differences and the Structure of Vocational
% Interests: The Importance of the People-Things Dimension
% future interesting idea: marital status might be an interesting variable

% RIASEC is a personality type based on activity interest
% a 2-factor PCA from RIASEC adds importantly to OCEAN
% and aligns with a personality analysis using the People-Things and Ideas-Data dimensions
% then we notice "very large" gender differences on the people-things axis, which is orthoganal to OCEAN
% https://openpsychometrics.org/tests/RIASEC/
%
% I conducted factor analyses on the six
% ipsatized RIASEC scores for all participants as well as for men
% and women separately (principal-components analysis, extraction of two factors, with orthogonal varimax rotation). Because
% the two factors that emerged from all three factor analyses were
% quite similar, only the results for men and women combined are
% reported here.
% https://towardsdatascience.com/principal-component-analysis-pca-from-scratch-in-python-7f3e2a540c51
% https://web-a-ebscohost-com.mutex.gmu.edu/ehost/pdfviewer/pdfviewer?vid=1&sid=056dfdd6-c42b-46ae-b687-4938a191ff87%40sdc-v-sessmgr03
%
% python and riasec https://www.learnpythonwithrune.org/pandas-explore-datasets-by-visualization-exploring-the-holland-code-riasec-test/

% 3. "5-Factor Conservatism" and Gender (could be it's own paper)
%     a. anti-innovation bias, or status quo bias; represents the gradualist aspect
%     b. antiforeign bias represents the nationalist aspect
% % new phrasing: https://www.econlib.org/archives/2006/03/framing_antifor.html
% % "I favor freer trade and migration with other nations"
%     c. regulatory measure for market orientation or fiscal conservatism.
%     d. prefer Christianity to religiosity for social conservatism a la United States
%         i. Agnostic or Atheist
%         ii. Spiritual or Theistic, No Specific Religion
%         iii. Religious, Not Christian
%         iv. Progressive Christian
%         v. Conservative or Evangelical Christian
%         vi. Other Christian
% % 190201-feb-survey-monkey has justification for non-preference of Christianity var
% % rebuttle to that justification: it's a poor construct
% % instead of

% Cultural effects include regional and ethnic effects.

% Non-cultural ideological effects include religiosity,
% christianity,
% favorability to regulation,
% favorability to AI (conservatism and anti-innovation bias proxy),
% and whether American education is important (nationalist / anti-foreign prox)


## the big battlecry
-> do not test any new theories, just outline them
-> this is a survey and a partial replication
    1. not a full replication
    2. if we had data to execute a full replication, why limit ourselves to the battlecry? why not test novel theories?
    3. the answer to 2 is: "paper brevity, paper clarity of thought / scope focus, and analysis and authoring time"
    4. however, we might still gather data to permit option 2, and yet execute the battlecry*****have a few papers in the tank

## gender fx theories
Big 3 Theories
1. differential personality theory (expected to relate to occupational preference)
    1. person-orientation vs object-orientation
    2. Grit+OCEAN cluster difference expected
2. differential risk tolerance / aversion
    a. differential norms orientation, ie preference for college
    b. differential favorability to non-conforming individuals [from skills paper survey https://papers.ssrn.com/sol3/papers.cfm?abstract_id=3829269]
    c. possibly endogenous to personality
3. differential education + work experience

minor theories (special mention territory)
4. formal social norms, ie public and private policy effects
    a. special attention to diversity hiring policies - i think there's external data
5. (informal) social norms theory (potentially out of scope)
    a. weak evidence from derived measures
        i. percent gender by industry; male or female stereotype theory? gender collapse?
        ii. men are more willing to work as a minority compared to females (minority employment as a risky choice + risk aversion)
        iii. do women say men have lower eq and men say women have higher eq? eq stereotyping norm
        iv. these once again seem to collapse into the Big 3
    b. stronger evidence would include a purpose-built questionnaire, rather than derived measures

Expected Endogenous Effects
4. differential skills
    a. expect to find it, but expect dominant endogeneity w/ personality theory, education, and work experience
    b. ie communication skill + teamwork + eq -> people oriented
5. covid favorability difference -> remote preferences -> expect dominant endogeneity
6. prestige valuation differences
    a. interesting result: are women harder to impress OR value different expression content than men (interview heuristic)

## antitheories
1. i don't think religiosity within-country (USA specifically) is important https://www.pewforum.org/2016/03/22/the-gender-gap-in-religion-around-the-world/
2. I believe credit by examination effects are captured by other measures of favorability to alternative credentials
3. veteran bootcamp disproportionately educate men - not really an antitheory just like "duh" and not worth measuring

***older stuff below


we have college student -> college-educated labor
what about alt ed -> alt ed labor?

course report 2014 - 2020 percent female bootcamp grads
https://www.coursereport.com/reports/2014-coding-bootcamp-outcomes-demographics-report

let's focus on coding bootcamps bc course report has the data, it's an important microcosm of alt creds, and IT wants women
(theoretically, minority gender demand commensurate w minority status and varies importantly by industry)

1. does enrollment track favorability?
2. does favorability track disparity?

What if we could combat the student debt crisis while at the same time enabling economic welfare, equity, and diversity?
Alternative credentials can accomplish all of the above in some cases, and many of the above in general.
A key unknown is whether alternative credentials drive diversity in general.
Research shows that students of alternative learning programs are disproportionately ethnic minorities, % this is sus tho...look into research
but such results constitute a superficial and unsatisfactory analysis of general diversity.
Employee diversity with respect to sex is the simplist possible diversity

https://www.giveagradago.com/news/2020/12/what-is-diversity-in-the-workplace/424
https://www.youtube.com/watch?v=3enoWw21j0Q
https://builtin.com/diversity-inclusion/types-of-diversity-in-the-workplace

---yes women are more risk averse
Byrnes JP, Miller DC, Schafer WD (1999) Gender differences in risk taking: A meta-analysis
https://onlinelibrary-wiley-com.mutex.gmu.edu/doi/pdfdirect/10.1111/joes.12069
Strong Evidence for Gender Differences in Risk Taking https://www.sciencedirect.com/science/article/abs/pii/S0167268111001521



---no women are not more risk averse
Financial Decision-Making: Are Women Really
More Risk-Averse? https://pubs-aeaweb-org.mutex.gmu.edu/doi/pdfplus/10.1257/aer.89.2.381
ARE WOMEN MORE RISK AVERSE? http://citeseerx.ist.psu.edu/viewdoc/summary?doi=10.1.1.500.1334
"Our findings suggest that when individuals have the same level of education irrespective of their
knowledge of finance, women are no more risk averse than men." (however
1. they suppose faculty have the same level of education and men outnumber women 11:3,
2. gender was significant among faculty (specifically when married)
3. the regression (table 5) has no education-gender interaction variable)
4. gender was significant in the pooled faculty test, but not in the split faculty test, indicating a sample size issue; the sign was positive across regs indicating
.../

above sides of the analytical debate are totally reconcilable imho as a yes with caveat


This paper is a metastudy with an eye toward out-of-sample prediction about the value of an early work strategy by gender.

1. we know there are personality differences with gender;
a. prosocial attitudes may lead to distaste for remote and technical learning
b. yet, openess may offset that
2. are women more risk averse? if so, that may reduce appetite for alternative credentials (directly and indirectly)
3. furtherkl


theoretic importance - gender is, in some sense, a post-discriminatory diversity target and a sucess story
theoretical question: how do we pick between wage parity and employment parity (that allegedly drives innovation through ideological diversity)?
(further, given preference differences, wage partity will likely and perhaps necessarily lead to employment differences)

Make a more interesting hook, but in boring terms:
1. this paper investigates the relationship between gender and alternative credentials
2. it's a metaanalysis and replication of prior papers
3. i look at attitudes but also utilization and impact
4. differing diversity from education and employer point of view;
women are the majority in college now but the minority in the work force generally,
but it varies importantly by industry
5. from an application standpoint - there are only so many standard sociological c
standard sociological controls: age, ethnicity, gender, education, income
6. sex presents one of the most straightforward diversity targets,
and it remains easy to achieve a high degree of success even when diversity targets are expanded to include non-binary gender identifcation.
in some sense it is the simplist of all diversity targets


less than a 1 percent expected change from 2000 to 2050? from 47 to 48? why? largely preferential not ability
https://www.bls.gov/opub/mlr/2002/05/art2full.pdf

labor force participation has been flat for 30 years:
https://www.dol.gov/agencies/wb/data/facts-over-time/women-in-the-labor-force#labor-force-participation-rate-of-women-by-age

so workforce gender discrimination has been a non-issue since 2000 and probably well before then


