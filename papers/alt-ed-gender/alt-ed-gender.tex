% using Elseveir template per https://www.elsevier.com/authors/author-schemas/latex-instructions
\documentclass[review]{elsarticle}
\usepackage{lineno,hyperref}
\modulolinenumbers[5]
\bibliographystyle{elsarticle-num}

\usepackage{booktabs}
\usepackage{graphicx}
\graphicspath{{../alt-ed-survey/figures-and-tables}}
\usepackage{hyperref}
\usepackage{threeparttable}
\usepackage{tikz}
\usetikzlibrary{calc,matrix}

\usepackage[capposition=top]{floatrow}

% ref: https://tex.stackexchange.com/questions/50747/options-for-appearance-of-links-in-hyperref
\hypersetup{
    breaklinks = true,   %%% not grammatical
    hidelinks = true,    %%% not grammatical
}

\begin{document}
\begin{frontmatter}

    \title{
        % Alternative Credentials and Gender Equity
        Coding Bootcamps and Gender Equity
    }

    \author[mymainaddress]{John Vandivier}
    \ead{john@afterecon.com}

    \begin{abstract}
        TODO
    \end{abstract}

    \begin{keyword}
        TODO %%% not grammatical
        \MSC[2010] TODO %%% not grammatical
    \end{keyword}

\end{frontmatter}

\pagebreak
\linenumbers

\section{Introduction}

This paper provides a survey of current knowledge and original empirical evidence to solve for the present gendered debt crisis.
Given that sixty percent of college students are female[1],
it is no surprise that about sixty percent of student debt is held by women in the United States[2].
The problem of debt payoff is a harder task for the average woman.
In 2020, the average female worker in the United States made eighty-four cents on the dollar compared to a male.
The female labor force participation rate is historically lower than the rate for men,
and further,
the recent pandemic has disproportionately impacted women, pushing labor force participation to a 33-year low in 2021[4].

Two naive solutions to the gendered debt crisis include
increasing earnings for women
and reducing the college debt obtained by women.
This paper will argue that these naive solutions are exactly appropriate.
I will argue that digital bootcamps are an ideal tool for equitable wage normalization,
and that reductions in educational debt subsidy solve the problem of excess debt retention.
It turns out that non-females benefit from the same activities.

% https://twitter.com/JohnVandivier/status/1447343762024419334

Glassdoor is a leading job search platform.
Glassdoor reports that in the United States and abroad, occupational sorting alone explains the majority of the wage gap[9].
The present paper provides original research on the degree to which this sorting is voluntary.
If sorting is involuntary or switching costs are high, we have identified the root cause of the issue.
If sorting is voluntary and switching costs are low, disparities in wages are not a problem to be solved.

Burning Glass is another job market analytics platform, similar to Glassdoor.
Burning Glass reports that in 2013,
the average entry-level STEM job paid 66,123 dollars,
with 52,299 dollars for the average non-STEM job.
This difference represents a twenty-six percent premium for a STEM job[5].
Some institutions including NASA have begun to support initiatives related to STEAM, or STEM plus art[6].
Artistic disciplines are lower-earning than ordinary STEM roles.
Institutional support of STEAM over STEM is a move in the wrong direction.
It is a move that enables perpetuation or  of existing wage gaps.

As the acronym grows, the salary advantage is diluted, and the reverse is also true.
Top-paying degree fields include computer-related, engineering, and mathematical fields[3].
Computer-related jobs are at the top in pay but at the bottom in female program enrollment and industry employment.
Pew notes that women make up half of those employed in STEM fields,
but only twenty-five percent of computer jobs
and fifteen percent of engineering jobs[7].

Software engineering and data engineering are at the intersection of computer-related jobs and engineering jobs.
A degree in computer science is the most common degree among software engineers[8].
Figure TODO provides historic data by gender on computer science degree enrollment,
software engineering as a profession,
and female graduation from coding bootcamps.

Software engineers are referred to by a wide variety of job titles on the market.
The division of Occupational Employment and Wage Statistics (OEWS)
within the Bureau of Labor Statistics (BLS) currently captures these roles using OES 15-1256,
15-1257, and 15-1251.
In 2013, the average salary for software developers was 92,820 dollars[10].
This is a forty percent premium compared to the average STEM job using the earlier Burning Glass data.
In short, specific facilitation of women into

TODO:
"girls drop out of STEM subjects in their teens, discouraged by a lack of role models and a plethora of gender stereotypes."

% TODO: some other study: correct for agreement bias
% people saying yes govt is good and yes ai is good and yes free trade is good...maybe better 'true feelings' if we normalize for
% 1. individual effects and
% 2. cross-individual agreement effects





% # the unhelpful STEM distinction
% 1. all fields are STEM. Specifically, technology cuts across every industry. Science is moderately useful and math isn't a professional field. Engineering is the only useful component of STEM.
% 2. The move towards STEAM, inclusive of art, is even worse in regards to the amibguity problem and meaningful salary differences.
% 3. my regressions show STEM is insignificant in a multiple regression and many people don't know whether their role or industry is STEM.
% 4. focus on what matters: industry is more predictive than STEM and more meaningful, useful, and specific. It also makes action clearer with respect to the inequality issues that we all care about.
Women earned 53 percent of STEM college degrees in 2018,
Women earned 53% of STEM college degrees in 2018, smaller than their 58% share of all college degrees. The gender dynamics in STEM degree attainment mirror many of those seen across STEM job clusters. For instance, women earned 85% of the bachelor’s degrees in health-related fields, but just 22% in engineering and 19% in computer science as of 2018.

[1]https://www.wsj.com/articles/college-university-fall-higher-education-men-women-enrollment-admissions-back-to-school-11630948233
[2]https://educationdata.org/student-loan-debt-by-gender
[3]https://fortune.com/2021/09/01/highest-paying-college-majors-starting-salaries/
[4]https://nwlc.org/resources/january-jobs-day-2021/
[5]https://www.burning-glass.com/research-project/stem/
[6]https://www.nasa.gov/press-release/nasa-announces-virtual-webb-steam-day-event-for-students-educators
[7]https://www.pewresearch.org/science/2021/04/01/stem-jobs-see-uneven-progress-in-increasing-gender-racial-and-ethnic-diversity/
[8]https://www.careerexplorer.com/careers/software-engineer/education/
[9]https://www.glassdoor.com/research/gender-pay-gap-2019/
[10]https://codesubmit.io/blog/the-evolution-of-developer-salaries/#from-2013-to-2019

assertiveness training:
 - https://www.shrm.org/resourcesandtools/hr-topics/behavioral-competencies/global-and-cultural-effectiveness/pages/study-gender-pay-gap-narrows-but-still-exists.aspx

% FED: "For practical applications, 10 percent, on average, is a good estimate of the return [per year]"
% https://www.stlouisfed.org/publications/regional-economist/january-2010/the-return-to-education-isnt-calculated-easily
% Bryan would say that's a massive overestimate: probably cut it in half.
% for one, Fed estimates presume the student is capable of completing another year of education, but according to Caplan this is unlikely
% I agree w Bryan but alternative pathways can easily beat 10 percent anyway.
% Table A1 for staters file:///D:/Downloads/.ptmp277819/completion.pdf
% women are better students than men
Alternative education allows for better return to education compared to the traditional bachelor's education.

graphic 1
1. from 2017 to 2021, stack overflow consistently shows that about 75 percent of professional programmers have a bachelor's degree or higher
  - https://insights.stackoverflow.com/survey/2021
  - this measure wasn't available in prior years (and neither was gender breakdown by pro dev status)
2. females among SO pro devs: 7.2 percent to 4.8 percent from 2017 to 2021
  - males: 89.5 percent to 92.8 percent from 2017 to 2021
  - other: 2.1 to 2.0
  - non-male within USA was 9.1 percent in 2021, although apples-to-apples trend data isn't present.
      * this is the highest percentage of all countries, but still quite asmmetrical
3. cs program female attendance over time
  - note this is only a loose correlation bc many programmers don't have a CS degree
  - https://codeorg.medium.com/women-computer-science-graduates-finally-surpass-record-set-17-years-ago-20a79a76275
4. course report bootcamp gender share over time
5. are social media, edtech, and digital marketing a larger share of female? what about people management in tech? (experience blocker)

- in tech, traditional credentials have particularly low value because employers in this industry are particularly quick learners
   - `https://econfaculty.gmu.edu/bcaplan/StudyGuide.pdf'
   - "Full catch up takes over 10 years"
   - But, Google says that degree-oriented models are insignificant predictors after 3 years
   - Since then, Google and many other tech firms have dropped the degree requirement. Coincidence? hardly.
   - So it's true that women are overrepresented in college and underrepresented in CS programs, but that is a less material fact in Third Age of Online Education

% TODO: validate whether prestige var independent from school favorability effects
% I don't have prestige var here...oops

single big question: what gender differences, if any, exist in favorability to alternative credentials?
a survey

bigger questions:
1. is the college degree going away?
2. should people prefer alternative credentials to the degree?
3. if people should, why are they not doing so?
4. can we create interesting predictions or interventions that generate value?
5. all of the above, with an eye toward gender.
[6. as a result, yes people-things distinction matters]

intro 1. review scope (lit review + limitted replication + contribute test for robustness over time)
pt 2 - do the lit review (review papers, factors, theories previously tested / established)
pt 3 - preview results

methodology

results

conclusions + future opportunities

data ref: alt-ed-metasurvey

I Want:
4. Prestige

I Have:
1. Skill
2. Grit + OCEAN
3. COVID
5. Standard sociological measures

% cite: weak correlation https://wol.iza.org/articles/measuring-individual-risk-preferences/long
% author above says `Survey measures lack a clear connection to theory' -> signaling model solves that (perceived risk > actual risk in hiring decision)
% 'enjoy taking risks' phrasing a. taken from above and b. consistent with programming career orientation (an alternative to people-thing orientation measure)
+ add a direct measure of risk tolerance?

% in our use case if we ultimately care about landing programmers a job, then RIASEC is not only overkill but it is also a noisey and indirect indicator
% future work which seeks to look at alternative credentials outside of programming should incorporate a full RIASEC,
% but i introduce a direct programming interest question instead for now.
% Not only does RIASEC + PCA introduce measurement error into thing-orientation,
% but reapplication of the construct onto programming creates an a secondary measurement error because programming is not a purely thing-oriented activity.
% my wording (enjoyable) is based on the open version https://openpsychometrics.org/tests/RIASEC
% further, my wording is meant to cause the respondent to think of the whole career, not the narrow activity
% further, my wording is meant to address the possibility that a respondent may already have a job programming, but I don't want them to anchor on their present job.

+ add a "people person" question -> person-orientation theory (scale of 1 to 10: beware Dunning-Kruger / overconfidence here) (extroversion)
caveat: speak to dunning-kruger; an actual eq test would be more accurate,
    correlation expected though, roughly linear,
    weakly cubic relation: https://www.researchgate.net/publication/12688660_Unskilled_and_Unaware_of_It_How_Difficulties_in_Recognizing_One's_Own_Incompetence_Lead_to_Inflated_Self-Assessments )

```https://onlinelibrary.wiley.com/doi/full/10.1111/j.1751-9004.2010.00320.x
Results show that gender differences in Big Five personality traits are ‘small’ to ‘moderate,’
with the largest differences occurring for agreeableness and neuroticism
(respective ds = 0.40 and 0.34; women higher than men). In contrast,
gender differences on the people–things dimension of interests are ‘very large’
(d = 1.18), with women more people-oriented and less thing-oriented than men.
...
Research on gender differences in personality needs to focus more attention on Big Five facets and on traits that may not have good representations in the Big Five model
'''

specific questions modeled after https://pubmed.ncbi.nlm.nih.gov/18712468/
```
These results suggest that biological factors may contribute to sex differences in personality and that culture plays a negligible to small role in moderating sex differences in personality.
'''
https://pubmed.ncbi.nlm.nih.gov/9569655/
http://psych.fullerton.edu/rlippa/
% https://www.hawaiipublicschools.org/DOE%20Forms/CTE/RIASEC.pdf
how do i complete people-things from RIASEC?
% Gender-Related Individual Differences and the Structure of Vocational
% Interests: The Importance of the People-Things Dimension
% future interesting idea: marital status might be an interesting variable

% RIASEC is a personality type based on activity interest
% a 2-factor PCA from RIASEC adds importantly to OCEAN
% and aligns with a personality analysis using the People-Things and Ideas-Data dimensions
% then we notice "very large" gender differences on the people-things axis, which is orthoganal to OCEAN
% https://openpsychometrics.org/tests/RIASEC/
%
% I conducted factor analyses on the six
% ipsatized RIASEC scores for all participants as well as for men
% and women separately (principal-components analysis, extraction of two factors, with orthogonal varimax rotation). Because
% the two factors that emerged from all three factor analyses were
% quite similar, only the results for men and women combined are
% reported here.
% https://towardsdatascience.com/principal-component-analysis-pca-from-scratch-in-python-7f3e2a540c51
% https://web-a-ebscohost-com.mutex.gmu.edu/ehost/pdfviewer/pdfviewer?vid=1&sid=056dfdd6-c42b-46ae-b687-4938a191ff87%40sdc-v-sessmgr03
%
% python and riasec https://www.learnpythonwithrune.org/pandas-explore-datasets-by-visualization-exploring-the-holland-code-riasec-test/

TODO?
% 3. "5-Factor Conservatism" and Gender (could be it's own paper)
%     a. anti-innovation bias, or status quo bias; represents the gradualist aspect
%     b. antiforeign bias represents the nationalist aspect
% % new phrasing: https://www.econlib.org/archives/2006/03/framing_antifor.html
% % "I favor freer trade and migration with other nations"
%     c. regulatory measure for market orientation or fiscal conservatism.
%     d. prefer Christianity to religiosity for social conservatism a la United States
%         i. Agnostic or Atheist
%         ii. Spiritual or Theistic, No Specific Religion
%         iii. Religious, Not Christian
%         iv. Progressive Christian
%         v. Conservative or Evangelical Christian
%         vi. Other Christian
% % 190201-feb-survey-monkey has justification for non-preference of Christianity var
% % rebuttle to that justification: it's a poor construct
% % instead of

% Cultural effects include regional and ethnic effects.

% Non-cultural ideological effects include religiosity,
% christianity,
% favorability to regulation,
% favorability to AI (conservatism and anti-innovation bias proxy),
% and whether American education is important (nationalist / anti-foreign prox)


## the big battlecry
-> do not test any new theories, just outline them
-> this is a survey and a partial replication
    1. not a full replication
    2. if we had data to execute a full replication, why limit ourselves to the battlecry? why not test novel theories?
    3. the answer to 2 is: "paper brevity, paper clarity of thought / scope focus, and analysis and authoring time"
    4. however, we might still gather data to permit option 2, and yet execute the battlecry*****have a few papers in the tank

## gender fx theories
Big 3 Theories
1. differential personality theory (expected to relate to occupational preference)
    1. person-orientation vs object-orientation
    2. Grit+OCEAN cluster difference expected
2. differential risk tolerance / aversion
    a. differential norms orientation, ie preference for college
    b. differential favorability to non-conforming individuals [from skills paper survey https://papers.ssrn.com/sol3/papers.cfm?abstract_id=3829269]
    c. possibly endogenous to personality
3. differential education + work experience

minor theories (special mention territory)
4. formal social norms, ie public and private policy effects
    a. special attention to diversity hiring policies - i think there's external data
5. (informal) social norms theory (potentially out of scope)
    a. weak evidence from derived measures
        i. percent gender by industry; male or female stereotype theory? gender collapse?
        ii. men are more willing to work as a minority compared to females (minority employment as a risky choice + risk aversion)
        iii. do women say men have lower eq and men say women have higher eq? eq stereotyping norm
        iv. these once again seem to collapse into the Big 3
    b. stronger evidence would include a purpose-built questionnaire, rather than derived measures

Expected Endogenous Effects
4. differential skills
    a. expect to find it, but expect dominant endogeneity w/ personality theory, education, and work experience
    b. ie communication skill + teamwork + eq -> people oriented
5. covid favorability difference -> remote preferences -> expect dominant endogeneity
6. prestige valuation differences
    a. interesting result: are women harder to impress OR value different expression content than men (interview heuristic)

## antitheories
1. i don't think religiosity within-country (USA specifically) is important https://www.pewforum.org/2016/03/22/the-gender-gap-in-religion-around-the-world/
2. I believe credit by examination effects are captured by other measures of favorability to alternative credentials
3. veteran bootcamp disproportionately educate men - not really an antitheory just like "duh" and not worth measuring

***older stuff below


we have college student -> college-educated labor
what about alt ed -> alt ed labor?

course report 2014 - 2020 percent female bootcamp grads
https://www.coursereport.com/reports/2014-coding-bootcamp-outcomes-demographics-report

let's focus on coding bootcamps bc course report has the data, it's an important microcosm of alt creds, and IT wants women
(theoretically, minority gender demand commensurate w minority status and varies importantly by industry)

1. does enrollment track favorability?
2. does favorability track disparity?

What if we could combat the student debt crisis while at the same time enabling economic welfare, equity, and diversity?
Alternative credentials can accomplish all of the above in some cases, and many of the above in general.
A key unknown is whether alternative credentials drive diversity in general.
Research shows that students of alternative learning programs are disproportionately ethnic minorities, % this is sus tho...look into research
but such results constitute a superficial and unsatisfactory analysis of general diversity.
Employee diversity with respect to sex is the simplist possible diversity

https://www.giveagradago.com/news/2020/12/what-is-diversity-in-the-workplace/424
https://www.youtube.com/watch?v=3enoWw21j0Q
https://builtin.com/diversity-inclusion/types-of-diversity-in-the-workplace

---yes women are more risk averse
Byrnes JP, Miller DC, Schafer WD (1999) Gender differences in risk taking: A meta-analysis
https://onlinelibrary-wiley-com.mutex.gmu.edu/doi/pdfdirect/10.1111/joes.12069
Strong Evidence for Gender Differences in Risk Taking https://www.sciencedirect.com/science/article/abs/pii/S0167268111001521



---no women are not more risk averse
Financial Decision-Making: Are Women Really
More Risk-Averse? https://pubs-aeaweb-org.mutex.gmu.edu/doi/pdfplus/10.1257/aer.89.2.381
ARE WOMEN MORE RISK AVERSE? http://citeseerx.ist.psu.edu/viewdoc/summary?doi=10.1.1.500.1334
"Our findings suggest that when individuals have the same level of education irrespective of their
knowledge of finance, women are no more risk averse than men." (however
1. they suppose faculty have the same level of education and men outnumber women 11:3,
2. gender was significant among faculty (specifically when married)
3. the regression (table 5) has no education-gender interaction variable)
4. gender was significant in the pooled faculty test, but not in the split faculty test, indicating a sample size issue; the sign was positive across regs indicating
.../

above sides of the analytical debate are totally reconcilable imho as a yes with caveat


This paper is a metastudy with an eye toward out-of-sample prediction about the value of an early work strategy by gender.

1. we know there are personality differences with gender;
a. prosocial attitudes may lead to distaste for remote and technical learning
b. yet, openess may offset that
2. are women more risk averse? if so, that may reduce appetite for alternative credentials (directly and indirectly)
3. furtherkl


theoretic importance - gender is, in some sense, a post-discriminatory diversity target and a sucess story
theoretical question: how do we pick between wage parity and employment parity (that allegedly drives innovation through ideological diversity)?
(further, given preference differences, wage partity will likely and perhaps necessarily lead to employment differences)

Make a more interesting hook, but in boring terms:
1. this paper investigates the relationship between gender and alternative credentials
2. it's a metaanalysis and replication of prior papers
3. i look at attitudes but also utilization and impact
4. differing diversity from education and employer point of view;
women are the majority in college now but the minority in the work force generally,
but it varies importantly by industry
5. from an application standpoint - there are only so many standard sociological c
standard sociological controls: age, ethnicity, gender, education, income
6. sex presents one of the most straightforward diversity targets,
and it remains easy to achieve a high degree of success even when diversity targets are expanded to include non-binary gender identifcation.
in some sense it is the simplist of all diversity targets


less than a 1 percent expected change from 2000 to 2050? from 47 to 48? why? largely preferential not ability
https://www.bls.gov/opub/mlr/2002/05/art2full.pdf

labor force participation has been flat for 30 years:
https://www.dol.gov/agencies/wb/data/facts-over-time/women-in-the-labor-force#labor-force-participation-rate-of-women-by-age

so workforce gender discrimination has been a non-issue since 2000 and probably well before then

\section{Description of Data and Methodology}

my empirical argument that sometimes, including in my use case, ar2>aic for model selection
1. ar2 selects vars w p-value < .4 (not sure always true, but true for samples in 30-3000 range)
2. i know from empirical evidence that industry is an important variable; including industry favored by ar2 and not by aic
3. i should prefer aicc, dic, or waic to aic based on my knowledge anyway bc K>n for me
4. I'm interested in factor identification and I consider a 'true model' to be one with no spurious or random variables; all vars must p<0.5

% test/train split LASSO and Elstatic Net were unable to find a decent model
% i've been told it's due to sample size https://stats.stackexchange.com/questions/548958/why-are-my-elastic-net-and-lasso-r-squared-measures-negative
% but small-n regularization/penalization is OK, standard rule ~30+ for t-tests is still advised
% https://stats.stackexchange.com/questions/200242/minimum-number-of-observations-needed-for-penalized-regression
% however cross-validation with leave-one-out over test/trial split
% https://stats.stackexchange.com/questions/397722/is-it-a-good-idea-to-do-cross-validation-for-lasso-with-a-small-sample-size

%%% a discussion on selection criteria
%%% AIC vs BIC vs AICc vs DIC vs WAIC vs MDL vs PCA vs stepwise
%%% maybe should be a completeley different paper
Economic Efficiency over Information-Theoretic Efficiency
AIC has been called an 'efficient selection criterion' but this has an information-theoretic meaning rather than an economic-theoretic meaning
it can be easily shown that a model with AIC informatic inefficiency is efficient under economic analysis.
Further, AIC can be shown to overfit even on information-theoretic ground
BIC is supposed to select a 'true model' and AIC makes no such claim but optimizes on error minimization instead
There seems to be a conse
# "Is the true model finite-dimensional or infinite-dimensional? There seems to be a consensus that for the
#   former case, BIC should be preferred and AIC should be chosen for the latter."
# 1) i think they use 'true model' differently than me, 2) I think my model is infinite-dimensional (and I would expect that for most models)
# http://users.stat.umn.edu/~yangx374/papers/Pre-Print_2003-10_Biometrika.pdf
for me, I consider a 'true model' to be one with neither omitted variable bias nor included variables bias, so p should be under 0.5
  - and it's risky to exclude variables with p < 0.2 (ordinal logic plus signifiance:
  we can naively dismiss due to possible model or measurement error
  but it's hard to dismiss when we are still significant after an ordinal correction)
% https://stats.stackexchange.com/questions/408267/justification-for-and-optimality-of-r2-adj-as-a-model-selection-criterion/548190


% # each page of survey has a factor group. they are:
% # 1. thoughts on alt creds
% # 2. thoughts on rulebreakers
% # 3. occupational information
% # 4. demographic information
% # 5. personality information
% # 6. ideological
% # 7. covid impact
% # 8. learning provider questions
% # 9. perceived skill questions
% 10th page is a thank you page that allows for paying participants

% pca and mca explored for dimensionality reduction of high-dimension categorical industry and state variables, but not feasible due to partial responses
% alternatives include: shrinkage (ridge, lasso, etc) or variable selection (best-subset, BFE), manual feature elimination (if a whole set is unimportant)

1. deskewing data drops 3 samples (nbd)
2. 2 other-gendered rows dropped bc sample's too small for meaningful analysis
3. TODO: I still need to find and drop the "non-serious respondents"
TODO: create flag for basically 'not serious respondent' and operationalize as "gave the same answer for all skill questions, with two or less exceptions"

Regression analysis and mean difference tests are used to analyze the questionnaire responses.
Differences in means are tested among variables relevant to various explanatory hypotheses.
Specifically, favor_programming_career, favor_seeking_risk, hirability, is_prefer_college_peer

Three other mean difference tests are used to test whether the present data set is consistent with other results found in the literature
Specifically, likelihood to work in information technology and grit
% https://www.sciencedirect.com/science/article/pii/S0732118X1930234X

also test gender*covid_impact significance bc we think women disproportionately impacted by covid

To ensure robustness, multiple regression methods are used with multiple independent variables.
Ordinary least squares regression and vector regression are used to explain hirability.
Logistical regression is used to explain employment in the information technology industry.
In both of these models, gender and gender interactions are inspected for utility as independent variables.

% TODO: mention cross-validation and also two samples w/ how much of raw is captured in deskewed


1. in non-pooled sample (Oct 2021) factors outnumber samples. so i tested models by factor group

\section{Results}

1. findings: replicated and disputed compared to prior results
    * OVERQUALIFICATION IS A FICTION - get Degreed guys to talk about this too; cross-validated in a new temporal context, see factor groups
2. novel results
    * no relative overqualification; simpler mental model without loss of (statistically significant or economically important) generality

%%% REVIEW PRIOR RESULTS, THEORIZE TRENDS
1. one would be the covid effect, so test covid_impact*gender
2.  Conformity  and  perceived  skillgaps  explain  about  one-third  of  the  hireability  variance -> alt ed matching effects 2.pdf
    * soft skills matter more than industry effects
    * conformity reduces hireability on average
    * Respondents tend to perceive ACNGcandidates as an even mix of high and low performers.  Evidence favors employerrisk aversion toward labor productivity as a preferred explanation of low ACNGdemand.

m18_a vs m18 for anti-AIC

%%% OCTOBER 2021 NOVEL RESULTS

1. no diff in means for hirability, favor_programming_career, favor_seeking_risk by gender
2. grit is closer to significance then those above, but still failed to detect (n=0.34)
    - may indicate our sample is underpowered compared to other literature, will include this anyway.
3. I do detect a strong mean diff with is_prefer_college_peer (men scored significantly higher)
4. is_tech mean diff by gender expected but not found
    - about 34 percent of my sample claims to be in information tech industry
    - but overall economy would expect 7 percent: https://www.prnewswire.com/news-releases/us-tech-employment-surpasses-12-million-workers-accounts-for-10-of-nations-economy-301044415.html
5. let's try to introduce gender. should it be interacted, added, both, or neither?
    - we know that gender influences industry choice, so it might already be 'embedded' adding little to overall model power
    % ref: m1-m4
    - finding: gender insig in naive simple reg
    - adding or interacting gender to industry reduces ar2 and aic





%%% POOLED / PANELED SAMPLES BELOW

Begin by testing industry alone. Prior research shows that gender plays a role in selecting industry.
Now test interacting gender. Notice this reduces AIC and AR2. In other words,

Now,

Begin by creating an interaction for gender and risk seeking
Then an interaction for gender and favor_programming_career
Finally, an interaction for gender and industry


\section{Conclusions}




\bibliography{./BibFile}

\end{document}
