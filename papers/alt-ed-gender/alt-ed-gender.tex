% using Elseveir template per https://www.elsevier.com/authors/author-schemas/latex-instructions
\documentclass[review]{elsarticle}
\usepackage{lineno,hyperref}
\modulolinenumbers[5]
\bibliographystyle{elsarticle-num}

\usepackage{booktabs}
\usepackage{graphicx}
\graphicspath{{../alt-ed-survey/figures-and-tables}}
\usepackage{hyperref}
\usepackage{threeparttable}
\usepackage{tikz}
\usetikzlibrary{calc,matrix}

\usepackage[capposition=top]{floatrow}

% ref: https://tex.stackexchange.com/questions/50747/options-for-appearance-of-links-in-hyperref
\hypersetup{
    breaklinks = true,   %%% not grammatical
    hidelinks = true,    %%% not grammatical
}

\begin{document}
\begin{frontmatter}

    \title{
        % Alternative Credentials and Gender Equity
        Digital Bootcamps and Gender Equity
    }

    \author[mymainaddress]{John Vandivier}
    \ead{john@afterecon.com}

    \begin{abstract}
        TODO
    \end{abstract}

    \begin{keyword}
        TODO %%% not grammatical
        \MSC[2010] TODO %%% not grammatical
    \end{keyword}

\end{frontmatter}

\pagebreak
\linenumbers

\section{Introduction}

single big question: what gender differences, if any, exist in favorability to alternative credentials?
a survey

bigger questions:
1. is the college degree going away?
2. should people prefer alternative credentials to the degree?
3. if people should, why are they not doing so?
4. can we create interesting predictions or interventions that generate value?
5. all of the above, with an eye toward gender.
[6. as a result, yes people-things distinction matters]

intro 1. review scope (lit review + limitted replication + contribute test for robustness over time)
pt 2 - do the lit review (review papers, factors, theories previously tested / established)
pt 3 - preview results

methodology

results

conclusions + future opportunities

data ref: alt-ed-metasurvey

I Want:
4. Prestige

I Have:
1. Skill
2. Grit + OCEAN
3. COVID
5. Standard sociological measures

% cite: weak correlation https://wol.iza.org/articles/measuring-individual-risk-preferences/long
% author above says `Survey measures lack a clear connection to theory' -> signaling model solves that (perceived risk > actual risk in hiring decision)
% 'enjoy taking risks' phrasing a. taken from above and b. consistent with programming career orientation (an alternative to people-thing orientation measure)
+ add a direct measure of risk tolerance?

% in our use case if we ultimately care about landing programmers a job, then RIASEC is not only overkill but it is also a noisey and indirect indicator
% future work which seeks to look at alternative credentials outside of programming should incorporate a full RIASEC,
% but i introduce a direct programming interest question instead for now.
% Not only does RIASEC + PCA introduce measurement error into thing-orientation,
% but reapplication of the construct onto programming creates an a secondary measurement error because programming is not a purely thing-oriented activity.
% my wording (enjoyable) is based on the open version https://openpsychometrics.org/tests/RIASEC
% further, my wording is meant to cause the respondent to think of the whole career, not the narrow activity
% further, my wording is meant to address the possibility that a respondent may already have a job programming, but I don't want them to anchor on their present job.

+ add a "people person" question -> person-orientation theory (scale of 1 to 10: beware Dunning-Kruger / overconfidence here) (extroversion)
caveat: speak to dunning-kruger; an actual eq test would be more accurate,
    correlation expected though, roughly linear,
    weakly cubic relation: https://www.researchgate.net/publication/12688660_Unskilled_and_Unaware_of_It_How_Difficulties_in_Recognizing_One's_Own_Incompetence_Lead_to_Inflated_Self-Assessments )

```https://onlinelibrary.wiley.com/doi/full/10.1111/j.1751-9004.2010.00320.x
Results show that gender differences in Big Five personality traits are ‘small’ to ‘moderate,’
with the largest differences occurring for agreeableness and neuroticism
(respective ds = 0.40 and 0.34; women higher than men). In contrast,
gender differences on the people–things dimension of interests are ‘very large’
(d = 1.18), with women more people-oriented and less thing-oriented than men.
...
Research on gender differences in personality needs to focus more attention on Big Five facets and on traits that may not have good representations in the Big Five model
'''

specific questions modeled after https://pubmed.ncbi.nlm.nih.gov/18712468/
```
These results suggest that biological factors may contribute to sex differences in personality and that culture plays a negligible to small role in moderating sex differences in personality.
'''
https://pubmed.ncbi.nlm.nih.gov/9569655/
http://psych.fullerton.edu/rlippa/
% https://www.hawaiipublicschools.org/DOE%20Forms/CTE/RIASEC.pdf
how do i complete people-things from RIASEC?
% Gender-Related Individual Differences and the Structure of Vocational
% Interests: The Importance of the People-Things Dimension
% future interesting idea: marital status might be an interesting variable

% RIASEC is a personality type based on activity interest
% a 2-factor PCA from RIASEC adds importantly to OCEAN
% and aligns with a personality analysis using the People-Things and Ideas-Data dimensions
% then we notice "very large" gender differences on the people-things axis, which is orthoganal to OCEAN
% https://openpsychometrics.org/tests/RIASEC/
%
% I conducted factor analyses on the six
% ipsatized RIASEC scores for all participants as well as for men
% and women separately (principal-components analysis, extraction of two factors, with orthogonal varimax rotation). Because
% the two factors that emerged from all three factor analyses were
% quite similar, only the results for men and women combined are
% reported here.
% https://towardsdatascience.com/principal-component-analysis-pca-from-scratch-in-python-7f3e2a540c51
% https://web-a-ebscohost-com.mutex.gmu.edu/ehost/pdfviewer/pdfviewer?vid=1&sid=056dfdd6-c42b-46ae-b687-4938a191ff87%40sdc-v-sessmgr03
%
% python and riasec https://www.learnpythonwithrune.org/pandas-explore-datasets-by-visualization-exploring-the-holland-code-riasec-test/

TODO?
% 3. "5-Factor Conservatism" and Gender (could be it's own paper)
%     a. anti-innovation bias, or status quo bias; represents the gradualist aspect
%     b. antiforeign bias represents the nationalist aspect
% % new phrasing: https://www.econlib.org/archives/2006/03/framing_antifor.html
% % "I favor freer trade and migration with other nations"
%     c. regulatory measure for market orientation or fiscal conservatism.
%     d. prefer Christianity to religiosity for social conservatism a la United States
%         i. Agnostic or Atheist
%         ii. Spiritual or Theistic, No Specific Religion
%         iii. Religious, Not Christian
%         iv. Progressive Christian
%         v. Conservative or Evangelical Christian
%         vi. Other Christian
% % 190201-feb-survey-monkey has justification for non-preference of Christianity var
% % rebuttle to that justification: it's a poor construct
% % instead of

% Cultural effects include regional and ethnic effects.

% Non-cultural ideological effects include religiosity,
% christianity,
% favorability to regulation,
% favorability to AI (conservatism and anti-innovation bias proxy),
% and whether American education is important (nationalist / anti-foreign prox)


## the big battlecry
-> do not test any new theories, just outline them
-> this is a survey and a partial replication
    1. not a full replication
    2. if we had data to execute a full replication, why limit ourselves to the battlecry? why not test novel theories?
    3. the answer to 2 is: "paper brevity, paper clarity of thought / scope focus, and analysis and authoring time"
    4. however, we might still gather data to permit option 2, and yet execute the battlecry*****have a few papers in the tank

## gender fx theories
Big 3 Theories
1. differential personality theory (expected to relate to occupational preference)
    1. person-orientation vs object-orientation
    2. Grit+OCEAN cluster difference expected
2. differential risk tolerance / aversion
    a. differential norms orientation, ie preference for college
    b. differential favorability to non-conforming individuals [from skills paper survey https://papers.ssrn.com/sol3/papers.cfm?abstract_id=3829269]
    c. possibly endogenous to personality
3. differential education + work experience

minor theories (special mention territory)
4. formal social norms, ie public and private policy effects
    a. special attention to diversity hiring policies - i think there's external data
5. (informal) social norms theory (potentially out of scope)
    a. weak evidence from derived measures
        i. percent gender by industry; male or female stereotype theory? gender collapse?
        ii. men are more willing to work as a minority compared to females (minority employment as a risky choice + risk aversion)
        iii. do women say men have lower eq and men say women have higher eq? eq stereotyping norm
        iv. these once again seem to collapse into the Big 3
    b. stronger evidence would include a purpose-built questionnaire, rather than derived measures

Expected Endogenous Effects
4. differential skills
    a. expect to find it, but expect dominant endogeneity w/ personality theory, education, and work experience
    b. ie communication skill + teamwork + eq -> people oriented
5. covid favorability difference -> remote preferences -> expect dominant endogeneity
6. prestige valuation differences
    a. interesting result: are women harder to impress OR value different expression content than men (interview heuristic)

## antitheories
1. i don't think religiosity within-country (USA specifically) is important https://www.pewforum.org/2016/03/22/the-gender-gap-in-religion-around-the-world/
2. I believe credit by examination effects are captured by other measures of favorability to alternative credentials
3. veteran bootcamp disproportionately educate men - not really an antitheory just like "duh" and not worth measuring

***older stuff below


we have college student -> college-educated labor
what about alt ed -> alt ed labor?

course report 2014 - 2020 percent female bootcamp grads
https://www.coursereport.com/reports/2014-coding-bootcamp-outcomes-demographics-report

let's focus on coding bootcamps bc course report has the data, it's an important microcosm of alt creds, and IT wants women
(theoretically, minority gender demand commensurate w minority status and varies importantly by industry)

1. does enrollment track favorability?
2. does favorability track disparity?

What if we could combat the student debt crisis while at the same time enabling economic welfare, equity, and diversity?
Alternative credentials can accomplish all of the above in some cases, and many of the above in general.
A key unknown is whether alternative credentials drive diversity in general.
Research shows that students of alternative learning programs are disproportionately ethnic minorities, % this is sus tho...look into research
but such results constitute a superficial and unsatisfactory analysis of general diversity.
Employee diversity with respect to sex is the simplist possible diversity

https://www.giveagradago.com/news/2020/12/what-is-diversity-in-the-workplace/424
https://www.youtube.com/watch?v=3enoWw21j0Q
https://builtin.com/diversity-inclusion/types-of-diversity-in-the-workplace

---yes women are more risk averse
Byrnes JP, Miller DC, Schafer WD (1999) Gender differences in risk taking: A meta-analysis
https://onlinelibrary-wiley-com.mutex.gmu.edu/doi/pdfdirect/10.1111/joes.12069
Strong Evidence for Gender Differences in Risk Taking https://www.sciencedirect.com/science/article/abs/pii/S0167268111001521



---no women are not more risk averse
Financial Decision-Making: Are Women Really
More Risk-Averse? https://pubs-aeaweb-org.mutex.gmu.edu/doi/pdfplus/10.1257/aer.89.2.381
ARE WOMEN MORE RISK AVERSE? http://citeseerx.ist.psu.edu/viewdoc/summary?doi=10.1.1.500.1334
"Our findings suggest that when individuals have the same level of education irrespective of their
knowledge of finance, women are no more risk averse than men." (however
1. they suppose faculty have the same level of education and men outnumber women 11:3,
2. gender was significant among faculty (specifically when married)
3. the regression (table 5) has no education-gender interaction variable)
4. gender was significant in the pooled faculty test, but not in the split faculty test, indicating a sample size issue; the sign was positive across regs indicating
.../

above sides of the analytical debate are totally reconcilable imho as a yes with caveat


This paper is a metastudy with an eye toward out-of-sample prediction about the value of an early work strategy by gender.

1. we know there are personality differences with gender;
a. prosocial attitudes may lead to distaste for remote and technical learning
b. yet, openess may offset that
2. are women more risk averse? if so, that may reduce appetite for alternative credentials (directly and indirectly)
3. furtherkl


theoretic importance - gender is, in some sense, a post-discriminatory diversity target and a sucess story
theoretical question: how do we pick between wage parity and employment parity (that allegedly drives innovation through ideological diversity)?
(further, given preference differences, wage partity will likely and perhaps necessarily lead to employment differences)

Make a more interesting hook, but in boring terms:
1. this paper investigates the relationship between gender and alternative credentials
2. it's a metaanalysis and replication of prior papers
3. i look at attitudes but also utilization and impact
4. differing diversity from education and employer point of view;
women are the majority in college now but the minority in the work force generally,
but it varies importantly by industry
5. from an application standpoint - there are only so many standard sociological c
standard sociological controls: age, ethnicity, gender, education, income
6. sex presents one of the most straightforward diversity targets,
and it remains easy to achieve a high degree of success even when diversity targets are expanded to include non-binary gender identifcation.
in some sense it is the simplist of all diversity targets


less than a 1 percent expected change from 2000 to 2050? from 47 to 48? why? largely preferential not ability
https://www.bls.gov/opub/mlr/2002/05/art2full.pdf

labor force participation has been flat for 30 years:
https://www.dol.gov/agencies/wb/data/facts-over-time/women-in-the-labor-force#labor-force-participation-rate-of-women-by-age

so workforce gender discrimination has been a non-issue since 2000 and probably well before then

\section{Description of Data and Methodology}

TODO: create flag for basically 'not serious respondent' and operationalize as "gave the same answer for all skill questions, with two or less exceptions"




\section{Results}




\section{Conclusions}




\bibliography{./BibFile}

\end{document}
