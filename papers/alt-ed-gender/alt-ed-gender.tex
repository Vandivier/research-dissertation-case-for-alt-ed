% using Elseveir template per https://www.elsevier.com/authors/author-schemas/latex-instructions
\documentclass[review]{elsarticle}
\usepackage{lineno,hyperref}
\modulolinenumbers[5]
\bibliographystyle{elsarticle-num}

\usepackage{booktabs}
\usepackage{graphicx}
\graphicspath{{../alt-ed-survey/figures-and-tables}}
\usepackage{hyperref}
\usepackage{threeparttable}  
\usepackage{tikz}
\usetikzlibrary{calc,matrix}

\usepackage[capposition=top]{floatrow}

% ref: https://tex.stackexchange.com/questions/50747/options-for-appearance-of-links-in-hyperref
\hypersetup{
    breaklinks = true,   %%% not grammatical
    hidelinks = true,    %%% not grammatical
}

\begin{document}
\begin{frontmatter}

    \title{
        Gender and Alternative Credentials
    }

    \author[mymainaddress]{John Vandivier}
    \ead{john@afterecon.com}

    \begin{abstract}
        TODO
    \end{abstract}

    \begin{keyword}
        TODO %%% not grammatical
        \MSC[2010] TODO %%% not grammatical
    \end{keyword}

\end{frontmatter}

\pagebreak
\linenumbers

\section{Introduction}

What if we could combat the student debt crisis while at the same time enabling economic welfare, equity, and diversity?
Alternative credentials can accomplish all of the above in some cases, and many of the above in general.
A key unknown is whether alternative credentials drive diversity in general.
Research shows that students of alternative learning programs are disproportionately ethnic minorities, % this is sus tho...look into research
but such results constitute a superficial and unsatisfactory analysis of general diversity.
Employee diversity with respect to sex is the simplist possible diversity

https://www.giveagradago.com/news/2020/12/what-is-diversity-in-the-workplace/424
https://www.youtube.com/watch?v=3enoWw21j0Q
https://builtin.com/diversity-inclusion/types-of-diversity-in-the-workplace

---yes women are more risk averse
Byrnes JP, Miller DC, Schafer WD (1999) Gender differences in risk taking: A meta-analysis
https://onlinelibrary-wiley-com.mutex.gmu.edu/doi/pdfdirect/10.1111/joes.12069
Strong Evidence for Gender Differences in Risk Taking https://www.sciencedirect.com/science/article/abs/pii/S0167268111001521



---no women are not more risk averse
Financial Decision-Making: Are Women Really
More Risk-Averse? https://pubs-aeaweb-org.mutex.gmu.edu/doi/pdfplus/10.1257/aer.89.2.381
ARE WOMEN MORE RISK AVERSE? http://citeseerx.ist.psu.edu/viewdoc/summary?doi=10.1.1.500.1334
"Our findings suggest that when individuals have the same level of education irrespective of their
knowledge of finance, women are no more risk averse than men." (however
1. they suppose faculty have the same level of education and men outnumber women 11:3,
2. gender was significant among faculty (specifically when married)
3. the regression (table 5) has no education-gender interaction variable)
4. gender was significant in the pooled faculty test, but not in the split faculty test, indicating a sample size issue; the sign was positive across regs indicating
.../

above sides of the analytical debate are totally reconcilable imho as a yes with caveat


This paper is a metastudy with an eye toward out-of-sample prediction about the value of an early work strategy by gender.

1. we know there are personality differences with gender;
a. prosocial attitudes may lead to distaste for remote and technical learning
b. yet, openess may offset that
2. are women more risk averse? if so, that may reduce appetite for alternative credentials (directly and indirectly)
3. furtherkl


theoretic importance - gender is, in some sense, a post-discriminatory diversity target and a sucess story
theoretical question: how do we pick between wage parity and employment parity (that allegedly drives innovation through ideological diversity)?
(further, given preference differences, wage partity will likely and perhaps necessarily lead to employment differences)

Make a more interesting hook, but in boring terms:
1. this paper investigates the relationship between gender and alternative credentials
2. it's a metaanalysis and replication of prior papers
3. i look at attitudes but also utilization and impact
4. differing diversity from education and employer point of view;
women are the majority in college now but the minority in the work force generally,
but it varies importantly by industry
5. from an application standpoint - there are only so many standard sociological c
standard sociological controls: age, ethnicity, gender, education, income
6. sex presents one of the most straightforward diversity targets,
and it remains easy to achieve a high degree of success even when diversity targets are expanded to include non-binary gender identifcation.
in some sense it is the simplist of all diversity targets


less than a 1 percent expected change from 2000 to 2050? from 47 to 48? why? largely preferential not ability
https://www.bls.gov/opub/mlr/2002/05/art2full.pdf

labor force participation has been flat for 30 years:
https://www.dol.gov/agencies/wb/data/facts-over-time/women-in-the-labor-force#labor-force-participation-rate-of-women-by-age

so workforce gender discrimination has been a non-issue since 2000 and probably well before then

\section{Description of Data and Methodology}




\section{Results}




\section{Conclusions}




\bibliography{./BibFile}

\end{document}
