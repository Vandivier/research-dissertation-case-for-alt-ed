
\chapter[Hirability and Educational Prestige]{\large Hirability and Educational Prestige}

\section{Introduction}

The accredited degree is an established means to individual-level employability, but the proliferation of the degree is associated with a variety of well-understood issues.
These issues include the student debt crisis, skill gaps, grade inflation, and low social return.
% and contribution to lack of diversity in particular labor markets.
% above, or
% and systematic demographic change to industrial labor. (which many perceive as problematic for a diverse labor pool)
Alternative credentials, or non-accredited credentials, are a broad category of offerings that exhibit greater variation intensity, price, and outcomes\cite{urdan_2020}.
Alternative credentials are often a signal of niche skills and expertise in a particular job family.
These characteristics combine to provide the benefit of high possible value addition to the labor market
with the cost of a value calculation problem shared by potential employers and education consumers.

This paper seeks to reduce the general difficulty of credential value calculation by testing a method of value normalization
with heuristics to identify those credentials likely to yield meaningful benefits to the typical job search.
This paper tests the lens of prestige as a tool to normalize value across accredited and alternative credentials.
This study leverages an original questionnaire to identify prestige levels of various credentials.
This paper tests the composite hypothesis that some level of prestige allows an alternative credential to compete with traditional credentials for employment.

Several specific lines of evidence are required to support the composite hypothesis.
Statistical evidence must demonstrate significant positive effects for accreditation and prestige on hirability.
The effect size for prestige must be sufficiently large to dominate the accreditation effect over the attainable range.
The questionnaire allows a prestige response on a 10-point scale, so the attainable range is from 1 to 10.
A vignette analysis can test whether a dominant range for prestige exists within the attainable window.
An ideal result would further show that one or more actual alternative credentials fall into this dominant range.

The motivation for the lens of prestige extends from the academic work in education economics and the economics of social norms.
Education economics provides two mainstream accounts of the value of a degree.
One account is the human capital model, and the other is the signaling model.
The human capital model explains that improved labor outcomes result from skills gained by a student in the course of education.

Stakeholders of various kinds prefer alternative credentials to the traditional degree for the attainment of specific technical skills\cite{craig2018new}.
For this reason, many college graduates supplement using alternative credentials.
Some alternative learning providers specifically target this market with a special kind of alternative education called last-mile training.
This presents an explanatory problem for the human capital model.
If better labor outcomes arise from skill enhancement,
then alternatively educated individuals should enjoy better wages, employment rates, and so on,
compared to college graduates.

The signaling model holds that credentials signal a basket of applicant qualities that employers value.
Proponents of the signaling model commonly argue that the college degree signals intelligence,
work ethic, and conformity\cite{caplan_2012}.
The signaling model presents an explanation for
the correlation of weak labor outcomes and alternative credentials,
even if alternative credentials endow students with better skills.
The explanation is that the alternative credential signals
an offsetting deficit of some kind.
This paper treats prestige as a signal rather than a matter of human capital.
This paper prefers the signaling approach to directly investigate prestige effects
with minimal theoretical baggage and without a need to test student skill.
% without concern for testing skill differences in students of various kinds.
% This paper hypothesizes that employers value prestige as a signal.
% Accredited degrees 
% A full description of such signal deficits is irrelevant to this study,
% but potential negative signals may include a deficit in conformity or work ethic.
% This study uses the signaling model as a framework for investigation
% because it appears 

% This paper hypothesizes that prestige is valued by employers as a signal.
% Prestige can be taken as a signal of conformity in part.
% Google is a prestigious employer and also an alternative learning provider.
% From the point of view of Google, their own credential is a preferred conformity signal as well as a signal of skill.
% The case of employer-provided credentials is interesting,
% but the component of prestige is 
% The main argument is that accreditation signals prestige,
% but there
% The simple hypothesis is that accredited degrees obtain higher prestige on average,
% but 
% While conformity and prestige intersect at times,
% this paper does not suppose they are identical nor generally correlated.
% Instead, this paper argues that these are two social characteristics that are valued by employers
% and a lack in one may be compensated for by the presence of the other.
% This paper hypothesizes that alternative credentials will have low average prestige,
% but that some particular credentials from prestigious providers like Google will prove valuable.

In a broad review of economics and norm types, hiring decisions exist within what Elster would identify as work norms\cite{elster1989social}.
Elster supports a rational model of work norms, with the caveat that social interactions may involve unobserved emotional effects.
Similarly, the rational model used in this paper may not extrapolate with accuracy into abnormal emotional situations.
This paper will also make use of the distinction between social and legal norms provided by Elster.

Rivera is one scholar within the economics of work norms to have recently operationalized social norms as prestige\cite{rivera2016pedigree}.
Rivera finds that prestige is important in her analysis, but her analytical scope focuses on traditional education and a few specific industries, including health and law.
The current paper extends the analysis of prestige and hiring norms
across many industries and to include alternative credentials.
% arguably I emphasize coding bootcamps / other bootcamp industries / information technology industry and perhaps sales

% To preview results,, statistical evidence confirms that prestige independently explains hirability better than accreditation alone,
% but accreditation fails to be explained away.
% Instead, models that use both factors produce superior estimates of willingness to hire.
% The independent importance of accreditation indicates that asymptotic improvements to alternative credentials are unlikely to totally outcompete the traditional education system.
% The failure of arbitrary technical and social gains in alternative credentials to fully crowd out traditional education
% points to a need to investigate legal norms for further remedy.
% The conclusion describes industry and policy solutions for the remainder of concerns in higher education.
% At the same time, there is a large level of economic value/arbitrage to be had/exploited from the socialization of alternative credentials
% before the binding constraint of legal stability is encountered.
% TODO: maybe estimate the level of economic growth (it would be proportional utility growth which is merely analgous tho)

\section{Description of Data and Methodology}

This paper investigates an original set of online questionnaire responses ($n = 454$).
Responses are cross-sectional data obtained in March of 2021.
Respondents are United States citizens at or over the age of eighteen.
Qualified respondents participated in the survey through the Amazon Mechanical Turk platform.

Appendix A contains the wording and response options for each question.
Appendix A also contains the wording for a priming message presented at the start of the survey.
The priming message lays out the definition of alternative credentials used in this study.
The message also provides several concrete examples of alternative credentials,
including ``a Certified Project Manager certification,
a portfolio of work, a Khan Academy profile, or a Nanodegree from Udacity.''

The dependent variable of interest is called hirability.
This variable measures individual response on a 10-point scale to the question,
``For many professions, alternative credentials can qualify a person for an entry-level position.''
The questionnaire is composed of three sections.
The first section collects respondent characteristics and baseline hirability.
The second section collects prestige responses with respect to nine real-world learning providers.
The third section collects hirability and prestige responses with respect to eight vignette learning providers.

% TODO: maybe describe real-world learning provider selection criteria here.

Investigation of the first section of the questionnaire uses ordinary least squares analysis.
Vignette data is analyzed as a panel in mixed models with individual random effects.
The vignette model allows comparison between prestige and accreditation coefficients.
Vignette analysis encounters a practical utility problem in that the schools are only vignettes rather than actual learning providers.
A comparison of descriptive statistics across vignettes and actual schools addresses this concern.

Half of the respondents randomly received an informational message about the nine real-world learning providers.
Appendix A includes the wording of this message.
The message provides rating data from two leading credential aggregator websites.
University ratings are US News ranking information for the 2021 school year.
Course Report provides the rating data for so-called coding bootcamps as of December 2020.

As an aside, inaccurate credential category labels contribute to the knowledge and value calculation problems that inhibit social adoption.
Coding bootcamps focus on roles in the information technology industry, but these roles are much broader in scope than the category label implies.
Moreover, the information technology industry is a special industry that cuts across all other industries.
Much of the academic, policy, and industry discussion on coding bootcamps misses that these institutions provide credentials that potentially compete with university degrees in nearly any subject.

For example, General Assembly is one of the particular coding bootcamps investigated in this study.
General Assembly provides credentials for user experience design, a set of skills involving market research, and applied technical art skills.
General Assembly provides credentials for product management.
Product management is a job family that competes for labor among business degree graduates.
The data science credential provides skills that compete with accredited labor in mathematics, statistics, economics, and even subjects in the hard sciences like computational biology.
Finally, there are credentials that relate to software development and compete with accredited degrees in computer science.

% TODO: write the below content into the paper when we get to results on concrete providers
% why do we not use a mixed model of concrete providers?
% three reasons:
%   first, the whole point of using an aggregator is to obtain individual-independent data
%   second, ratings are heterogenous so there would be some noise introduced by indexing them together
%   third, i use stipulated prestige categories as a pseudo-index
%       the pseudo-index can be used via simpler descriptive stats,
%       and the pseudo-index has evidence supporting it; that is, high / low stipulated prestige is sig correlated to response prestige.
% if all those objections fail, then fine we can do it but i suspect the answer will be:
%   'findings are insignificant bc of low variation and injected noise not bc no such level exists'

Respondent characteristics are categorical variables.
Hirability and prestige are 10-point Likert-type responses.
Prestige takes a second representation as a stipulated boolean.
Stipulating prestige enables the application of results to a real job search.
If stipulated prestige is highly correlated to prestige response,
and if prestige response is correlated to improved hirability,
then the selection criteria for stipulated prestige can be applied in an actual job search to potentially improve outcomes.

To illustrate the method of two-way prestige validation,
suppose that a vignette school is stipulated as high prestige.
This situation is represented in regression as a dummy variable for stipulated high prestige with a value of true.
The respondent reads that the vignette school is known to be prestigious.
After reading this, the respondent provides a prestige response rating on a 10-point scale.
Investigation of all responses allows an analyst to determine an average prestige response level which is associated with
the stipulated high prestige criteria.

To preview results, stipulated high prestige turns out to be strongly correlated with high prestige response.
Interestingly, there are cases where a respondent gives a low response rating to, for example,
the University of Chicago, a school with high stipulated prestige based on aggregator website ratings.
This result indicates the importance of some analysis that accounts for individual effects.
% the person could just be generally stingy with hirability points, they might be particularly opposed to accredited degrees, have a special issue w this school...

Two-way representation of prestige enables the application of findings into an actual job search.
In an actual job search, individuals can easily access aggregator website data.
In the real world, an individual cannot readily access questionnaire results for many credentials.
Results from this paper include the identification of rules of thumb
that a person can use to identify actual learning providers as high prestige.
To ensure clarity of results, stipulated prestige always refers to the dummy variable, and prestige response refers to the 10-point measure.

The vignette section and the section on actual schools use stipulated prestige.
All other variables are either 10-point Likert-type responses or categorical variables\footnote{
    It is an accepted practice to treat Likert-type responses as either categorical or continuous for regression analysis.
    Jaccard and Wan provide support for continuous analysis of Likert-type data.
    They note that severe departures from the assumptions on cardinality ``do not seem to affect Type I and Type II errors dramatically,''
    particularly when the Likert scale is five or more points\cite{jaccard1996lisrel}.
    This paper treats responses on a 10-point scale as continuous.
}.
Categorical variables are exclusively respondent characteristics.
Four other respondent measures are Likert-type responses.
Vignette responses include responses for hirability and prestige,
while actual schools only receive responses for hirability.
% Prestige is measured two-ways in the vignette section, but it is only stipulated in the section on actual schools.

Respondent characteristics include eight standard controls and four questions unique to this study.
The eight standard controls include
age, gender, ethnicity, income,
level of education, employment status, the industry of occupation, and state of residence.
A unique question on work norms records whether the respondent tends ``to work more closely with coworkers at your company or customers and external business partners.''
The motivation for this question is to test whether prestige disproportionately impacts roles that are outward or client-facing.
Respondents are also directly asked whether they
``prefer to hire or work with a person that has a college degree rather a person that holds a reputable certification or non-college credential.''

Another unique control is support for online education.
This control allows analysis to separate hirability effects due to online education preference
from hirability effects due to unaccredited education preference.
In practice, many alternative credentials involve online learning,
but accredited learning is also increasingly taking place online.

The fourth control is expected conventionality.
% This is an important representation of perceived social norms.
This variable measures whether the respondent believes that
``It will soon become common for high school graduates to obtain alternative credentials instead of going to college.''
This is a useful correction variable for two reasons.
First, it separates willingness to hire based on respondent preference
from indirect willingness to hire based on perceived social norms.
Individual preferences and social norms are certainly correlated,
but the correlation is small enough that failure to separate these effects leads to nontrivial statistical noise.

Second, surveys sometimes overreport demand effects because of the lack of cost constraint on respondent expression.
This bias is sometimes called budget constraint bias or omitted budget constraint bias\cite{ahlheim1998contingent, pachali2020omitted}.
% This is also in part corrected for by collecting income...so there isn't an 'unobserved budget' really...see sources cited
Without a cost constraint, respondents tend to exaggerate demand responses like the willingness to hire.
Budget constraint bias affects both hirability and expected conventionality,
so conventionality operates in part as a bias control.

Vignette question formatting follows Atzm{\"u}ller and Steiner\cite{atzmuller2010experimental}.
Each vignette stipulates whether a school is accredited,
whether the respondent should imagine the school as impressive,
and whether the respondent should imagine that other people consider the school impressive.
Each stipulated factor can take a value of true or false,
resulting in eight vignette questions.

This study uses multiple regression and descriptive statistics to generate results.
% \footnote{
%     While the data for this analysis is not public, the analytical code is open-source.
%     See \url{https://github.com/Vandivier/research-dissertation-case-for-alt-ed/tree/master/papers/alt-ed-prestige}
% }.
Multiple regression is conducted using ordinary least squares (OLS)
for baseline hirability analysis
and linear mixed models (LMM)
are used for vignette analysis.
OLS specification of vignette data is inappropriate because repeated measures of hirability
from a single participant introduce an individual-level bias into resulting coefficients.
LMM models are able to account for these individual-level effects.
Following Magezi\cite{magezi2015linear},
linear mixed models in this paper use a within-participant random factor,
or individual random effects,
to correct for individual-level repeated measures bias.
LMM yields linear coefficients, so the interpretation of LMM coefficients is similar to OLS.
One difference of note is that adjusted r-squared is not available for an LMM model.
% For this reasons, an OLS model is optimized for baseline hirability,
% and then that specification is trivially modified into an LMM model for further analysis.
% formula for LMM at https://www.statsmodels.org/stable/mixed_linear.html
% individual random effects https://www.statsmodels.org/stable/examples/notebooks/generated/mixed_lm_example.html#Growth-curves-of-pigs

% 2. how was nonresponse bias addressed? - maybe not at all
% - main way to address nonresponse bias is to explicitly capture and correct for all of the individual characteristics that matter: ethnicity, age, income...
% - it would not be enough to show nonresponse bias exists;
% - it would need to be shown that it exists in the direction of some effect that moves the relation of interest in a predictable and meaningful way;
% - else the criticism is an argument from ignorance which due dilligence has been undertaken to preclude.
% - https://forum.effectivealtruism.org/posts/a6LMQcER6Awhawtqq/using-amazon-s-mechanical-turk-for-animal-advocacy-studies
% - above indicates overstatement of effects...i would want more info...there is a paper internally cited
% - above also deflates income nonresponse bias consern (these don't pay much so systematic bias from rich ppl) also i explicitly capture income anyway
% - "AMT was found to be a reliable source of data and to diminish the potential for non-response error in online research"
% - https://www.ncbi.nlm.nih.gov/pmc/articles/PMC4397064/
% - https://duckofminerva.com/2013/07/mechanical-turk-and-experiments-in-the-social-sciences.html
% - https://www.tandfonline.com/doi/abs/10.1080/10967494.2016.1276493
% 3. How were ratings subjects selected? min 2*2*2 (isQuality)*(isBootcamp)*(isKnown) social and individual ratings [10 point likert-type unit]
% 4. a few correction variables based on literature review and computed norm factors how

\section{Results}

% top line results; alt creds are lower prestige on average but still important
Results ($n = 454$) indicate that accredited degrees are generally higher in prestige compared to alternative credentials.
Alternative credentials are meaningfully associated with hirability,
and in certain situations, they are preferred to accredited degrees.

Competitive status indicates that a credential is correlated with hirability to a similar or greater extent compared to an accredited degree.
Results provide evidence for three cases in which alternative credentials are competitive.
First, specific alternative credentials are of particularly high prestige.
This study finds that a credential from Google is sufficiently prestigious to be competitive without a requirement of supplementary conditions.

Second, some individuals award prestige preferentially to alternative learning providers.
% analysis_5_transfer table a5.1
In a comparison among nine actual learning providers in this study,
71 percent of respondents prefer at least one alternative credential to at least one university degree.
The proportion increases to about 75 percent when respondents view rating data from the online review aggregators Course Report and US News.
% These sites include US News and Course Report, and they aggregate learning providers,
% report standard information about those providers,
% and allow users to leave reviews.

% TODO: maybe re-introduce the term 'alternativel credentialed non-college graduated' / ACNG.
Third, certain independent factors in hiring decision models support the hirability of alternatively credentialed job candidates.
Industry and state effects are two such compensating factors that can add up to overcome the average comparative preference for accredited labor to alternatively credentialed job candidates.
% For example, the state effect for California is positive on hirability
% and it retains a magnitude that compensates almost exactly for the hirability penalty from non-accreditation.

% summary statistics
% a2.1
Baseline hirability is the institution-agnostic hirability measure.
The mean response for baseline hirability is 7.58 on a 10-point scale, and the median response is 8.
Table \ref{tab:desc_stats} gives average hirability and prestige for interesting segments of respondents.
Four basic results in the table are worth noting.
First, stipulated prestige always moves with prestige response as expected.
Second, as expected, the hirability and prestige effects for accredited schools are generally higher than non-accredited schools.

Third, the difference in average hirability between high and low prestige providers
is more than twice the difference in hirability between accredited and unaccredited providers.
This supports the possibility of an actual competitive alternative credential in the attainable range of prestige.
% alternative education becomes competitive with traditional education.
% The fourth result is that the average actual school with stipulated high prestige
% is too low in prestige to be competitive with the average actual school with an accredited status.
The fourth result is an initial attempt at a prestige rule of thumb.
For both vignette and actual schools,
if a school can obtain a prestige score of 7 or more,
it will be at least as prestigious as the average accredited school.

\begin{table}
    \caption{Average Hirability and Prestige}
    \resizebox{\columnwidth}{!}{
        \input{./figures-and-tables/table-prestige-summary-stats.tex}
    }
    \label{tab:desc_stats}
\end{table}

% The differences reported in Table \ref{tab:desc_stats} are significant ($p < 0.1$).
% Smaller differences between actual schools and vignette schools are also significant.
% The minimum difference of 0.14 between unaccredited actual and vignette schools is significant ($p < 0.1$).

Google is the only unaccredited learning provider to achieve a competitive status on the basis of this initial rule.
The mean prestige response for Google was 7.10, and the median response was 7.
Two lower bars for competitive status are interesting.
First, an alternative provider can be described as moderately competitive if it fails to beat the average university,
but it succeeds in beating at least one university on average.
The lowest average prestige response for an accredited university is 6.34 for the University of Nebraska.

Second, an alternative provider can be described as weakly competitive if it fails to beat any university on average,
but it succeeds in beating at least one university in a significant percentage of individual responses.
No alternative credentials investigated in this study meet the criteria for moderate competitiveness.
App Academy, General Assembly, and Google are the three alternative learning providers with stipulated high prestige.
All stipulated high prestige learning providers are at least weakly competitive.

When asked directly, 41.6 percent of respondents indicated that they would not prefer to
work with a person that holds an accredited credential instead of ``a person that holds a reputable certification or non-college credential.''
When examining prestige response instead of asking directly, over 70 percent of respondents reveal a preference for
at least one actual alternative credential to at least one university credential.
Over half of respondents preferred at least one actual alternative credential with stipulated high prestige
to at least one university credential with stipulated high prestige.
After excluding Google, over one-quarter of respondents continue to prefer
at least one actual alternative credential with stipulated high prestige to at least one university credential with stipulated high prestige.

% TODO: Does zety belong in the conclusion?
Zety is an online platform that facilitates job search.
Zety reports that one in six job applicants in the United States receive an interview,
and the average conversion rate from interview to offer was 19.78 in 2016\cite{turczynski_2021}.
Assuming rejections are independent enables naive estimation that most job searches consist of at least four interviews\footnote{
    Four independent games that each include an eighty percent chance of rejection yields $0.8^4 = 0.4096$.
    The associated probability of having at least one offer result from four interviews would be about $1 - 0.41 = 0.59$,
    or 59 percent, which is more likely than not.
} and dozens of applications.
Given the rates at which respondents prefer alternative credentials to accredited degrees,
a job search of typical length is likely to include several applications and at least one interview
with one or more employers that would prefer an alternative credential with stipulated high prestige to an accredited degree.

% More than half of respondents prefer a high prestige alternative credential to at least one high prestige accredited degree.
% After excluding the highest prestige alternative credential from Google,
% more than one-quarter of respondents still prefer one of the remaining high prestige alternative credentials to at least one high prestige accredited degree.
% % a2.2
% When asked directly, about 42 percent of respondents state that they do not prefer
% to work with a person that has a college degree rather than a person that holds a reputable non-college credential.

{
\def\sym#1{\ifmmode^{#1}\else\(^{#1}\)\fi}
\begin{center}
    {
        % \fontsize{8pt}{7pt}\selectfont
        \tabcolsep=2pt
        \begin{longtable}{l*{3}{c}}
            \caption{Table of Regression Results}
            \label{tab:table_regs}                                                                                                       \\

            \toprule
                                               & \multicolumn{1}{c}{Model 1} & \multicolumn{1}{c}{Model 2} & \multicolumn{1}{c}{Model 3} \\
            \midrule
            \endfirsthead

            \multicolumn{4}{c}%
            {{\bfseries \tablename\ \thetable{} -- Continued}}                                                                           \\
            \addlinespace
            \toprule
                                               & \multicolumn{1}{c}{Model 1} & \multicolumn{1}{c}{Model 2} & \multicolumn{1}{c}{Model 3} \\
            \midrule
            \endhead

            \addlinespace
            \hline
            \multicolumn{4}{|c|}{{Continued on Next Page}}                                                                               \\
            \hline
            \endfoot

            \hline \hline
            \endlastfoot

            % \hline
            %                                    & Model 1 & Model 2  & Model 3  \\
            % \hline
            Age, 45-60                         & 0.61***                     & 0.10                        &                             \\
            % \addlinespace
            External Facing, High              & 1.23***                     & 0.13                        &                             \\
            External Facing, Low               & 1.16***                     & 0.10                        &                             \\
            External Facing, Medium            & 1.16***                     & 0.13                        &                             \\
            Expected Conventionality           & 0.32***                     & 0.14***                     & 0.17***                     \\
            Income, 0-9999                     & 0.88                        & -0.87**                     & -1.22***                    \\
            Income, 100,000-124,999            & 1.25***                     & 0.47**                      & 0.41*                       \\
            Income, 175,000-199,999            & 1.58*                       & 0.40                        &                             \\
            Income, 200,000+                   & 1.14                        & -1.09*                      &                             \\
            Income, 25,000-49,999              & 0.57**                      & 0.19                        &                             \\
            Income, 50000-74999                & 0.51**                      & 0.26*                       & 0.18                        \\
            Income, 75000-99999                & 0.81***                     & 0.29*                       &                             \\
            Industry, Education                & 0.66**                      & 0.40**                      &                             \\
            Industry, Finance                  & 0.34                        & -0.07                       &                             \\
            Industry, Information Technology   & 0.46**                      & 0.05                        &                             \\
            Industry, Manufacturing            & 0.34                        & 0.17                        &                             \\
            Industry, Other                    & 0.37                        & 0.37**                      &                             \\
            Is Accredited                      &                             & 1.23***                     & 1.27***                     \\
            (Is Accredited)(Prestige Response) &                             & -0.09***                    & -0.10***                    \\
            Is Stipulated High Prestige        &                             &                             & 0.14**                      \\
            Is Stipulated Other Impressed      &                             & 0.64***                     & 0.59***                     \\
            Is Stipulated Self Impressed       &                             & -0.05                       &                             \\
            Online Ed Favorability             & 0.34***                     & 0.09***                     & 0.07**                      \\
            Prefers Traditional Coworker       & -0.22                       & 0.19*                       & 0.19*                       \\
            Prestige Response                  &                             & 0.55***                     & 0.53***                     \\
            State, Arizona                     & 1.35**                      & 0.69**                      &                             \\
            State, California                  & 0.44**                      & 0.27**                      & 0.37**                      \\
            State, Connecticut                 & 0.72                        & -0.11                       &                             \\
            State, Florida                     & 0.79***                     & 0.16                        &                             \\
            State, Georgia                     & -0.88*                      & -0.22                       &                             \\
            State, Kansas                      & 1.76                        & 0.52                        &                             \\
            State, Maryland                    & 0.92**                      & 0.31                        &                             \\
            State, Massachusetts               & 1.43**                      & 0.49                        &                             \\
            State, Michigan                    & 1.35***                     & 0.26                        &                             \\
            State, Mississippi                 & 1.77***                     & 0.45                        &                             \\
            State, Missouri                    & 0.81*                       & 0.34                        &                             \\
            State, Nebraska                    & -1.04                       & -0.75                       &                             \\
            State, New Mexico                  & 1.76*                       & 0.10                        &                             \\
            State, Pennsylvania                & 0.44                        & 0.44**                      &                             \\
            State, Tennessee                   & 0.74                        & -0.13                       &                             \\
            State, Texas                       & 0.39                        & -0.10                       &                             \\
            State, West Virginia               & -1.31                       & -0.92                       &                             \\
            Intercept                          & 0.30                        & 0.14                        & 0.50*                       \\
            % \hline
            \midrule
            R-squared                          & 0.47                        &                             &                             \\
            R-squared Adj.                     & 0.42                        &                             &                             \\
            N                                  & 454                         & 3600                        & 3600                        \\
            Measures Per Respondent            & 1                           & 8                           & 8                           \\
            % \addlinespace
            % Constant                         & 5.036\sym{***}        & 5.356\sym{***}        & 4.755\sym{***}     \\
            % \midrule
            % Adjusted R-sqr                   & 0.2181                & 0.2512                & 0.2331             \\
            % R-sqr                            & 0.3253                & 0.3539                & 0.3310             \\
            % p(F)                             & 0.0000                & 0.0000                & 0.0000             \\
            \hline
            % \addlinespace
            \multicolumn{4}{l}{\footnotesize \sym{*} \(p<0.10\), \sym{**} \(p<0.05\), \sym{***} \(p<.01\)}                               \\
        \end{longtable}
    }
\end{center}
}


Table \ref{tab:table_regs} gives three models.
The first model is an ordinary least squares model of baseline hirability.
Backward elimination to the point of adjusted r-squared maximization yields Model 1.
Adding factors of accreditation and prestige to Model 1,
then adapting the model to a linear mixed model (LMM) yields Model 2.
Model 3 results from additional backward elimination on Model 2.

Four individuals that completed the first section of the questionnaire
did not complete the entire questionnaire.
The remaining 450 respondents each report hirability for the eight vignette schools,
yielding 3,600 observations for the mixed models.

Because LMM does not permit computation of r-squared,
the termination criteria for the factor elimination process in Model 3
was to retain all factors with a p-value under 0.5.
This is a permissive criterion intended to guard against overfitting.
The logical basis for this rule is that each observed effect is
more likely to exist than to not exist when $p < 0.5$.
Despite permissive criteria, only one insignificant factor for income exists in Model 3.

% Comparison of coefficients across specifications improves confidence in the coefficients in all but two cases.
% The factor for the income range under ten thousand dollars per year
% and the factor for preference in a traditional coworker flip signs,
% but other factors are fairly consistent in their effects.

Model 2 and Model 3 have one other interesting difference.
Model 3 includes the boolean for whether a school was stipulated as high prestige.
For vignette schools with high prestige, the participant viewed two statements about the vignette.
The questionnaire instructs the participant to imagine a school they consider to be impressive.
The questionnaire also instructs the participant to imagine that other people consider the school to be impressive.
This situation is technically equivalent to an interaction of the two subcomponents.
Because Model 2 includes both stipulated high prestige subcomponents and the accreditation dummy, including high prestige generates perfect multicollinearity.
Backward elimination of Model 2 drops the factor for own stipulated prestige, so subsequent insertion of high prestige is nonproblematic.

Model 3 is the preferred model.
Prestige and accreditation effects are positive and significant.
These two effects also interact with a significant and negative coefficient.
The values of these coefficients of interest are consistent across Model 2 and Model 3.
The dummy variable for accreditation is about two and a half times larger than the prestige response,
but the average prestige response is near seven.
This indicates that the prestige response explains a larger share of hirability variance compared to accreditation.
% The negative interaction indicates both a decreasing marginal benefit for prestige among the accredited,
% and also a decreasing marginal penalty for prestige among the unaccredited.

An application of Model 3 is another approach to the identification of competitive alternative credentials.
Hold factors other than accreditation and prestige constant.
Let the hirability level of school $k$ be called $H_k$.
Let $X_{ka}$ be accreditation status,
$X_{kp}$ is prestige response,
$X_{kh}$ is the dummy for stipulated high prestige,
and $X_{ko}$ is the dummy for whether other people consider the school prestigious.

Let $H_1$ be an unaccredited school with high stipulated prestige.
Let $H_2$ be an accredited school without high stipulated prestige.
% Let $H_2$ receive a prestige response equal to the average for an accredited vignette.
Let $X_{2p} = 6.49$, which is the prestige response equal to the average for an accredited vignette,
as reported in Table \ref{tab:desc_stats}.
This system of equations is described in equations \ref{eq1} through \ref{eq5}:

\begin{subequations}
    \begin{equation}
        H_k = 1.27X_{ka} - \num{0.1}X_{ka}X_{kp} + 0.53X_{kp} + 0.14X_{kh} + 0.59X_{ko}
        \label{eq1}
    \end{equation}
    \begin{equation}
        H_1 = 0.53X_{kp} + 0.14 + 0.59
        \label{eq2}
    \end{equation}
    \begin{equation}
        H_2 = 1.27 - \num{0.1}(6.49) + 0.53(6.49)
        \label{eq3}
    \end{equation}
    \begin{equation}
        X_{kp} = (1.27 - \num{0.1}(6.49) + 0.53(6.49) - 0.14 - 0.59) / 0.53
        \label{eq4}
    \end{equation}
    \begin{equation}
        X_{kp} \approx 6.28
        \label{eq5}
    \end{equation}
\end{subequations}

Equation \ref{eq5} indicates that an alternative credential
with stipulated high prestige
and a prestige response of 6.28 or higher is approximately competitive with the average accredited vignette.
Table \ref{tab:desc_stats} indicates that the prestige response for the average vignette school is 6.21.
This is a significant difference compared to the average actual school prestige response of 6.50.
Coincidentally, additive and proportional compensation of 6.28 both yield 6.57.

This prestige requirement exceeds the low bar set by comparison to the University of Nebraska.
Google remains the only alternative provider to obtain general competitive status without the presence of other preferential factors.
App Academy and General Assembly both have average prestige responses close to 5.8.
Models reveal several situations in which other factors overcome this deficit,
but many of these offsetting factors are difficult to determine and leverage prior to a hiring decision.
The California state effect is an interesting exception that an actual job search could exploit.

Alternative credentials provide a source of potential diverse labor to employers.
Interestingly, neither ethnicity nor gender was significantly associated with hirability.
There is little evidence for the thesis that client-facing roles preferentially benefit from credential prestige or accreditation.
Respondent client exposure on the job was associated with a slightly larger baseline willingness to hire an alternatively educated candidate.
The extent of client contact was insignificant in mixed models.

\begin{figure}[h!]
    \centering
    \caption{Prestige Response Distribution for Actual Schools}
    \begin{tikzpicture}[element/.style={minimum width=1.75cm, minimum height=0.85cm}]
        \node (n1) {\includegraphics[width=1\textwidth]{./figures-and-tables/context-graph-massaged.png}};
    \end{tikzpicture}
    \label{fig:var_results}
\end{figure}

% This effect does not enter in to any regression
Finally, Figure \ref{fig:var_results} visualizes the prestige response distribution
for actual schools.
The four subplots describe whether a respondent randomly received information from review site aggregators
and how they evaluated credential accreditation.
Exposure to aggregated review information is associated with fewer responses at the positive
and negative extrema of the response distribution for accredited and unaccredited schools.
On average, alternative education prestige rose,
and accredited education prestige declined when a respondent received review aggregator site information.

\section{Conclusions}

This study hypothesized that some level of prestige allows an alternative credential to compete with traditional credentials for employment.
Results provide evidence in favor of this hypothesis.
Regression results show meaningful positive correlations of prestige and accreditation on hirability.
A range of hirability responses that include the average response and some below-average responses
find a dominant explanation in prestige effects over accreditation alone.

While prestige explains a larger share of hirability variance than accreditation, accreditation robustly maintains a meaningful effect on its own.
The robust importance of accreditation indicates that arbitrary improvements to alternative credential
quality and social acceptability are not likely to displace the higher education system in expectation.
This study began with the assertion that alternative credentials are a source of unexploited technical value.
The study validated a partial explanation from prestige as a representation of social norms.
The introduction noted an important distinction between legal and social norms from Elster.
By elimination, legal norm change is an important candidate to allow alternative credentials the opportunity
to fully outcompete the hirability effects of accreditation.

% While prestige explains a larger share of hirability variance than accreditation,
% accreditation robustly maintains a meaningful effect on its own.
% The robust importance of accreditation indicates that arbitrary improvements to alternative credential
% quality and social acceptability are not likely to displace the higher education system in expectation.
% % quality and social acceptability will not displace the higher education system in expectation.
% % quality and social acceptability will not displace the higher education system in expectation.
% Synthesis of the distinction from Elster between social and legal norms with the results of this study
% point to a need for legal norm change in order to create a competitive environment between traditional and alternative providers.
% % A change in legal norms appears to be required in order to create an even competitive environment between traditional and alternative providers.

In 2012, The Heritage Foundation called for two policy changes that are worth considering.
First, the Foundation proposed that the government should directly accredit courses rather than organizations\cite{burke2012accreditation}.
Second, they also called for a decoupling of accreditation and federal funding.
An additional option would be to replace legal requirements for formal education could be replaced with skill assessments.
With a legal requirement that prefers skills to degrees,
the public sector gains the ability to transfer formal accreditation duties to a market model with no loss of labor quality control.

There are several reasons to be pessimistic about the feasibility of these policy changes.
Reductions to education spending are unpopular with voters in the United States.
% Education is compulsory in the United States.
Over ninety percent of K-12 students in the United States attended a public school in 2016\cite{us2019digest},
and there is a systematized pipeline from public school to the traditional university system.
Education represents an example of an entangled political economy\cite{wagner2014entangled}.
Robust political economy points out additional reasons to doubt rapid innovation in this space\cite{boettke2004liberalism}.
Reduced political entanglement is associated with the absence of compulsory education.
However, after they exist, the elimination of compulsory laws also appears intractable.
The removal of compulsory education is a qualitative change that does not appear any less subject to the path dependency,
lock-in, ratchet, and other effects that inhibit contraction in the quantitative process of appropriations.
% Ratchet Effect
% One way to weaken the degree of political entanglement entanglement would be the absence of compulsory education laws,

An interesting alternative to formal legislative change is the emerging model of public-private partnerships in education.
In 2013, Georgia Tech formally partnered with Udacity to produce an accredited online graduate degree in Computer Science\cite{empson_2013}.
Udacity was able to facilitate an improved online learning experience at scale with an affordable price.
Georgia Tech offered branding, legitimacy, and accreditation, which supported a higher price point compared to the other offerings from Udacity.

In other cases, the hybridization of traditional and alternative education is indirect and informal.
Prior learning assessments and portfolio reviews are two of many processes by which a university can award credit to a student
without formal requirements connected to the source of student learning\cite{conrad2008building}.
University support for prior learning is an implementation pattern for course-level accreditation that does not require legislative action.
Formal and informal partnerships between traditional and alternative institutions can yield increased market surplus for producers and consumers.

Finally, this paper evaluated practical alternative credential selection strategies.
One strategy is to leverage credentials from industry leaders.
In this study, Google represented an alternative learning provider that is also an industry leader.
Fortune 50 membership is a rule of thumb used in this study to select an industry-leading firm.
A credential from Google was the only alternative credential to be identified as generally competitive with an accredited degree.

The second strategy is to use credential review aggregator sites to identify high prestige credentials.
This paper used Course Report as an aggregator to search for alternative credentials.
App Academy and General Assembly were identified by applying search criteria that include a rating of 4.25 or better on a 5-point scale and a minimum of four hundred reviews.
The combination of results with information on typical job search length from Zety indicated
that these credentials provide meaningful job search benefits,
albeit with significantly less efficacy than an accredited degree or a credential from Google.

