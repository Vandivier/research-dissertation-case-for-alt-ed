% using Elseveir template per https://www.elsevier.com/authors/author-schemas/latex-instructions
\documentclass[review]{elsarticle}
\usepackage{lineno,hyperref}
\modulolinenumbers[5]
\bibliographystyle{elsarticle-num}

\usepackage{booktabs}
\usepackage{graphicx}
\graphicspath{{../alt-ed-survey/figures-and-tables}}
\usepackage{hyperref}
\usepackage{threeparttable}  
\usepackage{tikz}
\usetikzlibrary{calc,matrix}

\begin{document}
\begin{frontmatter}

    \title{
        COVID-19 and Alternative Postsecondary Learning
        % \tnoteref{titlenotes}
    }
    % \tnotetext[titlenotes]{
    %     Go to \url{https://github.com/Vandivier/research-dissertation-case-for-alt-ed/tree/master/papers/alt-ed-survey}
    %     for additional materials including the online appendix,
    %     survey data, and data analysis source code.
    % }

    \author[mymainaddress]{John Vandivier}
    \address[mymainaddress]{4400 University Dr, Fairfax, VA 22030}
    \ead{jvandivi@masonlive.gmu.edu}
    % \fntext[authorlinefootnote]{
    %     Vandivier: George Mason University,
    %     4400 University Dr, Fairfax, VA 22030,
    %     jvandivi@masonlive.gmu.edu.
    % }

    \begin{abstract}
        As a result of the coronavirus pandemic,
        many individuals now engage in remote activities that they would otherwise not.
        While other research has assessed the impact of coronavirus on K-12 education,
        this paper fills a gap in the literature regarding
        the impact to professional certifications and other unaccredited postsecondary credentials.
        This paper investigates the results of an original online questionnaire (n=350)
        to understand the effects of COVID-19
        on support for alternative postsecondary learning.
        Respondents are U.S. citizens over the age of 18.
        Cross-sectional analysis using ordinary least squares (OLS) indicates that
        individual perception of a large negative impact from coronavirus
        is significantly correlated with
        higher favorability to alternative credentials.
        Some important industrial, ethnic, and state effects are also identified.
        Independent factors including the level of education, income, age, and gender were identified as insignificant.
    \end{abstract}

    \begin{keyword}
        education economics, alternative education, coronavirus             %%% not grammatical
        \MSC[2010] I21, I22, J20 % Unused at AEL: D12, J23, I24, I25, I26   %%% not grammatical
    \end{keyword}

\end{frontmatter}

\pagebreak
\linenumbers

\section{Introduction}

This study is concerned with postsecondary alternative credentials.
This category includes professional certifications, coding bootcamps, portfolios of work, and other proof of education other than traditional credentials.
Traditional credentials include a high school diploma or an accredited degree from a university.
Many alternative credentials are obtained as a result of education that involves a substantial component of remote learning.
This study tests the hypothesis that the impact of coronavirus is associated with increased favorability toward alternative credentials.
% This study also examines whether coronavirus-induced remote activity is associated with favorability to alternative credentials.
Results favor the hypothesis, with a few interesting caveats.

There are three theoretical reasons to suppose that a pandemic would make alternative postsecondary credentials more attractive.
The first is that a pandemic would induce increased remote learning.
This causes exposure, and exposure effects are generally positive on favorability.
Secondly, alternative learning providers tend to be smaller organizations with increased market exposure and less regulation
compared to the traditional postsecondary provider of education, the accredited university.
As a result, alternative learning providers will adapt more quickly during a pandemic.
Quicker adaptation will lead to a higher quality product at a lower price, and consumers are then expected to improve in favorability accordingly.

The third theory is that a pandemic is a time when normal strangeness increases across the economy.
This would theoretically reduce any relative stigma which the alternatively educated invididual might face.
The stigma is reduced by two mechanisms.
First, society changes many norms during a pandemic, so the general strength of social norms is reduced.
In this way, the general preference for traditional over alternative education is reduced.
Secondly, consuming alternative education is specifically reasonable as a pandemic adaptation.
During a pandemic, many of the attractions of university life are unusable,
remote learning provides protection against becoming ill,
and pinching pennies is more understandable than it is during times of flourishing.
% When an individual has such good reasons for pursuing alternative education, it becomes unreasonable to impose a stigma at hiring decision time.
% TODO: the above is related to mitigation tactics for disabled people during interview; stigma management...idk the exact keywords look it up

% The third reason that a pandemic could make a massive shift to alternative learning more reasonable
% is simply that massive shifts are less spectacular in a time of pandemic.
% Society has already changed many norms in the face of the pandemic.
% Some of these norm changes are directly linked while others are indirect.
% Social distancing, wearing masks, fewer shaken hands, and so on, are some direct measure.
% The lack of fresh bread at my local coffee shop is an indirect change resulting from supply chain cost impact of the pandemic.
% Both are understandable changes.
% In this pandemic context an empathetic employer would look at an alternatively educated individual as making a reasonable adaptation,
% whereas previously they might face some stigma for making a strange choice.
% In a time where many norms are being disrupted, there may be an external effect which weakens norms in general,
% and therefore the general preference of traditional to alternative education would.

While exposure to some stigma generally increases favorability, there are several special cases where it declines instead.
It could be the case that coronavirus-induced remote learning and work are an exceptional case where decline is observed.
This paper is valuable in part because it replaces this ambiguity with empirical evidence.
Pandemic exposure is an interesting case of exposure because it is a harmful, unwanted, or forced exposure.
It is not clear that coronavirus-induced exposure to remote learning and work will result in
Remote work and learning which are taken as a result of the pandemic might be perceived negatively by association,
or they might be perceived as a useful option that becomes more valuable in light of an unfortunate pandemic.
% TODO: cite these biases
Familiarity bias and mere-exposure effects are generally positive on favorability to some stimulus,
but these exposures are generally voluntary. Unwanted exposure can generate positive or negative favorability changes,
and when the exposure involves harm it tends to reduce favorability.
Backfire, boomerang, and blowback effects are some examples of negative favorability reactions to exposure.
One study hits remarkably close to COVID-19 pandemic relevance because it found a backfire effect in efforts to market flu vaccine usage,
unintentionally resulting in reduced willingness to take the flu vaccine [1].
Even repeated unwanted exposure to harm can eventually lead to positive favorability through as documented in the work on Stockholm syndrome.

The actual effect of coronavirus could mimick a combination of the above effects.
As a result, it is not obvious a priori whether the actual effect is positive, negative, or insignificant,
and this ambiguity calls for empirical investigation.
Individual favorability to alternative credentials is also like to vary for a variety of personal reasons which are unrelated to the pandemic.
This paper uses multiple regression to identify the net effect of coronavirus on favorability while holding constant these other sources of variation.

% [1] https://www.theatlantic.com/health/archive/2014/12/vaccine-myth-busting-can-backfire/383700/
% https://pubmed.ncbi.nlm.nih.gov/15974346/
% https://en.wikipedia.org/wiki/Belief_perseverance

There are already several papers examining the impact of coronavirus on the education system.
These papers focus on education from kindergarten through high school,
but it is reasonable to expect postsecondary education to be impacted in a similar way.

% TODO: cite four papers and show how they relate

\section{Description of Data and Methodology}

This paper uses an original set of response data (n = 350) obtained through the administration of an online questionnaire.
This cross-sectional data was obtained at the beginning of February in 2021.
Respondents are United States citizens at or over the age of eighteen.
The Amazon Mechanical Turk platform was used to recruit qualified participants.

% TODO: footnote that the python code for analysis and visualization is open source, although the data is confidential as requested by a relevant regulatory body.
Responses are investigated using least squares multiple regression.
Appendix A contains the exact wording and response options for each question.
Appendix A also contains the wording for a priming message presented at the start of the survey.
This message lays out the definition of alternative credentials for the purposes of this study.
The message also provides several concrete examples, including "a Certified Project Manager certification,
a portfolio of work, a Khan Academy profile, or a Nanodegree from Udacity."

The questionnaire is composed of fourteen questions.
There is one for the dependent variable of interest, favorability,
one for the independent variable of interest, coronavirus impact,
ten control factors,
and two questions on causality.

The variables of interest,
causality questions,
and two control variables are Likert-type responses.
% TODO: which are continuous vs categorical
\footnote{
    It is an accepted practice to treat Likert-type responses as either categorical or continuous for regression analysis.
    Jaccard and Wan provide support for continuous analysis of Likert-type data.
    They note that severe departures from the assumptions on cardinality "do not seem to affect Type I and Type II errors dramatically,"
    particularly when the Likert scale is five or more points\cite{jaccard1996lisrel}.
    This paper treats responses on a 10-point scale as continuous.
    This paper treats responses on a 4-point scale as categorical.
}.

Eight of the ten control factors are categorical measures for
for age, gender, ethnicity, income,
level of education, employment status, industry of occupation, and state of residence.

% prior work shows this is a comparatively good innovation bias proxy
Favorability to artificial intelligence technology is observed.
This is interpreted as a measure of innovation bias.
Innovation bias is interpreted as isomorphic to inverse status quo bias.

Two other important right-hand variables exist.
Respondents are asked whether they have heard of five popular alternative learning providers.
Familiarity is the count of confirmed known providers.
The expectation variable is a response from 1 to 10 to the question,
"It will soon become fairly conventional for high school graduates to obtain alternative credentials instead of going to college."

% The analytical strategy of this analysis is to address the hypothesis through the partialling out of certain effects.
%
% We know that personality and political ideology explain each other,
% so including for personality should partial out some or all of the political ideological effect.
% A surprising result is that the coefficient on regulatory favorability increases when we do this,
% and this is consistent with early regressions that corrected for more ideological variables and found an even
% higher coefficient. It seems the effect is neither a matter of reactionary response, personality, nor political ideology.
% What we seem to be left with is either:
% 1) It's a non-political ideological response,
% or 2) It's a rational response, even if the rationale follows some unobserved logical structure

\section{Results}

% TODO: results:
% 1. what is effect of covid impact?
% 2. what is the average effect of covid impact?
% 3. what is effect of other two covid vars?
% 4. how do interpret counterintuitive covid_ind_fav_online?
% 5. what is the average total effect across all three vars?
% 6. what is the average favorability?
% 7. what about skew and kurtosis and robustness? reg of interest 3 and 4

Table \ref{tab:table_new_ols} provides selected coefficients across four models of interest.
Selected variables emphasize representation of each category of effect,
significant effects,
and variables shared across models.
M-2018 and M-2019 are reprinted with permission from a prior paper.
These baseline models use data from the Attitudinal Survey on Alternative Credentials\cite{vandivier2020preliminary}.
M-2019-2 is a replication of M-2019 using new data obtained for the present research.

M-2019-2 involves a larger sample size compared to M-2019.
Coefficient significance and direction of effect is replicated.
Coefficient magnitude varies with a general lack of importance.
The coefficient on being a college graduate changes notably, but it is not a significant factor.

M-2020 introduces factors of personality.
The effect of mental effort is insignificant.
Being a college graduate is a significant and important factor in this specification.
Including factors of personality improves total and adjusted explanatory power by about 5 percent.
Grit, conscientiousness, and openness were important in the model.
These factors were significant at the $p<.18$ level.

Partialling out personality modifies the pro-regulatory effect in a direction opposite expectation.
The coefficient on regulatory favorability is amplified, rather than attenuated.
The result is significant and falls within the range of prior estimates.
Evidence does not indicate that conservative opposition to alternative credentials is an effect of personality or constrained mental effort.
The general concept of conservatism does seem to apply to the problem.
The social category of conservatism also does not seem to provide an explanation.
Correction for religiosity in 2018 shows that social conservatives tend to support alternative education.

\begin{table}
    \caption{Table of Multiple Regression on Enrollment, Selected Variables}
    \resizebox{\columnwidth}{!}{
        {
\def\sym#1{\ifmmode^{#1}\else\(^{#1}\)\fi}
% \begin{center}
{
    \fontsize{8pt}{7pt}\selectfont
    % \begin{small}
    \tabcolsep=3pt
    \begin{tabular}{l*{4}{c}}
        \toprule
        \multicolumn{1}{c}{Effect Group} & \multicolumn{1}{c}{Adj R-Sqr} & \multicolumn{1}{c}{R-Sqr} & \multicolumn{1}{c}{Max p-value} \\
        \midrule
        Absolute Gap                     & 0.0615                        & 0.0703                    & 0.097                           \\
        \addlinespace
        Comparative Gap                  & 0.0176                        & 0.0298                    & 0.687                           \\
        % \addlinespace
        % Rulebreaker                           & 0.1432                        & 0.1554                    & 0.053                           \\
        \addlinespace
        Industry                         & 0.0303                        & 0.0454                    & 0.958                           \\
        \addlinespace
        \addlinespace
        Other Factors                    & 0.0072                        & 0.0288                    & 0.537                           \\
        \addlinespace
        Rulebreaker                      & 0.0783                        & 0.0869                    & 0.127                           \\
        \addlinespace
        State                            & 0.0469                        & 0.1033                    & 0.772                           \\
        % \addlinespace
        % State, Semi-Robust                    & 0.0034                        & 0.0648                    & 0.831                           \\
        \bottomrule
    \end{tabular}
    % \end{center}
}
}

% TODO: maybe a count of k factors in group
% TODO: maybe distinguish strong and weak effects for industry, state, and gaps
% TODO: maybe other controls / other factors section doesn't matter
% TODO: maybe combine skill gaps

    }
    \label{tab:table_new_ols}
\end{table}

The most significant personality factor is an interaction between grit and familiarity ($p<0.005$).
If this factor is replaced with simple grit,
the negative direction of effect is maintained, but the significance is reduced ($p<0.17$).
% Marginal familiarity effects also become insignificant and exhibit an unexpected negative sign in that case.
Notice that the negative direction of effect is opposite in sign when compared with conscientiousness.

Familiarity bias is usually associated with favorability.
This bias is reproduced in work on alternative credentials,
but the favorability response is heterogeneous by personality.
Specifically, concurrently higher grit and familiarity yield lower favorability to alternative credentials.
The interaction effect is not reproduced when conscientiousness is substituted for grit.
% I really don't have any good intuition about why the interaction is so effective
% my guesses seem invalidated; that this could reflect lower IQ, online educ preference,
% or higher traditional ability eg less need of alternatives
% those are invalidated by testing inclusion of:
% p*invalid, n*online*1, ceduc*1, 

% Duckworth notes that grit includes a facet involving passion towards a long term goal and a facet involving reseliancy\cite{perkins2013significance}.
A well-cited meta-study in 2017 interpreted grit as a new label for conscientiousness\cite{schmidt2018same}.
That paper found that grit was strongly related to conscientiousness ($\rho = 0.84$).
The consistency facet of grit showed greater independence ($\rho = 0.61$).
The strong correlation between conscientiousness and grit is replicated in the present data ($\rho = 0.73$).
The present data also shows a strong correlation between grit and neuroticism ($\rho = -0.66$).

\section{Conclusions}

% TODO: conclusion:
% 1. it's a bit speculative, but do we think this bump is transient or permanent? why?
% 2. how does this relate to covid impact to school choice results?
% 3. what open questions remain?
%     a. collecting more samples and samples over time would allow for more confidence bc multi-specification checks, forward-testing, more factor confidence.
%     b. identifying underlying patterns within-group for state, industry, and ethnic effects could prove useful for modelling and also instructive for policy.
%          i. my skill gap survey dives into industry effects.
%     c. alternative credentials are extremely diverse. a useful study would disaggregate this category and ask about alternative credentials of different kinds.
%          i. my prestige study does this, and the skill gap study to some degree.
% 4. are there any implications for consumers, policymakers, firms/hiring managers, or alternative education providers?

This study introduced controls for mental effort and personality into an estimate of favorability of alternative postsecondary learning.
The main hypothesis was that these controls would deflate an apparent paradox in conservative opposition to a market solution.
Contrary to expectation, the paradoxical pro-regulatory effect was amplified with significance.
Personality factors were identified with importance and contributed to superior model power.

Conscientiousness and openness were important Big Five traits.
Grit was independently important in a multiple regression over and above conscientiousness.
Individuals high in grit experienced weaker familiarity bias.

Robustness of the pro-regulatory effect may be explained using a combination of three alternative hypotheses.
First, the pro-regulatory effect may represent an unobserved logical structure.
This hypothesis makes sense of improved effect identification resulting from added controls.
This hypothesis also makes sense of the lack of important mental effort effects.

A second hypothesis is that the measure of status quo bias is ineffective.
This explanation holds that status quo bias in education is particularly strong.
After correcting for the status quo proxy, there could be residual status quo bias remaining in the estimate.

In this study, the favorability of artificial intelligence is used as an innovation proxy.
Low favorability is taken to indicate status quo bias.
As artificial intelligence becomes normal,
favorability tends to become a poor tool to distinguish innovation from the status quo.
It seems plausible that for some respondents,
artificial intelligence is less a deviation from the status quo
compared to unaccredited learning.

The hypothesis of proxy failure may dovetail with an explanation from an unobserved logical structure.
That is, some conservatives may carve out education as a logical-ideological exception to general market favorability.

A third hypothesis is that there is systematic variation in the sample.
This explanation leverages an unexpected difference in the favorability to artificial intelligence
in the current sample compared to prior periods.
This variation can be taken as random,
but it might also be attributable in part to a recent massive social adoption of new technologies.
COVID-19 has forced massive social change to technology use.
This may contribute to an unexpectedly rapid normalization of artificial intelligence.
This third hypothesis need not exclude some effect from the other two.

\bibliography{./BibFile}

\end{document}
