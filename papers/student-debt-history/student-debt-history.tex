% note: model paper 20 pages, Annaert et al, Long-run stock returns: evidence from Belgium 1838-2010

% using Elseveir template per https://www.elsevier.com/authors/author-schemas/latex-instructions
\documentclass[review]{elsarticle}

\usepackage{lineno,hyperref}
\modulolinenumbers[5]

\journal{Journal of \LaTeX\ Templates}
\bibliographystyle{elsarticle-num}

\usepackage{booktabs}
\usepackage{graphicx}
\graphicspath{{../alt-ed-survey/figures-and-tables}}
\usepackage{hyperref}
\usepackage{threeparttable}  
\usepackage{tikz}
\usetikzlibrary{calc,matrix}

\begin{document}
\begin{frontmatter}

\title{
    New Digital Education as the Market Solution to the Student Debt Crisis
    \tnoteref{titlenotes}
}

% \tnotetext[titlenotes]{
%     Go to \url{https://github.com/Vandivier/research-dissertation-case-for-alt-ed/tree/master/papers/student-debt-history}
%     to obtain source materials for this paper.
% }

\author[mymainaddress]{John Vandivier}
\address[mymainaddress]{4400 University Dr, Fairfax, VA 22030}
\ead{jvandivi@masonlive.gmu.edu}

\begin{abstract}
    New Digital Education is an employer-lead, online, hybrid learning approach.
    This pattern of learning constitutes a market solution to the student debt crisis.
    This paper develops an empirical-historical argument to map New Digital Education
    to a location in the hype cycle. The key finding is that New Digital Education
    has survived disillusionment and is becoming a sustained institution in the US labor economy.
    A market solution to the student debt crisis undercuts the need for policy action.
\end{abstract}

\begin{keyword}
education economics, debt crisis, digital education, hype cycle
\MSC[2010] I21 % TODO: fix
\end{keyword}

\end{frontmatter}

\pagebreak
\linenumbers
        
    \section{Introduction}
    The prescient Van Dusen published in 1979 on The Coming Crisis in Student Aid\cite{van1979coming}.
    Discussion on the issue has proceeded continuously.
    Forbes noted in 2019\cite{friedman2018student} that “Student loan
    debt in 2019 is the highest ever...There are more than 44 million borrowers
    who collectively owe \$1.5 trillion in student loan debt in the U.S.
    alone.”


    In December 2019, Ryan Craig announced the Third Age of Online Education.
    The announcement is problematic in that online education was supposed to cut costs, but the debt problem persists.

    \section{Historical context}
    \section{Data}

    \section{Results}
    \section{Conclusions}

    \bibliography{./BibFile}

    \end{document}
