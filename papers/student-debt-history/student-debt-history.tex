% note: model paper 20 pages, Annaert et al, Long-run stock returns: evidence from Belgium 1838-2010

% using Elseveir template per https://www.elsevier.com/authors/author-schemas/latex-instructions
\documentclass[review]{elsarticle}

\usepackage{lineno,hyperref}
\modulolinenumbers[5]

\journal{Journal of \LaTeX\ Templates}
\bibliographystyle{elsarticle-num}

\usepackage{booktabs}
\usepackage{graphicx}
\graphicspath{{../alt-ed-survey/figures-and-tables}}
\usepackage{hyperref}
\usepackage{threeparttable}  
\usepackage{tikz}
\usetikzlibrary{calc,matrix}

\begin{document}
\begin{frontmatter}

\title{
    New Digital Education as the Market Solution to the Student Debt Crisis
    \tnoteref{titlenotes}
}

% \tnotetext[titlenotes]{
%     Go to \url{https://github.com/Vandivier/research-dissertation-case-for-alt-ed/tree/master/papers/student-debt-history}
%     to obtain source materials for this paper.
% }

\author[mymainaddress]{John Vandivier}
\address[mymainaddress]{4400 University Dr, Fairfax, VA 22030}
\ead{jvandivi@masonlive.gmu.edu}

\begin{abstract}
    New Digital Education is an employer-lead, online, hybrid learning approach.
    This pattern of learning constitutes a market solution to the student debt crisis.
    This paper develops an empirical-historical argument to map New Digital Education
    to a location in the hype cycle. The key finding is that New Digital Education
    has survived disillusionment and is becoming a sustained institution in the US labor economy.
    A market solution to the student debt crisis undercuts the need for policy action.
\end{abstract}

\begin{keyword}
education economics, debt crisis, digital education, hype cycle
\MSC[2010] I21 % TODO: fix
\end{keyword}

\end{frontmatter}

\pagebreak
\linenumbers
        
    \section{Introduction}

    The prescient Van Dusen published in 1979 on The Coming Crisis in Student Aid\cite{van1979coming}.
    Discussion on the issue has proceeded continuously.
    Forbes\cite{friedman2019student} noted that "Student loan debt in 2019 is the highest ever,"
    months before Ryan Craig would announce the Third Age of Online Education had begun\cite{craig2019welcome}.
    The fact of increasing debt in part casts doubt on the value proposition of online education.
    This paper contributes to the literature by mapping online education to the hype cycle, in addition to empirical and qualitative analysis.
    It turns out that cutting edge digital education is substantively different than the state of the art even a decade ago.
    This has implications for forecasting, policy, and the use of prior research.

    \section{Historical context}

    The Institute for Higher Education Policy and Lumina Foundation have well-documented the historical origin of the student debt crisis\cite{foundation_2017}.
    Ryan Craig provides an interesting tripartite history of online learning which I review and augment. In the end, I'm happy to accept his temporal markers,
    although my reasons are occassionally substantively different, and complimentary, for doing so.
    
    The First Age runs from about 2000 to 2012. Craig states this is the age of the for-profit online university.
    He states that quasi-for-profits and online program managers (OPMs) took over in roughly 2012.
    I appreciate the year 2000 as an important marker for a few different reasons.
    First, in 2000, 55 percent of two and four year institutions offered distance education\cite{tabs2003distance}.
    This is the first year where most institutions offered distanced education.
    In 1997, 44 percent offered distance education\cite{sikora2002profile}.

    Second, 2000 is the height of the dot-com boom, with the bust lasting through 2002\cite{wollscheid2012rise}.
    From a hype cycle perspective, technologies which survive this bust are emerging out of the trough of disillusionment and become sustainable.
    Third, NCES notes that 8 percent of students took at least one online course in 2000\cite{radford2011learning}.
    The figure had increased to 20 percent by 2008.
    5 or 10 percent are common Schelling points of significant adoption.
    Fourth, 2000 is simply a memorable, rounded number.
    Ferrer gives an interesting history of what might be called the Zeroth Age of Online Education, prior to the turn of the millennium\cite{ferrer_2019}.

    Craig sees 2012 as the beginning of the OPM era, and I don't disagree, but other important works were realized roughly contemporaneously.
    Studying the cases of Khan Academy and Udacity is illustrative.
    In 2008, Clayton Christiansen and Michael Horn applied the theory of disruption to digital education\cite{horn2008disrupting}.
    Khan Academy was founded the same year\cite{tucker_2018}.
    Dan Friedman\cite{friedman2014mooc} notes that Udacity rolled out classes in 2012, with Coursera and edX enrolling their first students the same year.
    The MOOC revolution began in 2012. That is what makes it a useful marker in my mind.
    Massive Online Open Courses (MOOCs) reported outcomes far weaker than expected through 2013 and 2014, but recovery and continued innovation proceeded quickly.

    In 2013, Udacity began offering some courses for college credit with San Jose State University\cite{shen_2015}.
    In 2014, Udacity entered into its first full-fledged program partnership with a university.
    Georgia Institute of Technology and Udacity announced an online Masters Degree in Computer Science for \$7000.
    This was 80 percent less than the on-campus equivalent\cite{onink2013georgia}.
    The same year, Udacity released its first signature alternative credential, the Nanodegree.

    A blended learning solution called Udacity Connect was piloted in 2016 and expanded in 2017.
    Udacity Connect's pilot achieved a 76 percent completion rate\cite{shah_2018}.
    As of 2019, Sebastian Thrun reports 4 percent is the MOOC completion rate,
    but Nanodegree programs have a 34 percent graduation rate, and using new personalized mentorship programs, cohorts commonly exceed 60 percent graduation rates.

    In 2016, Khan Academy applied for the \$100 million dollar grant by 100&Change in order to create a globally recognized secondary education diploma.
    1904 organizations applied for the grant\cite{conrad_2016}.
    When decisions were rendered in 2017, Khan Academy's proposal earned an honorable mention as one of the top ten in the education category,
    but it did not earn a financial award\cite{cushing_2017}.

    Like Udacity, Khan Academy is an online learning provider which went through a period of immense excitement followed by failure,
    and also like Udacity, Khan Academy achieved a remarkable success on a different project during the same calander year as their disenchanting loss.
    In 2017, Khan Academy released the results of a study they conducted with the College Board.
    It showed that studying for the SAT using Khan Academy is associated with 115-point average score increase\cite{khan_academy_sat_2017}.
    Khan Academy became the official practice partner for AP exams in 2017\cite{khan_academy_partner_2017}.

    Not only do Udacity and Khan Academy share a Jungian hero typology,
    they have both evolved from traditional learning competitors to traditional learning allies
    Like Coursera, edX, and others, the best-of-breed alternative learning providers of today are not substituting for traditional education providers,
    they are integrating with them. Likewise, the best-of-breed traditional providers are not rejecting new learning approaches,
    they are partnering with them, awarding credit to students for alternative learning, and even supplying online education providers with content.

    Summarily, there are three important changes which occur in the Second Age of Online Education, which I date from 2012-2017, in addition to the proliferation of OPMs.
    First, the MOOC revolution occurs through digitally-native, non-accredited online learning providers.
    Second, digitally-native learning providers go through a hero's journey. Providers like Coursera and Udacity initially fail to meet expectations, then improving outcomes significantly in a short period.
    Third, the education market integrates and equilibrates to some degree. Hybridization becomes a new normal.
    This hybrid norm is true of the industry leaders, but it will be explored quantitatively in the subsequent section on data.
    More traditional providers are brought online, and digitally native organizations begin exploring offline education.

    The Third Age of Online Education is characterized by essentially free college, subsidized by an employer.
    A new industry of intermediaries is emerging which operates as a platform to integrate employers and education providers.
    Employers are learning that investing in education as an employee benefit is generating substantial returns on investment.
    % Accenture + Cigna found 129 percent ROI https://www.luminafoundation.org/files/resources/talent-investments-pay-off-cigna-full.pdf
    % Guild, 200%+. They also unlocked hire utilization by providing no out of pocket model
    % in 2014, SHRM reported 60% orgs have reimbursement https://www.shrm.org/hr-today/news/hr-magazine/Pages/0614-tuition-assistance.aspx
    % as of 2018, talentCulture reports 85% do https://talentculture.com/tuition-assistance-look-like-2018/
    % Walmart is a case study here, as well as USA#1 employer; and indeed it looks like SMEs provide the benefit less frequently

    \section{Data}

    % This hybrid norm is true of the industry leaders, but it will be explored quantitatively in the subsequent section on data.
    % Fall 2012, more than 10% students exclusively online https://nces.ed.gov/pubs2014/2014023.pdf
    % In 2017, online learning grew why overall enrollment shrank https://www.insidehighered.com/digital-learning/article/2018/11/07/new-data-online-enrollments-grow-and-share-overall-enrollment
    % net tuition is largely flat since 2000; what about before that?
    % https://research.collegeboard.org/trends/college-pricing/figures-tables/average-net-price-sector-over-time
    % what about debt per student?
    % datasource = IPEDS, NCES
    % Guild touts that clients are achieving a $2.44 return for every $1 spent on tuition reimbursement
    % Cigna+Accenture https://www.cfo.com/training/2016/04/cigna-gets-129-roi-tuition-reimbursement-costs/
    % people are digital natives now
        %Second, only 10 percent of Americans are not internet-using\cite{anderson2019}.
        %Today, the national population essentially is internet-using.
    % can go to school for under 5250 deduction; additional deductions
    % spread of true cost instead of average sticker cost
    % argue out of pocket cost = 0; straighterline, credit by exam, University of the People

    % maybe some survey data or media analysis, also efficacy meta analysis

    \section{Results}

    % yes, we are thru hype cycle
    % online is already normal but it will continue to be even more normal
    
    \section{Conclusions}

    % it's a problem that will take care of itself, so long as we don't make things worse

    \bibliography{./BibFile}

    \end{document}
