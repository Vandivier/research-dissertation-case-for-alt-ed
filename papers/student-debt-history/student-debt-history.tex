% note: model paper 20 pages, Annaert et al, Long-run stock returns: evidence from Belgium 1838-2010

% using Elseveir template per https://www.elsevier.com/authors/author-schemas/latex-instructions
\documentclass[review]{elsarticle}

\usepackage{lineno,hyperref}
\modulolinenumbers[5]

\journal{Journal of \LaTeX\ Templates}
\bibliographystyle{elsarticle-num}

\usepackage{booktabs}
\usepackage{graphicx}
\graphicspath{{../alt-ed-survey/figures-and-tables}}
\usepackage{hyperref}
\usepackage{threeparttable}  
\usepackage{tikz}
\usetikzlibrary{calc,matrix}

\begin{document}
\begin{frontmatter}

\title{
    New Digital Education as the Market Solution to the Student Debt Crisis
    \tnoteref{titlenotes}
}

% \tnotetext[titlenotes]{
%     Go to \url{https://github.com/Vandivier/research-dissertation-case-for-alt-ed/tree/master/papers/student-debt-history}
%     to obtain source materials for this paper.
% }

\author[mymainaddress]{John Vandivier}
\address[mymainaddress]{4400 University Dr, Fairfax, VA 22030}
\ead{jvandivi@masonlive.gmu.edu}

\begin{abstract}
    New Digital Education is an employer-lead, online, hybrid learning approach.
    This pattern of learning constitutes a market solution to the student debt crisis.
    This paper develops an empirical-historical argument to map New Digital Education
    to a location in the hype cycle. The key finding is that New Digital Education
    has survived disillusionment and is becoming a sustained institution in the US labor economy.
    A market solution to the student debt crisis undercuts the need for policy action.
\end{abstract}

\begin{keyword}
education economics, debt crisis, digital education, hype cycle
\MSC[2010] I21 % TODO: fix
\end{keyword}

\end{frontmatter}

\pagebreak
\linenumbers
        
    \section{Introduction}

    The prescient Van Dusen published in 1979 on The Coming Crisis in Student Aid\cite{van1979coming}.
    Discussion on the issue has proceeded continuously.
    Forbes\cite{friedman2019student} noted that "Student loan debt in 2019 is the highest ever,"
    months before Ryan Craig would announce the Third Age of Online Education had begun\cite{craig2019welcome}.
    The fact of increasing debt in part casts doubt on the value proposition of online education.
    This paper contributes to the literature by mapping online education to the hype cycle, in addition to empirical and qualitative analysis.
    It turns out that cutting edge digital education is substantively different than the state of the art even a decade ago.
    This has implications for forecasting, policy, and the use of prior research.

    \section{Historical context}

    The Institute for Higher Education Policy and Lumina Foundation document the origin of the student debt crisis\cite{foundation_2017}.
    Ryan Craig argues that online learning has proceeded in three ages. I'm generally happy with his temporal markers,
    but here I review some substantively different, and some complimentary, reasons selecting these dates.
    
    The First Age runs from about 2000 to 2012. Craig argues that this is the age of the for-profit online university.
    He argues that a second age dominated by quasi-for-profits and online program managers (OPMs) begins in 2012.
    I note several other reasons that the year 2000 is important.
    In 2000, 55 percent of two and four year institutions offered distance education\cite{tabs2003distance}.
    This is the first year where most institutions offered distanced education.
    In 1997, 44 percent offered distance education\cite{sikora2002profile}.

    Secondly, 2000 begins the dot-com bust, which would last through 2002\cite{wollscheid2012rise}.
    From a hype cycle perspective, the bust is a trough of disillusionment.
    To survive the bust indicates economic durability, institutional fitness, and a positive continued growth outlook.
    NCES notes that 8 percent of students took at least one online course in 2000\cite{radford2011learning}.
    This essentially achieves a certain Schelling point of 5 to 10 percent for importance or significance.
    The figure had increased to 20 percent by 2008.
    A fourth reason for the date is simply that 2000 is a memorable, rounded number.
    One might argue for beginning at the founding of Western Governors University in 1997, or after the end of the dot-com bust in 2002, but all of these plausible dates round to 2000.
    Ferrer gives an interesting history of what might be called the Zeroth Age of Online Education, exploring the developmental phase of remote learning prior to the turn of the millennium\cite{ferrer_2019}.

    Craig argues for 2012 as the beginning of the OPM era, and I now  important works were realized roughly contemporaneously.
    Studying the cases of Khan Academy and Udacity is illustrative.
    In 2008, Clayton Christiansen and Michael Horn applied the theory of disruption to digital education\cite{horn2008disrupting}.
    Khan Academy was founded the same year\cite{tucker_2018}.
    Dan Friedman\cite{friedman2014mooc} notes that Udacity rolled out classes in 2012, with Coursera and edX enrolling their first students the same year.
    The MOOC revolution began in 2012. That is what makes it a useful marker in my mind.
    Massive Online Open Courses (MOOCs) reported outcomes far weaker than expected through 2013 and 2014, but recovery and continued innovation proceeded quickly.
    Uber was founded in 2009 and Facebook went public in 2012, so this second age also captures a fundamentally different culture where online activity is a basic social skill and entirely new labor models are being enabled.

    In 2013, Udacity began offering some courses for college credit with San Jose State University\cite{shen_2015}.
    In 2014, Udacity entered into its first full-fledged program partnership with a university.
    Georgia Institute of Technology and Udacity announced an online Masters Degree in Computer Science for \$7000.
    This was 80 percent less than the on-campus equivalent\cite{onink2013georgia}.
    The same year, Udacity released its first signature alternative credential, the Nanodegree.

    A blended learning solution called Udacity Connect was piloted in 2016 and expanded in 2017.
    Udacity Connect's pilot achieved a 76 percent completion rate\cite{shah_2018}.
    As of 2019, Sebastian Thrun reports 4 percent is the MOOC completion rate,
    but Nanodegree programs have a 34 percent graduation rate, and using new personalized mentorship programs, cohorts commonly exceed 60 percent graduation rates.

    In 2016, Khan Academy applied for the \$100 million dollar grant by 100&Change in order to create a globally recognized secondary education diploma.
    1904 organizations applied for the grant\cite{conrad_2016}.
    When decisions were rendered in 2017, Khan Academy's proposal earned an honorable mention as one of the top ten in the education category,
    but it did not earn a financial award\cite{cushing_2017}.

    Like Udacity, Khan Academy is an online learning provider which went through a period of immense excitement followed by failure,
    and also like Udacity, Khan Academy achieved a remarkable success on a different project during the same calander year as their disenchanting loss.
    In 2017, Khan Academy released the results of a study they conducted with the College Board.
    It showed that studying for the SAT using Khan Academy is associated with 115-point average score increase\cite{khan_academy_sat_2017}.
    Khan Academy became the official practice partner for AP exams in 2017\cite{khan_academy_partner_2017}.

    Not only do Udacity and Khan Academy share a Jungian hero typology,
    they have both evolved from traditional learning competitors to traditional learning allies
    Like Coursera, edX, and others, the best-of-breed alternative learning providers of today are not substituting for traditional education providers,
    they are integrating with them. Likewise, the best-of-breed traditional providers are not rejecting new learning approaches,
    they are partnering with them, awarding credit to students for alternative learning, and even supplying online education providers with content.

    Summarily, there are three important changes which occur in the Second Age of Online Education, which I date from 2012-2017, in addition to the proliferation of OPMs.
    First, the MOOC revolution occurs through digitally-native, non-accredited online learning providers.
    Second, digitally-native learning providers go through a hero's journey. Providers like Coursera and Udacity initially fail to meet expectations, then improving outcomes significantly in a short period.
    Third, the education market integrates and equilibrates to some degree. Hybridization becomes a new normal.
    This hybrid norm is true of the industry leaders, but it will be explored quantitatively in the subsequent section on data.
    More traditional providers are brought online, and digitally native organizations begin exploring offline education.

    The Third Age of Online Education is characterized by affordable college, driven by employer financing.
    In 2013, SHRM reported that 61 percent of employers offer tuition assistance\cite{cherry2014rejuvenating}.
    In 2017, World at Work found that 85 percent of employers offered such a benefit,
    with another 7 percent offering non-reimbursement tuition assistance, such as upfront tuition discounts\cite{talentculture_2018}.
    In 2019, reimbursement was negligibly higher with an offering at 86 percent of employers\cite{worldatwork_2019}.
    This increase in tuition assistance is invigorated in part by the emergence of a new kind of employee benefit intermediary,
    often partnering with online education providers.

    Guild Education is an example of this new kind of benefits intermediary. Walmart is the largest employer in the United States.
    Walmart and Guild partnered to provide higher education to Walmart employees, including part-time employees, forming an interesting case study.
    Walmart's program began in 2018 and provides college education for one dollar per day\cite{walmart_2018}.
    As of January 2020, Walmart's program allows access to around 40 degrees through about half a dozen online providers\cite{guild_walmart_2020}.
    Providers include the University of Florida and Southern New Hampshire University.
    Walmart's education benefit also provides a high school completion program and free ACT and SAT preparation assistance.

    Employers are motivated to provide this service because they generate return through such vectors as improved employee retention.
    Guild clients see 120 to 210 percent return to education spend\cite{hunter_2019}.
    This is consistent with the 129 percent return observed by Accenture and the Lumina Foundation,
    who published return to education spend informatino for Cigna's education reimbursement program\cite{mccann_2016}.

    While online learning has become normal, perception as normal lags slightly behind.
    Surveys from 2017 and 2018 show that students, employers, and the general public generally believe
    an online degree is equal to or better than on-campus\cite{venable_2019}.
    The general public is the least positive of these three groups.
    While most employers support online learning,
    about 18 percent of students state that employer perception of an online degree is a concern.

    Taking advantage of employer-financed education presupposes working while learning.
    This turns out to also be an established norm in higher education.
    In 2015, a report from Georgetown University showed that more than 70 percent of college students
    worked while learning over the prior 25 years\cite{carnevale2015learning}.
    For most students the academic effects are not important,
    but they find an important academic threat to low-income working learners who work more than 15 hours each week.
    Darolia\cite{darolia2014working} introduces additional controls and concludes that working
    marginally more hours does not negatively affect grades, although it does slow credit accumulation.

    \section{Data}

    I demonstrate hype cycles using two kinds of Google search data.
    The first is ordinary search volume and the second is Google Trends data.
    Both sources are intended to proxy social attention, or hype.
    There are several important caveats when using these sources.
    First, Google Trends data extends as far back as 2004, but as we will see this appears to be too late to capture some hype around the dot-com boom.
    Relatedly, Google News Archive Search became available in 2006.
    While Google retroactively added older news, I will shortly demonstrate serious coverage gaps with this information.
    As a result, general search data is used instead of media data in particular.

    A final caveat is that these search-based measures of hype ignore sentiment and focuses only on volume.
    Sentiment analysis of the underlying media coverage is quite feasible,
    but it is hardly necessary because it has been qualitatively covered in the section on historical context
    and also because the sustained growth in the market associated with hype already implies positive sentiment on average.

    Google Trends provides an index of search volume over time.
    The index is not an exact search frequency.
    The index takes a value between 0 and 100, inclusive, and is normalized for each series against itself, not across series.
    The key result is the establishment of a hype cycle in the trend of a series in relation to itself over time, not a comparison across series.
    Google Trends data is available using Google-defined topics as well as keyword matching.
    Google-defined topics are generally preferred as more precise, and that is the form of data which I use.
    Keyword matching is less precise because there is a boolean or statement used by the querying engine, resulting in many false positive results and significant noise in the data.
    With Google-defined topics, Google utilizes machine intelligence to ensure relevant, apples-to-apples results are contained in a trend signal.
    Western Governors University is a great example of why this distinction is important.
    As a keywords search, search volume would be reported for any search using any of those words.
    Irrelevant results for Western Union, Best Western, and University of Texas would all be included.
    
    Western Governors University was founded in 1997.
    Launching a well-funded organization like WGU is expected to generate some hype.
    This provides an edge case which helps validate the quality of Google News data.
    Google News data is revaled to be problematic immediately when a search from January 1990 through December 2001 yields zero results.
    This concern is bolstered when it is revealed that a search for the word "water" results in only 81 results over the same time period.
    Further inspection shows that about 23 of the 81 results were actually published during the specified time period.
    
    In contrast with Google News data, an ordinary Google search for water, filtered to the dates 1990 through 2001, yield hundreds of results.
    A cursory look indicates that these web pages were in fact published during the specified date range.
    Searching for Western Governors University in the ordinary way presents with more than zero results.
    While ordinary Google search data has plenty of peculiarities, it seems preferred due to this preliminary validation.
    Google Trends data is preferred to ordinary search results, but the founding of WGU precedes available Google Trends data.

    \section{Results}

    The first result is an illustration of Google Trend data over time.
    
    % google trends becomes relevant at this point; see the other chart where gradual increase shown

    \footnote{
        Technically, the boolean query "Western Governor's University" OR "Western Governors University" was used.
        Plenty of online discussion on WGU involves this typo.
        The result number excludes articles which Google excluded by default.
        Google excludes some web results by default, generally because the excluded results seem to be copies of results already shown.
    }

    % yes, we are thru hype cycle
    % online is already normal and it will continue to be even more normal
    % students, employers, policymakers shouldn't fear online education
    % it may slow graduation, but as long as you are working while learning the strategy still nets out a financial win
    %   -> let's do an example comparison of education ROI to prove the above line, and contrast that with Caplan's return estimates or the rest of the literature.

    % sustainable growth and non-need for policy intervention consistent with below...
    % Fall 2012, more than 10% students exclusively online https://nces.ed.gov/pubs2014/2014023.pdf
    % In 2017, online learning grew why overall enrollment shrank https://www.insidehighered.com/digital-learning/article/2018/11/07/new-data-online-enrollments-grow-and-share-overall-enrollment
    % net tuition is largely flat since 2000; what about before that?
    % https://research.collegeboard.org/trends/college-pricing/figures-tables/average-net-price-sector-over-time
    % what about debt per student?
    % datasource = IPEDS, NCES
    % people are digital natives now
        %Second, only 10 percent of Americans are not internet-using\cite{anderson2019}.
        %Today, the national population essentially is internet-using.
    % can go to school for under 5250 deduction; additional deductions
    % spread of true cost instead of average sticker cost
    % argue out of pocket cost = 0; straighterline, credit by exam, University of the People
    
    \section{Conclusions}

    % it's a problem that will take care of itself, so long as we don't make things worse
    % example of outdated paper: https://www.westga.edu/~distance/ojdla/spring121/columbaro121.html
    % obscelescence of much research in turn creates a demand for new research in the area: How do employers feel today,
    % and how should we expect them to feel in the future in light of new information?
    % high level positivity is expected based on this paper, given hype cycle dynamics and aggregate market growth,
    % but that is essentially an aggregate answer that could be improved by accounting for employer and learner heterogeneity

    \bibliography{./BibFile}

    \end{document}
