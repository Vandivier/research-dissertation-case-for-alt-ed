% using Elseveir template per https://www.elsevier.com/authors/author-schemas/latex-instructions
\documentclass[review]{elsarticle}

\usepackage{lineno,hyperref}
\modulolinenumbers[5]
\journal{Journal of \LaTeX\ Templates}
\bibliographystyle{elsarticle-num}

\usepackage{booktabs}
% allowing images as figures
% ref: https://tex.stackexchange.com/questions/19176/how-to-insert-an-image-into-latex-document
\usepackage{graphicx}
\graphicspath{{../alt-ed-survey/figures-and-tables}}
\usepackage{hyperref}
\usepackage{threeparttable}  
\usepackage{tikz}
\usetikzlibrary{calc,matrix}

\begin{document}

\begin{frontmatter}

    \title{
        Perceived Skills Gaps in Alternative Postsecondary Education as Determinants of Hireability
        \tnoteref{titlenotes}
    }
    \tnotetext[titlenotes]{
        Go to \url{https://github.com/Vandivier/research-dissertation-case-for-alt-ed/tree/master/papers/alt-ed-survey}
        for additional materials including the online appendix,
        survey data, and data analysis source code.
    }

    \author[mymainaddress]{John Vandivier}
    \address[mymainaddress]{4400 University Dr, Fairfax, VA 22030}
    \ead{jvandivi@masonlive.gmu.edu}

    \begin{abstract}
        This paper explores X to understand Y
        This paper explores a novel data set (n = 1190) to understand trends in public
        disposition toward alternative postsecondary learning, with a focus on employers.
        Results indicate that public favorability is positive and will remain flat over the next year.
        Employer attitudes are not meaningfully different from the general public.
    \end{abstract}

    \begin{keyword}
        education economics, alternative education, candidate fit, job fit, candidate matching     %%% not grammatical
        \MSC[2010] I21, I22, J20                                                                   %%% not grammatical
    \end{keyword}

\end{frontmatter}

\pagebreak
\linenumbers

\section{Introduction}

% a missing link for future research: favorability only correlates with actual hiring decisions it isn't a hiring decision
Hiring decisions reflect boundedly rational demand for skilled labor.
The college degree and alternative credentials provide two qualitatively distinct signals of skilled labor.
The hireability of individuals in possession of these credentials has been studied,
but the underlying determinants are not clear.
This paper hypothesizes that perceived skill gaps are important determinants of willingness to hire.
This paper further hypothesizes that perceived skill gaps are qualitatively different between college graduates and others.
% In particular, this paper hypothesizes that a noncollege stigma is obtained for candidates without a degree in pursuit of roles typically filled by degree holders.
% soft skill bias in particular

Alternative credentials refer to credentials other than the undergraduate degree\cite{brown2017complex}.
The category includes industry certifications,
vocational schooling, noncollege courses,
and even portfolios produced during self-study.
Individuals pursuing alternative credentials typically intend to leverage the credential in order to obtain employment.
That is, they typically have the same ends as those pursuing traditional education.

% Alternative credentials can be obtained quickly and cheaply relative to college.
Obtaining a college degree signals intelligence, conscientiousness, and conformity,
but it may not signal technical skill\cite{horton_2020}.
Alternative credentials signal technical skill.
As such, they provide an effective supplement to the degree.

This paper is concerned with another use case in which the degree is entirely substituted.
In that situation, employers may apply a noncollege stigma.
This is particularly the case for roles which are typically occupied by degree holders.
Noncollege stigma is a presumption, expectation, or bias toward perception of a skill gap of a certain kind.
Whether the gap exists in fact is out of the scope of the present paper.

Technical skill generally implies intelligence.
Alternative credentials, then, fail to signal two qualities compared to the degree.
Alternative credentials fail to signal conscientiousness and conformity.
Interestingly, some employers may demand some level of nonconformity.
Employers may also presume a certain lack of soft skill on the part of highly technical applicants.
% Finally, employers may use alternative credentials as a proxy for other employee characteristics like income, education, race, and gender.

% three interesting follow-on questions:
%   1. do employers have such a bias
%   2. is such a soft skills gap presumption actually true
%   3. if true, due employers overvalue the soft skills gap
% related paradox: most people won't be in a job for 4-5 years,
% so why do they need to show concientiousness and conformity towards the 4-5 year bachelor's goal line?
% hard skill stigma: in my experience, people who are highly technical are hard to work with
% soft skill bias: I am favorable bc i think u have soft skills (and maybe this is efficient...enter eq/iq discussion)

\subsection{Process Explanations of Suboptimal Wages}

Basic price theory holds that an employer should pay wages equal to the marginal revenue product of labor.
In the real world, measuring candidate productivity at hiring time is costly and imperfect.
% This produces a technical error which assumes alignment between the goals of the firm and a hiring team.
A further issue is identified when the hiring team is scrutinized for principal-agent problems.
The hiring team is composed of individuals with preferences, calculative limitations, and other biases.
Monitoring and correcting for these problems is expensive,
so firms will heterogenously realize some aggregation of these individual definiciencies.

Exacerbating the already necessarily imperfect hiring process are candidate-side problems.
Firms must hire among a finite, potentially small, number of candidates.
Risk aversion to time expense and other search costs may lead a firm to approve a suboptimal candidate\cite{simon1976substantive}.
In some cases, candidate pools may be systematically problematic.
In law and medicine, for example, extensive education and training are legally required.
These policies further restrict the candidate pool, inflate expected wages, and systematically alter the content of education in a politically-motivated manner.
Market forces implement hiring as a lumpy expenditure process to begin with, but certification requirements, wage regulation, and other policies extend the problem.

The prior discussion highlights many locations of hiring process inefficiency.
The practical importance of magnitudes and kinds of such effects are described in a legion of related papers.
A meager sampling of five such effects would include the attractiveness effect and many other issues related to gender bias\cite{quereshi1986physical},
agentic behavioral stigma\cite{steffens2009feminization},
and complex biases related to communication style\cite{brouer2017gender, nijs2019effects, sampugnaro1983nonverbal}.
Sung et al find that impression management meaningfully weakens disability stigma\cite{sung2017disclose}.
These tactics are transferable in part to noncollege stigma mitigation.
Finally, there are a wealth of concerns about the effects of social media.
For one, it presents a channel for the revival of religious discrimination\cite{esposito2018signaling}.

In the face of so many important inefficiencies, one begins to wonder whether the original theory holds any water at all.
Papers which identify matching effects, including the present paper,
serve to limit the proportion of explanation attributable to bias and redeem the elementary price theory story to some extent.
Prior work demonstrates the important of matching effects in the form of norm compliance\cite{francesco1981gender}.
Meta-accuracy is a kind of matching measure, and it has been shown to move positively with hireability\cite{renier2018no}.

\section{Description of Data}

% missing industrial variables: hospitality (restaurant/hotels) and entertainment / media, and sales!

how is skill gap measured? typically, skill of candidate compared to ideal; but this produces on overestimate of the skill gap imo.
The typical employee also has a skill gap compared to ideal, so:
1. if the candidate is as skilled as an actual employee, they should be hireable (well, maybe not if org wants to upskill or correct for onboarding costs)
2. if the organization routinely hires recent college graduates, an alt ed candidate should be higherable if their gaps are similar to a college grad.

% (horse racing): https://www.afterecon.com/economics-and-finance/kitchen-sink-regression-and-horse-racing/
% should probably randomly split sample and out-of-sample test with factors to combat overfit
% 3 different explanatory constructs are explored, but only the winner is reported in the paper:
% 1. alt ed to ideal
% 2. alt ed to typical [not interesting for this paper]
% 3. alt ed to college grad
% 4. alt ed to ideal without overqualification
% 5. alt ed to typical without overqualification [not interesting for this paper]
% 6. alt ed to college grad
overqualification concern?
aggregate excess attractiveness by recent college grads against ideal.
aggregate excess willingness to break rules by alt ed noncollege grads.
many non-aggregate, or respondent-level, cases of alt ed overqualification; in fact, some such responses for every question kind (the 13 types)

Only unassailable approach is to compare alt ed to ideal; bc typical employees and recent grads are not always theoretically hireable.
Left hand param is favorability.
Optional but interesting: college grad to ideal or college grad to alt ed; so that we can indirectly associate favorability to actual propensity to hire. (which we have for college grads)

% objective of analysis
% how much does matching explain (caveat: not multiply regressed, so matching effect is likely overstated in this paper, and possibly partially partialled-in to prior work)
% does noncollege stigma exist
% "alternative education is different how?"
% 'explaining favorability'

% simple match effect: those that prefer technical talent will tend to support alternative credentials.
% complex match effect: a match profile will have significantly and importantly more explanatory power compared to but consistent with a simple match effect.

% quality question meta: 1 to 10: disagree to agree
% ---
% An ideal candidate would have this quality...
% A typical employee would have this quality...
% A college graduate would have this quality...
% A credentialed or certified non-college graduate would have this quality...
% [later] Someone who is self-taught (without a credential or portfolio) would have this quality...
% [later] A typical junior-level high school student would have this quality...

% some notes, mainly out of scope
% ---
% hiring error awareness increases in a few ways
%  1. [passive search] participant observation. As an interviewer, interviewee, hiring manager, or other professional involved in the process, I simply notice a problem
%  2. [passive search] passive company and individual level search into HR best practices; an industry newsletter says hey Griggs v Duke happened so don't use IQ tests anymore.
%  3. [passive search]: audit compliance (legal+required, or optional audits from firms that certify quality, for example)
%       example: Supreme Court case Griggs v Duke had an industry-wide effect thru this means
%  4. [active search] intrapraneurship / policy change championing begins with an individual saying hey let's investigate this thing. what would motivate such individual? (maybe due to 1 or 2).
%
% my prior work has shown that we can predict (r2 0.5 - 0.7 and ar2 0.3 - 0.6) alternative education favorability from employer factors alone - without concern to matching

\section{Results}

TODO: Table 1 should have skill gaps from preferred model, model 5. so-called semi-robust skill gaps.

% n=212
\begin{table}
    \caption{Table of Multiple Regression on Favorability, Selected Variables}
    \resizebox{\columnwidth}{!}{
        {
\def\sym#1{\ifmmode^{#1}\else\(^{#1}\)\fi}
% \begin{center}
{
    \fontsize{8pt}{7pt}\selectfont
    % \begin{small}
    \tabcolsep=3pt
    \begin{tabular}{l*{4}{c}}
        \toprule
        \multicolumn{1}{c}{Effect Group} & \multicolumn{1}{c}{Adj R-Sqr} & \multicolumn{1}{c}{R-Sqr} & \multicolumn{1}{c}{Max p-value} \\
        \midrule
        Absolute Gap                     & 0.0615                        & 0.0703                    & 0.097                           \\
        \addlinespace
        Comparative Gap                  & 0.0176                        & 0.0298                    & 0.687                           \\
        % \addlinespace
        % Rulebreaker                           & 0.1432                        & 0.1554                    & 0.053                           \\
        \addlinespace
        Industry                         & 0.0303                        & 0.0454                    & 0.958                           \\
        \addlinespace
        \addlinespace
        Other Factors                    & 0.0072                        & 0.0288                    & 0.537                           \\
        \addlinespace
        Rulebreaker                      & 0.0783                        & 0.0869                    & 0.127                           \\
        \addlinespace
        State                            & 0.0469                        & 0.1033                    & 0.772                           \\
        % \addlinespace
        % State, Semi-Robust                    & 0.0034                        & 0.0648                    & 0.831                           \\
        \bottomrule
    \end{tabular}
    % \end{center}
}
}

% TODO: maybe a count of k factors in group
% TODO: maybe distinguish strong and weak effects for industry, state, and gaps
% TODO: maybe other controls / other factors section doesn't matter
% TODO: maybe combine skill gaps

    }
    \label{tab:table_new_ols}
\end{table}

% do matching effects work?
% do ideal match gaps or normative match gaps have more explanatory power?
% is there a significant expected wage differential?
% college skill gaps are things; how big are they and how are they qualitatively different compared to alternative credential skill gaps?

This paper acknowledges that own analysis proceeds through a technocentric lens.
This is an important anchoring point for the analysis, and it may skew application of results in low-technology or low-skill sectors.
The technocentric lens is an important caveat and anchoring point, but I argue that it is about as proper as any anchoring point.
In economics, after all, technology operationalizes the theory of innovation per se.
All skills can be viewed as point-in-time innovations, so that if there was no innovation then neither would there be a need for any skill.
By the same token, a technocentric lens at the present seems close to a cross-industry lens at a future time.
Anchoring to any other industry would be both asymmetric and unusuful in the future.
Perhaps this analysis is slightly skewed, but at least it is skewed only against the past, and will be increasingly useful in the future without partiality to any particular industry.
In addition, we did check for industrial effects, but the analytical skew may persist pass the data.

% This orientation occurs because New Alternative Education first flourished for roles in the information technology sector, and only later did roles like sales, business, art, nursing, and more join in.
%     [can refer to my New Digital Education] - https://papers.ssrn.com/sol3/papers.cfm?abstract_id=3530647

% Compare directly to bootcamp results from Indeed: https://www.indeed.com/lead/what-employers-think-about-coding-bootcamp

% n=212
\begin{table}
    \caption{Table of Multiple Regression on Favorability, Selected Variables}
    \resizebox{\columnwidth}{!}{
        % derived from analysis-5-regs-table.do
{
\def\sym#1{\ifmmode^{#1}\else\(^{#1}\)\fi}
\begin{tabular}{l*{5}{c}}
\toprule
                         &\multicolumn{1}{c}{1}&\multicolumn{1}{c}{2}&\multicolumn{1}{c}{3}&\multicolumn{1}{c}{4}&\multicolumn{1}{c}{Model 5}\\
\midrule
Is Employed Non-Manager  &      -0.336         &      -0.383\sym{*}  &      -0.497\sym{**} &      -0.471\sym{**} &      -0.451\sym{**} \\
\addlinespace
Is STEM Worker           &      -0.491\sym{**} &      -0.529\sym{**} &      -0.525\sym{**} &      -0.557\sym{**} &      -0.564\sym{**} \\
\addlinespace
Employees 51-200         &      -0.475\sym{*}  &      -0.480\sym{**} &      -0.364         &      -0.459\sym{*}  &      -0.457\sym{*}  \\
\addlinespace
Industry Credentials Legally Required&       0.706\sym{*}  &       0.722\sym{**} &       0.374         &       0.378         &       0.375         \\
\addlinespace
Industry Credentials Normal&       0.932\sym{**} &       0.926\sym{**} &       0.487\sym{*}  &       0.436\sym{*}  &       0.448\sym{*}  \\
\addlinespace
Industry Credentials Sometimes Used&       0.467         &       0.475         &                     &                     &                     \\
\addlinespace
Industry Credentials Unknown&       0.641\sym{*}  &       0.684\sym{**} &                     &                     &                     \\
\addlinespace
Industry, Agriculture    &       1.368         &       1.619\sym{*}  &                     &                     &                     \\
\addlinespace
Industry, Energy         &      -1.277\sym{*}  &      -1.190\sym{*}  &      -1.200\sym{*}  &      -1.442\sym{**} &      -1.448\sym{**} \\
\addlinespace
Industry, Finance, Investment, or Accounting&      -0.811\sym{***}&      -0.783\sym{***}&      -0.712\sym{***}&      -0.715\sym{***}&      -0.717\sym{***}\\
\addlinespace
Industry, Information Technology&       0.335         &       0.264         &       0.438\sym{*}  &       0.306         &       0.337         \\
\addlinespace
Industry, Law            &      -1.813\sym{***}&      -1.670\sym{**} &      -1.935\sym{***}&      -1.876\sym{***}&      -1.857\sym{***}\\
\addlinespace
Industry, Transportation &       1.512\sym{*}  &       1.643\sym{**} &       1.216         &       1.403\sym{*}  &       1.350\sym{*}  \\
\addlinespace
State, Arizona           &      -1.157\sym{**} &      -1.048\sym{**} &      -0.755         &      -0.823\sym{*}  &      -0.790         \\
\addlinespace
State, Arkansas          &      -2.690\sym{***}&      -2.817\sym{***}&      -2.489\sym{***}&      -2.664\sym{***}&      -2.770\sym{***}\\
\addlinespace
State, California        &      -0.575\sym{*}  &      -0.570\sym{**} &      -0.488\sym{*}  &      -0.435         &      -0.446         \\
\addlinespace
State, Colorado          &      -1.446\sym{**} &      -1.423\sym{**} &      -1.463\sym{**} &      -1.521\sym{***}&      -1.508\sym{***}\\
\addlinespace
State, Connecticut       &      -1.401         &      -1.550         &                     &                     &                     \\
\addlinespace
State, Florida           &      -0.444         &      -0.454         &                     &                     &                     \\
\addlinespace
State, Hawaii            &      -3.232\sym{***}&      -3.271\sym{***}&      -2.884\sym{***}&      -2.869\sym{***}&      -2.812\sym{***}\\
\addlinespace
State, Illinois          &      -0.637         &      -0.699\sym{*}  &      -0.596         &      -0.675\sym{*}  &      -0.698\sym{*}  \\
\addlinespace
State, Kansas            &      -3.283\sym{**} &      -3.486\sym{**} &      -2.923\sym{*}  &      -3.116\sym{**} &      -3.101\sym{*}  \\
\addlinespace
State, Kentucky          &      -3.143\sym{***}&      -3.167\sym{***}&      -2.583\sym{***}&      -2.729\sym{***}&      -2.679\sym{***}\\
\addlinespace
State, Louisiana         &      -1.455\sym{*}  &      -1.255\sym{*}  &      -0.915         &      -0.941         &      -0.898         \\
\addlinespace
State, Maryland          &      -0.596         &      -0.642         &                     &                     &                     \\
\addlinespace
State, Nebraska          &      -2.037\sym{*}  &      -2.167\sym{*}  &      -1.391         &      -1.655         &      -1.596         \\
\addlinespace
State, Nevada            &      -1.406         &      -1.470         &      -1.465         &      -1.434         &      -1.409         \\
\addlinespace
State, New Jersey        &      -1.145         &      -1.139         &      -0.976         &      -0.936         &      -0.963         \\
\addlinespace
State, New York          &      -0.692\sym{**} &      -0.640\sym{*}  &      -0.617\sym{*}  &      -0.595\sym{*}  &      -0.590\sym{*}  \\
\addlinespace
State, Ohio              &      -3.943\sym{***}&      -4.024\sym{***}&      -4.051\sym{***}&      -3.808\sym{***}&      -3.761\sym{***}\\
\addlinespace
State, Pennsylvania      &      -0.752         &      -0.687         &      -0.608         &      -0.534         &      -0.539         \\
\addlinespace
State, South Carolina    &      -1.183         &      -1.243         &      -1.361         &      -1.310         &      -1.347         \\
\addlinespace
State, Tennessee         &      -1.878\sym{**} &      -1.909\sym{**} &      -1.545\sym{*}  &      -1.843\sym{**} &      -1.799\sym{**} \\
\addlinespace
State, Texas             &      -0.906\sym{**} &      -0.851\sym{**} &      -0.797\sym{**} &      -0.790\sym{**} &      -0.789\sym{**} \\
\addlinespace
State, Washington        &      -0.817         &      -0.863\sym{*}  &      -0.880\sym{*}  &      -0.996\sym{**} &      -1.003\sym{**} \\
\addlinespace
Duration                 &       0.666\sym{**} &       0.634\sym{**} &       0.811\sym{***}&       0.744\sym{**} &       0.719\sym{**} \\
\addlinespace
cduration2               &     -0.0884\sym{**} &     -0.0857\sym{**} &      -0.113\sym{***}&      -0.103\sym{**} &     -0.1000\sym{**} \\
\addlinespace
WOQ, Gap, Attractiveness &      -0.161\sym{***}&                     &                     &                     &                     \\
\addlinespace
WOQ, Gap, Body Language-IT&       0.100         &                     &                     &                     &                     \\
\addlinespace
WOQ, Gap, Conscientiousness&     -0.0657         &                     &                     &                     &                     \\
\addlinespace
WOQ, Gap, EQ             &     -0.0966         &                     &                     &                     &                     \\
\addlinespace
Rule Breakers Risky      &      0.0732\sym{*}  &      0.0715\sym{*}  &      0.0880\sym{**} &      0.0747\sym{*}  &      0.0762\sym{*}  \\
\addlinespace
Rule Breakers Rockstars  &       0.133\sym{**} &       0.128\sym{**} &       0.147\sym{**} &       0.141\sym{**} &       0.140\sym{**} \\
\addlinespace
Rule Breakers Culture Add&      0.0905         &      0.0974\sym{*}  &       0.115\sym{**} &       0.112\sym{**} &       0.110\sym{**} \\
\addlinespace
Gap, Attractiveness      &                     &      -0.367\sym{***}&                     &      -0.350\sym{***}&      -0.358\sym{***}\\
\addlinespace
Gap, Body Language-IT    &                     &       0.132         &                     &       0.106         &      0.0874         \\
\addlinespace
Gap, Conscientiousness   &                     &     -0.0845         &                     &      -0.132\sym{**} &      -0.134\sym{**} \\
\addlinespace
Gap, EQ                  &                     &     -0.0952         &                     &                     &                     \\
\addlinespace
Comparative, Attractiveness&                    &                     &      -0.185\sym{*}  &                     &                     \\
\addlinespace
Comparative, Conscientiousness&                     &                     &      -0.140         &                     &                     \\
\addlinespace
Comparative, Customer Service&                     &                     &       0.138         &       0.142\sym{*}  &       0.145\sym{*}  \\
\addlinespace
Comparative, EQ          &                     &                     &     -0.0955         &                     &                     \\
\addlinespace
Comparative, Willing to Work Odd Hours&                     &                     &      -0.177\sym{*}  &      -0.255\sym{***}&      -0.260\sym{***}\\
\addlinespace
Comparative, Teamwork    &                     &                     &      -0.196\sym{*}  &      -0.242\sym{**} &      -0.251\sym{**} \\
\addlinespace
Comparative, Written Communication&                     &                     &       0.128         &      0.0920         &      0.0934         \\
\addlinespace
Comparative, Rulebreaker &                     &                     &                     &                     &      0.0182         \\
\addlinespace
Gap, Rule Breaker        &                     &                     &                     &                     &      0.0574         \\
\addlinespace
Constant                 &       5.036\sym{***}&       5.356\sym{***}&       4.755\sym{***}&       5.327\sym{***}&       5.343\sym{***}\\
\midrule
R-sqr                    &      0.3253         &      0.3539         &      0.3310         &      0.3706         &      0.3721         \\
p(F)                     &      0.0000         &      0.0000         &      0.0000         &      0.0000         &      0.0000         \\
N                        &         322         &         322         &         322         &         322         &         322         \\
\bottomrule
\multicolumn{6}{l}{\footnotesize Standard errors in parentheses}\\
\multicolumn{6}{l}{\footnotesize \sym{*} \(p<0.10\), \sym{**} \(p<0.05\), \sym{***} \(p<.01\)}\\
\end{tabular}
}

    }
    \label{tab:table_new_ols}
\end{table}

\section{Conclusions}

In David Blake's approach / Degreed's Approach skills are 1-8 and there is no notion of 'overqualification' (for better or worse)
https://degreed.com/skill-certification
(in this idea, overqualified candidates are qualified; discounts overqualification as detrimental,
ie hiring manager doesn't want to hire a report with many years of mgr experience)
...
The Expertise Economy
measure skill gap as skills quotient: https://www.expertiseeconomy.com/speaking

...

% Out of scope for this paper, but important:
% 1. aggregate social, legal, political, and economic movements (aggregate study is wanting, we know states, time, industry all matter)
% 2. applicant personal effects, and interviewer-applicant interaction effects
% despite those caveats, we can reasonably explain employer willingness to hire their imagined candidate based on matching effects

Notice that the alternatively credentialed individual doesn't need the average employer to value him or her.
He or she simply needs some significant chance of being hired, and that certainly exists.
Moreover, the average employer is already favorable to alternative credentials.
As more alternatively credentialed individuals are highered and promoted through society,
there is reason to think the number of opportunities afforded to alternatively educated individuals may grow.
The problem doesn't seem to be about whether alternative credentials work, but whether they exist in a given industrial context,
and whether an individual would like to pay the college premium for better favorability when both options are feasible.

\bibliography{./BibFile}

\end{document}
