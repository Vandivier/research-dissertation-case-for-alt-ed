% using Elseveir template per https://www.elsevier.com/authors/author-schemas/latex-instructions
\documentclass[review]{elsarticle}

\usepackage{lineno,hyperref}
\modulolinenumbers[5]
% \journal{Journal of \LaTeX\ Templates}
\bibliographystyle{elsarticle-num}

\usepackage{booktabs}
\usepackage{graphicx}
\graphicspath{{../alt-ed-survey/figures-and-tables}}
\usepackage{hyperref}
\usepackage{threeparttable}  
\usepackage{tikz}
\usetikzlibrary{calc,matrix}

\begin{document}

\begin{frontmatter}

    \title{
        Conformity and Soft Skills as Determinants of Alternatively Credentialed Non-College Graduate Hireability
    }

    \author[mymainaddress]{John Vandivier}
    \address[mymainaddress]{4400 University Dr, Fairfax, VA 22030}
    \ead{jvandivi@masonlive.gmu.edu}

    \begin{abstract}
        % el author guidelines: https://www.elsevier.com/journals/economics-letters/0165-1765/guide-for-authors
        % heswbl author guidelines: https://www.emeraldgrouppublishing.com/journal/heswbl#author-guidelines
        % While you are welcome to submit a PDF of the document alongside the Word file, PDFs alone are not acceptable.
        % LaTeX files can also be used but only if an accompanying PDF document is provided. Acceptable figure file types are listed further below.
        % Articles should be between 5000  and 7000 words in length. This includes all text, for example, the structured abstract, references, all text in tables, and figures and appendices. 
        % Please allow 280 words for each figure or table.
        Despite targeting technical skills,
        vocational school graduates are paid less than college graduates.
        This paper hypothesizes that nonconformity stigma and a perceived deficit in soft skills substantially explain reduced alternatively credentialed non-college graduate (ACNG) hireability.
        % This paper explores an original data set
        % to understand the influence of employer skill gap perception on the demand for labor.
        % Original survey data is investigated to 
        % Specifically, this study tests the hypothesis that nonconformity stigma is a key factor of reduced hireability for an alternatively credentialed non-college graduate (ACNG).
        Results from an original survey in the United States indicates that willingness to break rules is a key factor of hireability,
        but the direction of effect is heterogenous by employer type.
        ACNG job candidates tend to be perceived favorably as creatives or as possible high performers.
        Selection of traditional candidates is better explained as an employer risk aversion behavior,
        rather than selection for conformity as a direct property of quality labor.
        % Conformity selection is better explained as an employer risk aversion behavior rather than because they see conformity as means to enhanced productivity.
        Perceived skill gaps are more important than widely recognized factors of hireability including industrial and state effects.
        Soft skills are particularly important.
        Recent college graduates and ACNGs are seen as similarly lacking in soft skills including work ethic.
        The population of the United States systematically comparatively devalues alternative postsecondary education.
        Results collectively indicate that nontraditional postsecondary education is undervalued.
        % Results collectively indicate that nontraditional postsecondary education is more valuable than would be expected in the absence of such results.
        % Marginally increased social demand would be socially beneficial.
    \end{abstract}

    \begin{keyword}
        education economics, alternative education, candidate fit, job fit, candidate matching     %%% not grammatical
        \MSC[2010] I21, I22, J20                                                                   %%% not grammatical
    \end{keyword}

\end{frontmatter}

\pagebreak
\linenumbers

\section{Introduction}

A substantial gap exists between the skills expected by employers and those possessed by college graduates\cite{mcgarry2016examination, malik2017great, abbasi2018analysis, gingras2000there}.
Experts view college alternatives including vocational school as useful for technical training, but the traditional college degree retains a wage premium over vocational education.
Unemployment, underemployment, and other negative labor outcomes follow a similar pattern\cite{smith_2011}.
This paper maintains the orthodox view that employers pay for perceived job candidate skill.
% or expected marginal revenue product of labor
To explain inferior labor outcomes,
this paper tests the hypothesis that employers expect an offsetting non-technical skill deficit when considering an alternatively credentialed non-college graduate (ACNG).

Alternative credentials refer to credentials other than the undergraduate degree\cite{brown2017complex}.
The category includes, for example,
industry certifications,
portfolios of work,
% transcripts of accredited or unaccredited course work,
digital badges, and other records of unaccredited learning and achievement.
Individuals pursuing alternative credentials typically intend to leverage the credential toward better employment.
That is, they typically have the same goals as a college student.
Many individuals obtain alternative credentials as a supplement to the college degree.
Such a situation is pareto-superior to degree attainment alone and is therefore intentionally excluded from analysis.
This paper focuses on factors of ACNG hireability in order to validate whether ACNG labor outcomes are a general problem,
or perhaps a problem limitted in scope to a specific set of skills.
If the gap is limitted to a particular set of skills, such as soft skills, then alternative credential suppliers could modify their credentialling requirements to overcome the outcome deficit.

% MAYBE TODO: could summarize results here. zb: IDK if it's a good habit or not, by I like to sprinkle my conclusions throughout the intro - this ain't a mystery novel.

% Alternative credentials can be obtained quickly and cheaply relative to college.
% Obtaining a college degree signals intelligence, conscientiousness, and conformity,
% but it may not signal technical skill\cite{horton_2020}.
% Alternative credentials signal technical skill.
% As such, they provide an effective supplement to the college degree.

% This paper is concerned with another use case in which the college degree is entirely substituted.
% In that situation, employers may apply a noncollege stigma.
% This is particularly the case for roles which are typically occupied by degree holders.
% Noncollege stigma is a presumption, expectation, or bias toward perception of a skill gap of a certain kind.
% Whether the gap exists in fact is out of the scope of the present paper.

% Technical skill generally implies intelligence.
% Alternative credentials, then, fail to signal two qualities compared to the college degree.
% Alternative credentials fail to signal conscientiousness and conformity.
% Interestingly, some employers may demand some level of nonconformity.
% Employers may also presume a certain lack of soft skill on the part of highly technical applicants.
% % Finally, employers may use alternative credentials as a proxy for other employee characteristics like income, education, race, and gender.

% % a missing link for future research: favorability only correlates with actual hiring decisions it isn't a hiring decision
% Hiring decisions reflect boundedly rational demand for skilled labor.
% The college degree and alternative credentials provide two qualitatively distinct signals of skilled labor.
% The hireability of individuals in possession of these credentials has been studied,
% but the underlying determinants are not clear.
% This paper hypothesizes that perceived skill gaps are important determinants of willingness to hire.
% This paper further hypothesizes that perceived skill gaps are qualitatively different between college graduates and others.
% % In particular, this paper hypothesizes that a noncollege stigma is obtained for candidates without a degree in pursuit of roles typically filled by degree holders.
% % soft skill bias in particular

% three interesting follow-on questions:
%   1. do employers have such a bias
%   2. is such a soft skills gap presumption actually true
%   3. if true, due employers overvalue the soft skills gap
% related paradox: most people won't be in a job for 4-5 years,
% so why do they need to show conscientiousness and conformity towards the 4-5 year bachelor's goal line?
% hard skill stigma: in my experience, people who are highly technical are hard to work with
% soft skill bias: I am favorable bc i think u have soft skills (and maybe this is efficient...enter eq/iq discussion)

% This paper tests the hypothesis that there is a lack of willingness to hire an ACNG (ACNG) is explained by an offsetting perceived lack in non-technical skills.
% In particular, this paper hypothesizes that ACNGs are seen as nonconformist and lacking in soft skills or non-technical skills.
% These deficits explain why an ACNG would not be a preferred source of labor in many cases,
% even if such a candidate does possess superior technical skill.

% technical skill has negative coefficient but magnitude and reliability (p-value/variance) are weaker; overall, less important effect
% hypothesis stems from signalling model.
% This paper proposes skill gaps are perceived in particular among soft skills for alternatively educated individuals.
% one might argue employers are mistaken here; technical work may involve higher, not lower, conscientiousness; ya maybe but out of scope.
% that is, we test social stigma and skill-level / decomposed stigma; an application of the signlaing model.

% Experts view college alternatives including vocational school as useful for technical training, but the traditional college degree retains a wage premium over vocational education.
% This paper hypothesizes that employers pay for skill.
% As a result, lower wages for technically skilled individuals are hypothesized to derive from an offsetting perception of skill deficit elsewhere.
% That is, this paper hypothesizes that employers view an ACNGs (ACNG) as lacking in soft skills.
% This paper hypothesizes that employers expect a skill deficit, although not a technical skill deficit, 
% This expected deficit explains the variance in labor outcomes.

% Sustained rising costs to higher education motivate periodic review of the return to the college degree.
% Despite rising costs, Americans have become more educated than expected over the past decade.
% Trades have contemporaneously seen a labor shortage.

% actually, trade school enrollment is increasing faster than undergraduate enrollment
% https://www.chronicle.com/newsletter/the-edge/2020-01-22
% By 2020, They Said, 2 Out of 3 Jobs Would Need More Than a High-School Diploma. Were They Right?
% overinvestment in college seems to cause a technical labor shortage, but the market is compensating by enrolling more technical folks too
% https://www.theatlantic.com/education/archive/2019/03/choosing-trade-school-over-college/584275/
% undergraduate enrollment has slowed recently and many employers have dropped the college degree requirement
% https://www.npr.org/2019/12/16/787909495/fewer-students-are-going-to-college-heres-why-that-matters
% it is not the case that employers are increasingly demanding the college degree, but it is the case that many do today. Let's examine their reasons.
% peak college?

% Faster and cheaper alternatives to college exist, but high schools prefer immediate college enrollment over alternative options at a rate of nearly 2:1.
% A New U: Faster + Cheaper Alternatives to College

% Faster and cheaper alternatives to college exist, but the typical student prefers to immediately enroll in college.
% Five student-oriented explanations include an inflated perception of the return to college,
% lack of awareness about alternative programs,
% social pressure to pursue college over alternatives,
% inability to confidently compare returns to alternatives,
% and risk aversion which favors college as a low-risk option despite high cost.

% students overestimate wage gap and employability gap college v technical school
% https://www.bigrentz.com/blog/trade-school-survey
% https://www.manufacturing.net/labor/news/13250675/survey-majority-of-students-underestimate-trade-school

% TODO: long paper food...insert model section header and expand mathematically on the signalling model vs the human capital model

% An undergraduate degree is a historically reputable investment.
The signalling model has become one of the two standard explanations of the value of the college degree.
Signalling theory provides three advantages over human capital theory for the purposes of the present study.
First, signalling theory is able to explain labor outcome variance when human capital is held constant.
% First, signalling theory is able to explain labor outcome variance across labor types when skills are totally equal.
% Under a human capital model, in contrast, a variety of labor outcomes would directly imply variance among input labor.
% The present paper expects that skills for the ACNG compared to other labor types are not totally equal, but this must emerge as a result rather than a presumption.

Second the signalling model empowers a survey research design.
In an idealized human capital model, the measures of human capital would correspond to production process inputs.
To establish a wide array of skill measures would be complicated and prone to measurement sensitivies, assumptions, and errors of various and potentially subtle kinds.
Survey measures could be used as a second-best proxy, but they would never be an ideal measure of human capital.
Signaling theory takes the reverse approach.
According to the signalling model, labor demand is formed on the basis of job candidate value as perceived by an employer.
Whether this corresponds to any concrete ability is secondary.
Employer perception can be assessed through a simple survey.
An additional benefit of using a questionnaire is the ability to ask hypothetical questions.
In pondering hypotheticals, employer evaluation of a credential or signal can be isolated from job candidate human capital variance.

Third, signalling theorists have already laid out a testable hypothesis for weak labor outcomes among non-college graduates.
Following this model, scholars claim that the college degree signals intelligence, work ethic, and conformity\cite{caplan2018case}.
Non-traditional education, in contrast, is hypothesized to signal nonconformity.
Non-traditional courses can also be completed in a shorter span of time and with reduced entry qualifications relative to the traditional degree.
For this reason, alternative credentials are thought not to signal concientiousness, or work ethic.

% human capital is hella outdated; as of milton friedman we know that perceived human capital matters not actual human capital
% signalling is just a mature version of this old neoclassical notion.
% the idea that the human capital model is the classic neoclassical explanation is just a mistake

% nonconformist intuition is structural; by virtue of not doing that which is traditional they signal a nonconfomirst quality.

% Proponents of the signalling model often prefer employer-oriented explanations of college enrollment.
% In this explanation, employers prefer college graduates because the college degree signals intelligence, conscientiousness, and conformity.
% While a technical credential signals intelligence and technical skill, the absence of a degree yields a perceived gap in the mind of the employer.
% There is a perceived comparative lack on the part of the non-college graduate with respect to conscientiousness and conformity.
% This paper tests this hypothesis.

% An agent-based explanation would be that high school graduates are not taught about these alternatives.
% The college degree is popular, has a well-defined return, and is low in risk.
% Particular alternative programs are obscure and often lack a well-defined return.

% Other research indicates that concientiousness and conformity are not always desirable labor qualities.
% There is some reason to doubt the hypothesis that lower perceived value is attributable to signaling differences in concientiousness and conformity.
% Research indicates that extreme values for either factor in either direction may be detrimental to productivity.

% TODO: long paper food...uncomment below section as it implies we should be doing marginal analysis. Then do marginal analysis, and K*K skill interactions
% TODO: maybe move to results section when we talk about concientiousness
Research indicates a goldilocks level or bliss point for both concientiousness and conformity is likely to exist.
Excess individual concientiousness can disturb team performance\cite{curcseu2019personality}.
Conformity can lead to a lack of innovation and suboptimal organizational practices\cite{symon2006neglected}.
Psychologists also state that conformity selection may occur through heuristic decisioning rather than conscious choice.

% Because a single measure operationalizes each of these effects and their own negation, a fixed sample size is relatively unlikely to identify an important coefficient.
% Because these factors are sometimes demanded and in other cases the inverse is demanded, a factor coefficient may be harder to identify and may only represent the average effect, even if the average effect is hardly predominant in practice.

% The psychological problem is related to but distinct from the pure logical problem that a totally conformed mind is necessarily incapable of innovation.
% Firm innovation requires an underlying capacity for individual innovation.
% Firms must have some capacity for innovation to sustain profit in a growing economy.
% Even if conformity selection is a correct explanation of ACNG aversion, then, it may not be an ideal practice when viewed through the lense of technical or economic efficiency.
% Risk aversion is compatible with a sometimes-concious, sometimes-heuristic decisioning model.

% Innovators, leaders, and high-performers are three kinds of virtuously nonconformant labor.
% Because conformity is sometimes undesirable and sometimes desirable,
% the effect may neutralize itself in an ordinary least squares stastistical analysis.
% The effect may not be identified as important or significant in any particular direction.

Risk aversion is another explanation for conformity selection.
An employer may not be able to value an alternative credential.
From the point of view of such an employer, an ACNG may range in value from a positive outlier to a negative outlier.
The employer may not prefer to hire such a candidate on the basis of risk aversion,
even if their point estimate for ACNG labor value is higher than their point estimate for a recent college graduate.
% To preview a particular result, this paper finds that large employers are particularly willing to hire an ACNG.
If employers with many employees are positively associated with ACNG hireability, this will add weight to an explanation based on risk aversion.
% Reasons for this include: 1) failure to deliver can be catastrophic, so low performers may be disproportionately untolerable.
% Revenue consistency, timeline, reputation, quantity produced targeting, large min skill labor cost and marginally small pay increase to achieve adequate production.
% Performance monitoring and turnover costs reinforce this
% zb states I'm assuming constant cost per hire...actually I allow that some portion of large-firm favorability is due to better ability to distinguish low vs high
% other than that effect, the remainder would be attributable to risk aversion.
% yes, the null hypothesis is no difference in cost per hire; but I can also support that which I suspect...turnover costs are proportionally smaller for large firms
% these firms are able to specialize and economize in hiring, plus they are more likely to have high-skill candidates that can better interview and recruitment tech+processes that scales
% TODO: long paper food... consider below articles and flesh out the risk aversion to firm size interaction thing
% note turnover cost calculation is complex but we proxy of just cost to hire. A + B below support cheaper for large firms. (C reinforces B)
%   A) "As stated in a study by the National Association of Colleges and Employers, hiring an employee in a company with 0-500 people costs an average of $7,645."
%   B) "Another study by the Society for Human Resource Management states that the average cost to hire an employee is $4,129, with around 42 days to fill a position."
%   C) "According to Glassdoor, the average company in the United States spends about $4,000 to hire a new employee, taking up to 52 days to fill a position."
% related but doesn't solve the issue: https://builtin.com/recruiting/cost-of-turnover
% https://toggl.com/blog/cost-of-hiring-an-employee
% The highest performing employers, however, will be able to distinguish desirable from undesirable candidates within the unconventional pool.
% Risk aversion varies naturally among firms. <- probably don't write this line in paper as a reviewer can always posit there is a further reason you are missing
% Some employers that are high in risk aversion will provide a net preference to ACNG due to nonconformity preference.

% TODO: long paper food...below section is part of intro or perhaps model...
% \subsection{Process Explanations of Suboptimal Wages}

% Basic price theory holds that an employer should pay wages equal to the marginal revenue product of labor.
% In the real world, measuring candidate productivity at hiring time is costly and imperfect.
% % This produces a technical error which assumes alignment between the goals of the firm and a hiring team.
% A further issue is identified when the hiring team is scrutinized for principal-agent problems.
% The hiring team is composed of individuals with preferences, calculative limitations, and other biases.
% Monitoring and correcting for these problems is expensive,
% so firms will heterogenously realize some aggregation of these individual definiciencies.

% Exacerbating the already necessarily imperfect hiring process are candidate-side problems.
% Firms must hire among a finite, potentially small, number of candidates.
% Risk aversion to time expense and other search costs may lead a firm to approve a suboptimal candidate\cite{simon1976substantive}.
% In some cases, candidate pools may be systematically problematic.
% In law and medicine, for example, extensive education and training are legally required.
% These policies further restrict the candidate pool, inflate expected wages, and systematically alter the content of education in a politically-motivated manner.
% Market forces implement hiring as a lumpy expenditure process to begin with, but certification requirements, wage regulation, and other policies extend the problem.

% The prior discussion highlights many locations of hiring process inefficiency.
% The practical importance of magnitudes and kinds of such effects are described in a legion of related papers.
% A meager sampling of five such effects would include the attractiveness effect and many other issues related to gender bias\cite{quereshi1986physical},
% agentic behavioral stigma\cite{steffens2009feminization},
% and complex biases related to communication style\cite{brouer2017gender, nijs2019effects, sampugnaro1983nonverbal}.
% Sung et al find that impression management meaningfully weakens disability stigma\cite{sung2017disclose}.
% These tactics are transferable in part to noncollege stigma mitigation.
% Finally, there are a wealth of concerns about the effects of social media.
% For one, it presents a channel for the revival of religious discrimination\cite{esposito2018signalling}.

% In the face of so many important inefficiencies, one begins to wonder whether the original theory holds any water at all.
% Papers which identify matching effects, including the present paper,
% serve to limit the proportion of explanation attributable to bias and redeem the elementary price theory story to some extent.
% Prior work demonstrates the important of matching effects in the form of norm compliance\cite{francesco1981gender}.
% Meta-accuracy is a kind of matching measure, and it has been shown to move positively with hireability\cite{renier2018no}.

\section{Methodology}

% The method of this study begins with a decomposition of the main hypothesis into three simple statistical tests.
% This study applies the three-step method to the case of the ACNG, but the three-step method itself is sufficiently general to apply to any measurably distinct job candidate type.
% First, skill gaps are identified as independent factors with a general ability to explain willingness to hire.
% Second, the coefficient for a
% Finally, gaps in soft skills are hypothesized to be more important than technical skill gaps.
% If the above three conditions hold, ACNG labor dispreference can be explained with or without comparing labor outcomes to the college graduate.

% Responses were submitted using the SurveyMonkey web application.
This study uses ordinary least squares regression analysis to estimate the effect of perceived skill gaps on willingness to hire.
Perceived skill gaps and willingness to hire are included in original response data collected by online survey ($n = 212$).
Observations are cross-sectional and taken at the individual level.
The data is available for replication or any other use\footnote{
    See \url{https://osf.io/8qtxf/?view_only=95b0c0b0c65e4b7983198cc87c2ab733}
    for data used in this study.
}.

Respondents were obtained through the Amazon Mechanical Turk crowdsourcing service.
Respondents were United States citizens at or over the age of eighteen.
Opt-in respondents were paid for participation and selected on a first-come, first-serve basis up to a quota of $200$.
The survey administration took place in July of 2020.
% MAYBE TODO: I could explain that AMT has been shown by my other paper to be bias-free for this right hand param
% and that I purchased 225 samples on the basis of 1) over 100 for large numbers to kick in, and
% 2) experience with the other survey hinted this would be on the lower end of what was needed for significance.

The survey includes 65 questions in two sections.
The first section of responses describe the respondent.
There are 13 questions in this section.
The second section identifies perceived skill gaps for 13 skills.
% MAYBE TODO: citation to "make response anchoring appropriate"
Each section begins with a contextual message to normalize response anchoring.
Questions are provided in nonrandom order for the same reason.
Appendix A provides a copy of the survey.
% MAYBE TODO: "Your data does not include an index that describes the data units/scale"

% TODO: should we give more description to section 1 questions? do we need to go through each factor...?
% for now I punt this to results section, so I only actually describe factors that have a substantial result.
% The dependent variable of interest is favorability to the hiring of an alternatively educated non-college graduate.
% missing industrial variables: hospitality (restaurant/hotels) and entertainment / media, and sales!

Data from the second section is used to calculate perceived skill gaps.
For each of 13 skills, the respondent is asked to imagine four types of candidates.
One type of candidate is an ideal candidate.
At a high level, skill gaps are calculated by differencing skill levels of the ideal candidate with others.

Perceived skill is reported on a scale from 1 to 10.
Perceived skill is reported for the ideal candidate,
the average actual employee,
the average recent college graduate,
and the average ACNG.
Each skill gap has two associated measures.
One measure allows for overqualification in a skill and the other does not.
Overqualification effects have been identified as important\cite{green2007there, raybould2005over}, but these effects are sometimes ignored during skill gap analysis\cite{blake_2018}.
% MAYBE TODO: cite more than 1 person who ignores overqualification

When overqualification is allowed, the skill gap is measured as a raw skill gap.
The raw skill gap is the skill level of the ideal candidate less the skill level of the actual candidate.
The skill gap without overqualification is calculated as the raw skill gap or zero if the raw skill gap value is negative.

% begin low importance comment...
% this paper compares alt ed to ideal, but other papers could compare other sets:
% 3 different explanatory constructs are explored, but only the winner is reported in the paper:
% 1. alt ed to ideal
% 2. alt ed to typical [not interesting for this paper]
% 3. alt ed to college grad
% 4. alt ed to ideal without overqualification
% 5. alt ed to typical without overqualification [not interesting for this paper]
% 6. alt ed to college grad
% (horse racing): https://www.afterecon.com/economics-and-finance/kitchen-sink-regression-and-horse-racing/
% should probably randomly split sample and out-of-sample test with factors to combat overfit
% aggregate excess attractiveness by recent college grads against ideal.
% aggregate excess willingness to break rules by alt ed noncollege grads.
% many non-aggregate, or respondent-level, cases of alt ed overqualification; in fact, some such responses for every question kind (the 13 types)
% Optional but interesting: college grad to ideal or college grad to alt ed; so that we can indirectly associate favorability to actual propensity to hire. (which we have for college grads)

% objective of analysis
% how much does matching explain (caveat: not multiply regressed, so matching effect is likely overstated in this paper, and possibly partially partialled-in to prior work)
% does noncollege stigma exist
% "alternative education is different how?"
% 'explaining favorability'

% simple match effect: those that prefer technical talent will tend to support alternative credentials.
% complex match effect: a match profile will have significantly and importantly more explanatory power compared to but consistent with a simple match effect.

% quality question meta: 1 to 10: disagree to agree
% ---
% An ideal candidate would have this quality...
% A typical employee would have this quality...
% A college graduate would have this quality...
% A credentialed or certified non-college graduate would have this quality...
% [later] Someone who is self-taught (without a credential or portfolio) would have this quality...
% [later] A typical junior-level high school student would have this quality...

% some notes, mainly out of scope
% ---
% hiring error awareness increases in a few ways
%  1. [passive search] participant observation. As an interviewer, interviewee, hiring manager, or other professional involved in the process, I simply notice a problem
%  2. [passive search] passive company and individual level search into HR best practices; an industry newsletter says hey Griggs v Duke happened so don't use IQ tests anymore.
%  3. [passive search]: audit compliance (legal+required, or optional audits from firms that certify quality, for example)
%       example: Supreme Court case Griggs v Duke had an industry-wide effect thru this means
%  4. [active search] intrapraneurship / policy change championing begins with an individual saying hey let's investigate this thing. what would motivate such individual? (maybe due to 1 or 2).
%
% my prior work has shown that we can predict (r2 0.5 - 0.7 and ar2 0.3 - 0.6) alternative education favorability from employer factors alone - without concern to matching

\section{Results}

% TODO: long paper food...
% Big size TODO:
% 1. use ordinal regression, not ordinary least squares regression (which would use cardinal values)
%   a. decent paper on ordinal independent variables: https://www3.nd.edu/~rwilliam/stats3/OrdinalIndependent.pdf
% 2. for all skills, study marginal effects and K*K skill interactions (including rulebreaker questions)

% This paper acknowledges that own analysis proceeds through a technocentric lens.
% This is an important anchoring point for the analysis, and it may skew application of results in low-technology or low-skill sectors.
% The technocentric lens is an important caveat and anchoring point, but I argue that it is about as proper as any anchoring point.
% In economics, after all, technology operationalizes the theory of innovation per se.
% All skills can be viewed as point-in-time innovations, so that if there was no innovation then neither would there be a need for any skill.
% By the same token, a technocentric lens at the present seems close to a cross-industry lens at a future time.
% Anchoring to any other industry would be both asymmetric and unusuful in the future.
% Perhaps this analysis is slightly skewed, but at least it is skewed only against the past, and will be increasingly useful in the future without partiality to any particular industry.
% In addition, we did check for industrial effects, but the analytical skew may persist pass the data.

% This orientation occurs because New Alternative Education first flourished for roles in the information technology sector, and only later did roles like sales, business, art, nursing, and more join in.
%     [can refer to my New Digital Education] - https://papers.ssrn.com/sol3/papers.cfm?abstract_id=3530647

% Compare directly to bootcamp results from Indeed: https://www.indeed.com/lead/what-employers-think-about-coding-bootcamp

ACNG (ACNG) hireability was generally positive.
The mean response was 7.5 on a scale from one to ten ($\sigma = 1.80$).
Employer status was not associated with a significant response effect.
Perceived skill gaps explained a significant and important portion of hireability variance.

Table \ref{tab:explantory_power} compares perceived skill gap explanatory power in a simple regression
to explanatory power of other simple regressions involving factors of known relevance to hireability.
Allowing for overqualification seems to weaken explanatory power.
Overqualification effects seem to be heterogeneously signed per skill, so generalizing weakens overall explanatory power relative to ignoring these effects.
With overqualification, perceived skill gaps explain about fifty percent more than industrial effects or robust state effects.
Without overqualification, the adjusted explanatory power of perceived skill gaps is about three times the adjusted explanatory power of industry or state effects.
Semi-robust state factors are dummy variables by state which are significant in a multiple regression.
Robust state factors are subset of semi-robust state factors which are additionally significant in a simple regression.

Table \ref{tab:explantory_power} also describes the explanatory power of so-called rulebreaker effects.
Whether the candidate is perceived as a rule breaker is a perceived skill gap, but employers evaluate this gap in a heterogenous and multispecific way.
In the first place, this heterogenous evaluation has sign and magnitude implications for the dependent variable of interest.
Secondarily, heterogenous evaluation implies a qualitatively different evaluation.
These differences are captured using three questions in the first section of the survey.

% TODO: should this table include k, the number of independent factors per factor group?
\begin{table}
    \caption{Factor Group Explanatory Power in a Simple Regression}
    \resizebox{\columnwidth}{!}{
        {
\def\sym#1{\ifmmode^{#1}\else\(^{#1}\)\fi}
% \begin{center}
{
    \fontsize{8pt}{7pt}\selectfont
    % \begin{small}
    \tabcolsep=3pt
    \begin{tabular}{l*{4}{c}}
        \toprule
        \multicolumn{1}{c}{Effect Group} & \multicolumn{1}{c}{Adj R-Sqr} & \multicolumn{1}{c}{R-Sqr} & \multicolumn{1}{c}{Max p-value} \\
        \midrule
        Absolute Gap                     & 0.0615                        & 0.0703                    & 0.097                           \\
        \addlinespace
        Comparative Gap                  & 0.0176                        & 0.0298                    & 0.687                           \\
        % \addlinespace
        % Rulebreaker                           & 0.1432                        & 0.1554                    & 0.053                           \\
        \addlinespace
        Industry                         & 0.0303                        & 0.0454                    & 0.958                           \\
        \addlinespace
        \addlinespace
        Other Factors                    & 0.0072                        & 0.0288                    & 0.537                           \\
        \addlinespace
        Rulebreaker                      & 0.0783                        & 0.0869                    & 0.127                           \\
        \addlinespace
        State                            & 0.0469                        & 0.1033                    & 0.772                           \\
        % \addlinespace
        % State, Semi-Robust                    & 0.0034                        & 0.0648                    & 0.831                           \\
        \bottomrule
    \end{tabular}
    % \end{center}
}
}

% TODO: maybe a count of k factors in group
% TODO: maybe distinguish strong and weak effects for industry, state, and gaps
% TODO: maybe other controls / other factors section doesn't matter
% TODO: maybe combine skill gaps

    }
    \label{tab:explantory_power}
\end{table}

The three rulebreaker questions measure respondent agreement on a scale from 1 to 10 with statements about rulebreakers, or "People who are willing to break formal or informal rules and norms."
The first statement indicates that rulebreakers present a risk to the reputation, productivity, or value of a company.
This statement received the least agreement and greatest response variance among three qualitatively different descriptions of people that are willing to break rules ($\mu = 6.40, \sigma = 2.55$).

The second statement states that people break rules which hold them back, and that rulebreakers "could just as easily be high performers as low performers."
% MAYBE TODO: standard error instead of or in addition to standard deviation. maybe make a table since there are at least three cases of similar report.
This statement received the most agreement and least comparative response variance as a rule breaker description ($\mu = 7.42, \sigma = 1.91$).
The agreement with this statement provides evidence against the thesis that employers value conformity for its own sake.
In turn, this adds weight to the theory that employers value conformity as a risk aversion tactic, but they actually believe nonconformity may signal positive outlier potential.
The third description of rulebreakers states that they tend to be gifted in the areas of innovation or creativity, and that such people may benefit the culture of a company ($\mu = 7.25, \sigma = 2.03$).

Each of the three rulebreaker effects turn out to be independently important.
These effects collectively explain more than three times as much response variance as do industrial or state effects.
Rulebreaker effects are about twice as important as perceived skill gaps.
This is not taken to be a general lesson about skill gaps.
Willingness to break rules is a special case of a behavior which is valuable under certain parameters.
This contrasts with something like poor communication skill which is generally detrimental.

Table \ref{tab:table_new_ols} reports selected factor statistics across five least squares multiple regressions.
The selected factors which are reported include any perceived skill gap which is important in any specification.
Factor importance is determined by the ability of a factor to improve model adjusted explanatory power.
Model 1 is a multiple regression using skill gaps that allow for overqualification.
Model 2 is a multiple regression without overqualification.

% n=212
\begin{table}
    \caption{Table of Multiple Regression on Favorability, Selected Variables}
    \resizebox{\columnwidth}{!}{
        % derived from analysis-5-regs-table.do
{
\def\sym#1{\ifmmode^{#1}\else\(^{#1}\)\fi}
\begin{tabular}{l*{5}{c}}
\toprule
                         &\multicolumn{1}{c}{1}&\multicolumn{1}{c}{2}&\multicolumn{1}{c}{3}&\multicolumn{1}{c}{4}&\multicolumn{1}{c}{Model 5}\\
\midrule
Is Employed Non-Manager  &      -0.336         &      -0.383\sym{*}  &      -0.497\sym{**} &      -0.471\sym{**} &      -0.451\sym{**} \\
\addlinespace
Is STEM Worker           &      -0.491\sym{**} &      -0.529\sym{**} &      -0.525\sym{**} &      -0.557\sym{**} &      -0.564\sym{**} \\
\addlinespace
Employees 51-200         &      -0.475\sym{*}  &      -0.480\sym{**} &      -0.364         &      -0.459\sym{*}  &      -0.457\sym{*}  \\
\addlinespace
Industry Credentials Legally Required&       0.706\sym{*}  &       0.722\sym{**} &       0.374         &       0.378         &       0.375         \\
\addlinespace
Industry Credentials Normal&       0.932\sym{**} &       0.926\sym{**} &       0.487\sym{*}  &       0.436\sym{*}  &       0.448\sym{*}  \\
\addlinespace
Industry Credentials Sometimes Used&       0.467         &       0.475         &                     &                     &                     \\
\addlinespace
Industry Credentials Unknown&       0.641\sym{*}  &       0.684\sym{**} &                     &                     &                     \\
\addlinespace
Industry, Agriculture    &       1.368         &       1.619\sym{*}  &                     &                     &                     \\
\addlinespace
Industry, Energy         &      -1.277\sym{*}  &      -1.190\sym{*}  &      -1.200\sym{*}  &      -1.442\sym{**} &      -1.448\sym{**} \\
\addlinespace
Industry, Finance, Investment, or Accounting&      -0.811\sym{***}&      -0.783\sym{***}&      -0.712\sym{***}&      -0.715\sym{***}&      -0.717\sym{***}\\
\addlinespace
Industry, Information Technology&       0.335         &       0.264         &       0.438\sym{*}  &       0.306         &       0.337         \\
\addlinespace
Industry, Law            &      -1.813\sym{***}&      -1.670\sym{**} &      -1.935\sym{***}&      -1.876\sym{***}&      -1.857\sym{***}\\
\addlinespace
Industry, Transportation &       1.512\sym{*}  &       1.643\sym{**} &       1.216         &       1.403\sym{*}  &       1.350\sym{*}  \\
\addlinespace
State, Arizona           &      -1.157\sym{**} &      -1.048\sym{**} &      -0.755         &      -0.823\sym{*}  &      -0.790         \\
\addlinespace
State, Arkansas          &      -2.690\sym{***}&      -2.817\sym{***}&      -2.489\sym{***}&      -2.664\sym{***}&      -2.770\sym{***}\\
\addlinespace
State, California        &      -0.575\sym{*}  &      -0.570\sym{**} &      -0.488\sym{*}  &      -0.435         &      -0.446         \\
\addlinespace
State, Colorado          &      -1.446\sym{**} &      -1.423\sym{**} &      -1.463\sym{**} &      -1.521\sym{***}&      -1.508\sym{***}\\
\addlinespace
State, Connecticut       &      -1.401         &      -1.550         &                     &                     &                     \\
\addlinespace
State, Florida           &      -0.444         &      -0.454         &                     &                     &                     \\
\addlinespace
State, Hawaii            &      -3.232\sym{***}&      -3.271\sym{***}&      -2.884\sym{***}&      -2.869\sym{***}&      -2.812\sym{***}\\
\addlinespace
State, Illinois          &      -0.637         &      -0.699\sym{*}  &      -0.596         &      -0.675\sym{*}  &      -0.698\sym{*}  \\
\addlinespace
State, Kansas            &      -3.283\sym{**} &      -3.486\sym{**} &      -2.923\sym{*}  &      -3.116\sym{**} &      -3.101\sym{*}  \\
\addlinespace
State, Kentucky          &      -3.143\sym{***}&      -3.167\sym{***}&      -2.583\sym{***}&      -2.729\sym{***}&      -2.679\sym{***}\\
\addlinespace
State, Louisiana         &      -1.455\sym{*}  &      -1.255\sym{*}  &      -0.915         &      -0.941         &      -0.898         \\
\addlinespace
State, Maryland          &      -0.596         &      -0.642         &                     &                     &                     \\
\addlinespace
State, Nebraska          &      -2.037\sym{*}  &      -2.167\sym{*}  &      -1.391         &      -1.655         &      -1.596         \\
\addlinespace
State, Nevada            &      -1.406         &      -1.470         &      -1.465         &      -1.434         &      -1.409         \\
\addlinespace
State, New Jersey        &      -1.145         &      -1.139         &      -0.976         &      -0.936         &      -0.963         \\
\addlinespace
State, New York          &      -0.692\sym{**} &      -0.640\sym{*}  &      -0.617\sym{*}  &      -0.595\sym{*}  &      -0.590\sym{*}  \\
\addlinespace
State, Ohio              &      -3.943\sym{***}&      -4.024\sym{***}&      -4.051\sym{***}&      -3.808\sym{***}&      -3.761\sym{***}\\
\addlinespace
State, Pennsylvania      &      -0.752         &      -0.687         &      -0.608         &      -0.534         &      -0.539         \\
\addlinespace
State, South Carolina    &      -1.183         &      -1.243         &      -1.361         &      -1.310         &      -1.347         \\
\addlinespace
State, Tennessee         &      -1.878\sym{**} &      -1.909\sym{**} &      -1.545\sym{*}  &      -1.843\sym{**} &      -1.799\sym{**} \\
\addlinespace
State, Texas             &      -0.906\sym{**} &      -0.851\sym{**} &      -0.797\sym{**} &      -0.790\sym{**} &      -0.789\sym{**} \\
\addlinespace
State, Washington        &      -0.817         &      -0.863\sym{*}  &      -0.880\sym{*}  &      -0.996\sym{**} &      -1.003\sym{**} \\
\addlinespace
Duration                 &       0.666\sym{**} &       0.634\sym{**} &       0.811\sym{***}&       0.744\sym{**} &       0.719\sym{**} \\
\addlinespace
cduration2               &     -0.0884\sym{**} &     -0.0857\sym{**} &      -0.113\sym{***}&      -0.103\sym{**} &     -0.1000\sym{**} \\
\addlinespace
WOQ, Gap, Attractiveness &      -0.161\sym{***}&                     &                     &                     &                     \\
\addlinespace
WOQ, Gap, Body Language-IT&       0.100         &                     &                     &                     &                     \\
\addlinespace
WOQ, Gap, Conscientiousness&     -0.0657         &                     &                     &                     &                     \\
\addlinespace
WOQ, Gap, EQ             &     -0.0966         &                     &                     &                     &                     \\
\addlinespace
Rule Breakers Risky      &      0.0732\sym{*}  &      0.0715\sym{*}  &      0.0880\sym{**} &      0.0747\sym{*}  &      0.0762\sym{*}  \\
\addlinespace
Rule Breakers Rockstars  &       0.133\sym{**} &       0.128\sym{**} &       0.147\sym{**} &       0.141\sym{**} &       0.140\sym{**} \\
\addlinespace
Rule Breakers Culture Add&      0.0905         &      0.0974\sym{*}  &       0.115\sym{**} &       0.112\sym{**} &       0.110\sym{**} \\
\addlinespace
Gap, Attractiveness      &                     &      -0.367\sym{***}&                     &      -0.350\sym{***}&      -0.358\sym{***}\\
\addlinespace
Gap, Body Language-IT    &                     &       0.132         &                     &       0.106         &      0.0874         \\
\addlinespace
Gap, Conscientiousness   &                     &     -0.0845         &                     &      -0.132\sym{**} &      -0.134\sym{**} \\
\addlinespace
Gap, EQ                  &                     &     -0.0952         &                     &                     &                     \\
\addlinespace
Comparative, Attractiveness&                    &                     &      -0.185\sym{*}  &                     &                     \\
\addlinespace
Comparative, Conscientiousness&                     &                     &      -0.140         &                     &                     \\
\addlinespace
Comparative, Customer Service&                     &                     &       0.138         &       0.142\sym{*}  &       0.145\sym{*}  \\
\addlinespace
Comparative, EQ          &                     &                     &     -0.0955         &                     &                     \\
\addlinespace
Comparative, Willing to Work Odd Hours&                     &                     &      -0.177\sym{*}  &      -0.255\sym{***}&      -0.260\sym{***}\\
\addlinespace
Comparative, Teamwork    &                     &                     &      -0.196\sym{*}  &      -0.242\sym{**} &      -0.251\sym{**} \\
\addlinespace
Comparative, Written Communication&                     &                     &       0.128         &      0.0920         &      0.0934         \\
\addlinespace
Comparative, Rulebreaker &                     &                     &                     &                     &      0.0182         \\
\addlinespace
Gap, Rule Breaker        &                     &                     &                     &                     &      0.0574         \\
\addlinespace
Constant                 &       5.036\sym{***}&       5.356\sym{***}&       4.755\sym{***}&       5.327\sym{***}&       5.343\sym{***}\\
\midrule
R-sqr                    &      0.3253         &      0.3539         &      0.3310         &      0.3706         &      0.3721         \\
p(F)                     &      0.0000         &      0.0000         &      0.0000         &      0.0000         &      0.0000         \\
N                        &         322         &         322         &         322         &         322         &         322         \\
\bottomrule
\multicolumn{6}{l}{\footnotesize Standard errors in parentheses}\\
\multicolumn{6}{l}{\footnotesize \sym{*} \(p<0.10\), \sym{**} \(p<0.05\), \sym{***} \(p<.01\)}\\
\end{tabular}
}

    }
    \label{tab:table_new_ols}
\end{table}

Models 3 and 4 are equivalent to models 1 and 2, respectively, after normalizing for industry, state, and company size effects.
These effects are normalized for robustness by retaining those factors which appear in both model 1 and model 2.
For example, some state effects are important in one specification and not in the other.
Such state effects are dropped in models 3 and 4.

Model 5 is specified as Model 4 plus two adjustments.
First, the factor for salary is dropped.
The salary factor improved adjusted explanatory power in Model 2, but it provided no such benefit in any other model.
Moreover, the p-value of this factor was unacceptably low in Model 4 ($p > 0.9$).

The second adjustment is to add a duration factor.
The duration factor is a measure of the length of time a respondent believes it takes to earn an alternative credential\footnote{
    Duration is a categorical variable which was important in both Models 1 and 2.
    As a categorical variable, it was decomposed into a boolean series for factor analysis.
    Models 1 and 2 retained one or more duration dummies, but none overlapped.
    As a result, duration was dropped from Models 3 and 4.
}.
The duration factor which indicates that the respondent believes it takes more than a year to obtain an alternative credential
is significantly and importantly associated with improved willingness to hire ($\beta = 0.875, p < 0.01$).

An individual is considered an employer if they state that they contribute to hiring and firing decisions.
Employer effects are positively signed in all five models, but the significance is lost after normalizing effects.
This suggests that employer favorability to alternative credentials is sensitive to industry, state of residence, and firm size, which are the normalized effects.

The preferred model is able to explain roughly one third of the willingness to hire.
Thirteen skill gaps were investigated, and seven contribute to the preferred model.
One of the skill gaps in the final model is technical skill.
The technical skill gap is statistically insignificant, but it is robust in sign across models and it does possess the expected negative sign.
The other six important factors are soft skills.
Perceived skill gaps in body language and work ethic are the most important factors in the model.
The relative importance of soft skill gaps, and work ethic in particular, adds weight to a revision of the usual signalling explanation as the most plausible story.

An important and complicated finding involves conscientiousness.
The effect is robustly positive in multiple specifications.
Simple intuition would indicate that a large conscientiousness gap is associated with reduced hireability.
A simple regression of conscientiousness on favorability does produce the expected negative coefficient.
There are two reasons for the sign change on work ethic in the multiple regression.
The first reason is bliss point seeking and the second is an explanation from attenuation.

As previously discussed, excess individual concientiousness can disturb team performance\cite{curcseu2019personality}.
It follows that there is some bliss point level of concientiousness.
As a result, the sign of the coefficient on concientiousness is sensitive to the relationship between the model constant and the bliss point.
If the model constant is above the bliss point, concientiousness is expected to be positively signed.
The current data is consistent with this explanation from bliss point seeking.
In a simple regression of work ethic in hireability, the model constant is about $7.5$, and the coefficient on work ethic is negative.
In the multiple regression, the model constant is near $0.5$, and the coefficient on work ethic is positive.

% Adding a quadratic term to Model 5 identifies a negative marginal relation between conscientiousness and teamwork skill.
% This replicates the other research which found a parabolic relationship between peer-rated team contribution and conscientiousness.
% In addition, conscientiousness exhibits a negative marginal relation with hireability overall.

% much of the below could be heavily edited or deleted pending hanson's feedback.
The second explanation is that the direct measure of work ethic is attenuating an overstatement of the effect in the multiple regression.
Conscientiousness is importantly cross-correlated with several factors including willingness to commute and customer service skill.
Conscientiousness also structurally interacts with rulebreaker effects.
In theory, a person that is high in concientiousness will tend not to break rules.
These factors are entered independently in the multiple regression, so a partial measure of conscientiousness is entered in redunantly.
The direct factor for conscientiousness corrects, or attenuates, the overstated effect which is represented in those the correlated independent factors.
Removal of other skill gap factors and rulebreaker factors from Model 5 demonstrates this by yielding a negative concientiousness coefficient ($\beta = -0.084, p < 0.31$).

The importance of conscientiousness does not add weight to the classic signalling explanation.
The conscientiousness gap is not a comparative gap between a recent college graduate and a non-college graduate.
It is a gap between an ideal job candidate and an ACNG.
An important note is that the conscientiousness gap among recent college graduates is statitistically no different from an ACNG.
Unsurprisingly, the candidate perceived to have a minimal conscientiousness gap is the typical employee already working in the labor force.
A comparative gap of note would be that college graduates were perceived to have better technical skills compared to the average ACNG.

Employer size was an important factor in the preferred model.
The largest category of employer is positively associated with willingness to hire an ACNG.
This matches the risk aversion model.
The largest category of employer has lower risk, and in fact generates comparative advantage, when hiring from a high-variance pool of candidate quality.

Some state and industrial effects are identified.
No particular relation among state effects was found, but further comparative policy research is encouraged.
With respect to industry, an interesting interaction between body language skill and employment in the information technology industry yields a positive coefficient.
Body language skill gaps on their own are associated with reduced willingness to hire.
This specifically indicates a reduced penalty for lacking body language communication skills in the information technology industry.
With less strength and more breadth, a positive coefficient to the interaction variable indicates a reduced penalty for generalized soft skill deficiency in the information technology industry.

A reduced penalty for soft skill deficit helps explain the particular flourishing of alternative credentials in the information technology industry.
The reduced penalty in this particular industry might be related to a relative lack of deregulation in the industry.
Another explanation is that the reduced penalty may be related to cultural norms in the industry.
% Alternatively, it might be related to cultural norms around the acceptability of an anti-social geek in a technology field.
% The cultural norm itself might be derived from deeper collective personality organization.
There is less technical need for social skill in programming, so introverts may naturally obtain a comparative advantage in this field.

\section{Conclusion}

This study provides evidence that skill signals are an important factor of hireability and are unique for the ACNG.
Perceived skill gaps do a better job of explaining willingness to hire than do other widely recognized effects including industry and state effects.
Employer factors better explain candidate hireability than do the perceived skill gaps themselves.
Technical skill gaps were identified with less relevance to the hiring decision when compared with soft skill gaps for the ACNG job candidate.

This paper provides evidence that some employers engage in conformity selection as a means of avoiding risk to the reputation, productivity, or value of a company.
% Conformity selection of this kind, however, is only a proximal explanation.
An explanation from risk aversion fully this kind of conformity selection and also explains other behavior.
Respondents were most favorable to the description of rulebreakers as individuals that could just as easily be high performers as low performers.
Aversion to this kind of labor is better explained as risk aversion rather than positive selection for conformity.

Risk aversion and conformity selection are both partially unconcious biases which lead to suboptimal organizational operation.
A practical recommendation is for organizations to implement bias controls with respect to ACNG evaluation.
An example control would be to provide human resource procedures for formal evaluation of particular credentials which are relevant to specified job families.
These procedures can be immediately executed among known credentials and job families.
% Adjustments can be applied to new hires or during performance reviews for current employees.
These procedures should be retained for ongoing application as new credentials are developed and encountered over time.

Some evidence on the role of misinformation is demonstrated in a survey on trade schooling taken in 2019\cite{arabia_2019}.
Only 27 percent of respondents correctly responded that lower debt is an advantage of enrolling in trade school relative to college.
Additionally, over 75 percent of respondents failed to notice that trade school graduates receive industry employment sooner
and receive specialized training when compared to a four-year college.
% The news that employers are generally favorable to alternative credentials should be shared far and wide.
% The current education system should be reformed to better inform students about non-college career entry.
Obtaining a college degree after obtaining some work experience will allow students to leverage employer tuition benefits.
% Government should emphasize job skills over the formal degree.
% Recent moves have begun such emphasis\cite{https://www.usatoday.com/story/news/politics/2020/06/26/trump-executive-order-stresses-skill-over-college-degree-hiring/3263074001/}

% Out of scope for this paper, but important:
% 1. aggregate social, legal, political, and economic movements (aggregate study is wanting, we know states, time, industry all matter)
% 2. applicant personal effects, and interviewer-applicant interaction effects
% despite those caveats, we can reasonably explain employer willingness to hire their imagined candidate based on matching effects

% Studying skill gaps provides an explanation with several gaps.
The preferred model explains about one third of willingness to hire.
Perceived skill gaps and rulebreaker effects account for most of the explanatory power in the model.
The explanatory power of this study can be meaningfully improved in a few ways.
This study uses a cross-sectional analysis to investigate a subject that varies over time.
The traditional system of accredited undergraduate education was itself at one time an innovation.
% Abnormal job candidates differentiated themselves using novel credentials, and broad employer adoption was achieved over time.
% This paper suggests that the alternative credentials of today will follow a similar pattern.
% also, role, firm, and industry-level supply, demand, employment, and so on...
Dynamic analysis would yield deeper understanding of such trends, achieve greater explanatory power, and provide better casual understanding.
% Dynamic analysis would also generate better opportunity for causal understanding.
Other research has conducted some dynamic analysis of the same dependent variable with different regressors\cite{vandivier2020preliminary}.
Integrated analysis would be useful for replication and the generation of new models of better explanatory power.
Specifications that allow for overqualification effects and heterogeneous nonlinear relations between skill gaps and hireability would improve not only the present paper,
but the state of the art in skill gap analysis.

If perceived skill is representative of actual skill,
then the current study concludes that employers should be more willing to hire an ACNG.
At the same time, this paper demonstrates plausible misalignment of perceived and actual skill in some cases.
Last mile training is a type of alternative education which has been specifically recommended as a remedy for technical skill gaps among recent college graduates.
It is surprising that the average recent college graduate in the sample is perceived to have better technical skill compared to the average ACNG.

Employers seem to be favorable to individuals with alternative credentials.
In many cases, employer-perceived skill gaps are not statistically different when comparing recent college gradutes with ACNG candidates.
Social preference for the college degree may be better explained by public ignorance about appropriate alternative programs,
a lack of appropriate programs for certain occupations,
and government financial and other policy which gives preference to accredited education.

% Notice that the alternatively credentialed individual doesn't need the average employer to value him or her.
% He or she simply needs some significant chance of being hired, and that certainly exists.
% Moreover, the average employer is already favorable to alternative credentials.
% As more alternatively credentialed individuals are highered and promoted through society,
% there is reason to think the number of opportunities afforded to alternatively educated individuals may grow.
% The problem doesn't seem to be about whether alternative credentials work, but whether they exist in a given industrial context,
% and whether an individual would like to pay the college premium for better favorability when both options are feasible.

\bibliography{./BibFile}

\end{document}
