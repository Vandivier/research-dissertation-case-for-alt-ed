% using Elseveir template per https://www.elsevier.com/authors/author-schemas/latex-instructions
\documentclass[review]{elsarticle}

\usepackage{amsmath}
\usepackage{lineno,hyperref}
\usepackage{booktabs}
\usepackage{hyperref}
\usepackage{siunitx}
\usepackage{tabularx}
\usepackage{threeparttable}  
\usepackage{tikz}

\bibliographystyle{elsarticle-num}
% \journal{Journal of \LaTeX\ Templates}
\modulolinenumbers[5]
\usetikzlibrary{calc,matrix}

\begin{document}

\begin{frontmatter}

    \title{
        Conformity and Soft Skills as Determinants of Alternatively Credentialed Non-College Graduate hirability
    }

    \author[mymainaddress]{John Vandivier}
    \address[mymainaddress]{4400 University Dr, Fairfax, VA 22030}
    \ead{jvandivi@masonlive.gmu.edu}

    \begin{abstract}
        % % Applied Economics author guidelines
        % % https://www.tandfonline.com/action/authorSubmission?show=instructions&journalCode=raec20#words
        % % Should contain an unstructured abstract of 200 words.
        % %   - exactly??? looking at top 3 articles in current issue, #44, abstract varies 172-212
        % %   - am treating 200 as a limit
        % % Graphical abstract / video abstract optional
        % % accepted ref paper (3) stats:
        % % Announcement effects in the cryptocurrency market
        % % abstract len = 172, pages = 15, word count (w extra intro page; overestimate ~100-200 words...plus some long tables...9893)
        % %
        % % Unconventional monetary policy and inequality: is Japan unique?
        % % abstract len = 192, pages = 12, word count (w extra intro page; overestimate ~100-200 words...7397)
        % %
        % % Market competition and corporate performance: empirical evidence from China listed banks with financial monopoly aspect
        % % abstract len = 212, pages = 12, word count (w extra intro page; overestimate ~100-200 words...7856)
        % %
        % % note - Announcement effects in the cryptocurrency market has acknowledgements and disclosure in paper
        % %
        % % el author guidelines: https://www.elsevier.com/journals/economics-letters/0165-1765/guide-for-authors
        % % heswbl author guidelines: https://www.emeraldgrouppublishing.com/journal/heswbl#author-guidelines
        % % While you are welcome to submit a PDF of the document alongside the Word file, PDFs alone are not acceptable.
        % % LaTeX files can also be used but only if an accompanying PDF document is provided. Acceptable figure file types are listed further below.
        % % Articles should be between 5000  and 7000 words in length. This includes all text, for example, the structured abstract, references, all text in tables, and figures and appendices. 
        % % Please allow 280 words for each figure or table.
        Despite targeting technical skills,
        vocational school graduates are paid less than college graduates.
        This paper hypothesizes that conformity selection and a perceived deficit in soft skills substantially explain reduced alternatively credentialed non-college graduate (ACNG) hirability.
        % % This paper explores an original data set
        % % to understand the influence of employer skill gap perception on the demand for labor.
        % % Original survey data is investigated to 
        % % Specifically, this study tests the hypothesis that nonconformity stigma is a key factor of reduced hirability for an alternatively credentialed non-college graduate (ACNG).
        % % Results from an original survey in the United States indicates that willingness to break rules is a key factor of hirability,
        % % Results from an original questionnaire administered in the United States indicates that willingness to break rules is a key factor of hirability,
        % % cool, concise words :D https://en.wikipedia.org/wiki/Microdata_(statistics)
        Microdata from the United States confirm a perceived soft skill deficit for ACNG labor.
        Results also indicate that conformity is a key factor of hirability,
        but the direction of effect is heterogenous by employer type.
        % This analysis demonstrates the importance of perceived skill gaps and conformity selection when compared to other widely recognized factors of hirability.
        Conformity and perceived skill gaps explain about one third of hirability variance.
        % Perceived soft skill gaps demonstrate importance by explaining about as much variance in hirability as the sum of the effects from state of residence and industry of occupation.
        Perceived soft skill gaps explain about as much as the sum of the effects from state of residence and industry of occupation.
        Opposite a conventional explanation, the results of this study suggest that hirability is negatively correlated to conformity on average.
        ACNG job candidates tend to be perceived as creative types and an even mix of high and low performers.
        % % the risk aversion explanation subsumes the explanation from positive conformity selection, but incentives for negative conformity selection remain.
        % % that is, conformity is not a direct production function input, but it is a signal of low labor productivity variance.
        % % risk aversion explains why employers would prefer less variance regardless of the average comparative productivity estimate.
        Evidence of risk aversion from employers with respect to labor productivity contributes to the explanation of low ACNG demand.
        % Evidence favors an explanation of low demand for ACNG labor
        % % compared to the recent college graduate 
        % based on risk aversion over positive conformity selection.
        %
        % % it's all about that body language skill!
        % % This analysis shows that perceived skill gaps provide more statistical explanation of hirability than
        % % Perceived skill gaps are more important than widely recognized factors of hirability including industrial and state effects.
        % % Soft skills are particularly important.
        % The skill of communication using body language is perceived as a key comparative advantage of recent college graduates over ACNG job candidates.
        % % Recent college graduates and ACNG job candidates share many of the same perceived skill gaps...
        % % in paper, discuss: sum rulebreaker_ideal rulebreaker_ngwac rulebreaker_recentcollegegraduat rulebreaker_typicalemployeeatmyc
        The conclusion incorporates discussion of public misperception on the cost of vocational school and suggests that nontraditional postsecondary education is undervalued in the United States.
        % % Results collectively indicate that nontraditional postsecondary education is more valuable than would be expected in the absence of such results.
        % % Marginally increased social demand would be socially beneficial.
    \end{abstract}

    \begin{keyword}
        education economics, alternative education, conformity, vocational                          %%% not grammatical
        \MSC[2010] I21, I22, J20                                                                    %%% not grammatical
    \end{keyword}

\end{frontmatter}

\pagebreak
\linenumbers

\section{Introduction}

% Some optional TODO:
% . talk a bit more about state and industry and see if we can draw a pattern out (STEM industry? high-pop / Democrat states?)
% . `...and employment in the information technology industry yields a positive coefficient`: give beta and sd
% . use table since there are 3 cases of beta and sd (potential long paper food...2 additional tables; also a summary stats table; perhaps also a diagram)
% . if allowed, another diagram to aid the model section and make the theory clear;

% MB protip: intro outline follows:
% --hook or tension
% --where we are now
% --the importance of your question
% --preview of results

A substantial gap exists between the skills expected by employers and those possessed by college graduates\cite{mcgarry2016examination, malik2017great, abbasi2018analysis, gingras2000there}.
Experts view college alternatives including vocational school as useful for technical training, but the traditional college degree retains a wage premium over vocational education.
Unemployment, underemployment, and other negative labor outcomes follow a similar pattern\cite{smith_2011}.
This paper seeks to resolve the apparent descrepancy between these outcomes while preserving the mainline view that employers pay for perceived job candidate skill.
% or expected marginal revenue product of labor
To explain the apparent descrepancy,
this paper tests the hypothesis that employers expect an offsetting non-technical skill deficit when considering an alternatively credentialed non-college graduate (ACNG).
I find evidence that the general population of the United States, including employers, does apply a stigma to the ACNG in which soft skills are assumed to be deficient.

Alternative credentials refer to credentials other than the undergraduate degree\cite{brown2017complex}.
The category includes, for example,
industry certifications,
portfolios of work,
% transcripts of accredited or unaccredited course work,
digital badges, and other records of unaccredited learning and achievement.
Individuals pursuing alternative credentials typically intend to leverage the credential toward better employment.
That is, they typically have the same goals as a college student.
Many individuals obtain alternative credentials as a supplement to the college degree.
Such a situation is pareto-superior to degree attainment alone and is therefore intentionally excluded from analysis.
This paper focuses on the comparatively interesting case of alternative credentials as a substitute to the college degree in order to diagnose comparative disadvantage at the skill level.
If a specific set of skills effectively explain labor outcome differences,
% as opposed to a broad set of skills each weakly explaning some difference,
then alternative learning providers could update products to affect better outcomes for their consumers.

% Alternative credentials can be obtained quickly and cheaply relative to college.
% Obtaining a college degree signals intelligence, conscientiousness, and conformity,
% but it may not signal technical skill\cite{horton_2020}.
% Alternative credentials signal technical skill.
% As such, they provide an effective supplement to the college degree.

% This paper is concerned with another use case in which the college degree is entirely substituted.
% In that situation, employers may apply a noncollege stigma.
% This is particularly the case for roles which are typically occupied by degree holders.
% Noncollege stigma is a presumption, expectation, or bias toward perception of a skill gap of a certain kind.
% Whether the gap exists in fact is out of the scope of the present paper.

% Technical skill generally implies intelligence.
% Alternative credentials, then, fail to signal two qualities compared to the college degree.
% Alternative credentials fail to signal conscientiousness and conformity.
% Interestingly, some employers may demand some level of nonconformity.
% Employers may also presume a certain lack of soft skill on the part of highly technical applicants.
% % Finally, employers may use alternative credentials as a proxy for other employee characteristics like income, education, race, and gender.

% % a missing link for future research: hirability only correlates with actual hiring decisions it isn't a hiring decision
% Hiring decisions reflect boundedly rational demand for skilled labor.
% The college degree and alternative credentials provide two qualitatively distinct signals of skilled labor.
% The hirability of individuals in possession of these credentials has been studied,
% but the underlying determinants are not clear.
% This paper hypothesizes that perceived skill gaps are important determinants of willingness to hire.
% This paper further hypothesizes that perceived skill gaps are qualitatively different between college graduates and others.
% % In particular, this paper hypothesizes that a noncollege stigma is obtained for candidates without a degree in pursuit of roles typically filled by degree holders.
% % soft skill bias in particular

% three interesting follow-on questions:
%   1. do employers have such a bias
%   2. is such a soft skills gap presumption actually true
%   3. if true, due employers overvalue the soft skills gap
% related paradox: most people won't be in a job for 4-5 years,
% so why do they need to show conscientiousness and conformity towards the 4-5 year bachelor's goal line?
% hard skill stigma: in my experience, people who are highly technical are hard to work with
% soft skill bias: I am favorable bc i think u have soft skills (and maybe this is efficient...enter eq/iq discussion)

% This paper tests the hypothesis that there is a lack of willingness to hire an ACNG (ACNG) is explained by an offsetting perceived lack in non-technical skills.
% In particular, this paper hypothesizes that ACNGs are seen as nonconformist and lacking in soft skills or non-technical skills.
% These deficits explain why an ACNG would not be a preferred source of labor in many cases,
% even if such a candidate does possess superior technical skill.

% technical skill has negative coefficient but magnitude and reliability (p-value/variance) are weaker; overall, less important effect
% hypothesis stems from signalling model.
% This paper proposes skill gaps are perceived in particular among soft skills for alternatively educated individuals.
% one might argue employers are mistaken here; technical work may involve higher, not lower, conscientiousness; ya maybe but out of scope.
% that is, we test social stigma and skill-level / decomposed stigma; an application of the signlaing model.

% Experts view college alternatives including vocational school as useful for technical training, but the traditional college degree retains a wage premium over vocational education.
% This paper hypothesizes that employers pay for skill.
% As a result, lower wages for technically skilled individuals are hypothesized to derive from an offsetting perception of skill deficit elsewhere.
% That is, this paper hypothesizes that employers view an ACNGs (ACNG) as lacking in soft skills.
% This paper hypothesizes that employers expect a skill deficit, although not a technical skill deficit, 
% This expected deficit explains the variance in labor outcomes.

% Sustained rising costs to higher education motivate periodic review of the return to the college degree.
% Despite rising costs, Americans have become more educated than expected over the past decade.
% Trades have contemporaneously seen a labor shortage.

% actually, trade school enrollment is increasing faster than undergraduate enrollment
% https://www.chronicle.com/newsletter/the-edge/2020-01-22
% By 2020, They Said, 2 Out of 3 Jobs Would Need More Than a High-School Diploma. Were They Right?
% overinvestment in college seems to cause a technical labor shortage, but the market is compensating by enrolling more technical folks too
% https://www.theatlantic.com/education/archive/2019/03/choosing-trade-school-over-college/584275/
% undergraduate enrollment has slowed recently and many employers have dropped the college degree requirement
% https://www.npr.org/2019/12/16/787909495/fewer-students-are-going-to-college-heres-why-that-matters
% it is not the case that employers are increasingly demanding the college degree, but it is the case that many do today. Let's examine their reasons.
% peak college?

% An undergraduate degree is a historically reputable investment.
The signalling model has become one of the two standard explanations of the value of the college degree.
Signalling theory provides three advantages over human capital theory for the purposes of the present study.
First, signalling theory is able to explain labor outcome variance when human capital is held constant.
% First, signalling theory is able to explain labor outcome variance across labor types when skills are totally equal.
% Under a human capital model, in contrast, a variety of labor outcomes would directly imply variance among input labor.
% The present paper expects that skills for the ACNG compared to other labor types are not totally equal, but this must emerge as a result rather than a presumption.

Second, the signalling model empowers a questionnaire research design.
In an idealized human capital model, the measures of human capital would correspond to production process inputs.
To establish a wide array of such skill measures would be complicated and prone to measurement sensitivies, assumptions, and errors of various kinds.
In this framework, a questionnaire is a second-best design which provides a proxy for the functional measure of skill.
Signaling theory takes the reverse approach.
According to the signalling model, labor demand is formed on the basis of job candidate value as perceived by an employer.
In this framework, a questionnaire is ideal.
% In this framework, a questionnaire is a direct measure of the functional construct, which is the opinion of the employer.
The manner is in which employer perception relates to job candidate technical skill, if at all, is secondary.
An additional benefit of using a questionnaire is the ability to ask hypothetical questions.
In pondering hypotheticals, employer evaluation of a credential or signal can be isolated from the human capital variance which occurs in actual job candidates.

Third, signalling theorists have laid out a testable hypothesis for weak labor outcomes among non-college graduates.
Following this model, scholars claim that the college degree signals intelligence, conscientiousness, and conformity\cite{caplan2018case}.
% Since immediate college enrollment is the normal course of action for high school graduates,
Non-traditional education, in contrast, is hypothesized to signal nonconformity.
Non-traditional courses can be completed in a shorter span of time and with reduced entry qualifications relative to the traditional degree.
For this reason, alternative credentials are thought to signal low concientiousness relative to the college degree.

% Proponents of the signalling model often prefer employer-oriented explanations of college enrollment.
% In this explanation, employers prefer college graduates because the college degree signals intelligence, conscientiousness, and conformity.
% While a technical credential signals intelligence and technical skill, the absence of a degree yields a perceived gap in the mind of the employer.
% There is a perceived comparative lack on the part of the non-college graduate with respect to conscientiousness and conformity.
% This paper tests this hypothesis.

% An agent-based explanation would be that high school graduates are not taught about these alternatives.
% The college degree is popular, has a well-defined return, and is low in risk.
% Particular alternative programs are obscure and often lack a well-defined return.

% Other research indicates that concientiousness and conformity are not always desirable labor qualities.
% There is some reason to doubt the hypothesis that lower perceived value is attributable to signaling differences in concientiousness and conformity.
% Research indicates that extreme values for either factor in either direction may be detrimental to productivity.

% TODO: long paper food...uncomment below section as it implies we should be doing marginal analysis. Then do marginal analysis, and K*K skill interactions
% TODO: maybe move to results section when we talk about concientiousness
% Research indicates a goldilocks level or bliss point for both concientiousness and conformity is likely to exist.
Research indicates the existence of bliss points for employee concientiousness and conformity from the point of view of an employer.
Excess individual concientiousness can disturb team performance\cite{curcseu2019personality}.
Conformity can lead to a lack of innovation and suboptimal organizational practices\cite{symon2006neglected}.
% Psychologists also state that
Conformity selection occurs in part through heuristic decisioning rather than conscious choice.

% Because a single measure operationalizes each of these effects and their own negation, a fixed sample size is relatively unlikely to identify an important coefficient.
% Because these factors are sometimes demanded and in other cases the inverse is demanded, a factor coefficient may be harder to identify and may only represent the average effect, even if the average effect is hardly predominant in practice.

% The psychological problem is related to but distinct from the pure logical problem that a totally conformed mind is necessarily incapable of innovation.
% Firm innovation requires an underlying capacity for individual innovation.
% Firms must have some capacity for innovation to sustain profit in a growing economy.
% Even if conformity selection is a correct explanation of ACNG aversion, then, it may not be an ideal practice when viewed through the lense of technical or economic efficiency.
% Risk aversion is compatible with a sometimes-concious, sometimes-heuristic decisioning model.

% Innovators, leaders, and high-performers are three kinds of virtuously nonconformant labor.
% Because conformity is sometimes undesirable and sometimes desirable,
% the effect may neutralize itself in an ordinary OLS stastistical analysis.
% The effect may not be identified as important or significant in any particular direction.

Risk aversion is a distinct explanation for conformity selection.
An employer may not be able to evaluate an alternative credential with confidence.
From the point of view of such an employer, an ACNG may range in value from a positive outlier to a negative outlier.
The employer may not prefer to hire such a candidate on the basis of risk aversion,
even if their point estimate for ACNG labor value is higher than their point estimate for a recent college graduate.
% To preview a particular result, this paper finds that large employers are particularly willing to hire an ACNG.
If employers with many employees are positively associated with ACNG hirability, this will add weight to an explanation based on risk aversion.
% Reasons for this include: 1) failure to deliver can be catastrophic, so low performers may be disproportionately untolerable.
% Revenue consistency, timeline, reputation, quantity produced targeting, large min skill labor cost and marginally small pay increase to achieve adequate production.
% Performance monitoring and turnover costs reinforce this
% zb states I'm assuming constant cost per hire...actually I allow that some portion of large-firm hirability is due to better ability to distinguish low vs high
% other than that effect, the remainder would be attributable to risk aversion.
% yes, the null hypothesis is no difference in cost per hire; but I can also support that which I suspect...turnover costs are proportionally smaller for large firms
% these firms are able to specialize and economize in hiring, plus they are more likely to have high-skill candidates that can better interview and recruitment tech+processes that scales
% TODO: long paper food... consider below articles and flesh out the risk aversion to firm size interaction thing
% note turnover cost calculation is complex but we proxy of just cost to hire. A + B below support cheaper for large firms. (C reinforces B)
%   A) "As stated in a study by the National Association of Colleges and Employers, hiring an employee in a company with 0-500 people costs an average of $7,645."
%   B) "Another study by the Society for Human Resource Management states that the average cost to hire an employee is $4,129, with around 42 days to fill a position."
%   C) "According to Glassdoor, the average company in the United States spends about $4,000 to hire a new employee, taking up to 52 days to fill a position."
% related but doesn't solve the issue: https://builtin.com/recruiting/cost-of-turnover
% https://toggl.com/blog/cost-of-hiring-an-employee
% The highest performing employers, however, will be able to distinguish desirable from undesirable candidates within the unconventional pool.
% Risk aversion varies naturally among firms. <- probably don't write this line in paper as a reviewer can always posit there is a further reason you are missing
% Some employers that are high in risk aversion will provide a net preference to ACNG due to nonconformity preference.

% TODO: long paper food...below section is part of intro or perhaps model...
% \subsection{Process Explanations of Suboptimal Wages}

% Basic price theory holds that an employer should pay wages equal to the marginal revenue product of labor.
% In the real world, measuring candidate productivity at hiring time is costly and imperfect.
% % This produces a technical error which assumes alignment between the goals of the firm and a hiring team.
% A further issue is identified when the hiring team is scrutinized for principal-agent problems.
% The hiring team is composed of individuals with preferences, calculative limitations, and other biases.
% Monitoring and correcting for these problems is expensive,
% so firms will heterogenously realize some aggregation of these individual definiciencies.

% Exacerbating the already necessarily imperfect hiring process are candidate-side problems.
% Firms must hire among a finite, potentially small, number of candidates.
% Risk aversion to time expense and other search costs may lead a firm to approve a suboptimal candidate\cite{simon1976substantive}.
% In some cases, candidate pools may be systematically problematic.
% In law and medicine, for example, extensive education and training are legally required.
% These policies further restrict the candidate pool, inflate expected wages, and systematically alter the content of education in a politically-motivated manner.
% Market forces implement hiring as a lumpy expenditure process to begin with, but certification requirements, wage regulation, and other policies extend the problem.

% The prior discussion highlights many locations of hiring process inefficiency.
% The practical importance of magnitudes and kinds of such effects are described in a legion of related papers.
% A meager sampling of five such effects would include the attractiveness effect and many other issues related to gender bias\cite{quereshi1986physical},
% agentic behavioral stigma\cite{steffens2009feminization},
% and complex biases related to communication style\cite{brouer2017gender, nijs2019effects, sampugnaro1983nonverbal}.
% Sung et al find that impression management meaningfully weakens disability stigma\cite{sung2017disclose}.
% These tactics are transferable in part to noncollege stigma mitigation.
% Finally, there are a wealth of concerns about the effects of social media.
% For one, it presents a channel for the revival of religious discrimination\cite{esposito2018signalling}.

% In the face of so many important inefficiencies, one begins to wonder whether the original theory holds any water at all.
% Papers which identify matching effects, including the present paper,
% serve to limit the proportion of explanation attributable to bias and redeem the elementary price theory story to some extent.
% Prior work demonstrates the important of matching effects in the form of norm compliance\cite{francesco1981gender}.
% Meta-accuracy is a kind of matching measure, and it has been shown to move positively with hirability\cite{renier2018no}.

% \section{Literature Review}
% TODO: long paper food...lit review exists in Announcement effects in the cryptocurrency market
% TODO: long paper food...expand on the signalling model vs the human capital model
% Faster and cheaper alternatives to college exist, but high schools prefer immediate college enrollment over alternative options at a rate of nearly 2:1.
% A New U: Faster + Cheaper Alternatives to College
% Faster and cheaper alternatives to college exist, but the typical student prefers to immediately enroll in college.
% Five student-oriented explanations include an inflated perception of the return to college,
% lack of awareness about alternative programs,
% social pressure to pursue college over alternatives,
% inability to confidently compare returns to alternatives,
% and risk aversion which favors college as a low-risk option despite high cost.


\section{Data and Methodology}
% note: description of data / model / methodology: one section in a short paper; can be broken into seperate sections for a long paper...or can it? seems kinda intertwined
% for this paper: methodology [i. model / method of analysis, ii. description of data collection iii. additional description of data as needed at factor level]
% I don't feel the need to spell out an OLS model equation unless requested by a reviewer
% I don't feel the need to spell out the risk aversion model equation unless requested by a reviewer (it's just a signal multiplier on actual skill which varies by credential and employer)
% the risk aversion model i feel is more interesting, but it would be an inappropriately high level of focus and detail for an item of secondary interest

% The method of this study begins with a decomposition of the main hypothesis into three simple statistical tests.
% This study applies the three-step method to the case of the ACNG, but the three-step method itself is sufficiently general to apply to any measurably distinct job candidate type.
% First, skill gaps are identified as independent factors with a general ability to explain willingness to hire.
% Second, the coefficient for a
% Finally, gaps in soft skills are hypothesized to be more important than technical skill gaps.
% If the above three conditions hold, ACNG labor dispreference can be explained with or without comparing labor outcomes to the college graduate.

The hypothesis in this paper is based on a simple model of demand for labor which is clarified in Equations \ref{eq1} and \ref{eq2}:

\begin{subequations}
    \begin{equation}
        % maybe S is not S_j showing S is a non-person-specific credential
        % but I like S_j because it is *those credentials possessed by j* which can be a unique collection + unique work history, other unique signal, etc...
        S_j = f(H_j)
        \label{eq1}
    \end{equation}
    \begin{equation}
        w_{ij} = E_i(MRP_j) = f_i(S_j)
        % alternatively, D_i(L_j) = E_i(MRP_j) = f(S_j, i)
        \label{eq2}
    \end{equation}
\end{subequations}

Job candidate $j$, generates a signal of productivity, $S_j$ from unobserved human capital, $H_j$.
Employer $i$, forms an expectation of the marginal revenue product of $j$ on the basis of $f_i(S_j)$, an employer-specific evaluation of $S_j$.
A specific employer is willing to pay a specific job candidate wages of $w_{ij}$.

This study uses ordinary least squares (OLS) regression analysis to estimate the effect of perceived skill gaps on hirability.
An employer is willing to pay more for a relatively hirable individual.
This makes hirability a proxy of demand for labor and $w_{ij}$.
% this is questionable because hirability is technically the willingness to execute a wage integrated over some expected time period (average employee tenure)
% but then, that integrated wage would just be wage multiplied by some constant and we would need to divide by some constant since hirability is a probability
% so, hirability should correlate directly to wage after all
% you also might say hirability is theoretically closer to an estimate of productivity...but who cares it's once again equal
% and we want to frame the overall model as a simple labor demand model.
%
% Equally, a respondent is making an expected productivity statement when scoring hirability
% This study presumes that employers are willing to pay less for an ACNG.
% This study also presumes that ACNG labor has better technical skill.
In order to explain reduced willingness to pay for ACNG labor relative to college graduate labor,
this paper hypothesizes that employers preferentially value soft skills in the course of $f_i(S_j)$.
To provide evidence for preferential evaluation of soft skills by employers,
one or more soft skills should yield a negative coefficient
% of an economically important magnitude
in a regression on hirability.

% The hypothesis that employers associate soft skill deficiency with ACNG labor is based on a simple model of labor demand.
% As such, the top-level model is a model of demand for labor.
% This is how we derive the hypothesis of offsetting deficiencies from a lower ACNG wage.

% To address this question,
Regression analysis in this study is conducted using original cross-sectional data from an online self-completed questionnaire ($n = 212$).
The data is available for replication or any other use\footnote{
    See \url{https://osf.io/8qtxf/?view_only=95b0c0b0c65e4b7983198cc87c2ab733}
    for data used in this study.
}.
% Responses were submitted using the SurveyMonkey web application.
Respondents were obtained through the Amazon Mechanical Turk crowdsourcing service.
Respondents were United States citizens at or over the age of eighteen,
paid for participation,
and selected on an opt-in, first-come, first-serve basis.
The survey administration took place in July of 2020.
% MAYBE TODO: above paragraph could be moved to after categorical and likert-type discussion
% MAYBE TODO: I could explain that AMT has been shown by my other paper to be bias-free for this right hand param
% and that I purchased 225 samples on the basis of 1) over 100 for large numbers to kick in, and
% 2) experience with the other survey hinted this would be on the lower end of what was needed for significance.

The survey includes 65 questions in two sections\footnote{See Appendix A for a full copy of the survey.}.
The first section captures respondent characteristics and the second section captures perceived skill relative to hypothetical job applicants.
Employer responses did not significantly differ from the general population,
so respondent characteristics are also interpreted as employer characteristics.

Regression variables in this study are categorical or Likert-type responses based on a scale from 1 to 10.
% \footnote{
%     OLS classically supposes cardinal, interval, or continuous data.
%     Likert-type responses are generally considered ordinal data, but they can be interpreted as cardinal for a few reasons.
%     First, rounding is an inevitable measurement constraint.
%     A continuous response from 0 to 1 which rounds to the first decimal is technically statistically and mathematically isomorphic to,
%     or indistinguible from,
%     a 10 point Likert-type response.
%     Jaccard and Wan\cite{jaccard1996lisrel} note that severe departures from the assumptions on cardinality "do not seem to affect Type I and Type II errors dramatically."
%     They find departures are particularly nonproblematic for Likert scales with 5 or more points.
%     Finally, treating Likert-type responses as continuous is structurally defensible in this particular study.
%     The notion that Likert-type responses are purely ordinal would make the notion of marginal effect absurd and incalculable,
%     but willingness to pay for labor, concientiousness, and other factors used in this study are known to exhibit marginal effects in the literature. % TODO: I could cite but it's pretty obv
%     See the results section where a marginal effect from concientiousness is identified and interpreted in a meaningful way.
%
%     % plenty of research treats likert-type units as continuous, and there is also structural justification in this case.
%     % Likert-type responses can be interpreted cardinally if certain assumptions hold, which I am happy to posit as constraints in this paper (even spacing of units).
%     % Even if the assumptions fail, and I don't think they do, research indicates that results seem to be generally robust to breaking such assumptions, and the practice is widespread
%     % additionally, in this case of this paper we have a structural justification in marginal effects (computed only for concientiousness but it works...)
%     % d. in my case i have structural justification in marginal effects.
%     % you can't have a marginal effect on a non-continuous measure; you can't take an ordinal derivative
%     % but i have marginal effects, they are stastistically valid, and they are theoretically meaningful and similar measures are used all over the place.
%     % i would be happy to use a categorical-like treatment if not for this, and the latter might even have better fit but it could be overfit.
%     % finally, an alternative approach would be to treat likert responses as categorical, but the continuous treatment has less fit and more structural justification
%     % categorical treatment is a relative overfit
%     % likert-type units can be considered direct psychological measures or economic proxies.
%     % The likert-type response curve can be thought of as a function of actual skill;
%     % b. decent paper on ordinal independent variables: https://www3.nd.edu/~rwilliam/stats3/OrdinalIndependent.pdf
%     % c. more on this: https://www.researchgate.net/post/Is_a_Likert-type_scale_ordinal_or_interval_data
% }.
Higher Likert-type values indicate greater agreement with a statement that varies by variable.
Categorical variables include state of residence,
industry of occupation,
employer status,
firm size,
and a measure called duration.

Duration measures the length of time the respondent believes it takes to obtain an alternative credential.
Employer status describes whether an individual makes hiring and firing decisions in the course of their employment.
The variable takes one of three values: yes, no, or unemployed.
Employer effects refer to the case where an individual states that they do make hiring and firing decisions.
State of residence refers to a state within the United States.
Respondents were allowed to select the District of Columbia as a state of residence,
but no such responses were obtained.

Three other factor groups are investigated in the regression analysis.
These variables are measured using Likert-type units and they include hirability,
rulebreaker effects,
and perceived skill gaps.
Hirability is the dependent factor and it indicates the degree of agreement that, "For many professions, alternative credentials can qualify a person for an entry-level position."

% The unit of factor coefficients for nonconformists and skill gaps is hirability per Likert-type unit, where hirability is also a Likert-type unit.
% Rulebreaker effects refer to a collection of three factors that describe the way employers think about nonconformists, or "People who are willing to break formal or informal rules and norms."
Rulebreaker effects refer to a collection of three factors that measure respondent agreement with statements about nonconformists, or "People who are willing to break formal or informal rules and norms."
% The three rulebreaker questions measure respondent agreement with statements 
The first statement indicates that nonconformists present a risk to the reputation, productivity, or value of a company.
This statement received the least agreement and greatest response variance among three qualitatively different descriptions of people that are willing to break rules ($\mu = 6.40, \sigma = 2.55$).

The second statement indicates that nonconformists are held back by rules and "could just as easily be high performers as low performers."
% The second statement indicates that people break rules which hold them back, and that nonconformists "could just as easily be high performers as low performers."
% MAYBE TODO: standard error instead of or in addition to standard deviation. maybe make a table since there are at least three cases of similar report.
This statement received the most agreement and least variance among rulebreaker statements ($\mu = 7.42, \sigma = 1.91$).
The agreement with this statement provides evidence against the thesis that employers value conformity for its own sake.
In turn, this adds weight to the theory that employers value conformity as a risk aversion tactic while knowing that nonconformity signals positive outlier potential.
The third description of nonconformists states that they tend to be gifted in the areas of innovation or creativity,
and that such people may benefit the culture of a company ($\mu = 7.25, \sigma = 2.03$).
% managers are wary of ACNG, having high correlation with "Rule Breakers Risky"
% reg _ismanager1 rulebreakersnormsmightbedoingsob rulebreakersnormsprobablyhaveaha rulebreakersnormstendtobegiftedi

% % MAYBE TODO: sentence below can be shortened and we can introduce likert-type unit as distinct from likert-type response since it's a computed value
% Perceived skill gaps are computed from perceived skill questions in the second section of the survey.
% % Respondents do not directly report perceived skill gaps.
% % Instead, responses indicate perceived skill level for particular skill and a particular type of job candidate.
% % 13 skills are analyzed and 4 job candidate types are surveyed, for a total of 52 questions in the second section on perceived skill.
% % MAYBE TODO: citation to "make response anchoring appropriate"
% % Each section begins with a contextual message to normalize response anchoring.
% % Questions are provided in nonrandom order for the same reason.
% % Data from the second section is used to calculate perceived skill gaps.
% For each of 13 skills, the respondent is asked to imagine four types of job candidate.
% One type of candidate is an ideal candidate.
% Raw perceived ACNG skill gaps are calculated by differencing the perceived skill of an ideal candidate with the perceived skill of an ACNG.
% % technically, the actual skill of an ideal candidate equals the perceived skill so the adjective is extraneous; but let's be consistent and not confuse reader.

Perceived skill gaps are computed two ways from perceived skill questions in the second section of the survey.
Perceived skill gaps are measured seperately with and without overqualification effects.
Overqualification effects have been identified as important in external research\cite{green2007there, raybould2005over}, but these effects are sometimes ignored during skill gap analysis\cite{blake_2018}.
% MAYBE TODO: cite more than 1 person who ignores overqualification

Perceived skill is a Likert-type response reporting agreement with the statement that a particular candidate has a particular skill.
For each of 13 skills, the respondent is asked to imagine and report skill levels for the ideal candidate,
the average actual employee,
the average recent college graduate,
and the average ACNG.
% For each of 13 skills,
Raw perceived ACNG skill gaps are calculated by differencing the perceived skill of an ideal candidate with the perceived skill of an ACNG.
The perceived skill gap with overqualification effects equals the raw perceived skill gap.
The perceived skill gap without overqualification effects is calculated as the raw skill gap or zero if the raw skill gap value is negative.
% MAYBE TODO: just have one dependent variable and don't mention the other. it's something i did at analysis time, but may be confusing in the paper.

Rulebreaker effects and perceived skill gaps are structurally linked.
Respondents are asked to evaluate the soft skill of nonconformity, or "willingness to break formal or informal rules and norms."
% This question is a measure of the soft skill of nonconformity.
%, which is the inverse of conformity.
% Structurally, nonconformity interacts with employer disposition to rulebreaking.
Nonconformity interacts with employer disposition to rulebreaking.
For this reason, discussions on the importance of skill gaps include discussion on rulebreaker effects.

% MAYBE TODO: delete this paragraph
These methods allow for identification of a preferred model that explains hirability using perceived ACNG skill gaps.
The identified model will support the hypothesis if soft skills are more important than technical skill gaps. % and negatively affect ACNG hirability.
The model will support the risk aversion explanation of ACNG hirability over an explanation from conformity selection if large employer size is positively associated with hirability.
% If employers preferentially value soft skills then soft skill gaps should be identified as important and negatively relate to ACNG hirability.
% MAYBE TODO: I said this earlier in this same section...am I referencing the main hypothesis too often or is it healthy?
% MAYBE TODO: I also expect signs to be generally negative and soft skills should be statistically significantly less for ACNG compared to recent college graduate
% MAYBE TODO: rulebreaker effects and firm size effects also contribute to solutioning the risk aversion hypothesis

Comparative analysis provides additional confidence in the data by replicating a hirability gap between ACNG labor and recent college graduates.
A comparative skill gap variable is constructed for each perceived skill gap that is important in the preferred hirability model.
Comparative skill gap variables are constructed by subtracting perceived recent college graduate skill from perceived ACNG skill.
Multiple regression of these comparison factors on hirability demonstrates which, if any, perceived skill gaps are important distinguishers of the ACNG from the college graduate.
Identification of significant differences with a negative total effect on hirability will replicate external data on the lower job market value of ACNG labor
and provide a diagnostic on which skill or skills must be better addressed through alternative learning programs.
% although the negation of the above sentence does not provide evidence against the main hypothesis...because hirability is wrt ACNG labor; we don't have recent grad hirability data in sample

% begin low importance comment...
% this paper compares alt ed to ideal, but other papers could compare other sets:
% 3 different explanatory constructs are explored, but only the winner is reported in the paper:
% 1. alt ed to ideal
% 2. alt ed to typical [not interesting for this paper]
% 3. alt ed to college grad
% 4. alt ed to ideal without overqualification
% 5. alt ed to typical without overqualification [not interesting for this paper]
% 6. alt ed to college grad
% (horse racing): https://www.afterecon.com/economics-and-finance/kitchen-sink-regression-and-horse-racing/
% should probably randomly split sample and out-of-sample test with factors to combat overfit
% aggregate excess attractiveness by recent college grads against ideal.
% aggregate excess willingness to break rules by alt ed noncollege grads.
% many non-aggregate, or respondent-level, cases of alt ed overqualification; in fact, some such responses for every question kind (the 13 types)
% Optional but interesting: college grad to ideal or college grad to alt ed; so that we can indirectly associate hirability to actual propensity to hire. (which we have for college grads)

% objective of analysis
% how much does matching explain (caveat: not multiply regressed, so matching effect is likely overstated in this paper, and possibly partially partialled-in to prior work)
% does noncollege stigma exist
% "alternative education is different how?"
% 'explaining hirability'

% simple match effect: those that prefer technical talent will tend to support alternative credentials.
% complex match effect: a match profile will have significantly and importantly more explanatory power compared to but consistent with a simple match effect.

% quality question meta: 1 to 10: disagree to agree
% ---
% An ideal candidate would have this quality...
% A typical employee would have this quality...
% A college graduate would have this quality...
% A credentialed or certified non-college graduate would have this quality...
% [later] Someone who is self-taught (without a credential or portfolio) would have this quality...
% [later] A typical junior-level high school student would have this quality...

% some notes, mainly out of scope
% ---
% hiring error awareness increases in a few ways
%  1. [passive search] participant observation. As an interviewer, interviewee, hiring manager, or other professional involved in the process, I simply notice a problem
%  2. [passive search] passive company and individual level search into HR best practices; an industry newsletter says hey Griggs v Duke happened so don't use IQ tests anymore.
%  3. [passive search]: audit compliance (legal+required, or optional audits from firms that certify quality, for example)
%       example: Supreme Court case Griggs v Duke had an industry-wide effect thru this means
%  4. [active search] intrapraneurship / policy change championing begins with an individual saying hey let's investigate this thing. what would motivate such individual? (maybe due to 1 or 2).
%
% my prior work has shown that we can predict (r2 0.5 - 0.7 and ar2 0.3 - 0.6) alternative education hirability from employer factors alone - without concern to matching

\section{Results}

% missing industrial variables: hospitality (restaurant/hotels) and entertainment / media, and sales!

% TODO: long paper food...
% 2. for all skills, study marginal effects and K*K skill interactions (including rulebreaker questions)

% This paper acknowledges that own analysis proceeds through a technocentric lens.
% This is an important anchoring point for the analysis, and it may skew application of results in low-technology or low-skill sectors.
% The technocentric lens is an important caveat and anchoring point, but I argue that it is about as proper as any anchoring point.
% In economics, after all, technology operationalizes the theory of innovation per se.
% All skills can be viewed as point-in-time innovations, so that if there was no innovation then neither would there be a need for any skill.
% By the same token, a technocentric lens at the present seems close to a cross-industry lens at a future time.
% Anchoring to any other industry would be both asymmetric and unusuful in the future.
% Perhaps this analysis is slightly skewed, but at least it is skewed only against the past, and will be increasingly useful in the future without partiality to any particular industry.
% In addition, we did check for industrial effects, but the analytical skew may persist pass the data.

% This orientation occurs because New Alternative Education first flourished for roles in the information technology sector, and only later did roles like sales, business, art, nursing, and more join in.
%     [can refer to my New Digital Education] - https://papers.ssrn.com/sol3/papers.cfm?abstract_id=3530647

% Compare directly to bootcamp results from Indeed: https://www.indeed.com/lead/what-employers-think-about-coding-bootcamp

\subsection{Identification of the Preferred Model}

Results confirm that employers, and the population in general, associate a soft skill deficit with ACNG candidates.
At the same time, ACNG hirability was generally positive.
The mean response was 7.5 on a scale from one to ten ($\sigma = 1.80$).
Hirability critically depends on rulebreaker effects.
Rulebreaker effects have more explanatory power than perceived skill gaps.
Evidence favors an explanation of ACNG hirability from risk aversion over conformity selection.
Employer status was associated with an insignificant positive coefficient.
% TODO: the last sentence feels out of place

Table \ref{tab:table_new_ols} reports selected factor statistics across five OLS multiple regressions.
The selected factors which are reported include any perceived skill gap which is important in any specification.
Factor importance is determined by the ability of a factor to improve adjusted explanatory power.
Model 1 is a multiple regression using skill gaps that allow for overqualification.
Model 2 is a multiple regression without overqualification.

% n=212
\begin{table}
    \caption{Table of Coefficients for Multiple Regressions on hirability, Selected Variables}
    \resizebox{\columnwidth}{!}{
        % derived from analysis-5-regs-table.do
{
\def\sym#1{\ifmmode^{#1}\else\(^{#1}\)\fi}
\begin{tabular}{l*{5}{c}}
\toprule
                         &\multicolumn{1}{c}{1}&\multicolumn{1}{c}{2}&\multicolumn{1}{c}{3}&\multicolumn{1}{c}{4}&\multicolumn{1}{c}{Model 5}\\
\midrule
Is Employed Non-Manager  &      -0.336         &      -0.383\sym{*}  &      -0.497\sym{**} &      -0.471\sym{**} &      -0.451\sym{**} \\
\addlinespace
Is STEM Worker           &      -0.491\sym{**} &      -0.529\sym{**} &      -0.525\sym{**} &      -0.557\sym{**} &      -0.564\sym{**} \\
\addlinespace
Employees 51-200         &      -0.475\sym{*}  &      -0.480\sym{**} &      -0.364         &      -0.459\sym{*}  &      -0.457\sym{*}  \\
\addlinespace
Industry Credentials Legally Required&       0.706\sym{*}  &       0.722\sym{**} &       0.374         &       0.378         &       0.375         \\
\addlinespace
Industry Credentials Normal&       0.932\sym{**} &       0.926\sym{**} &       0.487\sym{*}  &       0.436\sym{*}  &       0.448\sym{*}  \\
\addlinespace
Industry Credentials Sometimes Used&       0.467         &       0.475         &                     &                     &                     \\
\addlinespace
Industry Credentials Unknown&       0.641\sym{*}  &       0.684\sym{**} &                     &                     &                     \\
\addlinespace
Industry, Agriculture    &       1.368         &       1.619\sym{*}  &                     &                     &                     \\
\addlinespace
Industry, Energy         &      -1.277\sym{*}  &      -1.190\sym{*}  &      -1.200\sym{*}  &      -1.442\sym{**} &      -1.448\sym{**} \\
\addlinespace
Industry, Finance, Investment, or Accounting&      -0.811\sym{***}&      -0.783\sym{***}&      -0.712\sym{***}&      -0.715\sym{***}&      -0.717\sym{***}\\
\addlinespace
Industry, Information Technology&       0.335         &       0.264         &       0.438\sym{*}  &       0.306         &       0.337         \\
\addlinespace
Industry, Law            &      -1.813\sym{***}&      -1.670\sym{**} &      -1.935\sym{***}&      -1.876\sym{***}&      -1.857\sym{***}\\
\addlinespace
Industry, Transportation &       1.512\sym{*}  &       1.643\sym{**} &       1.216         &       1.403\sym{*}  &       1.350\sym{*}  \\
\addlinespace
State, Arizona           &      -1.157\sym{**} &      -1.048\sym{**} &      -0.755         &      -0.823\sym{*}  &      -0.790         \\
\addlinespace
State, Arkansas          &      -2.690\sym{***}&      -2.817\sym{***}&      -2.489\sym{***}&      -2.664\sym{***}&      -2.770\sym{***}\\
\addlinespace
State, California        &      -0.575\sym{*}  &      -0.570\sym{**} &      -0.488\sym{*}  &      -0.435         &      -0.446         \\
\addlinespace
State, Colorado          &      -1.446\sym{**} &      -1.423\sym{**} &      -1.463\sym{**} &      -1.521\sym{***}&      -1.508\sym{***}\\
\addlinespace
State, Connecticut       &      -1.401         &      -1.550         &                     &                     &                     \\
\addlinespace
State, Florida           &      -0.444         &      -0.454         &                     &                     &                     \\
\addlinespace
State, Hawaii            &      -3.232\sym{***}&      -3.271\sym{***}&      -2.884\sym{***}&      -2.869\sym{***}&      -2.812\sym{***}\\
\addlinespace
State, Illinois          &      -0.637         &      -0.699\sym{*}  &      -0.596         &      -0.675\sym{*}  &      -0.698\sym{*}  \\
\addlinespace
State, Kansas            &      -3.283\sym{**} &      -3.486\sym{**} &      -2.923\sym{*}  &      -3.116\sym{**} &      -3.101\sym{*}  \\
\addlinespace
State, Kentucky          &      -3.143\sym{***}&      -3.167\sym{***}&      -2.583\sym{***}&      -2.729\sym{***}&      -2.679\sym{***}\\
\addlinespace
State, Louisiana         &      -1.455\sym{*}  &      -1.255\sym{*}  &      -0.915         &      -0.941         &      -0.898         \\
\addlinespace
State, Maryland          &      -0.596         &      -0.642         &                     &                     &                     \\
\addlinespace
State, Nebraska          &      -2.037\sym{*}  &      -2.167\sym{*}  &      -1.391         &      -1.655         &      -1.596         \\
\addlinespace
State, Nevada            &      -1.406         &      -1.470         &      -1.465         &      -1.434         &      -1.409         \\
\addlinespace
State, New Jersey        &      -1.145         &      -1.139         &      -0.976         &      -0.936         &      -0.963         \\
\addlinespace
State, New York          &      -0.692\sym{**} &      -0.640\sym{*}  &      -0.617\sym{*}  &      -0.595\sym{*}  &      -0.590\sym{*}  \\
\addlinespace
State, Ohio              &      -3.943\sym{***}&      -4.024\sym{***}&      -4.051\sym{***}&      -3.808\sym{***}&      -3.761\sym{***}\\
\addlinespace
State, Pennsylvania      &      -0.752         &      -0.687         &      -0.608         &      -0.534         &      -0.539         \\
\addlinespace
State, South Carolina    &      -1.183         &      -1.243         &      -1.361         &      -1.310         &      -1.347         \\
\addlinespace
State, Tennessee         &      -1.878\sym{**} &      -1.909\sym{**} &      -1.545\sym{*}  &      -1.843\sym{**} &      -1.799\sym{**} \\
\addlinespace
State, Texas             &      -0.906\sym{**} &      -0.851\sym{**} &      -0.797\sym{**} &      -0.790\sym{**} &      -0.789\sym{**} \\
\addlinespace
State, Washington        &      -0.817         &      -0.863\sym{*}  &      -0.880\sym{*}  &      -0.996\sym{**} &      -1.003\sym{**} \\
\addlinespace
Duration                 &       0.666\sym{**} &       0.634\sym{**} &       0.811\sym{***}&       0.744\sym{**} &       0.719\sym{**} \\
\addlinespace
cduration2               &     -0.0884\sym{**} &     -0.0857\sym{**} &      -0.113\sym{***}&      -0.103\sym{**} &     -0.1000\sym{**} \\
\addlinespace
WOQ, Gap, Attractiveness &      -0.161\sym{***}&                     &                     &                     &                     \\
\addlinespace
WOQ, Gap, Body Language-IT&       0.100         &                     &                     &                     &                     \\
\addlinespace
WOQ, Gap, Conscientiousness&     -0.0657         &                     &                     &                     &                     \\
\addlinespace
WOQ, Gap, EQ             &     -0.0966         &                     &                     &                     &                     \\
\addlinespace
Rule Breakers Risky      &      0.0732\sym{*}  &      0.0715\sym{*}  &      0.0880\sym{**} &      0.0747\sym{*}  &      0.0762\sym{*}  \\
\addlinespace
Rule Breakers Rockstars  &       0.133\sym{**} &       0.128\sym{**} &       0.147\sym{**} &       0.141\sym{**} &       0.140\sym{**} \\
\addlinespace
Rule Breakers Culture Add&      0.0905         &      0.0974\sym{*}  &       0.115\sym{**} &       0.112\sym{**} &       0.110\sym{**} \\
\addlinespace
Gap, Attractiveness      &                     &      -0.367\sym{***}&                     &      -0.350\sym{***}&      -0.358\sym{***}\\
\addlinespace
Gap, Body Language-IT    &                     &       0.132         &                     &       0.106         &      0.0874         \\
\addlinespace
Gap, Conscientiousness   &                     &     -0.0845         &                     &      -0.132\sym{**} &      -0.134\sym{**} \\
\addlinespace
Gap, EQ                  &                     &     -0.0952         &                     &                     &                     \\
\addlinespace
Comparative, Attractiveness&                    &                     &      -0.185\sym{*}  &                     &                     \\
\addlinespace
Comparative, Conscientiousness&                     &                     &      -0.140         &                     &                     \\
\addlinespace
Comparative, Customer Service&                     &                     &       0.138         &       0.142\sym{*}  &       0.145\sym{*}  \\
\addlinespace
Comparative, EQ          &                     &                     &     -0.0955         &                     &                     \\
\addlinespace
Comparative, Willing to Work Odd Hours&                     &                     &      -0.177\sym{*}  &      -0.255\sym{***}&      -0.260\sym{***}\\
\addlinespace
Comparative, Teamwork    &                     &                     &      -0.196\sym{*}  &      -0.242\sym{**} &      -0.251\sym{**} \\
\addlinespace
Comparative, Written Communication&                     &                     &       0.128         &      0.0920         &      0.0934         \\
\addlinespace
Comparative, Rulebreaker &                     &                     &                     &                     &      0.0182         \\
\addlinespace
Gap, Rule Breaker        &                     &                     &                     &                     &      0.0574         \\
\addlinespace
Constant                 &       5.036\sym{***}&       5.356\sym{***}&       4.755\sym{***}&       5.327\sym{***}&       5.343\sym{***}\\
\midrule
R-sqr                    &      0.3253         &      0.3539         &      0.3310         &      0.3706         &      0.3721         \\
p(F)                     &      0.0000         &      0.0000         &      0.0000         &      0.0000         &      0.0000         \\
N                        &         322         &         322         &         322         &         322         &         322         \\
\bottomrule
\multicolumn{6}{l}{\footnotesize Standard errors in parentheses}\\
\multicolumn{6}{l}{\footnotesize \sym{*} \(p<0.10\), \sym{**} \(p<0.05\), \sym{***} \(p<.01\)}\\
\end{tabular}
}

    }
    \label{tab:table_new_ols}
\end{table}

Models 3 and 4 are equivalent to models 1 and 2, respectively, after normalizing for industry, state, and firm size effects.
These effects are normalized for robustness by retaining those factors which appear in both models 1 and 2.
For example, Alabama had a significant effect when skills are measured with overqualification in Model 1.
Alabama was not significant in Model 2, so it was dropped in models 3 and 4.

Model 5 is specified as Model 4 plus two adjustments.
First, the factor for salary is dropped.
The salary factor improved adjusted explanatory power in Model 2, but it provided no such benefit in any other model.
Moreover, the p-value of this factor was unacceptably low in Model 4 ($p > 0.9$).

The second adjustment is to insert a variable for duration\footnote{
    Duration is a categorical variable which was important in both Models 1 and 2.
    As a categorical variable, it was decomposed into a boolean series for factor analysis.
    Models 1 and 2 retained one or more duration boolean factors, but none overlapped.
    As a result, duration was dropped from Models 3 and 4.
}.
% The duration factor which indicates that the respondent believes it takes more than a year to obtain an alternative credential
The belief that it takes more than a year to obtain an alternative credential
is importantly and positively associated with willingness to hire ($\beta = 0.875, p < 0.01$).
Employer effects are positively signed in all five models, but the significance is lost after normalizing effects.
This suggests that ACNG hirability is sensitive to industry, state of residence, and firm size, which are the normalized effects.

The preferred model is able to explain roughly one third of the variance in hirability.
Rulebreaker effects are identified as significant regardless of specification.
Seven perceived skill gaps are included in the preferred model.
The perceived technical skill gap is included in the preferred model.
The coefficient is statistically insignificant, but it is robust in sign across models and it does possess the expected negative sign.
The other six perceived skill gap factors in the preferred model are soft skills.
% The relative importance of soft skill gaps, and conscientiousness in particular, adds weight to a revision of the usual signalling explanation as the most plausible story.

In the classic signalling explanation of ACNG aversion, the ACNG is expected to have a deficit of concientiousness and conformity.
The explanation of ACNG aversion due to conformity selection is constrained in explanatory power because nonconformity is generally valued.
The perceived concientiousness gap is not significantly different between an ACNG and the average recent college graduate.
The data does validate that soft skills altogether are importantly different for ACNG and recent college graduate labor.
Concientiousness and body language are the two most important skill gaps in the model.
Other soft skills are less significant, but the coefficients of all soft skills in the preferred model are strictly larger than the coefficient for technical skill.
% MAYBE TODO: isn't there some chi test for this? where i say hey, one factor isn't diff but like this collection is significantly different...
% I kinda think that's just an F-test which my model does in the footnote but not sure

An important and complicated finding involves conscientiousness.
The effect is robustly positive in multiple specifications.
Simple intuition would indicate that a large conscientiousness gap is associated with reduced hirability.
A simple regression of conscientiousness on hirability does produce the expected negative coefficient.
There are two reasons for the sign change on conscientiousness in the multiple regression.
The first reason is bliss point seeking and the second is an explanation from attenuation.

As previously discussed, excess individual concientiousness can disturb team performance\cite{curcseu2019personality}.
It follows that there is some bliss point level of concientiousness.
As a result, the sign of the coefficient on concientiousness is sensitive to the relationship between the model constant and the bliss point.
If the model constant is above the bliss point, concientiousness is expected to be positively signed.
The current data is consistent with this explanation from bliss point seeking.
In a simple regression of conscientiousness in hirability, the model constant is about $7.5$, and the coefficient on conscientiousness is negative.
In the multiple regression, the model constant is near $0.5$, and the coefficient on conscientiousness is positive.

% Adding a quadratic term to Model 5 identifies a negative marginal relation between conscientiousness and teamwork skill.
% This replicates the other research which found a parabolic relationship between peer-rated team contribution and conscientiousness.
% In addition, conscientiousness exhibits a negative marginal relation with hirability overall.

% much of the below could be heavily edited or deleted pending hanson's feedback.
The second explanation is that the direct measure of conscientiousness is attenuating an overstatement of the effect in the multiple regression.
Conscientiousness is importantly cross-correlated with several factors including willingness to commute and customer service skill.
Conscientiousness also structurally interacts with rulebreaker effects.
In theory, a person that is high in concientiousness will tend not to break rules.
These factors are entered independently in the multiple regression, so a partial measure of conscientiousness is entered in redunantly.
The direct factor for conscientiousness corrects, or attenuates, the overstated effect which is represented in those the correlated independent factors.
Removal of other skill gap factors and rulebreaker factors from Model 5 demonstrates this by yielding a negative concientiousness coefficient ($\beta = -0.084, p < 0.31$).

The importance of conscientiousness does not add weight to the classic signalling explanation.
The ACNG concientiousness gap and the recent college graduate concientiousness gap are not significantly different.
% The average typical employee has a meaningfully smaller perceived concientiousness gap, but it is also not statistically significantly different
% because the sample size is just too small.
% A comparative gap of note would be that college graduates were perceived to have better technical skills compared to the average ACNG.

Employer size is an important factor in the preferred model.
The largest category of employer is positively associated with willingness to hire an ACNG.
This matches the risk aversion explanation of ACNG hirability.
% The largest category of employer has lower risk, and in fact generates comparative advantage, when hiring from a high-variance pool of candidate quality.

Some state and industrial effects are identified.
No particular relation among state effects was found, but further comparative policy research is encouraged.
With respect to industry, an interesting interaction between body language skill and employment in the information technology industry yields a positive coefficient.
Body language skill gaps on their own are associated with reduced willingness to hire.
This specifically indicates a reduced penalty for lacking body language communication skills in the information technology industry.
With less confidence and more generality, a positive coefficient to this interaction variable indicates a reduced penalty for generalized soft skill deficiency in the information technology industry.
% With less strength and more breadth, a positive coefficient to the interaction variable indicates a reduced penalty for generalized soft skill deficiency in the information technology industry.

A reduced penalty for soft skill deficit helps explain the particular flourishing of alternative credentials in the information technology industry.
The reduced penalty in this particular industry might be related to a relative lack of deregulation in the industry.
Another explanation is that the reduced penalty may be related to cultural norms in the industry.
% Alternatively, it might be related to cultural norms around the acceptability of an anti-social geek in a technology field.
% The cultural norm itself might be derived from deeper collective personality organization.
There is less technical need for social skill in programming, so introverts may obtain a comparative advantage in this field.

\subsection{Explanatory Share of Perceived Gaps}

The preferred model explains about one third of hirability variance, but how much of the explanatory power is attributable to perceived skill gaps?
Table \ref{tab:explantory_power} provides evidence on the importance of perceived skill gaps and rulebreaker effects.
This table compares the explanatory power of selected factor groups.
Industry and state effects are widely regarded as important in explaining labor outcomes.
The table shows that perceived skill gaps and rulebreaker effects are even more important in explaining hirability.
Rulebreaker effects collectively explain more than three times as much response variance as do industrial or state effects.

Allowing for overqualification seems to weaken explanatory power.
Overqualification effects seem to be heterogeneously signed per skill, so generalizing weakens overall explanatory power relative to ignoring these effects.
With overqualification, perceived skill gaps explain about fifty percent more than industrial effects or robust state effects.
Without overqualification, the adjusted explanatory power of perceived skill gaps is about three times the adjusted explanatory power of industry or state effects.
Semi-robust state effects are state effects which are significant in any multiple regression described in Table \ref{tab:table_new_ols}.
Robust state effects are the significant factors in a simple regression of semi-robust state effects on hirability.
An example of a state which appears in a multiple regression from Table \ref{tab:table_new_ols} but is not significant in a simple regression of semi-robust state effects on hirability would be [TODO].

Table \ref{tab:explantory_power} also describes the explanatory power of so-called rulebreaker effects.
Whether the candidate is perceived as a rulebreaker is a perceived skill gap, but employers evaluate this gap in a heterogenous and multispecific way.
In the first place, this heterogenous evaluation has sign and magnitude implications for the dependent variable of interest.
Secondarily, heterogenous evaluation implies a qualitatively different evaluation.
These differences are captured using three questions in the first section of the survey.

% TODO: should this table include k, the number of independent factors per factor group?
\begin{table}
    \caption{Factor Group Explanatory Power in a Simple Regression}
    \resizebox{\columnwidth}{!}{
        {
\def\sym#1{\ifmmode^{#1}\else\(^{#1}\)\fi}
% \begin{center}
{
    \fontsize{8pt}{7pt}\selectfont
    % \begin{small}
    \tabcolsep=3pt
    \begin{tabular}{l*{4}{c}}
        \toprule
        \multicolumn{1}{c}{Effect Group} & \multicolumn{1}{c}{Adj R-Sqr} & \multicolumn{1}{c}{R-Sqr} & \multicolumn{1}{c}{Max p-value} \\
        \midrule
        Absolute Gap                     & 0.0615                        & 0.0703                    & 0.097                           \\
        \addlinespace
        Comparative Gap                  & 0.0176                        & 0.0298                    & 0.687                           \\
        % \addlinespace
        % Rulebreaker                           & 0.1432                        & 0.1554                    & 0.053                           \\
        \addlinespace
        Industry                         & 0.0303                        & 0.0454                    & 0.958                           \\
        \addlinespace
        \addlinespace
        Other Factors                    & 0.0072                        & 0.0288                    & 0.537                           \\
        \addlinespace
        Rulebreaker                      & 0.0783                        & 0.0869                    & 0.127                           \\
        \addlinespace
        State                            & 0.0469                        & 0.1033                    & 0.772                           \\
        % \addlinespace
        % State, Semi-Robust                    & 0.0034                        & 0.0648                    & 0.831                           \\
        \bottomrule
    \end{tabular}
    % \end{center}
}
}

% TODO: maybe a count of k factors in group
% TODO: maybe distinguish strong and weak effects for industry, state, and gaps
% TODO: maybe other controls / other factors section doesn't matter
% TODO: maybe combine skill gaps

    }
    \label{tab:explantory_power}
\end{table}

Employer status is significantly and positively related to each of the rulebreaker effects.
% reg _ismanager1 rulebreakersnormsmightbedoingsob rulebreakersnormsprobablyhaveaha rulebreakersnormstendtobegiftedi
Rulebreaker effects are about twice as important as perceived skill gaps.
% Rulebreaker effects are a characteristic of the employer, not the job candidate.
These results provide evidence that ACNG hirability depends importantly on the way a particular employer views nonconformists.
% indicates that the skill gap in willingness to break rules is highly sensitive to the particular employer
% TODO: long paper food...it also indicates we should be testing interaction variables but meh and sample size etc
% This is not taken to be a general lesson about skill gaps.
% Willingness to break rules is a special case of a behavior which is valuable under certain parameters.
% This contrasts with something like poor communication skill which is generally detrimental.

\subsection{Perceived Gaps Compared to the Recent College Graduate}

Table \ref{tab:gap_of_gaps} provides two models of hirability based only on comparative skill gaps.
Model 6 is specified with differences in each perceived skill gap that was identified as important in Model 5.
Model 7 consolidates factors in order to optimize adjusted explanatory power.
The four factors that were eliminated in the consolidation had a p-value greater than 0.55.
This indicates all four eliminated factors were more likely not to have a relation than to have a relation.
The maximum p-value in Model 7 is less than 0.16.

\begin{table}
    \caption{Multiple Regression of Comparative Skill Gap on Hirability}
    \resizebox{\columnwidth}{!}{
        % ref: analysis-7-diff-of-gaps.do

{
\def\sym#1{\ifmmode^{#1}\else\(^{#1}\)\fi}
\begin{tabular}{l*{4}{c}}
    \toprule
                          & \multicolumn{1}{c}{Model 6} & \multicolumn{1}{c}{Model 7} & \multicolumn{1}{c}{Model 8} & \multicolumn{1}{c}{Model 9} \\
    \midrule
    Body Language         & -3.295e-01\sym{*}           & -3.395e-01\sym{*}           & 4.656e-02                   &                             \\
                          & (1.138e-01)                 & (1.049e-01)                 & (2.091e-01)                 &                             \\
    \addlinespace
    Commute               & 1.498e-01                   & 1.574e-01                   & 6.999e-02                   &                             \\
                          & (1.101e-01)                 & (1.069e-01)                 & (1.789e-01)                 &                             \\
    \addlinespace
    Conscientiousness     & 1.416e-01                   & 1.508e-01                   & 9.795e-02                   &                             \\
                          & (1.119e-01)                 & (1.061e-01)                 & (1.955e-01)                 &                             \\
    \addlinespace
    Customer Service      & -1.493e-02                  &                             &                             &                             \\
                          & (1.143e-01)                 &                             &                             &                             \\
    \addlinespace
    Technical             & 4.955e-02                   &                             &                             &                             \\
                          & (1.164e-01)                 &                             &                             &                             \\
    \addlinespace
    Teamwork              & 1.552e-02                   &                             &                             &                             \\
                          & (1.008e-01)                 &                             &                             &                             \\
    \addlinespace
    Nonconformity         & -5.822e-02                  &                             &                             &                             \\
                          & (1.108e-01)                 &                             &                             &                             \\
    \addlinespace
    Body Language$^2$     &                             &                             & 9.276e-02                   & 8.301e-02\sym{**}           \\
                          &                             &                             & (4.268e-02)                 & (2.108e-02)                 \\
    \addlinespace
    Commute$^2$           &                             &                             & -2.075e-02                  & -3.213e-02                  \\
                          &                             &                             & (3.644e-02)                 & (2.151e-02)                 \\
    \addlinespace
    Conscientiousness$^2$ &                             &                             & -2.525e-02                  & -4.187e-02                  \\
                          &                             &                             & (4.063e-02)                 & (2.227e-02)                 \\
    % \addlinespace
    % \_cons                            & 7.613e+00\sym{**}     & 7.629e+00\sym{**}     & 7.658e+00\sym{**}     & 7.661e+00\sym{**}     \\
    %                                   & (1.264e-01)           & (1.221e-01)           & (1.221e-01)           & (1.211e-01)           \\
    \midrule
    Adj R-sqr             & 0.0311                      & 0.0474                      & 0.0499                      & 0.0715                      \\
    R-sqr                 & 0.0633                      & 0.0609                      & 0.0867                      & 0.0847                      \\
    p(F)                  & 0.0610                      & 0.0044                      & 0.0046                      & 0.0004                      \\
    % N                                 & 212                   & 212                   & 212                   & 212                   \\
    \bottomrule
    \multicolumn{5}{l}{\footnotesize Standard errors in parentheses}                                                                              \\
    \multicolumn{5}{l}{\footnotesize \sym{*} \(p<.01\), \sym{**} \(p<.001\)}                                                                      \\
\end{tabular}
}


    }
    \label{tab:gap_of_gaps}
\end{table}

The preferred model demonstrates that a perceived soft skill deficit explains ACNG hireability,
but it does show that the soft skill deficit is a comparative difference between ACNG labor and the recent college graduate.
% Neither of these models is preferred, and comparative analysis per se is not the main intent of this paper.
% Instead, these models are useful to demonstrate
Models 6 and 7 demonstrate that a few soft skills do constitute a significant difference between groups.
Model 7 in combination with summary statistics on the average value of each factor also demonstrates that
the total hireability effect for the collection of these three factors is negative on average\footnote{
    % per analysis-7-diff-of-gaps.do
    The respective mean values for the comparative gaps in perceived body language skill,
    willingness to commute,
    and conscientiousness
    are -0.3395, 0.1574, and 0.1508.
    The total avarage hireability effect attributable to these effects is described by the equation:
    $Y = -0.3395*0.1415 - 0.1575*0.0943 + 0.1508*0.0330 = -0.0579$.
}.
This replicates external research which shows lower demand for ACNG labor.
% but notice that replication failure wouldn't invalidate other findings bc hireability is just for ACNG; it isn't really ACNG vs college grad hireability

Model 7 also provides a skill-level diagnostic in service of closing the demand gap.
This model indicates that the soft skill of communication using body language is the only skill difference associated with reduced hireability.
It is also the most significant and largest in magnitude of such differences.
This result should not be interpreted as a hireability difference due to generalized communication skill.
Perceived skill gaps for verbal and written communication are insignificant.

% I could talk about how the conscientiousness differential gap may be a spurious result per analysis-7-diff-of-gaps.do

\section{Conclusion}

This study provides evidence that skill signals are an important factor of hirability and are unique for the ACNG.
Perceived skill gaps do a better job of explaining willingness to hire than do other widely recognized effects including industry and state effects.
Employer factors better explain candidate hirability than do the perceived skill gaps themselves.
Technical skill gaps were identified with less relevance to the hiring decision when compared with soft skill gaps for the ACNG job candidate.

This paper provides evidence that some employers engage in conformity selection as a means of avoiding risk to the reputation, productivity, or value of a company.
% Conformity selection of this kind, however, is only a proximal explanation.
An explanation from risk aversion fully this kind of conformity selection and also explains other behavior.
Respondents were most favorable to the description of nonconformists as individuals that could just as easily be high performers as low performers.
Aversion to this kind of labor is better explained as risk aversion rather than positive selection for conformity.

Risk aversion and conformity selection are both partially unconcious biases which lead to suboptimal organizational operation.
A practical recommendation is for organizations to implement bias controls with respect to ACNG evaluation.
An example control would be to provide human resource procedures for formal evaluation of particular credentials which are relevant to specified job families.
These procedures can be immediately executed among known credentials and job families.
% Adjustments can be applied to new hires or during performance reviews for current employees.
These procedures should be retained for ongoing application as new credentials are developed and encountered over time.

Some evidence on the role of misinformation is demonstrated in a survey on trade schooling taken in 2019\cite{arabia_2019}.
Only 27 percent of respondents correctly responded that lower debt is an advantage of enrolling in trade school relative to college.
Additionally, over 75 percent of respondents failed to notice that trade school graduates receive industry employment sooner
and receive specialized training when compared to a four-year college.
% The news that employers are generally favorable to alternative credentials should be shared far and wide.
% The current education system should be reformed to better inform students about non-college career entry.
Obtaining a college degree after obtaining some work experience will allow students to leverage employer tuition benefits.
Because ACNG hirability varies importantly by particular employer, ACNG job candidates are advised to apply to a substantial number of employers at the outset of the job search.
Insight into ACNG hirability at a particular firm can be obtained prior to application through social networking with employees of the firm,
online research into the policies of the target employer,
or consulting a recruiter that specializes in the target employment industry.
% Government should emphasize job skills over the formal degree.
% Recent moves have begun such emphasis\cite{https://www.usatoday.com/story/news/politics/2020/06/26/trump-executive-order-stresses-skill-over-college-degree-hiring/3263074001/}

% Out of scope for this paper, but important:
% 1. aggregate social, legal, political, and economic movements (aggregate study is wanting, we know states, time, industry all matter)
% 2. applicant personal effects, and interviewer-applicant interaction effects
% despite those caveats, we can reasonably explain employer willingness to hire their imagined candidate based on matching effects

% Studying skill gaps provides an explanation with several gaps.
The preferred model explains about one third of willingness to hire.
Perceived skill gaps and rulebreaker effects account for most of the explanatory power in the model.
The explanatory power of this study can be meaningfully improved in a few ways.
% This study uses a cross-sectional analysis to investigate a subject that varies over time.
% The traditional system of accredited undergraduate education was itself at one time an innovation.
% Abnormal job candidates differentiated themselves using novel credentials, and broad employer adoption was achieved over time.
% This paper suggests that the alternative credentials of today will follow a similar pattern.
% also, role, firm, and industry-level supply, demand, employment, and so on...
% Dynamic analysis would improve on the cross-sectional analysis in this paper.
Longitudinal study would allow for causal analysis and improve forecasting of ACNG hirability in the future.
Other research has conducted some dynamic analysis of the same dependent variable with different regressors\cite{vandivier2020preliminary}.
Integrated analysis would be useful for replication and the generation of new models of better explanatory power.
Analysis that includes overqualification effects and heterogeneous nonlinear relations between skill gaps and hirability would improve estimates of hirability for a candidate of a particular skill profile.

% TODO: this paragraph not needed in short paper
This paper noted that large employers and the information technology industry has a peculiar suceptibility to alternative credentials,
so recent changes implemented by Google may be indicative of future trends.
Google has not required a college degree since prior to 2013\cite{bryant2013head}.
Laszlo Bock, then Senior Vice President of People Operations at Google, stated the following in 2013:
"After two or three years, your ability to perform at Google is completely unrelated to how you performed when you were in school, because the skills you required in college are very different."
In 2020, Google added three new certificate programs to an existing set and declared that all of its certificates would be treated as the equivalent of an undergraduate degree for their hiring purposes\cite{hess_2020}.

If perceived skill is representative of actual skill,
then the current study concludes that employers should be more willing to hire an ACNG.
At the same time, this paper provides evidence that perceived and actual skill levels sometimes do not align.
For example, the average recent college graduate in the sample is perceived to have better technical skill compared to the average ACNG.
This is surprising because last mile training, a kind of alternative education, has been specifically recommended in popular literature as a remedy for the technical skill gaps that exist among recent college graduates.
% Further study of the differences between perceived and actual skill would be valuable.

Employers seem to be favorable to individuals with alternative credentials.
In many cases, employer-perceived skill gaps are not statistically different when comparing recent college gradutes with ACNG candidates.
Social preference for the college degree may be better explained by public ignorance about appropriate alternative programs,
a lack of appropriate programs for certain occupations,
and government financial and other policy which gives preference to accredited education.

% Notice that the alternatively credentialed individual doesn't need the average employer to value him or her.
% He or she simply needs some significant chance of being hired, and that certainly exists.
% Moreover, the average employer is already favorable to alternative credentials.
% As more alternatively credentialed individuals are highered and promoted through society,
% there is reason to think the number of opportunities afforded to alternatively educated individuals may grow.
% The problem doesn't seem to be about whether alternative credentials work, but whether they exist in a given industrial context,
% and whether an individual would like to pay the college premium for better hirability when both options are feasible.

\bibliography{./BibFile}

\end{document}
