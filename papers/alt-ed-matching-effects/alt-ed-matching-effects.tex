% using Elseveir template per https://www.elsevier.com/authors/author-schemas/latex-instructions
\documentclass[review]{elsarticle}

\usepackage{lineno,hyperref}
\modulolinenumbers[5]
\journal{Journal of \LaTeX\ Templates}
\bibliographystyle{elsarticle-num}

\usepackage{booktabs}
% allowing images as figures
% ref: https://tex.stackexchange.com/questions/19176/how-to-insert-an-image-into-latex-document
\usepackage{graphicx}
\graphicspath{{../alt-ed-survey/figures-and-tables}}
\usepackage{hyperref}
\usepackage{threeparttable}  
\usepackage{tikz}
\usetikzlibrary{calc,matrix}

\begin{document}

\begin{frontmatter}

    \title{
        Perceived Skills Gaps in Alternative Postsecondary Education as Determinants of Hireability
        \tnoteref{titlenotes}
    }
    \tnotetext[titlenotes]{
        Go to \url{https://github.com/Vandivier/research-dissertation-case-for-alt-ed/tree/master/papers/alt-ed-survey}
        for additional materials including the online appendix,
        survey data, and data analysis source code.
    }

    \author[mymainaddress]{John Vandivier}
    \address[mymainaddress]{4400 University Dr, Fairfax, VA 22030}
    \ead{jvandivi@masonlive.gmu.edu}

    \begin{abstract}
        This paper explores X to understand Y
        This paper explores a novel data set (n = 1190) to understand trends in public
        disposition toward alternative postsecondary learning, with a focus on employers.
        Results indicate that public favorability is positive and will remain flat over the next year.
        Employer attitudes are not meaningfully different from the general public.
    \end{abstract}

    \begin{keyword}
        education economics, alternative education, candidate fit, job fit, candidate matching     %%% not grammatical
        \MSC[2010] I21, I22, J20                                                                   %%% not grammatical
    \end{keyword}

\end{frontmatter}

\pagebreak
\linenumbers

\section{Introduction and Description of Data}

% a missing link for future research: favorability only correlates with actual hiring decisions it isn't a hiring decision
Hiring decisions reflect boundedly rational demand for skilled labor.
The college degree and alternative credentials provide two qualitatively distinct signals of skilled labor.
The hireability of individuals in possession of these credentials has been studied,
but the underlying determinants are not clear.
This paper hypothesizes that perceived skill gaps are important determinants of willingness to hire.
This paper further hypothesizes that perceived skill gaps are qualitatively different between college graduates and others.
% In particular, this paper hypothesizes that a noncollege stigma is obtained for candidates without a degree in pursuit of roles typically filled by degree holders.
% soft skill bias in particular

Alternative credentials refer to credentials other than the undergraduate degree\cite{brown2017complex}.
The category includes industry certifications,
vocational schooling, noncollege courses,
and even portfolios produced during self-study.
Individuals pursuing alternative credentials typically intend to leverage the credential in order to obtain employment.
That is, they typically have the same ends as those pursuing traditional education.

% Alternative credentials can be obtained quickly and cheaply relative to college.
Obtaining a college degree signals intelligence, conscientiousness, and conformity,
but it may not signal technical skill\cite{horton_2020}.
Alternative credentials signal technical skill.
As such, they provide an effective supplement to the degree.

This paper is concerned with another use case in which the degree is entirely substituted.
In that situation, employers may apply a noncollege stigma.
This is particularly the case for roles which are typically occupied by degree holders.
Noncollege stigma is a presumption, expectation, or bias toward perception of a skill gap of a certain kind.
Whether the gap exists in fact is out of the scope of the present paper.

Technical skill generally implies intelligence.
Alternative credentials, then, fail to signal two qualities compared to the degree.
Alternative credentials fail to signal conscientiousness and conformity.
Interestingly, some employers may demand some level of nonconformity.
Employers may also presume a certain lack of soft skill on the part of highly technical applicants.
% Finally, employers may use alternative credentials as a proxy for other employee characteristics like income, education, race, and gender.

% three interesting follow-on questions:
%   1. do employers have such a bias
%   2. is such a soft skills gap presumption actually true
%   3. if true, due employers overvalue the soft skills gap
% related paradox: most people won't be in a job for 4-5 years,
% so why do they need to show concientiousness and conformity towards the 4-5 year bachelor's goal line?
% hard skill stigma: in my experience, people who are highly technical are hard to work with
% soft skill bias: I am favorable bc i think u have soft skills (and maybe this is efficient...enter eq/iq discussion)

\subsection{Process Explanations of Suboptimal Wages}

Basic price theory holds that an employer should pay wages equal to the marginal revenue product of labor.
In the real world, measuring candidate productivity at hiring time is costly and imperfect.
% This produces a technical error which assumes alignment between the goals of the firm and a hiring team.
A further issue is identified when the hiring team is scrutinized for principal-agent problems.
The hiring team is composed of individuals with preferences, calculative limitations, and other biases.
Monitoring and correcting for these problems is expensive,
so firms will heterogenously realize some aggregation of these individual definiciencies.

Exacerbating the already necessarily imperfect hiring process are candidate-side problems.
Firms must hire among a finite, potentially small, number of candidates.
Risk aversion to time expense and other search costs may lead a firm to approve a suboptimal candidate\cite{simon1976substantive}.
In some cases, candidate pools may be systematically problematic.
In law and medicine, for example, extensive education and training are legally required.
These policies further restrict the candidate pool, inflate expected wages, and systematically alter the content of education in a politically-motivated manner.
Market forces implement hiring as a lumpy expenditure process to begin with, but certification requirements, wage regulation, and other policies extend the problem.

Prior discussion highlights the many locations of hiring process inefficiency.
The practical importance of magnitudes and kinds of such effects are described in a legion of related papers.
A meager sampling of five such effects would include the attractiveness effect,
disability stigma,
agentic behavior stigma,
and complex biases related to communication style.
Finally, there are a wealth of concerns about the new role of social media.
For one, it presents a channel for the revival of religious discrimination.


% related papers on hireability are legion:
% https://scholar.google.com/scholar?start=0&q=hireability&hl=en&as_sdt=0,47
% disability stigma and how to combat it: https://onlinelibrary.wiley.com/doi/full/10.1111/epi.13619
% agentic behavioral stigma: https://psycnet.apa.org/record/2009-06359-002
% interview non-verbal behaviors and communication style effects: https://eric.ed.gov/?id=ED238071
% attractiveness and interaction effects https://www.ingentaconnect.com/content/sbp/sbp/1986/00000014/00000001/art00014
% communication, language, accent is valued and intermingled with that are stigma around non-native english speakers and anti-foreign bias https://theses.ubn.ru.nl/handle/123456789/7949
%
% self-similarity doesn't hold for attractiveness https://digitalcommons.bucknell.edu/honors_theses/18/
% 
% social media provides an additional channel for previously stated hiring concerns (attractiveness, communication style)
% https://scholar.utc.edu/rcio/2016/sessions/18/
% https://scholarspace.manoa.hawaii.edu/handle/10125/41424
% but social media channels provide many other signals are used in many complex ways: 
%
% perhaps attractiveness is more important in marketing roles ?

% totally unrelated, but feels true: academics may exhibit pro-complexity bias which makes them overestimate plausibility of implausible explanations
%               could vary by field.
%
% what are my questions?
% 1. demand for technical skills
% certification or degree required by law, required by corporate policy, not required
%     1. rate for: level I desire, level I expect from a college grad (who may or may not have a degree in the subject), level I expect from a bootcamp grad (who may or may not have a college degree)
% 2. demand for heterogenous (5-8) non-technical skills and qualities
%     1. attractiveness (look for marketing industry effect)
% 3. firm size effects (principal-agent)
% 4. job title / industry
% 5. wages (and replacement cost assumed to be some monotonically increasing function thereof)
% 6. favorability toward alternative education: asked first or last effect; two sample populations
% 7. about employer's gender
%       a. one of the most stable effects which is important and significant
%       b. allows us to compare matching effects to others in a limitted multi-regressy ish meh kinda way
%       c. (could ask a `check all of these true for you' to get more cheap info...up to 10 check boxes here...is college educated, is minority, is an immigrant or child thereof)

% matching effects as gender norm compliance: https://onlinelibrary.wiley.com/doi/abs/10.1111/j.1471-6402.1981.tb01098.x
% meta-accuracy is a kind of matching measure and it is positive with hireability: https://journals.aom.org/doi/abs/10.5465/AMBPP.2018.13955abstract

\section{Description of Data}





% there can be a large set of particular candidate types that may fit a job,
% and in such case the assessed value of candidate properties which are acceptibly subject to variation should be considered less important.
% proper low importance survey response might fail to obtain if survey respondents are improperly anchoring.
% That is, they have a single candidate type in mind, although many types might fit the job.
% questions: an ideal candidate, myself, someone with a college degree, someone without a college degree but with an alternative credential



% in a trivial economic model, employers pay for employee productivity
% a slightly more complicated story acknowledges that measuring productivity is accomplished through a finite set of processes with heterogeneous cost, error, and bias.
% an even more complicated story says that companies will only pay to reduce the error or bias they observe, and that is itself a subset of total error or bias.
% organizations with relatively long authority chains for hiring decisions face multi-layered principal-agent concerns which further decouple productivity from firm willingness to pay [large corporation effect]
% hiring error awareness increases in a few ways
%  1. [passive search] participant observation. As an interviewer, interviewee, hiring manager, or other professional involved in the process, I simply notice a problem
%  2. [passive search] passive company and individual level search into HR best practices; an industry newsletter says hey Griggs v Duke happened so don't use IQ tests anymore.
%  3. [passive search]: audit compliance (legal+required, or optional audits from firms that certify quality, for example)
%       example: Supreme Court case Griggs v Duke had an industry-wide effect thru this means
%  3. [active search] intrapraneurship / policy change championing begins with an individual saying hey let's investigate this thing. what would motivate such individual? (maybe due to 1 or 2).
%
% because the above means are imperfect, some share of both random error and also systematic bias are expected to exist
% such error and bias are expected to manifest in a way that is heterogenous and multi-specific by employer
% this means that even if there is some general alternative education stigma, it is expected to vary in magnitude by employer, and might even turn negative for some employers.
% this altogether yields hope to the alternatively educated candidate; it is likely that some employer will hire them (perhaps, ironically, even if the candidate is unskilled)
%
% my prior work has shown that we can predict (r2 0.5 - 0.7 and ar2 0.3 - 0.6) alternative education favorability from employer factors alone - without concern to matching
% how much does matching explain (caveat: not multiply regressed, so matching effect is likely overstated in this paper, and possibly partially partialled-in to prior work)
%
% this paper investigates the share of explanatory power in hireability among a few effects:
% 1. demand for technical skills
% 2. demand for heterogenous (5-8) non-technical skills and qualities
% 3. firm size effects (principal-agent)
% 4. job title / industry
% 5. wages (and replacement cost assumed to be some monotonically increasing function thereof)
%
%
% in addition to the above, here are some other things that matter:
% 1. aggregate social, legal, political, and economic movements (aggregate study is wanting, we know states, time, industry all matter)
% 2. applicant personal effects, and interviewer-interviewee interaction effects
%
%
% despite those caveats, we can reasonably explain employer willingness to hire their imagined candidate based on matching effects

\section{Results}



\section{Conclusions}

Notice that the alternatively credentialed individual doesn't need the average employer to value him or her.
He or she simply needs some significant chance of being hired, and that certainly exists.
Moreover, the average employer is already favorable to alternative credentials.
As more alternatively credentialed individuals are highered and promoted through society,
there is reason to think the number of opportunities afforded to alternatively educated individuals may grow.
The problem doesn't seem to be about whether alternative credentials work, but whether they exist in a given industrial context,
and whether an individual would like to pay the college premium for better favorability when both options are feasible.



\bibliography{./BibFile}

\end{document}
